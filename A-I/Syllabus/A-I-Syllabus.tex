\documentclass[a4paper, 12pt]{article}

\usepackage[T1,T2A]{fontenc}
\usepackage[utf8]{inputenc}
\usepackage[english, russian]{babel}
\usepackage{fullpage}
\usepackage{amssymb}
\usepackage{amsmath}
\usepackage{enumitem}
\usepackage{indentfirst}
\usepackage{titlesec}
\usepackage{hyperref}
\usepackage{tabularx}
\usepackage{longtable}
\usepackage{multirow}
\usepackage{makecell}

\titleformat*{\section}{\large\bfseries\scshape}

\hypersetup{
     colorlinks,
     citecolor=black,
     filecolor=black,
     linkcolor=black,
     urlcolor=black
 }

\setcounter{tocdepth}{2}
\setcounter{secnumdepth}{2}

%reduce spaces before eqs
\expandafter\def\expandafter\normalsize\expandafter{%
    \normalsize%
    \setlength\abovedisplayskip{2pt}%
    \setlength\belowdisplayskip{2pt}%
    \setlength\abovedisplayshortskip{-2pt}%
    \setlength\belowdisplayshortskip{2pt}%
}

%declare and redeclare math operators
%TBD
%\DeclareMathOperator{\ord}{ord}
\DeclareMathOperator{\chr}{char}
\DeclareMathOperator{\Mod}{mod}
\DeclareMathOperator{\Gal}{Gal}
\DeclareMathOperator{\N}{N}
\DeclareMathOperator{\Tr}{Tr}
\DeclareMathOperator{\SL}{SL}

\begin{document}

\begin{titlepage}

    \centering

    Федеральное государственное бюджетное образовательное учреждение\\
    высшего образования \\
    Московский государственный университет имени М.В. Ломоносова\\
    Факультет вычислительной математики и кибернетики

    \vspace{24pt}

    \begin{flushright}
        \textbf{
        УТВЕРЖДАЮ\\
        декан факультета вычислительной\\
        математики и кибернетики\\
        \vspace{12pt}
        \underline{\hspace{2.5cm}} /И.А. Соколов/\\
        \flqq \underline{\hspace{1cm}}\frqq \underline{\hspace{3cm}} 2024 г.}
    \end{flushright}

    \vspace{24pt}

    \textbf{\textsc{Рабочая программа дисциплины}}

    \vspace{8pt}

    \textbf{Введение в теорию чисел}

    \underline{\hspace{15cm}}

    \vspace{24pt}

    \textbf{Уровень высшего образования:\\
    магистратура}

    \vspace{16pt}

    \textbf{Направление подготовки / специальность:\\
    01.04.02 "Прикладная математика и информатика" (3++)}

    \vspace{16pt}

    \textbf{Направленность (профиль) ОПОП:\\
    Кибербезопасность}

    \vspace{16pt}

    \textbf{Форма обучения:\\
    очная}

    \vspace{48pt}

    \begin{flushright}

        Рабочая программа рассмотрена и утверждена\\
        на заседании  Ученого совета факультета ВМК\\
        (протокол №\underline{\hspace{0.5cm}} от \underline{\hspace{3cm}})

    \end{flushright}

\vspace*{\fill}

Москва 2025

\end{titlepage}

Рабочая программа дисциплины (модуля) разработана в соответствии с самостоятельно разрабатываемым образовательным стандартом МГУ имени М.В. Ломоносова для реализуемых основных профессиональных образовательных программ высшего образования по направлению подготовки 01.04.02 \flqq Прикладная математика и информатика\frqq (3++).

\newpage \tableofcontents

\newpage

\section{Место дисциплины (модуля) в структуре ОПОП ВО}

Настоящая дисциплина включена в учебный план по направлению 01.04.02 \flqq Прикладная математика и информатика\frqq (3++), профиль Кибербезопасность и входит в Базовую часть программы (Модуль \flqq Программное обеспечение современных вычислительных систем\frqq).
Дисциплина является кафедральным (вариативным)  курсом и изучается по выбору студента.
Дисциплина рассчитана на студентов, знакомых с основными понятиями и результатами линейной алгебры, действительного и комплексного анализа, а также владеющих основами языка программирования Python.

\section{Цели и задачи дисциплины}

Курс является введением в современную теорию чисел с акцентом на приложения. Даётся теоретический материал, необходимый для последующего изучения прикладных курсов по криптографии и теории кодирования, а также продвинутых курсов по современной теории чисел, алгебраической геометрии и алгебраической топологии. Теоретический материал подкрепляется вычислительными задачами и примерами с использованием системы компьютерной алгебры SageMath. Курс не является введением в криптографию и теорию кодирования, хотя на семинарах и рассматриваются некоторые криптографические схемы и элементы теории кодов. Курс также не является введением в вычислительные алгоритмы теории чисел и алгебры.

\section{Результаты обучения по дисциплине (модулю)}

Компетенции выпускников, частично формируемые при реализации дисциплины (модуля):

{\bf Содержание и код компетенции}
\begin{itemize}[noitemsep,topsep=0pt]
    \item {\bf ОПК-1.} Способность формулировать и решать актуальные задачи в области фундаментальной и прикладной математики.
    
    \item {\bf ОПК-4.} Способность комбинировать и адаптировать современные информационно-коммуникационные технологии для решения задач в области профессиональной деятельности с учетом требований информационной безопасности.
    
    \item {\bf ПК-2.} Способность в рамках задачи, поставленной специалистом более высокой квалификации, проводить научные исследования и (или) осуществлять разработки в области прикладной математики и информатики с получением научного и (или) научно-практического результата;
    
    \item {\bf СПК-ВТЧП-1М.} Способность формулировать и решать задачи в области теории чисел и её приложений, используя современные информационно-комму\-никационные технологии.
\end{itemize}

{\bf Индикатор (показатель) достижения компетенции}
\begin{itemize}[noitemsep,topsep=0pt]
    \item {\bf СПК-ВТЧП-1М.1.} Владение основными понятиями и методами элементарной теории чисел
    \item {\bf СПК-ВТЧП-1М.2.} Владение основными понятиями и результатами связанными с конечными полями и многочленами на ними
    \item {\bf СПК-ВТЧП-1М.3.} Знакомство с основными понятиями и методами алгебраической теории чисел (числовые поля и p-адические числа)
    \item {\bf СПК-ВТЧП-1М.4.} Знакомство с методами аналитической теории чисел (ряды Дирихле и оценки тригонометрических сумм)
    \item {\bf СПК-ВТЧП-1М.5.} Владение системой компьютерной алгебры SageMath для решения задач теории чисел и алгебры
    \item {\bf СПК-ВТЧП-1М.6.} Понимание приложений теории чисел для задач криптографии
    \item {\bf СПК-ВТЧП-1М.7.} Понимание приложений теории чисел для задач теории кодирования
\end{itemize}

{\bf Планируемые результаты обучения по 
дисциплине, сопряженные с индикаторами   
достижения компетенций}
\begin{itemize}[noitemsep,topsep=0pt]
    \item {\bf Знать}
    \begin{itemize}[noitemsep,topsep=0pt]
        \item Основные понятия, определения и результаты элементарной теории чисел
        \item Основные понятия, определения и результаты теории конечных полей
        \item Основные понятия, определения и результаты алгебраической теории чисел
        \item Основные направления приложений теории чисел в криптографии и теории кодирования
    \end{itemize}
    \item {\bf Уметь}
    \begin{itemize}[noitemsep,topsep=0pt]
        \item Решать задачи теории чисел, используя элементарные, комбинаторные,  аналитические и алгебраические методы
        \item Применять системы компьютерной алгебры и символьных вычислений для решения задач алгебры и теории чисел
        \item Применять методы теории чисел к формализации постановок прикладных задач, включая криптографию и теорию кодирования
    \end{itemize}
    \item {\bf Владеть}
    \begin{itemize}[noitemsep,topsep=0pt]
        \item Навыками работы в системе компьютерной алгебры SageMath
    \end{itemize}
\end{itemize}

\section{Формат обучения и объём дисциплины (модуля)}

Формат обучения: занятия проводятся с использованием меловой или маркерной доски, интерактивные материалы демонстрируются с помощью ноутбука и проектора.

Объем дисциплины (модуля) составляет 136 академических часов, в том числе 68 академических часов, отведенных на контактную работу обучающихся с преподавателем, 68 академических часов на самостоятельную работу обучающихся.


\section{Содержание дисциплины (модуля), структурированное по темам (разделам) с указанием отведённого на них количества академических часов и видов учебных занятий}

\subsection{Структура дисциплины (модуля) по темам (разделам) с указанием отведенного на них количества академических часов и виды учебных занятий (в строгом соответствии с учебным планом)}

\noindent
\begin{longtable}{ | >{\raggedright}p{5cm} | p{1.5cm}| p{1.5cm} | p{1.5cm} | p{1.5cm} | p{2cm} | } 
    \hline
     & \multicolumn{3}{c|}{\textbf{\makecell[c]{Номинальные трудо-\\затраты обучающегося,\\ академические часы}}} & & \\
    \hline
    \textbf{Наименование разделов и тем дисциплины (модуля), Форма промежуточной аттестации по дисциплине (модулю)} & Кон\-такт\-ная работа, занятия лекционного типа & Кон\-такт\-ная работа, занятия семинарского типа & Самос\-тоятель\-ная работа обучающегося &  \textbf{Всего академических часов} & \textbf{Форма текущего контроля успеваемости* (наименование)} \\
    \hline
    \hline
    \multicolumn{6}{|l|}{\textit{Раздел 1. Элементарная теория чисел}} \\ \hline
    Тема 1. Простые числа & 2 & 2 & 4 & 8 & \\ \hline
    Тема 2. Сравнения & 2 & 2 & 4 & 8 & \\ \hline
    Тема 3. Первообразные корни & 2 & 2 & 4 & 8 & \\ \hline
    Тема 4. Квадратичные вычеты & 2 & 2 & 4 & 8 & \\ \hline
    \multicolumn{6}{|l|}{\textit{Раздел 2. Конечные поля и тригонометрические суммы}} \\ \hline
    Тема 5. Конечные поля. Расширения полей & 2 & 2 & 4 & 8 & \\ \hline
    Тема 6. Группа автоморфизмов. Норма и след & 2 & 2 & 4 & 8 & \\ \hline
    Тема 7. Корни из единицы. Круговой многочлен & 2 & 2 & 4 & 8 & \\ \hline
    Тема 8. Характеры. Суммы Гаусса & 2 & 2 & 4 & 8 & \\ \hline
    Тема 9. Тригонометрические суммы. Уравнения над конечными полями & 2 & 2 & 4 & 8 & \\ \hline
    Тема 10. Дзета-функция Артина & 2 & 2 & 4 & 8 & \\ \hline
    \multicolumn{6}{|l|}{\textit{Раздел 3. p-адические числа}} \\ \hline
    Тема 11. p-адические числа: элементарное определение и свойства & 2 & 2 & 4 & 8 & \\ \hline
    Тема 12. Аксиоматическое определение поля p-адических чисел, метризованные поля & 2 & 2 & 4 & 8 & \\ \hline
    Тема 13. Лемма Гензеля, сравнения и кольцо целых p-адических чисел & 2 & 2 & 4 & 8 & \\ \hline
    \multicolumn{6}{|l|}{\textit{Раздел 4. Числовые поля}} \\ \hline
    Тема 14. Кольцо целых гауссовых чисел, числовые поля & 2 & 2 & 4 & 8 & \\ \hline
    Тема 15. Делимость в кольцах целых алгебраических чисел & 2 & 2 & 4 & 8 & \\ \hline
    Тема 16. Квадратичное поле и круговое поле & 2 & 2 & 4 & 8 & \\ \hline
%    \multicolumn{6}{|l|}{\textit{Раздел 5. Теорема Дирихле}} \\ \hline
%    Тема 16. Ряды Дирихле & 2 & 2 & 4 & 8 & \\ \hline
%    Тема 17. Распределение простых чисел в арифметической прогрессии & 2 & 2 & 4 & 8 & \\ \hline
    \multicolumn{5}{|l|}{\textbf{Итоговая аттестация}} & \textbf{зачет} \\ \hline
    \textbf{Итого, академические часы} & \textbf{32} & \textbf{32} & \textbf{64} & \textbf{128} & \\ \hline
\end{longtable}

\subsection{Содержание разделов (тем) дисциплины}

Курс состоит из четырёх разделов. Первый раздел является введением и посвящен элементарной теории чисел. Рассматриваются свойства делимости в кольце целых рациональных чисел, изучается кольцо классов вычетов, исследуются свойства основных арифметических функций, рассматриваются вопросы разрешимости некоторых сравнений и диофантовых уравнений.

Во втором разделе изучается один из основных объектов теоретико-числовых приложений – конечные поля. Работа с многочленами над конечными полями также является ключевым навыком для приложений в криптографии и теории кодирования. Вводится понятие характера конечной абелевой группы, доказываются некоторые факты о полных и неполных тригонометрических суммах. В последней лекции второго раздела затрагиваются вопросы алгебраической геометрии над конечными полями: вводится понятие дзета-функции алгебраической поверхности над конечным полем, на примере одной гиперповерхности элементарными методами доказывается рациональность соответствующей дзета-функции.

Третий раздел посвящен введению в p-адические числа и метризованные поля. p-адический анализ является важным инструментом современной криптографии, а область p-адических динамических систем имеет много других приложений. В первой лекции раздела дается прямое построение кольца целых p-адических чисел, рассматриваются его основные арифметические свойства, изучаются вопросы сходимости p-адических последовательностей и рядов. Далее рассматривается общий случай метризованных полей и их пополнений. Доказывается теорема Островского о классификации всех пополнений поля рациональных чисел. Разбирается несколько случаев леммы Гензеля, которая может быть использована в качестве инструмента для решения алгебраических уравнений в p-адических числах и сравнений. Формулируется теорема Минковского-Хассе и локально-глобальный принцип (принцип Хассе).

В четвертом разделе рассматриваются некоторые вопросы числовых полей и колец. Порядки и кольца алгебраических чисел играют важную роль при изучении теории решеток, а также арифметики эллиптических кривых. Доказывается однозначность разложения на множители в кольце гауссовых чисел. Даются примеры колец целых алгебраических чисел, в которых разложение на простые множители не является однозначным. Вводится понятие евклидовых и дедекиндовых колец. Обзорно приводятся результаты приложения теории Галуа к вопросам делимости в порядках числовых полей. Исследуется связь с представимостью чисел суммами квадратов и разрешимостью некоторых диофантовых уравнений. Более подробно изучаются случаи квадратичного и кругового поля. Вводится понятие алгебры кватернионов.

%Заключительный пятый раздел выступает введением в аналитическую теорию чисел. Дается определение характеров и L-функций Дирихле, доказывается теорема Дирихле о распределении простых чисел в арифметических прогрессиях. Вводится понятие дзета-функции Дедекинда, Обзорно приводятся результаты для числа классов идеалов числового поля.

\vspace{8pt}

\noindent
\begin{longtable}{ | >{\raggedright}p{6cm} | p{9cm} | } 
    \hline
    \textbf{Наименование разделов (тем) дисциплины} & \textbf{Содержание разделов (тем) дисциплин} \\
    \hline
    \hline
    \textit{Раздел 1. Элементарная теория чисел} & \textit{Источники: \cite{Vin}, главы 1-5; \cite{IR}} \\ \hline
    Тема 1. Простые числа & Делимость в кольце целых чисел, НОД, НОК, Алгоритм Евклида. Однозначное разложение целых чисел на простые множители. Бесконечность числа простых чисел. Мультипликативные функции, функции Эйлера и Мёбиуса. Формула обращения Мёбиуса. Неравенства Чебышева. \\ \hline
    Тема 2. Сравнения & Сравнения и их свойства. Полная и приведенная система вычетов. Число решений линейного сравнения. Теоремы Эйлера и Ферма. Китайская теорема об остатках. Повторение основных определений алгебры: группа, кольцо, идеал, фактор-группа и фактор-кольцо, поле. Кольцо классов вычетов (определения и свойства выше на языке колец). Делители нуля. Кольцо многочленов над любым полем. Однозначность разложения в кольце многочленов. \\ \hline
    Тема 3. Первообразные корни & Теорема Лагранжа (число корней многочлена над произвольным полем не превосходит степени многочлена). Группа единиц кольца вычетов по модулю простого. Структура группы единиц кольца вычетов по произвольному модулю. Первообразные корни и индексы. Степенные вычеты. \\ \hline
    Тема 4. Квадратичные вычеты & Квадратичные вычеты и невычеты. Символ Лежандра, символ Якоби, их свойства. Теорема Вильсона и ее обобщение. Лемма Гаусса. Квадратичный закон взаимности. Равномерное распределение последовательностей mod n (обзорно). \\ \hline
    \textit{Раздел 2. Конечные поля и тригонометрические суммы} & \textit{Источники: \cite{Step}, главы I-II; \cite{IR}, главы 7-8; \cite{Serre}, глава I; \cite{LN}, главы 2,5; \cite{BSh}, \S\S I.1-I.2, АД.2-АД.3} \\ \hline
    Тема 5. Конечные поля. Расширения полей & Конечное поле $\mathbb{F}_q$ как фактор-кольцо кольца многочленов $\mathbb{F}_p[x]$. Количество неприводимых многочленов над $\mathbb{F}_p$. Цикличность мультипликативной группы $\mathbb{F}_q^*$. Конечные расширения полей, cвойства расширения $\mathbb{F}_q/\mathbb{F}_p$. \\ \hline
    Тема 6. Группа автоморфизмов. Норма и след & Подполя конечных полей. Автоморфизм Фробениуса, цикличность группы Галуа конечного поля. Соответствие Галуа для башни конечных полей. Норма и след для произвольных конечных расширений полей. Дискриминант базиса расширения. Свойства нормы и следа в случае конечных полей.  \\ \hline
    Тема 7. Корни из единицы. Круговой многочлен & Круговой многочлен, круговое поле. Цикличность группы корней из единицы. Примитивные корни из единицы. Произведение круговых многочленов. Разложение кругового многочлена над конечным полем. Число неприводимых многочленов над конечным полем. Факты о неприводимых многочленах. Критерий неприводимости Эйзенштейна. \\ \hline
    Тема 8. Характеры. Суммы Гаусса & Характеры конечных абелевых групп. Двойственная группа. Свойства ортогональности. Аддитивные и мультипликативные характеры конечного поля. Суммы Гаусса, обобщённые суммы Гаусса. Соотношение Хассе-Дэвенпорта. \\ \hline
    Тема 9. Тригонометрические суммы. Уравнения над конечными полями & Связь тригонометрических сумм с числом решений уравнений в конечных полях. Суммы Якоби и их приложения для числа решений уравнений. Теоремы Варинга и Шевалле. Суммы характеров. Неполные суммы характеров. Теорема Виноградова-Пойя. \\ \hline
    Тема 10. Дзета функция Артина & Элементы алгебраической геометрии над конечным полем: аффинное и проективное пространства, уравнения как гиперповерхности. Дзета функция Артина, критерий её рациональности.  Дзета функция гиперповерхности заданной формой $a_0 x_0^m + \dots + a_n x_n^m$ и её рациональность. \\ \hline
    \textit{Раздел 3. p-адические числа} & \textit{Источники: \cite{BSh}, глава I; \cite{Serre}, глава II; \cite{Gouv}, главы 1-4]} \\ \hline
    Тема 11. $p$-адические числа: элементарное определение и свойства & Элементарное определение кольца и поля $p$-адических чисел. Единицы кольца $\mathbb{Z}_p$. $p$-показатель и $p$-адическая метрика. Свойства вложения $\mathbb{Q}$ в $\mathbb{Q}_p$. Делимость и сравнения в $\mathbb{Z}_p$. Сходимость последовательностей и рядов $p$-адических чисел. \\ \hline
    Тема 12. Аксиоматическое определение поля $p$-адических чисел, метризованные поля & Метризованные поля, пополнение по метрике, теорема о существовании и единственности пополнения метризованного поля (без доказательства). Эквивалентность метрик. Метрики поля рациональных чисел, Теорема Островского. Некоторые топологические свойства связанные с неархимедовыми метриками.  \\ \hline
    Тема 13. Лемма Гензеля, сравнения и кольцо целых $p$-адических чисел & Несколько вариантов формулировки Леммы Гензеля. Связь разрешимости сравнений по модулю степени простого с разрешимостью в поле $p$-адических чисел. Оценка числа решений сравнения по модулю степени простого. Теорема Минковского-Хассе и локально-глобальный принцип (обзорно). \\ \hline
    \textit{Раздел 4. Числовые поля} & \textit{Источники: \cite{IR}, главы 9,12-13; \cite{BSh}, глава III; \cite{Marc}, главы 2-5; \cite{Cox}, \S I.4, \S II.7; \cite{DSV}, глава 2} \\ \hline
    Тема 14. Кольцо целых гауссовых чисел, числовые поля & Однозначность разложения на множители в кольцах $\mathbb{Z}[i]$, $\mathbb{Z}[\omega]$. Кубический и биквадртичный характеры. Высшие законы взаимности. Суммы двух квадратов, суммы четырёх квадратов. Кватернионы. \\ \hline
    Тема 15. Делимость в кольцах целых алгебраических чисел & Числовое поле. Норма, след, дискриминант. Порядки и кольца алгебраических целых чисел, Дедекиндовы кольца. Однозначность разложения в кольце алгебраических целых чисел. Элементы теории Галуа числовых полей (обзорно). Применение теории Галуа к разложению на простые идеалы в башнях полей и колец. \\ \hline
    Тема 16. Квадратичное поле и круговое поле & Квадратичное числовое поле $\mathbb{Q}(\sqrt{d})$ и его кольцо целых. Порядки в квадратичном поле. Разложимость на простые идеалы. Связь с квадратичными формами. Число классов идеалов квадратичного поля (обзорно). Круговое поле $\mathbb{Q}(\zeta_m)$, простые идеалы соответствующего кольца целых. Приложения для кругового многочлена. \\ \hline
%    \textit{Раздел 5. Теорема Дирихле} & \textit{Источники: \cite{Serre}, глава VI; \cite{IR}, глава 16, \cite{BSh}, глава V; \cite{Marc}, главы 7-8} \\ \hline
%    Тема 16. Ряды Дирихле & Характеры Дирихле. Ряды Дирихле. Аналитическое продолжение. Дзета функция Римана. Асимптотический закон распределения простых чисел (обзорно). \\ \hline
%    Тема 17. Распределение простых чисел в арифметической прогрессии & Теорема Дирихле о распределении простых чисел в арифметической прогрессии. L-функции и производящие функции. Дзета функция Дедекинда. Связь с числом классов идеалов числового поля. \\ \hline
    
\end{longtable}

\subsection{Примеры задач для семинаров и самостоятельной работы \label{problems}}

\begin{center} {\bf Тема 1. Простые числа} \end{center}

\noindent\textbf{Упражнения и задачи}

\begin{enumerate}[topsep=0pt]
    \item Докажите свойства делимости: 
    \begin{itemize}[noitemsep,topsep=0pt]
        \item $a | a$, $a \neq 0$;
        \item $a|b, b|a \implies a=\pm b$;
        \item $a|b, b|c \implies a|c$;
        \item $a|b, a|c \implies a| b\pm c$.
    \end{itemize}
    \item Алгоритм Евклида: Пусть $a,b \in\mathbb{Z} \setminus \{0\}, a>b$, определим последовательность $b>r_1>r_2>\dots > r_n$ следующим образом: $a=bq_0+r_1$, $b=r_1q_1+r_2$, $r_1=r_2q_2+r_3$, \dots $r_n=r_{n-1} q_{n-1}+r_n$. Докажите, что $\exists n: r_{n-1}=r_nq_n$ и $r_n=(a,b)$. (Сначала докажите , что $(a,b)=(b,r_1)$).
    \item Докажите, что $\sqrt{2}$ --- иррациональное число, т.е. что не $\exists$ рационального $r=a/b$ ($a,b \in \mathbb{Z}$) такого, что $r^2=2$.
    \item Пусть $\alpha \in \mathbb{R}$, $b\in\mathbb{Z}_+$, докажите, что $\left[\dfrac{[\alpha]}{b}\right]=\dfrac{\alpha}{b}$.
    \item Пусть $(a,b)=1$, докажите, что $(a+b,a-b)=1$ или $2$.
    \item Пусть $a,b,c \in \mathbb{Z}$, докажите что уравнение $ax+by=c$ разрешимо в целых числах $\iff$ $d=(a,b)|c$. Докажите также, что если $x_0,y_0$ --- решение этого уравнения, то все решения имеют вид $x=x_0+t\dfrac{b}{d}$, $y=y_0-t\dfrac{b}{d}$, где  $t \in \mathbb{Z}$.
    \item Докажите следующие свойства:
    \begin{itemize}[noitemsep,topsep=0pt]
        \item $\nu_p([a,b])=\max(\nu_p(a),\nu_p(b))$;
        \item $\nu_p(a+b)\geqslant \min(\nu_p(a),\nu_p(b))$, причем $\nu_p(a+b)= \min(\nu_p(a),\nu_p(b))$, если $\nu_p(a) \neq \nu_p(b)$;
        \item $(a,b)[a,b]=ab$;
        \item $(a+b,[a,b])=(a,b)$.
    \end{itemize}
    \item Докажите, что $\nu_p(n!)= \left[\frac{n}{p}\right] + \left[\frac{n}{p^2}\right] + \left[\frac{n}{p^3}\right] + \dots $
    \item Пусть $a,b,c,d \in \mathbb{Z}$, $(a,b)=1$, $(c,d)=1$. Докажите, что если $\dfrac{a}{b}+\dfrac{c}{d} \in \mathbb{Z}$, то $b=\pm d$.
    \item Пусть $n \in \mathbb{Z}$, $n>2$ Докажите, что числа
    $$
    \dfrac{1}{2}+\dfrac{1}{3}+\dots+\dfrac{1}{n};\ 
        \dfrac{1}{3}+\dfrac{1}{5}+\dots+\dfrac{1}{2n+1}
    $$
    не являются целыми.
    \item Пусть $f(n)$ --- мультипликативная функция. Докажите, что функции
    $$
        g(n)=\sum\limits_{d|n} f(d),\ 
        h(n)=\sum\limits_{d|n} \mu(\frac{n}{d})f(d)
    $$
    также мультипликативны.
    \item Докажите, что $\forall n \in \mathbb{Z}$
    $$
    \sum\limits_{d|n} \mu(\frac{n}{d}) \nu(d) = 1,\ 
        \sum\limits_{d|n} \mu(\frac{n}{d}) \sigma(d) = n.
    $$
    \item Докажите, что $\forall m,n \in \mathbb{Z}$
    \begin{itemize}[noitemsep,topsep=0pt]
        \item $\varphi(n)\varphi(m)=\varphi((n,m))\varphi([n,m])$;
        \item $\varphi(mn)\varphi((m,n))=(m,n)\varphi(m)\varphi(n)$.
    \end{itemize}
    \item Пусть $P,Q\in\mathbb{Z}_+$ --- нечетные, $(P,Q)=1$. Докажите, что 
    $$
    \sum_{0<x<\frac{Q}{2}}\left[\frac{P}{Q}x\right] + \sum_{0<y<\frac{P}{2}}\left[\frac{Q}{P}y\right] = \frac{P-1}{2}\frac{Q-1}{2}.
    $$
    (Используйте подсчет целых точек в некоторой ограниченной области на плоскости).
    
\end{enumerate}

\noindent\textbf{SageMath}

\begin{itemize}[topsep=0pt]

    \item Исследуйте основные теоретико-числовые функции в SageMath:
    \begin{itemize}[noitemsep,topsep=0pt]
        \item НОД, НОК: \texttt{gcd(), lcm()};
        \item разложение на множители: \texttt{factor(), valuation()};
        \item простые числа: \texttt{is\_prime(), next\_prime(), previous\_prime()};
        \item делители: \texttt{divisors(), prime\_divisors() };
        \item функции Эйлера и Мёбиуса, число и сумма делителей: \texttt{euler\_phi(), moebius(), sigma()};
        \item число простых чисел: \texttt{prime\_pi()}.
    \end{itemize}
    
\end{itemize}

%\noindent\textbf{Темы для самостоятельного изучения}

\begin{center} {\bf Тема 2. Сравнения} \end{center}

\noindent\textbf{Упражнения и задачи}

\begin{enumerate}[topsep=0pt]
    \item Докажите, что $\ \equiv \pmod{m}$ задаёт отношение эквивалентности в кольце $\mathbb{Z}$, то есть, 1) $a \equiv a\ (m)$; 2) $a \equiv b\ (m) \Rightarrow b \equiv a\ (m)$; 3) $a \equiv b\ (m),\ b \equiv c\ (m) \Rightarrow a \equiv c\ (m)$.

    \item Докажите утверждения про классы вычетов: 
    \begin{itemize}
        \item $\bar{a} = \bar{b} \Leftrightarrow a \equiv b\ (m)$;
        \item $\bar{a} \neq \bar{b} \Leftrightarrow \bar{a} \cap \bar{b}=\varnothing$;
        \item $|\{\bar{a}: a\in \mathbb{Z}\}|=m$.
    \end{itemize}

    \item Докажите, что операции $\bar{a}+\bar{b}$, $\bar{a} \cdot \bar{b}$ на множестве классов вычетов корректно определены, то есть не зависят от выбора представителей классов $\bar{a}$ и $\bar{b}$.

    \item Докажите, что множество $F[x]$ многочленов с коэффициентами из поля $F$ является кольцом.

    \item Докажите, что $\forall f\in F[x]$, $\deg f \geqslant 1$, $f$ раскладывается в произведение неприводимых многочленов.

    \item Докажите, что в кольце $F[x]$ возможно деление с остатком, т.е. $\forall f,g \in F[x]$, $g \neq 0$, $\exists h,r\in F[x]:$ $f=hg+r$, где либо $\deg r < \deg g$, либо $r=0$.

    \item Докажите следующие утверждения про делимость в кольце многочленов:
    \begin{itemize}
        \item $f,g\in F[x]$ --- взаимно простые (т.е. $(f,g)=(1)$), $f|gh$ $\Rightarrow$ $f|h$;
        \item $p \in F[x]$ --- неприводимый, $p|fg$ $\Rightarrow$ $p|f$ или $p|g$.
    \end{itemize}

    \item Докажите, что в кольце многочленов $F[x]$ имеет место однозначнось разложения на неприводимые множители.
    
    \item Используя сравнимость $\mod n$  докажите, что уравнения $3x^2+2=y^2$ и $7x^3+2=y^3$ не разрешимы в целых числах.
    \item Пусть $p,q$ --- различные нечетные простые такие что $p-1|q-1$, докажите, что если $(n,pq)=1$ то $n^{q-1} \equiv 1 (pq)$.
    \item Пусть $a,b,c$ --- решение диофантова уравнения $a^2+b^2=c^2$, $a,b,c \in \mathbb{Z}$, $(a,b)=(b,c)=(c,a)=1$. Докажите, что существуют целые числа $u,v$ такие, что $c-b=2u^2$, $c+b=2v^2$, $(u,v)=1$, и, как следствие, $a=2uv$, $b=v^2-u^2$, $c=v^2+u^2$.
    \item Пусть $m,a,b$ --- целые, $m>1$, $(a,m)=1$. Докажите, что
    \begin{itemize}
        \item $\sum\limits_{x \pmod m} \left\{ \dfrac{ax+b}{m} \right\} = \dfrac{1}{2}(m-1)$;
        \item $\sum\limits_{\substack{x \pmod m\\(x,m)=1}} \left\{ \dfrac{ax}{m} \right\} = \dfrac{1}{2}\varphi(m)$.
    \end{itemize}
    
\end{enumerate}

\noindent\textbf{SageMath}
\begin{itemize}[topsep=0pt]

    \item Исследуйте основные классы и функции SageMath релевантные материалу лекции:
    \begin{itemize}[noitemsep,topsep=0pt]
        \item Кольцо вычетов и модулярная арифметика: \texttt{Integers()};
        \item Китайская теорема об остатках \texttt{crt()};
        \item Кольцо многочленов: \texttt{PolynomialRing()};
        \item Неприводимость многочлена: \texttt{is\_irreducible()};
        \item Разложение многочлена на множители: \texttt{factor()};
        \item Корни многочлена: \texttt{roots()};
        \item Рассмотрите примеры поведения разложения многочлена на множители над $\mathbb{Z}, \mathbb{Q}$ и различными кольцами вычетов $\mathbb{Z}/N\mathbb{Z}$.
    \end{itemize}
    
\end{itemize}

\noindent\textbf{Темы для самостоятельного изучения}
\begin{itemize}[topsep=0pt]
    \item Быстрое возведение в степень в $\mathbb{Z}/n\mathbb{Z}$, [Stein-ent], \S 2.3.
    \item Проверка на простоту, [Stein-ent], \S 2.4.
\end{itemize}




\begin{center} {\bf Тема 3. Первообразные корни} \end{center}

\noindent\textbf{Упражнения и задачи}

\begin{enumerate}[topsep=0pt]
    \item Путь $p$ --- простое, докажите, что $p|\binom{n}{k}$ для $1 \leqslant k < p$.
    \item Путь $p>2$ --- простое, $l\geqslant 2$. Докажите, что $\forall a \in \mathbb{Z}$ $(1+ap)^{p^{l-2}} \equiv 1+ap^{l-1}\ (p^l)$.
    \item Пусть $p>2$ --- простое, $g$ --- первообразный корень $\Mod p^n$. Докажите, что тогда $g$ --- первообразный корень $\Mod p$. %[IR, p48, Ex3]
    \item Пусть $p$ --- простое, $p \equiv 1 (\Mod 4)$. Докажите что $g$ --- первообразный корень $(\Mod p)$ $\Leftrightarrow$ $-g$ --- первообразный корень $(\Mod p)$. %[IR, p48, Ex4]
    \item Пусть p --- простое, $p \equiv 3 (\Mod 4)$. Докажите что $g$ --- первообразный корень $(\Mod p)$ $\Leftrightarrow$ $-g$ имеет порядок $(p-1)/2$. %[IR, p48, Ex5]
    \item Докажите, что $3$ --- первообразный корень простого числа вида $p=2^n+1$. %[IR, p48, Ex6]
    \item Пусть $p>2$ --- простое. Докажите, что $g$ --- первообразный корень $\Mod p$ $\Leftrightarrow$\\ $a^{(p-1)/q}\not\equiv 1 (p)$ для всех простых делителей $q\ |\ p-1$. %[IR, p48, Ex8]
    \item Докажите, что ${\prod\limits_{g}}'g \equiv (-1)^{\varphi(p-1)}\ (p)$, где $\prod'$ --- произведение по всем $0\leqslant g \leqslant p-1$, $g$ --- первообразный корень $\Mod p$. %[IR, p48, Ex9]
    \item Пусть $g$ --- первообразный корень $\Mod p$, $d|(p-1)$. Докажите, что $g^{(p-1)/d}$ имеет порядок $d$, а также что $a$ является $d$-ой степенью $\Leftrightarrow$ $a \equiv g^{kd} (p)$ для некоторого $k$. %[IR, p48, Ex21]
    \item Пусть $G$ --- конечная циклическая группа порядка $n$, $g$ --- образующая $G$. Докажите, что все образующие имеют вид $g^k$, $(k,n)=1$. %[IR, p48, Ex13]
    \item Пусть $G$ --- конечная абелева группа, $a,b$ – элементы порядков $m,n$ соответственно. Докажите, что если $(m,n)=1$ то порядок элемента $ab$ равен $mn$. %[IR, p48, Ex14]

\end{enumerate}

\noindent\textbf{SageMath}
\begin{itemize}[topsep=0pt]

    \item Исследуйте основные классы и функции SageMath релевантные материалу лекции:
    \begin{itemize}[noitemsep,topsep=0pt]
        %\item Корни многочлена: \texttt{roots()};
        \item Первообразные корни: \texttt{primitive\_root(), is\_primitive\_root()};
        \item Образующие группы единиц: \texttt{unit\_gens()};
        \item Порядок элемента в кольце вычетов: \texttt{multiplicative\_order()};
        \item Индекс и дискретный логарифм в кольце вычетов: \texttt{log()};
        \item Абелевы группы \texttt{AbelianGroup()}, образующие и порядки \texttt{gens(), gens\_orders()}.
    
    \end{itemize}

    \item Пусть $a$ --- наименьшее положительное число являющееся первообразным корнем $\Mod p$. Постройте частотную таблицу для $a$, что можно заметить?
    \item Пусть $a\neq -1$ и не является полным квадратом. Постройте примеры последовательностей простых, для которых $a$ является первообразным корнем (согласно гипотезе Артина таких простых бесконечно много, также можно оценить плотность их распределения).
    
\end{itemize}

\noindent\textbf{Темы для самостоятельного изучения}
\begin{itemize}[topsep=0pt]
    \item Структура группы единиц $U(\mathbb{Z}/2^l\mathbb{Z})$ ([IR, глава 4], [Вин, глава 6]).
    \item Критерии разрешимости сравнения $x^n\equiv a (\Mod n)$ ([IR, глава 4]).
    \item Основы криптографии с открытым ключем: протокол Диффи--Хеллмана и RSA, [Stein-ent], глава 3.
\end{itemize}



\begin{center} {\bf Тема 4. Квадратичные вычеты} \end{center}

\noindent\textbf{Упражнения и задачи}

\begin{enumerate}[topsep=0pt]
    \item Докажите, что существует бесконечно много простых $p\equiv 1\ (4)$ и $p\equiv 3\ (4)$.
    \item Докажите, что $\left \lceil \frac{p-1}{4} \right \rceil $  четно $\Leftrightarrow$ $p=8k \pm 1$.
    \item Докажите свойства символа Якоби:
    \begin{itemize}[noitemsep,topsep=0pt]
        \item $a\equiv b\ (P)$ $\Rightarrow$ $\left(\dfrac{a}{P}\right)=\left(\dfrac{b}{P}\right)$;
        \item $\left(\dfrac{ab}{P}\right)=\left(\dfrac{a}{P}\right)\left(\dfrac{b}{P}\right)$;
        \item $\left(\dfrac{a}{PQ}\right)=\left(\dfrac{a}{P}\right)\left(\dfrac{a}{Q}\right)$.
    \end{itemize}
    \item Пусть $\alpha$ --- иррациональное число. Докажите, что последовательность $(\{n\alpha\})_{n=1}^\infty$ равномерно распределена $\Mod 1$.
    \item Пусть $p$ --- простое, $(a,p)=1$. Докажите, что число решений сравнения\\ $ax^2+bx+c \equiv 0 \pmod{p}$ равно $1+\left(\frac{b^2-4ac}{p}\right)$. %[IR, p63, Ex2-3]
    \item Докажите, что если $(a,p)=1$ то $\sum_{x \Mod p} \left(\frac{ax+b}{p}\right)=0$. %[IR, p63, Ex4-5]
    \item Используя замену переменных, докажите, что число решений сравнения $x^2-y^2 \equiv a \pmod{p}$ равно $p-1$, если $(a,p)=1$, и $2p-1$, если $p|a$. Выразите число решений этого сравнения через сумму с символом Лежандра. Используя эти выражения, найдите значение для суммы $\sum_{y \Mod p} \left(\frac{y^2+a}{p}\right)$. %[IR, p63, Ex6-8]
    \item Докажите, что если $(a,p)=1$ то $\sum_{x \Mod p} \left(\frac{x(x+a)}{p}\right)=-1$. %[В, стр82, Упр 8.a]
    \item Пусть $r_1, \dots, r_{(p-1)/2}$ --- квадратичные вычеты в промежутке $[1;p]$. Докажите, что их произведение $\equiv 1\ (p)$, если $p \equiv 3\ (4)$, и  $\equiv -1\ (p)$, если $p \equiv 1\ (4)$. %[IR, p63, Ex10]
    
    \item Пусть $p \equiv 1\ (4)$ --- простое, $(a,p)=1$, $S(a) = \sum_{x \Mod p} \left(\frac{x(x^2+a)}{p}\right)$. Докажите, что 
    \begin{itemize}[noitemsep,topsep=0pt]
        \item $S(a)\equiv 0\ (2)$;
        \item $S(at^2)= \left(\frac{t}{p}\right) S(a)$;
        \item если $r, n$ --- такие, что $\left(\frac{r}{p}\right)=1$, $\left(\frac{n}{p}\right)=-1$, то $p=\left(\frac{1}{2}S(r)\right)^2+\left(\frac{1}{2}S(n)\right)^2$.
    \end{itemize}
    %[В, стр83, Упр 9.c]

    \item Пусть $f(x) \in \mathbb{Z}[x]$. Будем говорить, что простое $p$ делит $f(x)$, если $\exists n \in \mathbb{Z}$ такое, что $p|f(n)$. Опишите простые делители многочленов $x^2+1$ и $x^2-2$. Докажите, что если $p$ делит $x^4-x^2+1$, то $p \equiv 1\ (12)$. %[IR, p63, Ex12-13]
    \item Пусть $D>0$ --- нечетное и свободное от квадратов. Докажите, что $\exists b\in \mathbb{Z}$, $(b,D)=1$ такое, что $\left(\frac{b}{D}\right)=-1$. Докажите также, что $\sum' \left(\frac{a}{D}\right) = 0$, где суммирование берется по приведенной системе вычетов $\Mod D$. %[IR, p64, Ex18-19]
    \item Пусть $p$ --- нечетное простое. Докажите, что 
    $$
    \left(\frac{2}{p}\right) = \prod_{j=1}^{(p-1)/2} 2\cos\left(\frac{2\pi j}{p}\right),
    $$
    а также, что если $p>3$ то
    $$
    \left(\frac{3}{p}\right) = \prod_{j=1}^{(p-1)/2} \left(3- 4\sin^2\left(\frac{2\pi j}{p}\right) \right).
    $$
    %[IR, p64, Ex32,34]
\end{enumerate}

\noindent\textbf{SageMath}
\begin{itemize}[topsep=0pt]
    \item Исследуйте основные функции SageMath связанные с вычислением квадратичных вычетов и символов Лежандра и Якоби:
    \begin{itemize}[noitemsep,topsep=0pt]
        \item Квадратичные вычеты: \texttt{quadratic\_residues()};
        \item Символы: \texttt{kronecker(), jacobi()}.
     \end{itemize}

    \item Пусть $r(p)$ --- наименьший квадратичный вычет $\Mod p$, $n(p)$ --- наименьший квадратичный невычет $\Mod p$, $d(p)$ --- максимальное расстояние между соседними квадратичными невычетами $\Mod p$. Постройте частотные таблицы для $r(p), n(p), d(p)$. Что можно заметить?\\
    (Cогласно гипотезам Виноградова, $\forall \varepsilon > 0$ $\frac{d(p)}{p^\varepsilon} \rightarrow 0$, $\frac{n(p)}{p^\varepsilon} \rightarrow 0$, $\frac{r(p)}{p^\varepsilon} \rightarrow 0$ при $p \rightarrow \infty$.)
    
    \item  Проведите численные эксперименты относительно равномерного распределения последовательностей, которые упоминались в лекции: \begin{itemize}[noitemsep,topsep=0pt]
        \item $(\{n\alpha\})_{n=1}^\infty$, $\alpha$ --- иррациональное;
        \item $(\{p\alpha\})_{p=1}^\infty$, $\alpha$ --- иррациональное, $p$ пробегает все простые;
        \item $(\{\frac{x_p}{p}\})_{p=1}^\infty$, $x_p$ --- решение сравнения $x^2 \equiv a\ (p)$, $p$ пробегает все простые.
     \end{itemize}
\end{itemize}

\noindent\textbf{Темы для самостоятельного изучения}
\begin{itemize}[topsep=0pt]
    \item Когда простое $q$ является квадратичным вычетом по модулю простого $p$? (Приложение квадратичного закона взаимности, [IR, \S 5.2, теорема 2]).
    \item Существует бесконечно много простых таких, что $\left(\frac{a}{p}\right)=-1$, где $a$ --- целое, отличное от квадрата. ([IR, \S 5.2, теорема 3]).
    \item Критерий разрешимости сравнения $x^2\equiv a\ (m)$ для произвольного $m$. ([IR, \S 5.1, предложение 5.1.1], [Вин, \S V.4]).
\end{itemize}


\begin{center} {\bf Тема 5. Конечные поля. Расширения полей} \end{center}

\noindent\textbf{Упражнения и задачи}

\begin{enumerate}[topsep=0pt]
    \item Завершите доказательство предложения: пусть $k$ --- поле характеристики $p$, тогда $\forall \alpha, \beta \in k$ $\forall d \in \mathbb{Z}_+$ $(\alpha+\beta)^{p^d}=\alpha^{p^d}+\beta^{p^d}$.
    \item Докажите, что если расширение $L/K$ конечной степени $[L:K]$, то $L/K$ --- алгебраическое.
    %\item Докажите, что $\Gal(L/K)$ --- группа и что $|\Gal(L/K)| \leqslant [L:K]$.
    \item Докажите, что
    \begin{itemize}[noitemsep,topsep=0pt]
        \item для $a\in\mathbb{Z}$ $a^l-1|a^m-1$ $\Leftrightarrow$ $l|m$
        \item в $\mathbb{F}_q[x]$ $x^l-1|x^m-1$ $\Leftrightarrow$ $l|m$
    \end{itemize}

    \item Пусть $p,q$ --- различные простые. Чему равно число неприводимых многочленов степени $q$ в $\mathbb{F}_p[x]$? %[IR, p87 ex7.17]
    
    \item Пусть $\sigma_j(f) = \sum_{g|f}' (Ng)^j$, где суммирование берется по неприводимым унитарным делителям $g$ (для $f\in\mathbb{F}_q[x]$ степени $\deg f = n$ $Nf=q^n$). Докажите, что
    \begin{itemize}[noitemsep,topsep=0pt]
        \item $\sum\limits_f \dfrac{\sigma_0(f)}{(Nf)^s} = \dfrac{1}{(1-q^{1-s})^2}$;
        \item $\sum\limits_f \dfrac{\sigma_1(f)}{(Nf)^s} = \dfrac{1}{(1-q^{1-s})(1-q^{2-s})}$.
    \end{itemize} %[IR, p87 ex7.20]

    \item Пусть $\alpha \in \mathbb{F}_q^*$. Докажите, что $x^n=\alpha$ разрешимо $\Leftrightarrow$ $\alpha^{(q-1)/d=1}$, где $d=(n,q-1)$, причем если разрешимо, то $d$ решений. %[IR, p80 prop 7.1.2]

    \item Как выглядит подгруппа всех квадратов в $\mathbb{F}_{2^n}$? %[IR, p86 ex7.8]

    \item Пусть $n|q-1$, докажите, что $G=\{\alpha\in\mathbb{F}_q^*: x^n=\alpha\ \text{--- разрешимо}\}$ --- подгруппа в $\mathbb{F}_q^*$, $|G|=\frac{q-1}{n}$. %[IR, p86 ex7.4]

    \item Пусть $n|q-1$, $F=\mathbb{F}_q$, $K/F$ --- расширение конечных полей, $[K:F] = n$. Докажите, что $\forall \alpha \in F^*$ уравнение $x^n=\alpha$ имеет $n$ решений в $K$. %[IR, p86 ex7.5]

    \item Пусть $K/F$ --- расширение конечных полей, $\chr F \neq 2$, $[K:F] = 3$. Докажите, что если $\alpha$ не является квадратом в $F$, то $\alpha$ не является квадратом и в $K$. %[IR, p86 ex7.6]

    \item Пусть $F=\mathbb{F}_q$, $K/F$ --- расширение конечных полей, $\alpha \in \mathbb{F}_q$, $n|q-1$ и $x^n=\alpha$ не разрешимо в $\mathbb{F}_q$. Тогда $x^n=\alpha$ не разрешимо в $K$, если $(n,[K:F])=1$. %[IR, p86 ex7.9]

    \item Пусть $F=\mathbb{F}_q$, $K/F$ --- расширение конечных полей, $[K:F] = 2$. Докажите, что $\forall \beta \in K$ $\beta^{1+q} \in F$. Более того, $\forall \alpha \in F$ $\exists \beta \in K$: $\alpha=\beta^{1+q}$. %[IR, p86 ex7.10]

    %\item Пусть $\mathbb{F}_{q^m}/\mathbb{F}_q$ --- расширение конечных полей, $\alpha \in \mathbb{F_{q^m}}$. Докажите, что сопряженные элементы $\alpha, \alpha^q, \alpha^{q^2}, \dots \alpha^{q^{m-1}}$ имеют один и тот же порядок в мультипликативной группе $\mathbb{F}_{q^m}^*$.   %[LN, p50 th2.18]
\end{enumerate}

\noindent\textbf{SageMath}
\begin{itemize}[topsep=0pt]
    \item Исследуйте основные функции SageMath связанные с заданием и свойствами конечных полей
    \begin{itemize}[noitemsep,topsep=0pt]
        \item Определение конечного поля: \texttt{FiniteField(), GF()};
        \item Неприводимый многочлен задающий конечное поле: \texttt{polynomial()}, опция \texttt{modulus} в \texttt{FiniteField()} для явного задания неприводимого многочлена модели конечного поля;
        \item Решение уравнения $x^n=\alpha$: \texttt{nth\_root()}.
     \end{itemize}
\end{itemize}

%\noindent\textbf{Темы для самостоятельного изучения}
%\begin{itemize}[topsep=0pt]
%    \item Поле $\mathbb{F}_q$, $q=p^n$, однозначно определено в $\bar{\mathbb{F}}_p$ как поле разложения многочлена $z^q-z$. Всякое конечное поле изоморфно одному и только одному $\mathbb{F}_q$. ([Степ], [LN])
%    \item Всякое конечное кольцо с единицей, в котором каждый ненулевой элемент обратим, является полем. ([LN], [The Book]).
%\end{itemize}


\begin{center} {\bf Тема 6. Группа автоморфизмов. Норма и след} \end{center}

\noindent\textbf{Упражнения и задачи}
\begin{enumerate}[topsep=0pt]
    \item Пусть $k$ --- поле, $f=a_nx^n+\dots+a_1x+a_0 \in k[x]$, $f'=n a_n x^{n-1}+\dots+a_1 \in k[x]$ --- формальная производная $f$. Докажите следующие свойства: 1) $(f+g)'=f'+g'$; 2) $(fg)'=f'g+fg'$.

    \item Пусть $k$ --- поле, $f \in k[x]$. докажите, что $\alpha \in k$ --- кратный корень $\Leftrightarrow f'(\alpha) = 0$.

    \item Пусть $f \in \mathbb{F}_{p^m}[x]$. Докажите, что $f\in \mathbb{F}_{p} \Leftrightarrow (f(x))^p=f(x^p)$.

    \item Докажите, что если $d|n$ и $\Phi_n$ определен, то $\Phi_n\ |\ \dfrac{x^n-1}{x^d-1}$.

    \item Пусть $f \in \mathbb{F}_q[x]$ --- неприводимый, $\deg f = m$. Докажите, что $f\ |\ x^{q^n}-x \Leftrightarrow m|n$.

    \item Докажите, что $\prod' f = x^{q^n}-x$, где произведение берется по всем неприводимым унитарным многочленам $f\in \mathbb{F}_q[x]$ таким, что $\deg f\ |\ n$. Сделайте вывод о числе неприводимых унитарных многочленов степени $d$ в $\mathbb{F}_q[x]$ (на этот раз для произвольного конечного поля, $q=p^n$).

    \item Докажите, что если $\chr k = p \not{|}\ n$, то $\Phi_n = \prod\limits_{d|n}(x^d-1)^{\mu(n/d)}$.
    
    \item Докажите, что $\mathbb{F}_q$ есть $(q-1)$-е круговое поле над любым своим подполем. %[LN, p63 Lemma 2.49]

    %3\item Докажите, что если $f \in \mathbb{F}_q[x]$ --- неприводимый, $\deg f = 2$, то $f$ раскладывается на линейные множители в $\mathbb{F}_{q^2}[x]$. %[LN p71 ex 2.17]

    \item Пусть $\alpha \in \mathbb{F}_q$, $n \in \mathbb{Z}$. Докажите, что $x^q-x+\alpha\ |\ x^{q^n}-x+n\alpha$. %[LN p71 ex 2.19]

    \item Пусть $f \in \mathbb{F}_q[x]$, $q=p^n$. Докажите, что $f'(x)=0$ $\Leftrightarrow$ $f=g^p$ для некоторого $g \in \mathbb{F}_q[x]$. %[LN p71 ex 2.23]
    
    \item Пусть $f\in \mathbb{F}_q[x]$, $q=p^n$, $\deg f = m \geqslant 1$, $f(0) \neq 0$. Докажите, что $\exists e \in \mathbb{Z_+}$ $e\leqslant q^m-1$ такое что $f(x)\ |\ x^e-1$. Наименьшее такое $e$ называется порядком многочлена $f(x)$ в $\mathbb{F}_q[x]$. Докажите также следующие свойства:
    \begin{itemize}[noitemsep,topsep=0pt]
        \item Пусть $f\in \mathbb{F}_q[x]$ --- неприводимый, тогда порядок $f$ равен порядку $\alpha \in \mathbb{F}_{q^m}^*$, $\alpha$ --- корень $f$;
        \item Пусть $f\in \mathbb{F}_q[x]$ --- неприводимый, тогда порядок $f$ делит $x^e-1$;
        \item Пусть $c\in \mathbb{Z}_+$, $e$ --- порядок $f$, тогда $f(x)\ |\ x^c-1$ $\Leftrightarrow$ $e|c$;
        \item Пусть $e_1, e_2\in \mathbb{Z}_+$, тогда наибольший общий делитель многочленов $x^{e_1}-1$, $x^{e_2}-1$ в $\mathbb{F}_q[x]$ равен $x^d-1$, где $d=(e_1,e_2)$.

        %\item Пусть $\mathbb{F}_{q^m}/\mathbb{F}_q$ --- расширение конечных полей, $\alpha \in \mathbb{F_{q^m}}$. Докажите, что сопряженные элементы $\alpha, \alpha^q, \alpha^{q^2}, \dots \alpha^{q^{m-1}}$ имеют один и тот же порядок в мультипликативной группе $\mathbb{F}_{q^m}^*$.   %[LN, p50 th2.18]
        
    \end{itemize} %[LN, p77-78 3.1-3.7]
\end{enumerate}


\noindent\textbf{SageMath}
\begin{itemize}[topsep=0pt]
    \item Исследуйте основные функции SageMath связанные с работой в кольцах многочленов над конечными полями:
    \begin{itemize}[noitemsep,topsep=0pt]
        \item Кольцо многочленов: \texttt{PolynomialRing()};
        \item Неприводимость многочлена: \texttt{is\_irreducible()};
        \item Разложение многочлена на множители: \texttt{factor()};
        \item Корни многочлена: \texttt{roots()};
        \item Круговой многочлен: \texttt{cyclotomic\_polynomial()};
        \item Поле разложения: \texttt{splitting\_field()};
        \item Расширение полей: \texttt{extension()}.
     \end{itemize}
        
\end{itemize}

\noindent\textbf{Темы для самостоятельного изучения}
\begin{itemize}[topsep=0pt]
    \item Изучите "элементарные" доказательство неприводимости кругового многочлена $\Phi_n(x)$ в $\mathbb{Z}[x]$.
    (см. например \texttt{https://www.lehigh.edu/~shw2/c-poly/several\_proofs.pdf})

\end{itemize}


\begin{center} {\bf Тема 7. Корни из единицы. Круговой многочлен} \end{center}

\noindent\textbf{Упражнения и задачи}
\begin{enumerate}[topsep=0pt]
    \item Пусть $k$ --- поле, $f=a_nx^n+\dots+a_1x+a_0 \in k[x]$, $f'=n a_n x^{n-1}+\dots+a_1 \in k[x]$ --- формальная производная $f$. Докажите следующие свойства:
    \begin{itemize}[topsep=0pt]
        \item $(f+g)'=f'+g'$;
        \item $(fg)'=f'g+fg'$.
    \end{itemize}

    \item Пусть $k$ --- поле, $f \in k[x]$. докажите, что $\alpha \in k$ --- кратный корень $\Leftrightarrow f'(\alpha) = 0$.

    \item Пусть $f \in \mathbb{F}_{p^m}[x]$. Докажите, что $f\in \mathbb{F}_{p} \Leftrightarrow (f(x))^p=f(x^p)$.

    \item Докажите, что если $d|n$ и $\Phi_n$ определен, то $\Phi_n\ |\ \dfrac{x^n-1}{x^d-1}$.

    \item Пусть $f \in \mathbb{F}_q[x]$ --- неприводимый, $\deg f = m$. Докажите, что $f\ |\ x^{q^n}-x \Leftrightarrow m|n$.

    \item Докажите, что $\prod' f = x^{q^n}-x$, где произведение берется по всем неприводимым унитарным многочленам $f\in \mathbb{F}_q[x]$ таким, что $\deg f\ |\ n$. Сделайте вывод о числе неприводимых унитарных многочленов степени $d$ в $\mathbb{F}_q[x]$ (на этот раз для произвольного конечного поля, $q=p^n$).

    \item Докажите, что если $\chr k = p \not{|}\ n$, то $\Phi_n = \prod\limits_{d|n}(x^d-1)^{\mu(n/d)}$.
    
    \item Докажите, что $\mathbb{F}_q$ есть $(q-1)$-е круговое поле над любым своим подполем. %[LN, p63 Lemma 2.49]

    %3\item Докажите, что если $f \in \mathbb{F}_q[x]$ --- неприводимый, $\deg f = 2$, то $f$ раскладывается на линейные множители в $\mathbb{F}_{q^2}[x]$. %[LN p71 ex 2.17]

    \item Пусть $\alpha \in \mathbb{F}_q$, $n \in \mathbb{Z}$. Докажите, что $x^q-x+\alpha\ |\ x^{q^n}-x+n\alpha$. %[LN p71 ex 2.19]

    \item Пусть $f \in \mathbb{F}_q[x]$, $q=p^n$. Докажите, что $f'(x)=0$ $\Leftrightarrow$ $f=g^p$ для некоторого $g \in \mathbb{F}_q[x]$. %[LN p71 ex 2.23]
    
    \item Пусть $f\in \mathbb{F}_q[x]$, $q=p^n$, $\deg f = m \geqslant 1$, $f(0) \neq 0$. Докажите, что $\exists e \in \mathbb{Z_+}$ $e\leqslant q^m-1$ такое что $f(x)\ |\ x^e-1$. Наименьшее такое $e$ называется порядком многочлена $f(x)$ в $\mathbb{F}_q[x]$. Докажите также следующие свойства:
    \begin{itemize}[noitemsep,topsep=0pt]
        \item Пусть $f\in \mathbb{F}_q[x]$ --- неприводимый, тогда порядок $f$ равен порядку $\alpha \in \mathbb{F}_{q^m}^*$, $\alpha$ --- корень $f$;
        \item Пусть $f\in \mathbb{F}_q[x]$ --- неприводимый, тогда порядок $f$ делит $x^e-1$;
        \item Пусть $c\in \mathbb{Z}_+$, $e$ --- порядок $f$, тогда $f(x)\ |\ x^c-1$ $\Leftrightarrow$ $e|c$;
        \item Пусть $e_1, e_2\in \mathbb{Z}_+$, тогда наибольший общий делитель многочленов $x^{e_1}-1$, $x^{e_2}-1$ в $\mathbb{F}_q[x]$ равен $x^d-1$, где $d=(e_1,e_2)$.

        %\item Пусть $\mathbb{F}_{q^m}/\mathbb{F}_q$ --- расширение конечных полей, $\alpha \in \mathbb{F_{q^m}}$. Докажите, что сопряженные элементы $\alpha, \alpha^q, \alpha^{q^2}, \dots \alpha^{q^{m-1}}$ имеют один и тот же порядок в мультипликативной группе $\mathbb{F}_{q^m}^*$.   %[LN, p50 th2.18]
        
    \end{itemize} %[LN, p77-78 3.1-3.7]
\end{enumerate}

\noindent\textbf{SageMath}
\begin{itemize}[topsep=0pt]
    \item Исследуйте основные функции SageMath связанные с работой в кольцах многочленов над конечными полями:
    \begin{itemize}[noitemsep,topsep=0pt]
        \item Кольцо многочленов: \texttt{PolynomialRing()};
        \item Неприводимость многочлена: \texttt{is\_irreducible()};
        \item Разложение многочлена на множители: \texttt{factor()};
        \item Корни многочлена: \texttt{roots()};
        \item Круговой многочлен: \texttt{cyclotomic\_polynomial()}.
     \end{itemize}
        
\end{itemize}

\noindent\textbf{Темы для самостоятельного изучения}
\begin{itemize}[topsep=0pt]
    \item Изучите "элементарные" доказательство неприводимости кругового многочлена $\Phi_n(x)$ в $\mathbb{Z}[x]$.
    (см. например\\ \texttt{https://www.lehigh.edu/~shw2/c-poly/several\_proofs.pdf})
    \item Теорема Веддербёрна: Всякое конечное кольцо с единицей, в котором каждый ненулевой элемент обратим, является полем. ([LN], [The Book]).
\end{itemize}


\begin{center} {\bf Тема 8. Характеры. Суммы Гаусса} \end{center}

\noindent\textbf{Упражнения и задачи}

\begin{enumerate}[topsep=0pt]
    \item Пусть $G$ --- конечная абелева группа, $H$ --- собственная подгруппа, $g\in G$, $g \not\in H$. Докажите, что существует характер $\chi$ группы $G$ такой что $\chi(g)\neq 1$ и $\forall h\in H$ $\chi(h)=1$. %[LN Ex 5.1]
    \item Пусть $G$ --- конечная абелева группа, $\widehat{G}$ --- группа характеров, $H < G$ ---  подгруппа, $A < \widehat{G}$ --- аннигилятор $H$: $A=\{\chi\in\widehat{G}: \forall h\in H\ \chi(h)=1 \}$. Докажите, что $A\cong G/H$ и что $H\cong \widehat{G}/A$. %[LN Ex 5.2]
    \item Пусть $G = G_1\times \cdots \times G_k$ --- прямое произведение конечных абелевых групп (множество $k$-кортежей с операцией $(g_1,\dots,g_k) (h_1,\dots,h_k) = (g_1 h_1, \dots, g_k h_k)$). Докажите, что $\widehat{G}\cong \widehat{G_1}\times \cdots \times \widehat{G_k}$. %[LN Ex 5.4]
    \item Основная теорема о стуктуре конечных абелевых групп утверждает, что каждая такая группа изоморфна прямому произведению конечного числа циклических групп. Выведете из этой теоремы, что если $G$ --- конечная абелева группа, то $\widehat{G}\cong G$. %[LN Ex 5.5]
    \item Пусть $G$ -- конечная абелева группа, $m>0$ --- целое. Докажите, что $g\in G$ является $m$-ой степенью в $G$ $\iff$ $\forall$ характера порядка $m$ выполняется $\chi(g)=1$. %???
    \item Покажите, что для аддитивных характеров $\psi_a$, $\psi_b$ поля $\mathbb{F}_q$ выполняется $\psi_a \psi_b = \psi_{a+b}$, и что из этого следует изоморфизм аддитивной группы $\mathbb{F}_q$ группе аддитивных характеров поля. %[LN Ex 5.6]
    \item Докажите, что для аддитивного характера $\psi=\psi_1$ поля $\mathbb{F}_q/\mathbb{F}_p$ для всех $\alpha\in\mathbb{F}_q$, $j\in\mathbb{Z}_+$ справедливо $\psi_1(\alpha^{p^j})=\psi_1(\alpha)$. %[LN Ex 5.7]
    \item Пусть $\chi'$ --- мультипликативный характер $\mathbb{F}_{q^s}$ порядка $m$, $\chi$ --- ограничение $\chi'$ на $\mathbb{F}_q$. Докажите, что $\chi$ --- мультипликативный характер $\mathbb{F}_q$ порядка $m/(m,(q^s-1)/(q-1))$. %[LN Ex 5.8]
    \item Пусть $\chi$ --- мультипликативный характер $\mathbb{F}_q$ порядка, $\chi'$ --- продолжение $\chi$ на $\mathbb{F}_{q^s}$. Докажите, что $\chi'(a)=\chi(q)^s$ $\forall a \in\mathbb{F}_q^{*}$. %[LN Ex 5.10]
    \item Пусть $p\neq 2$, $ab\not\equiv 0\,(p)$, $\chi$ --- квадратичный характер $\mathbb{F}_p^*$. Докажите, что
    \begin{itemize}[topsep=0pt]
        \item $G(\chi,\psi_a)G(\chi,\psi_b) = \left(\dfrac{-ab}{p}\right) p$; %[БШ стр 26 упр 5]
        \item $\sum_a G(\chi,\psi_a)=0$. %[БШ стр 26 упр 6]
    \end{itemize}
    \item Пусть $p>2$, $G = \sum_{x=0}^{p-1} e^{2\pi i x^2/p}$ --- сумма Гаусса для квадратичного характера, $A = (a_{st})_{0\leqslant s,t \leqslant p-1}$ --- $p\times p$ матрица с элементами $a_{st}=e^{2\pi i st/p}$. Докажите, что:
    \begin{itemize}[topsep=0pt]
        \item если $\lambda_0,\dots,\lambda_{p-1}$ – характеристические числа матрицы $A$, то $\sum_{k=0}^{p-1} \lambda_k = G$;
        \item характеристический многочлен матрицы $A^2$ имеет вид: $(t-p)^{(p+1)/2} (t+p)^{(p-1)/2}$;
        \item для определителя матрицы $A$ справедливо $\det A = i^{p(p-1)/2}p^{p/2}$.
    \end{itemize} %???
    \item Пусть $q=p^n, p>2$. Определим аналог символа Лежандра для $\mathbb{F}_q$: $\left(\frac{\alpha}{q}\right)=1$, если $\alpha$ --- квадрат в $\mathbb{F}_q$; $\left(\frac{\alpha}{q}\right)=1$, если $\alpha$ не является квадратом в $\mathbb{F}_q$; $\left(\frac{0}{q}\right)=0$. Докажите следующие свойства этого символа: 
    \begin{itemize} [topsep=0pt]
        \item $\left(\frac{\alpha\beta}{q}\right) = \left(\frac{\alpha}{q}\right)\left(\frac{\beta}{q}\right)$, $\alpha, \beta \in \mathbb{F}_q$; 
        \item $\sum_{\alpha\in\mathbb{F}_q} \left(\frac{\alpha}{q}\right) = 0$; 
        \item $\left(\frac{\alpha}{q}\right)=\left(\frac{\N_{\mathbb{F}_q/\mathbb{F}_p}(\alpha)}{p}\right)$ --- обычный символ Лежандра $\pmod p$.
    \end{itemize} %[Степ Задача I.2.9]
    \item Докажите свойства обобщенных сумм Гаусса для конечного подя $\mathbb{F}_q/\mathbb{F}_p$:
    \begin{itemize}[topsep=0pt]
        \item $G(\chi,\psi_{ab})=\chi(a) G(\chi, \psi_b)$, $a\in\mathbb{F}_q^*$, $b\in \mathbb{F}_q$;
        \item $G(\chi,\bar{\psi}) = \chi(-1) G(\chi,\psi)$;
        \item $G(\bar{\chi},\psi) = \chi(-1) \overline{G(\chi,\psi)}$;
        \item $G(\chi,\psi)G(\bar{\chi},\psi)=\chi(-1) q$, $\chi\neq\chi_0$, $\psi\neq\psi_0$;
        \item $G(\chi^p,\psi_b)=G(\chi,\psi_{\sigma(b)})$, $b\in\mathbb{F}_q$, $\sigma$ --- автоморфизм Фробениуса.
    \end{itemize} %[LN Th 5.12]
    \item Пусть $f:\mathbb{F}_q \rightarrow \mathbb{C}$, $\hat{f} = \frac{1}{q}\sum_{t\in\mathbb{F}_q} f(t)\overline{\psi(st)}$ --- конечное преобрахование Фурье. Докажите, что $f(t)=\sum_{s\in\mathbb{F}_q} \hat{f}(s) \psi(st)$. %[IR Ex 10.23]
\end{enumerate}

\noindent\textbf{SageMath}
\begin{itemize}[topsep=0pt]
    \item Исследуйте основные функции SageMath связанные с группой характеров конечных абелевых групп:
    \begin{itemize}[noitemsep,topsep=0pt]
        \item \texttt{character\_table()}.
     \end{itemize}
\end{itemize}

\noindent\textbf{Темы для самостоятельного изучения}
\begin{itemize}[topsep=0pt]
    \item Докажательство квадратичного закон взаимности через суммы Гаусса. [IR], \S 7.3; [LN], \S 5.2.
\end{itemize}


\begin{center} {\bf Тема 9. Тригонометрические суммы.\\ Уравнения над конечными полями} \end{center}

\noindent\textbf{Упражнения и задачи}

\begin{enumerate}[topsep=0pt]
    
    \item Пусть $F_1(x_1,\dots, x_n), \dots, F_m(x_1,\dots, x_n)$ --- многочлены с целыми коэффициентами степеней $r_1, \dots, r_m$. Докажите, что если $r_1+\dots+r_m < n$, то число решений системы сравнений $F_i(x_1,\dots, x_n) \equiv 0\ (p), 1\leqslant i \leqslant m$, делится на $p$. %[БШ стр 14 теорема 4]
    
    \item Пусть $p$ --- простое, $F(x_1, \dots, x_n)\in \mathbb{Z}[x_1,\dots,x_n]$ --- многочлен с целыми коэффициентами, $\deg F = r < n(p-1)$. Докажите, что $p^a\ |\ \sum' F(x_1, \dots, x_n)$, где в сумме $x_i$ пробегают независимо друг от друга полную систему вычетов $\Mod p$, и $a=n-[r/(p-1)]$. %[БШ стр16 задача 1]

    \item Пусть $m$ --- натуральное, $f(x) \in \mathbb{Z}[x]$, $S_a = \sum\limits_{x \Mod m} e^{2\pi i \frac{af(x)}{m}}$. Докажите, что
    $$
        \sum\limits_{a \Mod m} |S_a|^2 = m \sum\limits_{c \Mod m} N(c)^2,
    $$
    где $N(c) = N_m\left(f(x) \equiv c\ (m)\right)$ --- число решений сравнения $f(x) \equiv c\ (m)$.
    %[БШ стр26 задача 9]

    \item Пусть $p$ --- простое, $S_a = \sum\limits_{x \in \mathbb{F}_p} e^{2\pi i \frac{a x^r}{p}}$, $d=(r,p-1)$. Докажите, что 
    \begin{itemize}[noitemsep,topsep=0pt]
        \item $\sum\limits_{a \in \mathbb{F}_p^*} |S_a|^2 = p(p-1)(d-1)$;
        \item $|S_a| < d \sqrt{p}$, при $a \neq 0$;
        \item и более точная оценка: $|S_a| \leqslant (d-1) \sqrt{p}$, при $a \neq 0$.
    \end{itemize} %[БШ стр26 задачи 10-12]

    \item Пусть $\chi$, $\lambda$ --- неглавные мультипликативные характеры $\mathbb{F}_p$, $\epsilon$ --- главный, $\tau(\chi)$ --- сумма Гаусса. Докажите свойства сумм Якоби:
        \begin{itemize}[topsep=0pt]
            \item $J(\epsilon,\epsilon)=p$;
            \item $J(\epsilon,\chi)=0$;
            \item $J(\chi,\chi^{-1})=-\chi(-1)$;
            \item $J(\chi,\lambda) = \tau(\chi)\tau(\lambda)/\tau(\chi\lambda)$ при $\chi\lambda \neq \epsilon$.
        \end{itemize} % [IR, \S 8.3 Th1]

    \item Пусть $\chi, \rho$ --- мультипликативные характеры $\mathbb{F}_p^*$, $\chi$ --- неглавный, $\rho$ --- порядка $2$. Докажите следующие утверждения:
    \begin{itemize}[topsep=0pt]
        \item $\sum_t \chi(1-t^2)=J(\chi,\rho)$; %[IR Ex 8.3]
        \item $\sum_t \chi(t(k-t))=\chi(k^2/4)J(\chi,\rho)$, $k\in\mathbb{F}_p^*$; %[IR Ex 8.4]
        \item $G(\chi)^2=\chi(2)^{-2}J(\chi,\rho) G(\chi^2)$ если $\chi^2$ --- неглавный; %[IR Ex 8.5]
        \item $J(\chi,\chi)=\chi(2)^{-2}J(\chi,\rho)$; %[IR Ex 8.6]
        \item если $p\equiv 1\, (4)$, $\chi$ --- порядка $4$, то $\chi^2=\rho$ и $J(\chi,\chi)=\chi(-1)^{-2}J(\chi,\rho)$; %[IR Ex 8.7]
        \item $\sum_t \chi(1-t^m) = \sum_{\lambda^m=\epsilon} J(\chi,\lambda)$; %[IR Ex 8.8]
        \item $|\sum_t \chi(1-t^m)| \leqslant (m-1) \sqrt{p}$. %[IR Ex 8.8]
    \end{itemize}

    \item Пусть $\chi_1, \chi_2, \dots, \chi_l$ --- мультипликативные характеры, $\varepsilon$ --- главный характер $\Mod p$,\\ $J = J(\chi_1, \chi_2, \dots, \chi_l) = \sum\limits_{t_1+\dots+t_l=1} \chi_1(t_1) \cdots \chi_l(t_l)$ --- обобщенная сумма Якоби,\\ $J_0 = J_0(\chi_1, \chi_2, \dots, \chi_l) = \sum\limits_{t_1+\dots+t_l=0} \chi_1(t_1) \cdots \chi_l(t_l)$. Докажите следующие свойства $J$ и $J_0$:
    \begin{itemize}[noitemsep,topsep=0pt]
        \item $J_0(\varepsilon, \dots, \varepsilon) = J(\varepsilon, \dots, \varepsilon) = p^{l-1}$;
        \item если некоторые, но не все, среди характеров $\chi_i$ являются главными, то $J_0 = 0$, $J = 0$;
        \item пусть $\chi_l \neq \varepsilon$, тогда если $\chi_1 \chi_2 \cdots \chi_l \neq \varepsilon$, то $J_0 = 0$, а если $\chi_1 \chi_2 \cdots \chi_l = \varepsilon$, то $J_0 (\chi_1, \chi_2, \dots, \chi_l)= \chi_l(-1(p-1))J(\chi_1, \chi_2, \dots, \chi_{l-1})$.
    \end{itemize} %[IR p99 prop 8.5.1]

    \item Пусть $\chi_1, \chi_2, \dots, \chi_l$ --- неглавные характеры $\Mod p$ такие что $\chi_1 \chi_2 \cdots \chi_l$ тоже неглавный, $\tau$ --- сумма Гаусса, $J$ --- обобщенная сумма Якоби. Докажите, что
    \begin{itemize}[noitemsep,topsep=0pt]
        \item $\tau(\chi_1) \cdots \tau(\chi_l) = J(\chi_1, \dots, \chi_l) \tau(\chi_1 \cdots \chi_l)$;
        \item $|J(\chi_1, \dots, \chi_l)| = p^{(l-1)/2}$.
    \end{itemize} %[IR p100 Th 3, 4]

    \item Путь $m>1$ --- целое, $K(a,b;m) = \sum'_{xy\equiv 1 (m)} e^{2\pi i \frac{ax+by}{m}}$, где $x$ пробегает приведеную систему вычетов $\Mod m$. $K(a,b;m)$ называется суммой Клоостермана, удобно также использовать запись $K(a,b;m) = \sum'_{x \Mod m} e^{2\pi i \frac{ax+bx^*}{m}}$, где $x^*$ обозначает вычет обратный к $x$. Докажите следующие свойства сумм Клоостермана:
    \begin{itemize}[noitemsep,topsep=0pt]
        \item $K(a,b;m)=K(b,a;m)$;
        \item если $(m,c)=1$, то $K(ac,b;m)=K(a,bc;m)$;
        \item если $m=m_1 m_2$, $(m_1,m_2)=1$, то $K(a,b;m)=K(n_2 a,n_2 b;m_1)K(n_1 a,n_1 b;m_2)$, где $n_1,n_2$ определены из $m_1 n_1 \equiv 1\ (m_2)$, $m_2 n_2 \equiv 1\ (m_1)$;
        \item если $m=p^{2\alpha}$, $(m,2a)=1$, то $K(a,a;m)=\sqrt{m}(e^{2\pi i \frac{2a}{m}}+e^{-2\pi i \frac{2a}{m}})$.
    \end{itemize} %[IW p59]

    \item Пусть $p$ --- простое, $(k,p)=1$, $S=\sum'_x \sum'_y \left(\frac{xy+k}{p}\right)$, где $x,y$ пробегают возрастающие последовательности из $X$ и $Y$ вычетов полной системы вычетов $\Mod p$. Докажите, что $|S|<\sqrt{XYp}$. %[В стр82 Задача 8c]

    \item Пусть $m>1$ --- целое, $(a,m)=1$, $S=\sum\limits_{x \Mod m}\sum\limits_{y \Mod m} \xi(x) \eta(y) e^{2\pi i \frac{axy}{m}}$, где $\xi,\eta$ --- такие, что $\sum\limits_{x \Mod m} |\xi(x)|^2=X$, $\sum\limits_{y \Mod m} |\eta(x)|^2=Y$. Докажите, что $|S|<\sqrt{XYm}$. %[В стр103 8]

    \item Пусть $p$ --- простое, $(a,p)=(b,p)=1$, n --- целое $0<n<p$, $S=\sum\limits_{x \in \mathbb{F}_p^*} e^{2\pi i \frac{ax^n+bx}{p}}$. Докажите, что $|S| < \frac{3}{2} n^{1/4} p^{3/4}$. %[В стр103 9]

    \item Пусть $p>60$ --- простое, $M,Q$ --- целые, $0<M<M+Q\leqslant p$, $\chi$ --- неглавный характер $\Mod p$, $S = \sum\limits_{x=M}^{M+Q-1} \chi(x)$. Докажите, что $|S|<\sqrt{p}\ (\log p - 1)$. %[В стр112 4]
  
\end{enumerate}

\noindent\textbf{SageMath}
\begin{itemize}[topsep=0pt]
    \item Сопроводите оценки тригонометрических сумм полученные в лекции и упраждениях экспериментальными оценками с помощью SageMath.
\end{itemize}


\noindent\textbf{Темы для самостоятельного изучения}

\begin{itemize}[topsep=0pt]
    \item Вывод числа решений уравнения $a_1 x_1^{l_1} + \dots + a_r x_r^{l_r}=b$ через суммы Якоби. [IR], глава 8.
    \item Теорема Бёрджесса. [Степ], \S II.1.
\end{itemize}


\begin{center} {\bf Тема 10. Дзета функция Артина} \end{center}

\noindent\textbf{Упражнения и задачи}

\begin{enumerate}[topsep=0pt]
    \item Докажите, что $|\mathbb{P}^n(\mathbb{F}_q)|=q^n+q^{n-1}+\dots+1$.
    \item Докажите, что для $f=-y_0^2+y_1^2+y_2^2+y_3^2$ дзета-функция имеет вид $Z_f(u) = (1-u)^{-1}(1-qu)^{-2}(1-q^2u)^{-1}$, если $-1$ --- квадрат в $\mathbb{F}_q$, и $Z_f(u) = (1-u)^{-1}(1-qu)^{-1}(1+qu)^{-1}(1-q^2u)^{-1}$ в противном случае. %[IR, p153]
    \item Докажите, что проективная $n$-мерная гиперплоскость в $\mathbb{P}^n(\mathbb{F}_q)$ (т.е. гиперповерхность, заданная однородным многочленом степени 1) имеет столько же точек сколько $n-1$-мерное проективное пространство $\mathbb{P}^{n-1}(\mathbb{F}_q)$. %[IR Ex 10.4]
    \item Пусть $f(x_0,x_1,x_2)\in \mathbb{F}_q[x_0,x_1,x_2]$ --- однорлный многочлен $\deg f = n$. $h = \in \mathbb{F}_q[x_0,x_1,x_2]$ --- линейная форма, такая что не каждый её нуль является нулём $f$. Докажите, что в $\mathbb{P}^2(\mathbb{F}_q)$ у $f$ и $h$ может быть не более $n$ общих нулей (т.е. плоская проективная кривая пересекается с проективной прямой в не более чем $n$ точках). %[IR Ex 10.5]
    \item Пусть $\SL_n(\mathbb{F}_q)$ --- множество $n\times n$ матриц с элементами из поля $\mathbb{F}_q$ и определителем равным $1$. Покажите, что $\SL_n(\mathbb{F}_q)$ можно рассматривать как гиперповерхность в $\mathbb{A}^{n^2}(\mathbb{F}_q)$ и что число её точек равно $(q-1)^{-1} (q^n-1) (q^n-q) \dots (q^n-q^{n-1})$. %[IR Ex 10.6]
    \item Пусть $\frac{\partial}{\partial x_i}$ --- операторы формальных производных на $\mathbb{F}_q[x_0,\dots,x_n]$ (например, для $f(x)=a_0 x^n +\dots + a_{n-1}x+a_n$ по определению $\frac{\partial}{\partial x} f=a_0 x^{n-1} +\dots + a_{n-1}$ и пусть $f\in\mathbb{F}_q[x_0,\dots,x_n]$ --- однородный многочлен $\deg f = m$. Докажите, что:
    \begin{itemize}[topsep=0pt]
        \item $\sum_{i=0}^n x_i \frac{\partial}{\partial x_i} = mf$; %[IR Ex 10.7]
        \item если $(m,p)=1$ ($p=\chr \mathbb{F}_q$) и для $a=(a_0,\dots,a_n)$ при всех $i$ выполняется $\frac{\partial}{\partial x_i} f(a)=0$, то $f(a)=0$. (Такая точка $a$ называется особой точкой гиперповерхности $f=0$). %[IR Ex 10.8]
    \end{itemize}
    \item Пусть $q=p^n$, $(m,p)=1$. Докажите, что гиперповерхность $a_0 x_0^m + \dots + a_n x_n^m=0$ не имеет особых точек в $\mathbb{P}^n(\mathbb{F}_q)$. %[IR Ex 10.9]
    \item Пусть $q=p^n$, $p\neq 2$. Рассмотрим кривую $ax^2+bxy+cy^22=1$, $a,b,c \in \mathbb{F}_q$. Докажите, что если $d=b^2-4ac$  не является квадратом в $\mathbb{F}_q$, то не существует бесконечно удаленных точек на кривой в $\mathbb{P}^n(\mathbb{F}_q)$, а если $d$ --- квадрат, то существует одна или две бесконечно удаленные точки, в зависимости от обращения $d$ в ноль. При этом если $d=0$, то бесконечно удаленная точка является особой точкой заданной кривой. %[IR Ex 10.13]
    \item Выпишите дзета-функцию кривой $x_0 x_1 - x_2 x_3=0$ над $\mathbb{F}_p$. %[IR Ex 11.4]
    \item Выпишите дзета-функцию для $f = a_0 x_0^2 + \dots + a_n x_n^2$ над $\mathbb{F}_q$ при $\chr(\mathbb{F}_q) \neq 2$. %[IR Ex 11.5]
    \item Покажите, что на кривой $x_0^3+x_1^3+x_2^3 = 0$ в $\mathbb{P}^2(\mathbb{F}_4)$ лежит девять точек. Выпишите дзета-функцию этой кривой. %[IR Ex 11.6]
    \item Выпишите дзета-функцию кривой $y^2=x^3+x^2$ над $\mathbb{F}_p$. %[IR Ex 11.9]
    \item Пусть $q \equiv 1\,(3)$, $\alpha\in\mathbb{F}_q^*$. Покажите, что дзетв-функция кривой $y^2=x^3+\alpha$ над $\mathbb{F}_q$ имеет вид $Z(u) = (1+au+qu^2)(1-u)^{-1}(1-qu)^{-1}$, где $a\in\mathbb{Z}$, $|a|\leqslant 2\sqrt{q}$. %[IR Ex 11.10]
    \item Пусть $C_1$ --- кривая над $\mathbb{F}_p$ заданная $y^2=x^3-Dx$, $D\neq 0$. Покажите, что подстановка $x=\frac{1}{2}(u+v^2)$, $y=\frac{1}{2}v(u+v^2)$ переводит $C_1$ в кривую $C_2$ заданную уравнением $u^2-v^4=4D$. Докажите, что для любого расширения $\mathbb{F}_q/\mathbb{F}_p$ для числа точек справедливо $|C_1(\mathbb{F}_q)| > |C_2(\mathbb{F}_q)|$. %[IR Ex 11.11]
\end{enumerate}

\noindent\textbf{SageMath}
\begin{itemize}[topsep=0pt]
    \item Исследуйте основные функции SageMath связанные с количеством точек на кривых над конечными полями:
    \begin{itemize}[noitemsep,topsep=0pt]
        \item Для эллиптических и гиперэллиптических кривых: \texttt{cardinality()}.
     \end{itemize}
\end{itemize}


\noindent\textbf{Темы для самостоятельного изучения}
\begin{itemize}[topsep=0pt]
    \item $L$-функции Артина. Суперэллиптическое уравнение. [Степ], \S\S I.3--I.4.
\end{itemize}

\begin{center} {\bf Тема 11. $p$-адические числа:\\ элементарное определение и свойства} \end{center}

\noindent\textbf{Упражнения и задачи}

\begin{enumerate}[topsep=0pt]

    \item Докажите, что различные канонические последовательности определяют различные целые $p$-адические числа. %[БШ]

    \item Докажите, что для целых $p$-адических чисел $\alpha$ и $\beta$ заданные в лекции операции $\alpha \beta$, $\alpha + \beta$ корректно определены (то есть результат не зависит от выбора последовательностей-представителей $\alpha \sim (x_n)$, $\beta \sim (y_n)$) и $\mathbb{Z}_p$ --- действительно коммутативное кольцо. %[БШ]

    \item Пусть $\alpha = \sum_{n=0}^{\infty} a_n p^n \in \mathbb{Z}_p$. Какой будет иметь вид разложение числа $-\alpha$? %[БШ, G]

    \item Докажите, что уравнение $x^2=2$ не имеет решений в $\mathbb{Q}_5$.

    \item Докажите, что $\forall \alpha \in \mathbb{Z}_p\ \exists a \in \mathbb{Z}:\ \alpha \equiv a\ \pmod{p^n}$. Для $a, b \in \mathbb{Z}$ $a \equiv b\ \pmod{p^n}$ как сравнение в $\mathbb{Z}_p$ $\Leftrightarrow$ $a \equiv b\ \pmod{p^n}$ как сравнение в $\mathbb{Z}$. %[БШ]

    \item Пусть $p\neq 2$, $c$ --- квадратичный вычет $\Mod p$. Докажите, что существует два различных $p$-адических числа $\alpha, \beta \in \mathbb{Q}_p:\ \alpha^2 = \beta^2 = c$. %[БШ]

    \item Пусть $p\neq 2$, $(m,p)=1$. Сформулируйте и докажите необходимое и достаточное условие разрешимости уравнения $x^2=m$ в $\mathbb{Q}_p$. Сделайте вывод, что $\mathbb{Q}_p$ не является алгебраически замкнутым.

    \item Докажите, что если $\xi_n \rightarrow \xi$ в $\mathbb{Q}_p$ и $\xi\neq 0$, то $1 / \xi_n \rightarrow 1/\xi$ в $\mathbb{Q}_p$. %[БШ]

    \item Докажите $p$-адический аналог утверждения из анализа: из всякой ограниченной последовательности можно выделить сходящуюся подпоследовательность. %[БШ]

    \item Докажите $p$-адический критерий Коши: последовательность $(\xi_n)$ сходится $\Leftrightarrow$\\ ${\nu_p(\xi_m-\xi_n) \rightarrow \infty}$, при $m,n \rightarrow \infty$. %[БШ]

    \item Пусть последовательность $(x_n)$ определена как $x_n = 1+p+\dots +p^{n-1}$. Докажите, что в $\mathbb{Q}_p$ $x_n \rightarrow 1/(1-p)$, $n \rightarrow \infty$. %[БШ]

    %\item Пусть $c \in \mathbb{Z}$, $p \nmid c$. Докажите, что последовательность $(c^{p^n})$ сходится в $\mathbb{Q}_p$, при этом для $\gamma = \lim c^{p^n}$ имеем $\gamma \equiv c \pmod{p}$, $\gamma^{p-1}=1$. Используя этот результат, докажите что в $\mathbb{Q}_p[t]$ многочлен $t^{p-1}-1$ раскладывается на линейные множители.
    
    \item Докажите, что для $0 \neq \xi \in \mathbb{Q}_p \cap \mathbb{Q}$ представление $\xi = \sum_{n=0}^{\infty} a_n p^n$, $0 \leqslant a_n \leqslant p-1$ имеет периодические коэффициенты (начиная с некоторого номера $k_0$, т.е. $\exists m: \forall k \ge k_0\ a_{m+k} = a_{k}$). Обратно всякий такой ряд представляет рациональное число.
    
\end{enumerate}

\noindent\textbf{SageMath}

\begin{itemize}[topsep=0pt]
    \item Исследуйте основные функции SageMath связанные с арифметикой $p$-адических чисел. Определение кольца и поле $p$-адических чисел: \texttt{Zp(n)}, \texttt{Qp(n)}. Рассмотрите примеры уравнения $x^2=m$ (используйте функцию \texttt{sqrt()}).
        
\end{itemize}

%\noindent\textbf{Темы для самостоятельного изучения}




\begin{center} {\bf Тема 12. Аксиоматическое определение поля $p$-адических чисел,\\ метризованные поля} \end{center}

\noindent\textbf{Упражнения и задачи}
\begin{enumerate}[topsep=0pt]

    \item Пусть $(k,\varphi)$ --- метризованное поле. Докажите следующие свойства:
    \begin{itemize}[topsep=0pt]
        \item $\varphi(\pm 1)=1$; $\varphi (-x) = \varphi (x)$;
        \item $\varphi (x - y) \leqslant \varphi(x) + \varphi(y)$;
        \item $\varphi(x \pm y) \geqslant |\varphi(x) - \varphi(y)|$;
        \item $\varphi(x/y) = \varphi(x)/\varphi(y)$, $y \neq 0$.
    \end{itemize}

    \item Пусть $(k,\varphi)$ --- метризованное поле, $d$ --- индуцированное расстояние: $d(x,y) = \varphi(x-y)$. Докажите, что операции поля ($+,-,\cdot,/$) являются непрерывными по отношению к $d$ (то есть $k$ --- топологическое поле).

    \item Пусть $(k,\varphi)$ --- метризованное поле. Докажите, что $\lim\limits_{n\rightarrow\infty} x_n = x$ $\Leftrightarrow$ каждое открытое множество содержащее $x$ содержит все кроме конечного числа элементы последовательности $x_n$.
    
    \item Пусть $k$ --- поле, на котором заданы две метрики (абсолютные величины) $\varphi_1$, $\varphi_2$. Докажите следующие импликации теоремы о критериях эквивалентности:
    \begin{itemize}[topsep=0pt]
        \item $\varphi_1$, $\varphi_2$ --- эквивалентны $\implies$ для любой сходящейся последовательности\\ ${\lim\limits_{n\rightarrow\infty}}^{(\varphi_1)}\ x_n = x$ если и только если ${\lim\limits_{n\rightarrow\infty}}^{(\varphi_2)}\ x_n = x$ (${\lim\limits}^{(\varphi)}$ означает предел по метрике $\varphi$);
        \item ${\lim\limits_{n\rightarrow\infty}}^{(\varphi_1)}\ x_n = {\lim\limits_{n\rightarrow\infty}}^{(\varphi_2)}\ x_n = x$  $\implies$ $\forall x \in k$ $\varphi_1(x)<1$ если и только если $\varphi_2(x)<1$;
        \item $\exists \alpha \in \mathbb{R}$: $\forall x \in k\ \varphi_1(x)=\varphi_2(x)^\alpha$ $\implies$ $\varphi_1$, $\varphi_2$ эквивалентны.
    \end{itemize}

    \item Пусть $k$ --- поле, $\varphi$ --- функция $k \rightarrow \mathbb{R}_{>0}$ такая что:
    \begin{itemize}[topsep=0pt]
        \item $\varphi(x) = 0 \Leftrightarrow x=0$,
        \item $\varphi(xy) = \varphi(x)\varphi(y)$,
        \item $\varphi(x) \leqslant 1 \Rightarrow \varphi(x-1) \leqslant 1$.
    \end{itemize}
    Докажите, что $\varphi$ является неархимедовой метрикой на $k$.

    \item Пусть $(k,\varphi)$ --- метризованное поле , $\varphi$ --- неархимедова метрика. Докажите, что $\varphi(x) \neq \varphi(y)$ $\Rightarrow$ $\varphi(x+y) = \max(\varphi(x),\varphi(y))$.

    \item Пусть $(k,\varphi)$ --- метризованное поле, $A$ --- образ $\mathbb{Z}$ в $k$. Докажите, что $\varphi$ --- неархимедова метрика $\Leftrightarrow$ $\forall a \in A$ $\varphi(a) \leqslant 1$. (Подсказка: сведите к утверждению $\varphi$ --- неархимедова метрика $\Leftrightarrow$ $\varphi(x+1) \leqslant \max(\varphi(x),1)$; рассмотрите $\varphi(x+1)$).

    \item Пусть $(k,\varphi)$ --- метризованное поле, $\varphi$ --- неархимедова метрика, $B(x,r)$ --- открытый шар радиуса $r$ с центром в $x$. Докажите следующие свойства:
    \begin{itemize}[topsep=0pt]
        \item $\forall y \in B(x,r)$ $B(x,r)=B(y,r)$;
        \item $\partial B(x,r) = \varnothing$ ($\partial B(x,r)$ обозначает множество граничных точек);
        \item $B(x,r) \cap B(y,s) \neq \varnothing$ $\Leftrightarrow$ $B(x,r) \subset B(y,s)$ или $B(y,s) \subset B(x,r)$.
    \end{itemize}
    Рассмотрите аналогичные утверждения для замкнутых шаров $\bar B(x,r)$.

    \item Пусть $\chr k = p$. Докажите, что всякая метрика $\varphi$ поля $k$ неархимедова.

    \item Пусть $k$ --- поле, $k(t)=\{f(t)/g(t): f,g \in k[t], g \neq 0\}$ --- поле рациональных функций над $k$. $\forall r \in k(t)^*$ определим $\varphi(r)=\rho^m$, где $m$ такое, что $r=f/g=t^m(f_0/g_0)$, где $f_0, g_0$ не делятся на $t$ как многочлены, $0 < \rho < 1$; для $r=0$ положим $\varphi(0)=0$. Докажите, что $\phi$ --- метрика поля $k(t)$.

    %\item Пуст $C$ --- Канторово множество. (напомним итеративное определение: $C_0=[0,1]$ --- единичный интервал, удалим среднюю часть длины $1/3$, получим $C_1=[0,\frac{1}{3}]\cup [\frac{2}{3},1]$, далее удалим средние части из каждого подынтервала, получим\\ $C_2=[0,\frac{1}{9}]\cup [\frac{2}{9},\frac{1}{3}]\cup [\frac{2}{3},\frac{7}{9}]\cup [\frac{8}{9},1]$; продолжая аналогичным образом, построим $C_n$, заметим, что $C_n = \frac{C_{n-1}}{3} \cup \left(\frac{2}{3} + \frac{C_{n-1}}{3} \right)$. Определим $C = \bigcap\limits_{n=0}^\infty C_n$).\\
    Докажите, что множество $2$-адических чисел $\mathbb{Z}_2$ с $2$-адической метрикой $|\cdot|_2$ гомеоморфно Канторову множеству $C$ с обычным модулем $|\cdot|=|\cdot|_\infty$.

\end{enumerate}

\noindent\textbf{SageMath}
\begin{itemize}[topsep=0pt]
    \item В контексте задач 11 и 12 ознакомьтесь с функцией \texttt{Zp(n).plot()}.
\end{itemize}

\noindent\textbf{Темы для самостоятельного изучения}
\begin{itemize}[topsep=0pt]
    \item Единственность пополнения поля по метрике. [БШ] \S I.4.
    \item $\forall$ простого $p$ множество целых $p$-адических чисел $\mathbb{Z}_p$ гомеоморфно множеству $2$-адических чисел $\mathbb{Z}_2$. [Kat], глава 2. %+источник который нашла Диана
\end{itemize}


\begin{center} {\bf Тема 13. Лемма Гензеля,\\ сравнения и кольцо целых $p$-адических чисел} \end{center}

\noindent\textbf{Упражнения и задачи}
\begin{enumerate}[topsep=0pt]

    \item Пусть $k$ --- произвольное поле, $F(X) \in k[x]$. Докажите формулу Тейлора для формальной производной $F'$.

    \item Докажите, что порядок фактор группы $\mathbb{Q}_2^*/(\mathbb{Q}_2^*)^2$ равен $8$. Укажите соответствующее множество представителей.

    \item Пусть $p \neq 2$, $\alpha, \beta \in \mathbb{Z}_p$, $p \nmid \alpha, p \nmid \beta$. Докажите, что разрешимость сравнения $\alpha x^p \equiv \beta\ (\Mod p^2)$ достаточна для разрешимости уравнения $\alpha x^p = \beta$ в $\mathbb{Q}_p$. %БШ задача 3 стр 58

    \item Пусть $p \neq 2$, $U=U(\mathbb{Z}_p)$ --- группа $p$-адических единиц, $U_n = 1+p^n \mathbb{Z}_p$. ($U_n = \{\alpha \in \mathbb{Z}_p: \nu_p(\alpha - 1) \geqslant n \}$, т.е. $U_n$ --- $p$-адические окрестности единичного элемента). Докажите следующие свойства:
    \begin{itemize}[topsep=0pt]
        \item $U_n/U_{n+1} = \mathbb{Z}/p\mathbb{Z}$, $U=\lim\limits_{\longleftarrow} U_n$;
        \item $U=U_1 V$, где $V$ --- циклическая подгруппа корней степени $p-1$ из единицы;
        \item Если $\alpha \in U_n \setminus U_{n+1}$, то $\alpha^p \in U_{n+1} \setminus U_{n+2}$;
        \item $U_1/U_n$ --- циклическая группа, $U_1 \cong \mathbb{Z}_p$.
    \end{itemize} 
    Сделайте вывод о структуре мультипликативной группы $\mathbb{Q}_p$: $\mathbb{Q}_p^* \cong \mathbb{Z} \times \mathbb{Z}_p \times \mathbb{Z}/(p-1)\mathbb{Z}$.
    %https://old.mccme.ru/ium/postscript/f10/AZ-Problems_3.pdf

    \item *Пусть $F(x_1,\dots, x_n) \in \mathbb{Z}_p[x_1,\dots, x_n]$, $N_m$ --- число решений сравнения $F \equiv 0 \pmod{p^m}$, $L_F(u) = \sum_{m=0}^\infty N_m u^m$ --- так называемый ряд Пуанкаре (вспомните дзета функцию Артина).
    \begin{itemize}[topsep=0pt]
        \item Найдите ряд Пуанкаре для $F = \alpha_1 x_1 + \dots + \alpha_n x_n$, где $\alpha_i \in \mathbb{Z}_p^*$. Убедитесь, что в этом случае $L_F(u)$ --- рациональная функция.
        \item Найдите ряд Пуанкаре для многочлена $F(x_1,\dots, x_n)$, обладающего свойством: для всякого решения сравнения $F(x_1,\dots, x_n) \equiv 0 \pmod{p}$ при некотором $i$ имеем $F'_{x_i}(x_1,\dots, x_n) \not\equiv 0 \pmod{p}$;
        \item Найдите ряд Пуанкаре для $F(x,y)=x^2-y^3$;
        \item Докажите рациональность ряда Пуанкаре для многочленов одной переменной. 
    \end{itemize}
    (Существует теорема (Игусы) о том, что $L_F$ всегда является рациональной функцией).
    %БШ задачи 9-12 стр 59

    \item Докажите $p$-адический критерий Эйзенштейна: пусть $f\in \mathbb{Z}_p[x]$, $f(x) = a_0 x^n+ \dots a_n$, $f$ -- неприводим, если $p \nmid a_0$, $p \mid a_i\ 1\leqslant i \leqslant n$, $p^2 \nmid a_n$.

    \item Верно ли следующее: $f\in \mathbb{Z}[x]$ неприводим в $f\in \mathbb{Q}[x]$ $\Leftrightarrow$ $f$ неприводим в $ \mathbb{Q}_p[x]$ $\forall p\leqslant\infty$? %G131

    \item Докажите, что уравнение $(x^2-2)(x^2-17)(x^2-34)=0$ разрешимо в $\mathbb{Q}_p$ $\forall p\leqslant \infty$ но не разрешимо в $\mathbb{Q}$.
  
\end{enumerate}

\noindent\textbf{SageMath}
\begin{itemize}[topsep=0pt]
    \item Исследуйте функции SageMath для работы с многочленами с $p$-адическими коэффициентами:
    \begin{itemize}[noitemsep,topsep=0pt]
        \item Разложение многочлена: \texttt{factor()}.
        \item В контексте Леммы Гензеля: \texttt{hensel\_lift()}.
    \end{itemize}
\end{itemize}

\noindent\textbf{Темы для самостоятельного изучения}
\begin{itemize}[topsep=0pt]
    \item Разрешимость уравнений с квадратичными формами над полем $p$-адических чисел ([БШ \S I.6 п.2).
    \item Приложение теоремы Минковского-Хассе для квадратичной формы от трёх переменных ([Gouv \S 4.8]).
\end{itemize}




\begin{center} {\bf Тема 14. Кольцо целых гауссовых чисел, числовые поля} \end{center}

\noindent\textbf{Упражнения и задачи}
\begin{enumerate}[topsep=0pt]

    \item Пусть $R$ --- евклидово кольцо (с нормой $N(\cdot)$). Докажите, что $u$ --- единица $R$ $\iff$ $N(u)=1$.
    
    \item Пусть $R$ --- кольцо. Докажите следующие утверждения (свойства делимости на языке идеалов):
    \begin{itemize}[noitemsep,topsep=0pt]
        \item $a|b$ $\Leftrightarrow$ $(b) \subseteq (a)$;
        \item $u$ --- единица $\Leftrightarrow$ $(u)=R$;
        \item $a,b$ --- ассоциированы $\Leftrightarrow$ $(a)=(b)$;
        \item $p$ --- простой элемент $\Leftrightarrow$ $ab \in (p)$ $\Rightarrow$ $a \in (p)$ или $b \in (p)$;
        \item $p$ --- неприводимый элемент $\Leftrightarrow$ $(p) \subseteq (a)$ $\Rightarrow$ $(a) = R$ или $(a) = (p)$.
    \end{itemize}

    \item Пусть $R$ --- кольцо главных идеалов, $a,b \in R$ $d$ --- НОД $a,b$. Докажите, что $\exists d \in R: (d)=(a,b)$. 
    
    \item Пусть $R$ --- кольцо главных идеалов. Докажите, что $p\in R$ --- неприводимый элемент $\iff$ $p$ --- простой.
    
    \item Докажите свойство показателя в кольце главных идеалов: если $p$ --- неприводимый элемент, $a,b \in R^*$, то $\nu_p(ab) = \nu_p(a)+\nu_p(b)$.
    
    \item Докажите теорему об однозначности разложения на простые множители в кольцах главных идеалов.
    
    \item Пусть $\pi\in\mathbb{Z}[i]$ --- простой элемент, $\nu_\pi(\alpha)$ --- соответствующий показатель, $|\alpha|_\pi = (\N p)^{-\nu_\pi(\alpha)}$ --- метрика заданная на $\mathbb{Z}[i]$ и $\mathbb{Q}(i)$. Опишите ограничение этой метрики на $\mathbb{Q}$. %[Gouv]
    
    \item Докажите, что $\mathbb{Z}[\omega]$ --- евклидово кольцо. Найдите единицы $\mathbb{Z}[\omega]$. %[IR]
    
    \item Докажите, что для функций определенных в лекции выполняется $d(n_1)=d_1(n)-d_3(n)$. %[DSV]
    
    \item Докажите оценку для числа представлений в виде суммы двух квадратов: $r_2(n) = \mathcal{O}_\varepsilon (n^\varepsilon)$. %[DSV]
  
\end{enumerate}

\noindent\textbf{SageMath}
\begin{itemize}[topsep=0pt]
    \item Рассмотрите примеры арифметики кольца гауссовых чисел: \texttt{ZZ[I]}, исследуйте базовые функции такие как \texttt{gcd(), xgcd(), factor(), ...}.
    \item Исследуйте функции для нахождения разложений целых чисел в виде суммы двух и четырёх квадратов, например, библиотека \texttt{sum\_of\_squares}.
\end{itemize}

\noindent\textbf{Темы для самостоятельного изучения}
\begin{itemize}[topsep=0pt]
    \item Арифметика кольца чисел Эйзенштейна $\mathbb{Z}[\omega]$, [IR], \S\S 9.1--9.2.
    \item Алгебра кватернионов, число представлений суммой четырёх квадратов, [DSV], \S\S 2.5--2.6.
\end{itemize}

\begin{center} {\bf Тема 15. Делимость в кольцах целых алгебраических чисел} \end{center} TODO
\begin{center} {\bf Тема 16. Квадратичное поле и круговое поле} \end{center} TODO

%\begin{center} {\bf Тема 16. Ряды Дирихле} \end{center} TODO
%\begin{center} {\bf Тема 17. Распределение простых чисел\\ в арифметической прогрессии} \end{center} TODO

\section{Фонд оценочных средств (ФОС, оценочные и методические материалы) для оценивания результатов обучения по дисциплине (модулю)}

\subsection{Типовые контрольные задания или иные материалы для проведения текущего контроля успеваемости}

Вопросы к зачёту:

\noindent
\begin{longtable}{ | p{0.8cm} | p{11cm} | p{2cm} | } 
    \hline
    \bf № & \textbf{Вопрос} & \textbf{Раздел и тема дисциплины} \\
    \hline
    \hline
    \endhead
    1 & Делимость в кольце целых чисел, алгоритм Евклида. Бесконечность числа простых чисел. Однозначное разложение целых чисел на простые множители & 1 (1.1) \\ \hline
    2 & Мультипликативные функции, функции Эйлера и Мёбиуса. Формула обращения Мёбиуса & 1 (1.1) \\ \hline
    3 & Неравенства Чебышева & 1 (1.1) \\ \hline
    4 & Сравнения, полная и приведенная система вычетов. Число решений линейного сравнения. Теоремы Эйлера и Ферма & 2 (1.2) \\ \hline
    5 & Китайская теорема об остатках & 2 (1.2) \\ \hline
    6 & Однозначность разложения в кольце многочленов & 2 (1.2) \\ \hline
    7 & Теорема Лагранжа о числе корней многочлена. Теорема Вильсона & 3 (1.3) \\ \hline
    8 & Первообразные корни и структура группы единиц по модулю m & 3 (1.3) \\ \hline
    9 & Степенные вычеты & 3 (1.3) \\ \hline
    10 & Квадратичные вычеты, символ Лежандра и его свойства & 4 (1.4) \\ \hline
    11 & Квадратичный закон взаимности & 4 (1.4) \\ \hline
    12 & Сравнения по двойному модулю, существование неприводимых многочленов произвольной степени над конечным полем & 5 (2.1) \\ \hline
    13 & Мультипликативная группа конечного поля & 5 (2.1) \\ \hline
    14 & Автоморфизм Фробениуса, группа Галуа конечного поля & 5 (2.1) \\ \hline
    15 & Круговое поле, группа корней из единицы & 6 (2.2) \\ \hline
    16 & Круговой многочлен, разложение кругового многочлена на неприводимые множители над конечным полем & 6 (2.2) \\ \hline
    17 & Норма и след, их свойства. Абсолютные и относительные норма и след & 7 (2.3) \\ \hline
    18 & Характеры абелевых групп & 7 (2.3) \\ \hline
    19 & Сумма Гаусса, значение её модуля & 7 (2.3) \\ \hline
    20 & Теоремы Варинга и Шевалле & 8 (2.4) \\ \hline
    21 & Теорема Виноградова-Пойя & 8 (2.4) \\ \hline
    22 & Суммы Якоби и их свойства & 8 (2.4) \\ \hline
    23 & Аффинное и проективное пространства над конечным полем & 9 (2.5) \\ \hline
    24 & Соотношение Хассе-Дэвенпорта & 9 (2.5) \\ \hline
    25 & Дзета функция Артина, критерий её рациональности & 9 (2.5) \\ \hline
    26 & Построение кольца целых $p$-адических чисел & 10 (3.1) \\ \hline
    27 & Делимость и единицы в кольце целых $p$-адических чисел, $p$-показатель и $p$-адическая метрика & 10 (3.1) \\ \hline
    28 & Ряды и последовательности p-адических чисел, их сходимость & 10 (3.1) \\ \hline
    29 & Метризованные поля, эквивалентность метрик пополнение по метрике & 11 (3.2) \\ \hline
    30 & Теорема Островского & 11 (3.2) \\ \hline
    31 & Неархимедовы метрики и их топологические свойства & 11 (3.2) \\ \hline
    32 & Кольцо и идеал показателя. Локальные кольца & 12 (3.3) \\ \hline
    33 & Лемма Гензеля & 12 (3.3) \\ \hline
    34 & Делимость и разложение на множители в кольцах $\mathbb{Z}[i]$, $\mathbb{Z}[\omega]$ & 13 (4.1) \\ \hline
    35 & Кубический и биквадртичный характеры. Высшие законы взаимности & 13 (4.1) \\ \hline
    36 & Числовые поля. Норма, след и дискриминант & 14 (4.2) \\ \hline
    37 & Дедекиндовы кольца, однозначность разложения на простые идеалы & 14 (4.2) \\ \hline
    38 & Квадратичное поле и его кольцо целых. Порядки в квадратичном поле & 15 (4.3) \\ \hline
    39 & Круговое поле и его кольцо целых. Круговой многочлен & 15 (4.3) \\ \hline
%    40 & Характеры Дирихле. Ряды Дирихле и их свойства & 16 (5.1) \\ \hline
%    41 & Аналитическое продолжение рядов Дирихле & 16 (5.1) \\ \hline
%    42 & Теорема Дирихле & 17 (5.2) \\ \hline
%    43 & Дзета-фукнция Дедекинда & 17 (5.2) \\ \hline

\end{longtable}

Примеры контрольных задач приведены в разделе \ref{problems}.

\section{Ресурсное обеспечение}

\subsection{Перечень основной и дополнительной литературы}

\begin{thebibliography}{Serre}
    
    \bibitem[Вин]{Vin}
    И.М. Виноградов, Основы теории чисел. Наука, 1981.

    \bibitem[IR]{IR}
    К. Айерленд, М. Роузен, Классическое введение в современную теорию чисел, МИР, 1987 (K. Ireland, M. Rosen, A Classical Introduction to Modern Number Theory. Springer, 1990).

    \bibitem[БШ]{BSh}
    З.И. Боревич, И.Р. Шафаревич, Теория чисел. Наука, 1972.

    \bibitem[Serre]{Serre}
    Ж.-П. Серр, Курс арифметики. МИР, 1972 (J.-P. Serre, Cours D’Arithmetique, Presses Universitaire de France, 1970).

    \bibitem[Степ]{Step}
    С.А. Степанов, Арифметика алгебраических кривых. Наука, 1991.

    \bibitem[ЛН]{LN}
    R. Lidl, H. Niederreiter, Introduction to Finite Fields and their Applications. Cambridge University Press, 1994.

    \bibitem[Gouv]{Gouv}
    F. Gouvêa, p-adic Numbers: An Introduction. Springer, 2020.

    \bibitem[Marc]{Marc}
    D.A. Marcus, Number Fields. Springer, 2018.

    \bibitem[Cox]{Cox}
    D.A. Cox, Primes of the Form x2+ny2. Wiley, 2013.

    \bibitem[DSV]{DSV}
    G. Davidoff, P. Sarnak, A. Valette, Elementary Number Theory, Group Theory, and Ramanujan Graphs, Cambridge University Press, 2003.

    \bibitem[Stein-ent]{Stein-ent}
    W. Stein, Elementary Number Theory: Primes, Congruences, and Secrets, 2017.

\end{thebibliography}

\subsection{Перечень лицензионного программного обеспечения, в том числе отечественного производства}

При реализации дисциплины может быть использовано следующее программное обеспечение:
\begin{itemize}
    \item Операционная система Linux (Свободно-распространяемое ПО) / MacOS / Windows;
    \item Язык программирования Python и система компьютерной алгебры SageMath. Свободно-распространяемое ПО;
    \item Среда разработки Jupyter / VS Code / Vim (Emacs), ... Свободно-распространяемое ПО;
    \item Издательская система LaTeX. Свободно-распространяемое ПО.
\end{itemize}

\subsection{Перечень профессиональных баз данных и информационных справочных систем}

\begin{enumerate}
    \item http://www.edu.ru --- портал Министерства образования и науки РФ;
    \item http://www.ict.edu.ru --- система федеральных образовательных порталов «ИКТ в образовании»;
    \item http://www.openet.ru --- Российский портал открытого образования;
    \item http://www.mon.gov.ru  --- Министерство образования и науки Российской Федерации;
    \item http://www.fasi.gov.ru --- Федеральное агентство по науке и инновациям.
\end{enumerate}

\subsection{Перечень ресурсов информационно-телекоммуникационной сети «Интернет»}

\begin{enumerate}
    \item Math-Net.Ru [Электронный ресурс] : общероссийский математический портал / Математический институт им. В. А. Стеклова РАН ; Российская академия наук, Отделение математических наук. - М. : [б. и.], 2010. - Загл. с титул. экрана. - Б. Ц. URL: http://www.mathnet.ru;
    \item Университетская библиотека Online [Электронный ресурс] : электронная библиотечная система / ООО "Директ-Медиа" . - М. : [б. и.], 2001. - Загл. с титул. экрана. - Б. ц.  URL: www.biblioclub.ru;
    \item Универсальные базы данных EastView [Электронный ресурс] : информационный ресурс / EastViewInformationServices. - М. : [б. и.], 2012. - Загл. с титул. экрана. - Б. Ц. URL: www.ebiblioteka.ru;
    \item Научная электронная библиотека eLIBRARY.RU [Электронный ресурс] : информационный портал / ООО "РУНЭБ" ; Санкт-Петербургский государственный университет. - М. : [б. и.], 2005. - Загл. с титул. экрана. - Б. Ц. URL: www.eLibrary.ru.
\end{enumerate}

\subsection{Описание материально-технического обеспечения}

Образовательная организация, ответственная за реализацию данной Программы, располагает соответствующей материально-технической базой, включая современную вычислительную технику, объединенную в локальную вычислительную сеть, имеющую выход в Интернет. Используются специализированные компьютерные классы, оснащенные современным оборудованием. Материальная база соответствует действующим санитарно-техническим нормам и обеспечивает проведение всех видов занятий (лекционных, практических, семинарских, лабораторных, дисциплинарной и междисциплинарной подготовки) и научно-исследовательской работы обучающихся, предусмотренных учебным планом.

\section{Методические рекомендации по организации изучения дисциплины}

\subsection{Формы и методы преподавания дисциплины}

\begin{itemize}
    \item Используемые формы и методы обучения: лекции и семинары, самостоятельная работа студентов.
    \item В процессе преподавания дисциплины преподаватель использует как классические формы и методы обучения (лекции и практические занятия), так и активные методы обучения. 
    \item При проведении лекционных занятий преподаватель использует аудиовизуальные, компьютерные и мультимедийные средства обучения, а также демонстрационные и наглядно-иллюстрационные (в том числе раздаточные) материалы.
    \item Семинарские (практические) занятия по данной дисциплине проводятся с использованием компьютерного и мультимедийного оборудования, при необходимости - с привлечением полезных Интернет-ресурсов и пакетов прикладных программ. 
\end{itemize}

\subsection{Методические рекомендации преподавателю}

Перед началом изучения дисциплины преподаватель должен ознакомить студентов с видами учебной и самостоятельной работы, перечнем литературы и интернет-ресурсов, формами текущей и промежуточной аттестации, с критериями оценки качества знаний для итоговой оценки по дисциплине. 
При проведении лекций, преподаватель:
\begin{itemize}[noitemsep,topsep=0pt]
    \item формулирует тему и цель занятия;
    \item излагает основные теоретические положения;
    \item сопровождает теоретические положения наглядными примерами (численные результаты и частные случаи);
    \item в конце занятия дает вопросы для самостоятельного изучения.
\end{itemize}

Во время выполнения заданий в учебной аудитории студент может консультироваться с преподавателем, определять наиболее эффективные методы решения поставленных задач. Если какая-то часть задания остается не выполненной, студент может продолжить её выполнение во время внеаудиторной самостоятельной работы.

Перед выполнением внеаудиторной самостоятельной работы преподаватель проводит инструктаж (консультацию) с определением цели задания, его содержания, сроков выполнения, основных требований к результатам работы, критериев оценки, форм контроля и перечня источников и литературы.

Для оценки полученных знаний и освоения учебного материала по каждому разделу и в целом по дисциплине преподаватель использует формы текущего, промежуточного и итогового контроля знаний обучающихся.

\vspace{8pt}
{\bf Для семинарских занятий}

Подготовка к проведению занятий проводится регулярно. Организация преподавателем семинарских занятий должна удовлетворять следующим требования: количество занятий должно соответствовать учебному плану программы, содержание планов должно соответствовать программе, план занятий должен содержать перечень рассматриваемых вопросов.

Во время семинарских занятий используются словесные методы обучения, как беседа и дискуссия, что позволяет вовлекать в учебный процесс всех слушателей и стимулирует творческий потенциал обучающихся. 

При подготовке семинарскому занятию преподавателю необходимо знать план его проведения, продумать формулировки и содержание учебных вопросов, выносимых на обсуждение. 

В начале занятия преподаватель должен раскрыть теоретическую и практическую значимость темы занятия, определить порядок его проведения, время на обсуждение каждого учебного вопроса. В ходе занятия следует дать возможность выступить всем желающим и предложить выступить тем слушателям, которые проявляют пассивность.

Целесообразно, в ходе обсуждения учебных вопросов, задавать выступающим и аудитории дополнительные и уточняющие вопросы с целью выяснения их позиций по существу обсуждаемых проблем, а также поощрять выступление с места в виде кратких дополнений. На занятиях проводится отработка практических умений под контролем преподавателя

\vspace{8pt}
{\bf Для практических занятий}

Подготовка преподавателя к проведению практического занятия начинается с изучения исходной документации и заканчивается оформлением плана проведения занятия.

На основе изучения исходной документации у преподавателя должно сложиться представление о целях и задачах практического занятия и о том объеме работ, который должен выполнить каждый обучающийся. Далее можно приступить к разработке содержания практического занятия. Для этого преподавателю (даже если он сам читает лекции по этому курсу) целесообразно вновь просмотреть содержание лекции с точки зрения предстоящего практического занятия. Необходимо выделить понятия, положения, закономерности, которые следует еще раз проиллюстрировать на конкретных задачах и упражнениях. Таким образом, производится отбор содержания, подлежащего усвоению.

Важнейшим элементом практического занятия является учебная задача (проблема), предлагаемая для решения. Преподаватель, подбирая примеры (задачи и логические задания) для практического занятия, должен представлять дидактическую цель: привитие каких навыков и умений применительно к каждой задаче установить, каких усилий от обучающихся она потребует, в чем должно проявиться творчество студентов при решении данной задачи.

Преподаватель должен проводить занятие так, чтобы на всем его протяжении студенты были заняты напряженной творческой работой, поисками правильных и точных решений, чтобы каждый получил возможность раскрыться, проявить свои способности. Поэтому при планировании занятия и разработке индивидуальных заданий преподавателю важно учитывать подготовку и интересы каждого студента. Педагог в этом случае выступает в роли консультанта,  способного вовремя оказать необходимую помощь, не подавляя самостоятельности и инициативы студента.

\subsection{Методические рекомендации студентам по организации самостоятельной работы}

Приступая к изучению новой учебной дисциплины, студенты должны ознакомиться с учебной программой, учебной, научной и методической литературой, имеющейся в библиотеке университета, встретиться с преподавателем, ведущим дисциплину, получить в библиотеке рекомендованные учебники и учебно-методические пособия, осуществить запись на соответствующий курс в среде электронного обучения университета.

Глубина усвоения дисциплины зависит от активной и систематической работы студента на лекциях и практических занятиях, а также в ходе самостоятельной работы, по изучению рекомендованной литературы. 

На лекциях важно сосредоточить внимание на ее содержании. Это поможет лучше воспринимать учебный материал и уяснить взаимосвязь проблем по всей дисциплине. Основное содержание лекции целесообразнее записывать в тетради в виде ключевых фраз, понятий, тезисов, обобщений, схем, опорных выводов. Необходимо обращать внимание на термины, формулировки, раскрывающие содержание тех или иных явлений и процессов, научные выводы и практические рекомендации. Желательно оставлять в конспектах поля, на которых делать пометки из рекомендованной литературы, дополняющей материал прослушанной лекции, а также подчеркивающие особую важность тех или иных теоретических положений. С целью уяснения теоретических положений, разрешения спорных ситуаций необходимо задавать преподавателю уточняющие вопросы. Для закрепления содержания лекции в памяти, необходимо во время самостоятельной работы внимательно прочесть свой конспект и дополнить его записями из учебников и рекомендованной литературы. Конспектирование читаемых лекций и их последующая доработка способствует более глубокому усвоению знаний, и поэтому являются важной формой учебной деятельности студентов.

\vspace{8pt}
{\bf Методические указания для обучающихся по подготовке к семинарским занятиям}

Для того чтобы семинарские занятия приносили максимальную пользу, необходимо помнить, что упражнение и решение задач проводятся по вычитанному на лекциях материалу и связаны, как правило, с детальным разбором отдельных вопросов лекционного курса. Следует подчеркнуть, что только после усвоения лекционного материала с определенной точки зрения (а именно с той, с которой он излагается на лекциях) он будет закрепляться на семинарских занятиях как в результате обсуждения и анализа лекционного материала, так и с помощью решения проблемных ситуаций, задач.

При этих условиях студент не только хорошо усвоит материал, но и научится применять его на практике, а также получит дополнительный стимул (и это очень важно) для активной проработки лекции.

При самостоятельном решении задач нужно обосновывать каждый этап решения, исходя из теоретических положений курса. Если студент видит несколько путей решения проблемы (задачи), то нужно сравнить их и выбрать самый рациональный. Полезно до начала вычислений составить краткий план решения проблемы (задачи). Решение проблемных задач или примеров следует излагать подробно, вычисления располагать в строгом порядке, отделяя вспомогательные вычисления от основных. Решения при необходимости нужно сопровождать комментариями, схемами, чертежами и рисунками. 

Следует помнить, что решение каждой учебной задачи должно доводиться до  окончательного логического ответа, которого требует условие, и по возможности с выводом. Полученный ответ следует проверить способами, вытекающими из существа данной задачи. Полезно также (если возможно) решать несколькими способами и сравнить полученные результаты. Решение задач данного типа нужно продолжать до приобретения твердых навыков в их решении. 

При подготовке к семинарским занятиям следует использовать основную литературу из представленного списка, а также руководствоваться приведенными указаниями и рекомендациями. Для наиболее глубокого освоения дисциплины рекомендуется изучать литературу, обозначенную как «дополнительная» в представленном списке.

\vspace{8pt}
{\bf Методические указания для обучающихся по подготовке к практическим занятиям}

Целью практических занятий по данной дисциплине является закрепление теоретических знаний, полученных при изучении дисциплины. 

При подготовке к практическому занятию целесообразно выполнить следующие рекомендации: изучить основную литературу; ознакомиться с дополнительной литературой, новыми публикациями в периодических изданиях: журналах, газетах и т. д.; при необходимости доработать конспект лекций. При этом учесть рекомендации преподавателя и требования учебной программы.

При выполнении практических занятий основным методом обучения является самостоятельная работа студента под управлением преподавателя. На них пополняются теоретические знания студентов, их умение творчески мыслить, анализировать, обобщать изученный материал, проверяется отношение студентов к будущей профессиональной деятельности.

Оценка выполненной работы осуществляется преподавателем комплексно: по результатам выполнения заданий, устному сообщению и оформлению работы. После подведения итогов занятия студент обязан устранить недостатки, отмеченные преподавателем при оценке его работы.

\vspace{8pt}
{\bf Методические указания для самостоятельной работы обучающихся}

Прочное усвоение и долговременное закрепление учебного материала невозможно без продуманной самостоятельной работы. Такая работа требует от студента значительных усилий, творчества и высокой организованности. В ходе самостоятельной работы студенты выполняют следующие задачи: дорабатывают лекции, изучают рекомендованную литературу, готовятся к практическим занятиям, к коллоквиуму, контрольным работам по отдельным темам дисциплины. При этом эффективность учебной деятельности студента во многом зависит от того, как он распорядился выделенным для самостоятельной работы бюджетом времени.

Результатом самостоятельной работы является прочное усвоение материалов по предмету согласно программы дисциплины. В итоге этой работы формируются профессиональные умения и компетенции, развивается творческий подход к решению возникших в ходе учебной деятельности проблемных задач, появляется самостоятельности мышления.

\end{document}