\documentclass[11pt]{article}
\usepackage[utf8]{inputenc}
\usepackage[russian]{babel}
\usepackage{amsmath}
\usepackage{amssymb}
\usepackage{amsthm}
\usepackage{algorithm}
\usepackage{algorithmic}
\usepackage{fullpage}
\usepackage{cancel}

% Окружения теорем (названия на русском)
\newtheorem{theorem}{Теорема}
\newtheorem{lemma}{Лемма}
\newtheorem{proposition}{Утверждение}
\newtheorem{corollary}{Следствие}
\newtheorem{fact}{Факт}
\newtheorem{definition}{Определение}
\newtheorem{remark}{Замечание}

% Обозначения
\newcommand{\F}{\mathbb{F}}
\newcommand{\R}{\mathbb{R}}
\newcommand{\N}{\mathbb{N}}
\newcommand{\E}{\mathbb{E}}
\newcommand{\Z}{\mathbb{Z}}
\newcommand{\wt}{\mathrm{wt}}
\newcommand{\Spec}{\mathrm{Spec}}
\newcommand{\Inv}{\mathrm{Inv}}
\newcommand{\codim}{\mathrm{codim}}
\newcommand{\sgn}{\mathrm{sgn}}

\begin{document}

\section{Конечные поля}
\begin{equation}
    \Z /p \Z = \F_p = GF(p)
\end{equation}
\begin{equation}
f,g \in \F_p[x]
\end{equation}
\begin{equation}
    f \in \F_p[x] \text{ - неприводим}
\end{equation}
\begin{equation}
    g |f \Rightarrow g\in \F_p or f= \alpha g
\end{equation}
\begin{equation}11
gcd(f,g) = d = d(x)
\end{equation}


\begin{equation}
    I: \forall f \in \F_p[x] \; f=c \prod f_i^{m_i} 
\end{equation}

$f_i$ - неприводимый унитарный

% всё ок
\begin{equation}
    \F_p[x] - \text{кольцо главных идеалов} \; \forall \text{идеала } I \exists \text{неприводимый } f : I=(f)
\end{equation}

\begin{theorem}
$f \in \F_p[x]$ - неприводимый. $F_p[x] / f$ - поле из $p^n$ элементов, $n=deg(f)$. Тогда $\F_p[x] /f, \; \Z/(p) = \F_p$
\end{theorem}
\begin{proof}
$a(x) \in \F_p[x]$
$\exists ! r(x) \in \F_p[x] : a \equiv r(f) \leftrightarrow f | a-r$
$\exists ! $ доказательство
Пусть $\exists r_1, r_2$
$r_1 \equiv a(f), r_e \equiv a(f) \Rightarrow $ вы стёрли.........


$r(x) = a_1x^{n-1} + ... + a_0$
$a_1 \in \F_p$
есть $p^n$ штук вот таких многочленов
$| \F_p(x) / (f)| = p^n$
Обозначим $O=(f)$
\begin{equation}
    E = \{ f \in \F_p[x] : g \equiv 1(f)
\end{equation}
$A \neq O$
$a(x) \in A \Leftrightarrow f \; x \; a \Leftrightarrow (f,a) =(1) = \F_p[x] \Leftrightarrow \exists u,v \in \F_p[x]$
$fu+av=1 \Rightarrow av=1(f)$
то есть $V= [v(x)]$ класс в $\F_p[x]$

Докажем единсвтенность

$AW=E, W \neq V$
$\exists w(x), w \cancel{\equiv} v(а)$
$aw=1(f); av=1(f)$
$a(w-v) = 0(f)$
$fxa \Rightarrow f | w-v$
\end{proof}

Отсюда $q=p^n, \F_q=\F_p[x]/(f)$. $def(f) = r, f - $ неприводим

\begin{theorem}
    \begin{equation}
        \forall n \ge 1 \exists f \in \F_p[x]
    \end{equation}
    неприводимый. $deg(f) = 1$
\end{theorem}

    Введём $N(g) = Ng=p(deg(g))$ для $g \in \F_p[x]$
    Эта функция мультипликативна. $N(fg)=Nf \; Ng$
    Аналог дзета функции. 
    $\zeta (s) = \prod_p^*(1 - \frac{1}{(Nf)^s})^{-1}$
    $\prod^*$ - произвдение по всем неприводимым унитарным многочленам из кольца
    Область сходимости  $\zeta (s) = \sum_{n=1}^{\inf}\frac{1}{n^3} = \prod_{p - \text{простые}} (1 - \frac{1}{p^3})^{-1}$ при $Re(s) > 1$

    \begin{equation}
        \zeta = \prod_p^*(1 + \sum_{m=1}^{\inf} \frac{1}{(Nf)^{ms}})^{-1} = 
        1 + \sum_{g}^{*} \frac{1}{(Nf)^{s}}
    \end{equation}
    \begin{equation}
        \sum^* - \text{сумма по унитарным } g \in \F_p[x]
    \end{equation}
    \begin{equation}
        = 1 \sum_{n=1}^{\inf} \sum_{g, deg(g) = n} = \frac{1}{(Ng)^s}
    \end{equation}
    \begin{equation}
        = 1 + \sum_{n=1}^{\inf} p^n \frac{1}{p^{ns}}
    \end{equation}
    \begin{equation}
        = (1-\frac{p}{p^s})^{-1}
    \end{equation}
Количество неприводимых унитарных мночленов где $deg(g) = n$ - $\nu (n) $

\begin{equation}
    \prod_{n=1}^{\inf} (1 - \frac{1}{p^{ns}})^{- \nu(n)} = 
    \prod_f^* = (1 - \frac{1}{(Nf)^s})^{-1}
\end{equation}
прологорифмируем:
\begin{equation}
    \sum_{n=1}^{\inf} (-\nu(n) log(1-\frac{1}{p^{ns}} = -log(1 - \frac{p}{p^s})
    \end{equation}
    \begin{equation}
    log(1-\tau) = \sum_{m=1}^{\inf} \frac{1}{m} \tau^m
\end{equation}

У доказательства есть бонус.
\begin{equation}
    \sum_{n=1}^{\inf} \nu(n) \sum_{l=1}^{\inf} \frac{1}{l} \frac{1}{p^{lns}} 
    = \sum_{m=1}^{\inf} \frac{1}{m}\frac{p^m}{p^{ms}} 
\end{equation}

\begin{equation}
    \sum_{n=1}^{\inf} \nu(n) \sum_{l=1}^{\inf} \frac{1}{l} \frac{1}{p^{lns}} 
    = \sum_{n} \sum_l \frac{\nu(n)}{l}\frac{1}{p^{ms}} = 
    \sum_{m=1}^{\inf} \sum _{n |m} \frac{\nu(n)}{m / n}\frac{1}{p^{ms}} = 
\end{equation}

\begin{equation}
     = (\sum _{n |m} \nu(n) n)\frac{1}{m} \frac{1}{p^{ms}} = 
\end{equation}
\begin{equation}
    \sum _{n |m} \nu(n) n = p^m
\end{equation}
\begin{equation}
    \Rightarrow \nu(n) = \frac{1}{n} \sum _{d |m} \mu (d) p^{n/d} \cancel{=} 0 
\end{equation}

\begin{theorem}
    \begin{equation}
        \forall z \in \F_q^*, q - p^n, \F_q = \F_p[x]/(f)
    \end{equation}
    Тогда
    \begin{equation}
        z^{q-1} -1 = 0
    \end{equation}
\end{theorem}
\begin{proof}
    Пусть
    \begin{equation}
        g \in \F_p[x], Fxf
    \end{equation}
    \begin{equation}
        \prod_r gr = \prod_r r(f)
    \end{equation}
    \begin{equation}
        (g^{q-1} - 1) \prod r \equiv 0(f)
    \end{equation}
    Следовательно
    \begin{equation}
        \prod_{z \in \F_q} (x -z) = x^q - x
    \end{equation}
\end{proof}

\begin{lemma}
    \begin{equation}
        f | x^q - x, deg(f) = d \Rightarrow
    \end{equation}
    $f$ имеет $d$ различных корней
\end{lemma}
\begin{proof}
    \begin{equation}
        x^q-x = f(x) g(x)
    \end{equation}
    f(x) - d корней, g(x) -  q-d корней

    если $<d $ корней у f $\Rightarrow d+(q^1d) = 1$ у $x^q - x$ 
\end{proof}
\begin{theorem}
    мультипликативная группа $\F_q^*$ циличская то содержит $\phi(q-1)$
\end{theorem}
\begin{proof}
    \begin{equation}
        m | q-1; \psi(m) - \text{число элементов поля}
    \end{equation}
    Если $\psi(m) > 0$, $\alpha \in \F^*_q -$ порядок $m$.
    $1, \alpha, ..., \alpha^{m-1}$ - различны корни $x^m - 1$
    это все корни $x^m - 1$
    \begin{equation}
        \forall \beta \in \F_q^* - \text{порядок }  | m
    \end{equation}
    \begin{equation}
        \beta = \alpha^s, r\le s \le m-1
    \end{equation}
    если $(s,m) =d, \alpha^s$ - порядок $m/d$ \\
    Было упражнение что 
    \begin{equation}
        \alpha^s \text{ порядок } m \Leftrightarrow (s,m)=1
    \end{equation}
    Таким образом если $\psi(m) > 0$ то $\psi(m) = \phi(m)$
    \begin{equation}
        \sum_{m | q-1} \psi(m) = q-1
    \end{equation}
    Мы знаем что
    \begin{equation}
        \sum_{m | q-1} \phi(m) = q-1
    \end{equation}
    Тогда
    \begin{equation}
        \sum_{m | q-1} (\phi(m) - \psi(m)) = 0 \Rightarrow 
        \psi(q-1)=\phi(q-1) 
    \end{equation}
\end{proof}

\end{document}
