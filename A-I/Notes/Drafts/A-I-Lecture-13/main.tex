\documentclass[12pt]{article}

\usepackage[utf8]{inputenc}
\usepackage[russian]{babel}
\usepackage{amsmath, amssymb, amsthm}
\usepackage{geometry}
\geometry{margin=2.5cm}

\theoremstyle{definition}
\newtheorem{definition}{Определение}[section]

\theoremstyle{plain}
\newtheorem{lemma}[definition]{Лемма}
\newtheorem{theorem}[definition]{Теорема}
\newtheorem{corollary}[definition]{Следствие}

\theoremstyle{remark}
\newtheorem*{remark}{Замечание}

\title{Лекция №13. Лемма Гензеля и принцип Хассе.}
\date{}
\author{}

\begin{document}
\maketitle

\section{Нормированные и ультраметрические поля}

\begin{definition}
Пусть $K$ — поле. Отображение $\varphi : K \to \mathbb{R}_{\ge 0}$ называется
\emph{нормой}, если для всех $x,y \in K$ выполнено:
\begin{enumerate}
\item $\varphi(x)=0 \iff x=0$;
\item $\varphi(xy)=\varphi(x)\varphi(y)$;
\item $\varphi(x+y)\le \varphi(x)+\varphi(y)$.
\end{enumerate}
\end{definition}

\begin{definition}
Норма $\varphi$ называется \emph{ультраметрической}, если вместо (3) выполнено
\[
\varphi(x+y)\le \max\{\varphi(x),\varphi(y)\}.
\]
\end{definition}

\begin{lemma}
Ультраметрическая норма индуцирует метрику
\[
d(x,y)=\varphi(x-y),
\]
удовлетворяющую усиленному неравенству треугольника.
\end{lemma}

\begin{proof}
Так как $\varphi$ — норма, то $d(x,y)=0 \iff x=y$, симметрия очевидна.
Для треугольника:
\[
d(x,z)=\varphi(x-z)=\varphi((x-y)+(y-z))\le
\max\{\varphi(x-y),\varphi(y-z)\}.
\]
\end{proof}

\section{p-адическая норма и поле $\mathbb{Q}_p$}

\begin{definition}
Пусть $p$ — простое число. Для $x\in\mathbb{Q}^\times$ представимого в виде
\[
x=p^n\frac{a}{b}, \quad (a,b,p)=1,
\]
определим \emph{p-адическую валюацию}
\[
v_p(x)=n.
\]
Полагаем $v_p(0)=+\infty$.
\end{definition}

\begin{definition}
\emph{p-адическая норма} определяется формулой
\[
|x|_p=p^{-v_p(x)}, \quad |0|_p=0.
\]
\end{definition}

\begin{lemma}
Функция $|\cdot|_p$ является ультраметрической нормой на $\mathbb{Q}$.
\end{lemma}

\begin{proof}
Свойства (1) и (2) следуют из свойств валюации.
Для суммы используем:
\[
v_p(x+y)\ge \min\{v_p(x),v_p(y)\},
\]
откуда
\[
|x+y|_p \le \max\{|x|_p,|y|_p\}.
\]
\end{proof}

\begin{definition}
Поле $\mathbb{Q}_p$ — пополнение $\mathbb{Q}$ по метрике $|\cdot|_p$.
\end{definition}

\section{Кольцо p-адических целых}

\begin{definition}
\[
\mathbb{Z}_p=\{x\in\mathbb{Q}_p : |x|_p\le 1\}.
\]
\end{definition}

\begin{lemma}
$\mathbb{Z}_p$ является кольцом, а множество
\[
\mathfrak m=\{x\in\mathbb{Q}_p : |x|_p<1\}
\]
— его максимальный идеал.
\end{lemma}

\begin{proof}
Если $|x|_p,|y|_p\le1$, то
\[
|x+y|_p\le \max(|x|_p,|y|_p)\le1,
\quad |xy|_p\le1.
\]
Максимальность $\mathfrak m$ следует из того, что любой $x\notin\mathfrak m$
обратим в $\mathbb{Z}_p$.
\end{proof}

\begin{corollary}
Поле вычетов $\mathbb{Z}_p/\mathfrak m \simeq \mathbb{F}_p$.
\end{corollary}

\section{Полнота}

\begin{theorem}
Поле $\mathbb{Q}_p$ полно.
\end{theorem}

\begin{proof}
По построению $\mathbb{Q}_p$ как пополнения метрического пространства
$\mathbb{Q}$ относительно $|\cdot|_p$.
\end{proof}

\begin{corollary}
$\mathbb{Z}$ плотно в $\mathbb{Z}_p$.
\end{corollary}

\section{Формальные производные}

\begin{definition}
Пусть
\[
f(x)=a_0+a_1x+\dots+a_nx^n.
\]
Формальная производная:
\[
f'(x)=a_1+2a_2x+\dots+na_nx^{n-1}.
\]
\end{definition}

\begin{lemma}[Формула Тейлора]
Для $f\in\mathbb{Z}_p[x]$ и $h\in\mathbb{Z}_p$:
\[
f(x+h)=f(x)+f'(x)h+\frac{f''(x)}{2!}h^2+\dots
\]
\end{lemma}

\begin{proof}
Доказательство стандартно и следует из биномиальной формулы.
\end{proof}

\section{Лемма Гензеля}

\begin{theorem}[Лемма Гензеля]
Пусть $f\in\mathbb{Z}_p[x]$ и существует $a_0\in\mathbb{Z}_p$ такое, что
\[
f(a_0)\equiv0\pmod p, \quad f'(a_0)\not\equiv0\pmod p.
\]
Тогда существует единственный $\alpha\in\mathbb{Z}_p$, для которого
\[
f(\alpha)=0, \quad \alpha\equiv a_0\pmod p.
\]
\end{theorem}

\begin{proof}
Построим последовательность
\[
a_{n+1}=a_n-\frac{f(a_n)}{f'(a_n)}.
\]
Индукцией доказывается:
\[
f(a_n)\equiv0\pmod{p^{2^n}}, \quad |a_{n+1}-a_n|_p\le p^{-2^n}.
\]
Последовательность фундаментальна, а полнота $\mathbb{Z}_p$ даёт предел $\alpha$.
Переход к пределу в равенстве $f(a_n)=0$ даёт $f(\alpha)=0$.
Единственность следует из ультраметричности.
\end{proof}

\section{Следствия}

\begin{corollary}
Если уравнение $f(x)\equiv0\pmod p$ имеет простой корень, то оно имеет корень
в $\mathbb{Z}_p$.
\end{corollary}

\begin{corollary}
Пусть $(m,p)=1$. Тогда уравнение $x^m=a$ разрешимо в $\mathbb{Q}_p$
тогда и только тогда, когда оно разрешимо по модулю $p$.
\end{corollary}

\begin{proof}
Рассматриваем $f(x)=x^m-a$ и применяем лемму Гензеля.
\end{proof}

\section{Принцип Хассе}

\begin{theorem}[Принцип Хассе для квадратичных форм]
Пусть $F$ — квадратичная форма над $\mathbb{Q}$.
Тогда уравнение $F=0$ имеет нетривиальное решение в $\mathbb{Q}$
тогда и только тогда, когда оно имеет решение в $\mathbb{R}$
и во всех $\mathbb{Q}_p$.
\end{theorem}

\end{document}
