\documentclass[12pt]{article}

\usepackage[utf8]{inputenc}
\usepackage[T2A]{fontenc}
\usepackage[russian]{babel}
\usepackage{amsmath,amssymb,amsfonts}
\usepackage{xcolor}

% Окружение "доказательство" с треугольником и квадратом
\newenvironment{prooftri}
{\par\noindent\(\triangleright\)\;}
{\hfill\(\blacksquare\)\par}

\begin{document}

\begin{center}
{\Large\bfseries Дзета-функция Артина}
\end{center}

Пусть $F$ --- поле.

Определим
\[
F^n = \mathbb{A}^n(F)=\{(a_1,\ldots,a_n): a_i\in F\}
\]
(афинное пространство).

В $\mathbb{A}^{n+1}(F)$ вводится отношение эквивалентности
\[
(a_0,\ldots,a_n)\sim (b_0,\ldots,b_n)
\quad\Longleftrightarrow\quad
\exists\,\gamma\in F:\ a_i=\gamma b_i,\ i=0,\ldots,n.
\]

Определим
\[
\mathbb{P}^n(F)
=
\big(\mathbb{A}^{n+1}(F)\setminus\{0\}\big)/\sim
=
\{[a_0:\ldots:a_n]\}.
\]

Если $F=\mathbb{F}_q$, то
\[
|\mathbb{A}^n(\mathbb{F}_q)| = q^n.
\]

Можно построить отображение
\[
\varphi=\varphi_0:\ \mathbb{P}^n(F)\longrightarrow \mathbb{A}^n(F),
\qquad
[x_0:\ldots:x_n]\longmapsto
\left(\frac{x_1}{x_0},\ldots,\frac{x_n}{x_0}\right),
\]
при условии $x_0\neq 0$.

Лемма. Пусть
\[
H_0=\big\{[x_0:\ldots:x_n]\in\mathbb{P}^n(F):x_0\neq0\big\}.
\]
Тогда ограничение
\[
\varphi_0:H_0\longrightarrow\mathbb{A}^n(F)
\]
является биекцией.

\begin{prooftri}
Если $[x_0:\ldots:x_n]\in H_0$, то
\[
[x_0:\ldots:x_n]
=
\left[1:\frac{x_1}{x_0}:\ldots:\frac{x_n}{x_0}\right].
\]
Пусть $\varphi_0([x])=\varphi_0([y])$. Тогда
\[
\frac{x_i}{x_0} = \frac{y_i}{y_0}
\quad\Longrightarrow\quad
x_i y_0 = y_i x_0
\]
для всех $i$, т.е. $x_i = \gamma y_i$ при $\gamma = x_0^{-1}y_0$.
Следовательно, $[x]=[y]$. Обратное отображение есть
\[
(x_1,\ldots,x_n)\longmapsto[1:x_1:\ldots:x_n].
\]
\end{prooftri}

Таким образом, $H_0\simeq\mathbb{A}^n(F)$.

Имеем разложение
\[
\mathbb{P}^n(F)
=
\mathbb{A}^n(F)\ \cup\ \mathbb{P}^{n-1}(F).
\]

Лемма.
\[
|\mathbb{P}^n(\mathbb{F}_q)| = q^n + q^{n-1} + \ldots + q + 1.
\]

\begin{prooftri}
Индукция по $n$, используя разложение
$\mathbb{P}^n = \mathbb{A}^n\cup\mathbb{P}^{n-1}$ и то, что
$|\mathbb{A}^n(\mathbb{F}_q)|=q^n$.
\end{prooftri}

Примеры.

1) $\mathbb{P}^0$ --- одна точка.

2) $\mathbb{P}^1 = \mathbb{A}^1 \cup \{\infty\}$,
где $\infty$ --- бесконечно удалённая точка (проективная прямая).

Определение. Многочлен
\[
f\in F[x_1,\ldots,x_n]
\]
называется \emph{однородным} степени $d$, если
\[
f = \sum a_I x_0^{i_0}\cdots x_n^{i_n},
\qquad i_0+\cdots+i_n=d.
\]


Определение.
\[
f\in F[x_1,\ldots,x_n], deg f = d,
\]
если возьмем 
\[
\overline{f} = y^d_0f(\frac{y_1}{y_0},\ldots,\frac{y_n}{y_n}),\in F[y_1,\ldots,y_n] -
\]
соответствующий однородный многочлен. 
Пример. 
\[
x_1^2+x_2^2-1
\]
не является однородным многочленом,
а
\[
y_1^2+y_2^2-y_3^2
\]
--- однородный.

Далее $F=\mathbb{F}_q$.

\[
E\setminus F - \text{расширение;}
\]

\[
H_f(E) = \{a\in\mathbb{A}^n(E):f(a)=0\} - \text{гиперповерхность над конечным полем.}
\]

\[
\overline H_{\overline f(E)} = \{a\in\mathbb{P}^n(E):\overline f(a)=0\}
\]
\[
|H_f| \leq |\overline H_{\overline f}|
\]
Пример.
\[
f(x_1,x_2)=x_1^2+x_2^2-1 \in \mathbb{F}_p[x_1,x_2].
\]

\[
|H_f(\mathbb{F}_p)| =
\begin{cases}
p-1 & \text{, если } p\equiv1\pmod4, \\
p+1 & \text{, если } p\equiv3\pmod4.
\end{cases}
\]


Проективное множество:
\[
H_f(\mathbb{F}_p)
=
\{[y_0:y_1:y_2]\in\mathbb{P}^2(\mathbb{F}_p): y_1^2+y_2^2=y_0^2\}.
\]

Если $y_0 = 0$, то 
\[
[0:y_1:y_2] \Rightarrow
y_1^2 + y_2^2 = 0 \Leftrightarrow (\frac{y_1}{y_0}) = -1
\]
\[
p\equiv1\pmod4 \Rightarrow (\frac{-1}{p}) = -1, a^2 = -1
\]
\[
[0:1:a], [0:1;-a]\in \overline H_{\overline f}
\]
\[
|\overline H_{\overline f}| =p + 1, p\equiv1\pmod4, \text{иначе} p\equiv3\pmod4 - \text{нет решений} [0:y_1:y_2]
\]
\[
\text{Тогда } |\overline H_{\overline f}| = p + 1
\]
\vspace{1em}

Далее рассматривается общий случай.

Пусть $f$ --- однородный многочлен степени $n$ над $\mathbb{F}_p$.
Тогда проективная кривая $\overline{H}_f$ над $\overline{\mathbb{F}}_p$ имеет
число $\mathbb{F}_p$-точек, удовлетворяющее оценке
\[
\bigl||\overline{H}_f(\mathbb{F}_p)| - (p+1)\bigr|
\le (n-1)(n-2)\sqrt{p}.
\]

Теорема (на следующей странице).  
Пусть $F=\mathbb{F}_q$, $r|q-1$, тогда
\[
|\overline{H}_{\overline{f}}| = q^{n-1} + \dots + 1 + R,
\qquad |R| = \frac{1}{q}G(\chi_0)*\dots*G(\chi_n),
\]
\[
\chi_i - \text{характер порядка r}:\chi_i^r=\epsilon - \text{главный характер}
\]
\[
|R|=Cq^{\frac{n}{2}-1} \text{для некоторой константы $C$}
\]

Пусть далее
\[
f(y_0,\ldots,y_n)\in \mathbb{F}_q[y_0,\ldots,y_n].
\]

Для расширения $\mathbb{F}_{q^m}/\mathbb{F}_q$ обозначим
\[
N_m = |\overline{H}_f(\mathbb{F}_{q^m})|
=
N_m(f).
\]

Определение (дзета-функция Артина).

Для $f = 0$ задаётся
\[
Z_f(u) = \exp\left(\sum_{m=1}^\infty \frac{N_m}{m}u^m\right),
\qquad u\in\mathbb{C}.
\]

Лемма. Функция $Z_f(u)$ определена при $|u|<q^{-n}$.

\begin{prooftri}
$|N_m|\le|\mathbb{P}^n(\mathbb{F}_q^m)|
\le (n+1) q^{mn}$, то тогда ряд $Z_f(u)$ сходится.
\end{prooftri}

\vspace{1em}
Почему называется дзета-функцией:\\
рассмотрим дзето-функцию Римана
\[
\zeta(s) = \sum_{n=1}^\infty \frac1{n^s} = \prod_{p} \frac1{1-p^{-s}}.
\]

Далее рассматриваются многочлены над $\mathbb{F}_q[x]$.\\

Дзета-функция Артина определяется
\[
\zeta_{\mathbb{F}_q[x]}(s)
=
\prod_{f} \bigl(1 - (N_f)^{-s}\bigr)^{-1},
\]
где $f$ - неприводимый многочлен.
\vspace{1em}
Рассмотрим $H_f(\mathbb{\overline F}_{q^d})$, $\alpha = (a_1,\dots,a_n)\in\mathbb{F}_q$,\\
$a_i\in\mathbb{F}_{q^d},\quad\alpha^q=(a_1^q,\dots,a_n^q)$\\
$f(\alpha^q)=f(\alpha)^q=0$

Определение.
\[
\text{Множество}\quad P=\{ \alpha_1,\alpha^q,\ldots,\alpha^{q^{d-1}}\} - 
\]
называется циклом, deg P = d  называется степенью цикла ($\alpha_1,\alpha^q,\ldots,\alpha^{q^{d-1}},\alpha^{q^{d}}$, $\alpha^{q^{d}}= \alpha$)

\vspace{1em}
Лемма.
Если обозначить через $n_d = |\{P - \text{циклов}:deg P=d\}|$, то
\[
N_m = \sum_{d\mid m} d\, n_d.
\]

\begin{prooftri}
$H_f(\mathbb{F}_{q^m})=\bigcup P, \alpha\in\mathbb{F}_{q^d}$.
$\mathbb{F}_{q^d}$ подполе $\mathbb{F}_{q^m} \Leftrightarrow d|m$.\\
$H_f(\mathbb{F}_{q^m})=\bigcup_{d|m} P,\quad P_1\bigcap P_2\neq\emptyset  \quad P_1\neq P_2. \alpha\in P_1\bigcap P_2 \Leftrightarrow \\ \alpha^{q^{d_1}-1}=1, \quad \alpha^{q^{d_2}-1}=1 \Rightarrow d_1=d_2 \Rightarrow p_1=p_2 \quad\bot$
\end{prooftri}

Теорема.

\[
Z_p(q^{-s})=\prod_{P}(1 - q^{-s\deg P}\bigr)^{-1}.
\]

Теорема.
\[
Z_f(u)\in\mathbb{C}(u),\qquad
Z_f(u)=\frac{P(u)}{Q(u)}.
\]

Теорема.
\[
Z_f(u) - \text{рац}\Leftrightarrow N_m = \sum_j{p_j^m}-\sum_i{d_i^m}.
\]


\end{document}
