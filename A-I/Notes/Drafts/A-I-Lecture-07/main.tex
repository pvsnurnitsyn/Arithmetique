\documentclass[12pt]{article}

% -------------------- Packages --------------------
\usepackage[T2A]{fontenc}
\usepackage[utf8]{inputenc}
\usepackage[russian]{babel}
\usepackage{amsmath, amssymb, amsthm}
\usepackage{mathtools}
\usepackage{geometry}
\usepackage{enumitem}
\usepackage{hyperref}
\usepackage{microtype}

\geometry{a4paper, margin=2.5cm}

% -------------------- Theorem environments --------------------
\theoremstyle{definition}
\newtheorem{definition}{Определение}[section]
\newtheorem{example}[definition]{Пример}

\theoremstyle{plain}
\newtheorem{theorem}[definition]{Теорема}
\newtheorem{lemma}[definition]{Лемма}
\newtheorem{proposition}[definition]{Утверждение}
\newtheorem{corollary}[definition]{Следствие}

\theoremstyle{remark}
\newtheorem{remark}[definition]{Замечание}

% -------------------- Commands --------------------
\newcommand{\Z}{\mathbb{Z}}
\newcommand{\Q}{\mathbb{Q}}
\newcommand{\R}{\mathbb{R}}
\newcommand{\CC}{\mathbb{C}}
\newcommand{\F}{\mathbb{F}}
\newcommand{\ord}{\operatorname{ord}}
\newcommand{\Gal}{\operatorname{Gal}}

\title{Корни из единицы. Круговой многочлен.}
\author{Конспект лекции}
\date{}

\begin{document}
\maketitle

\tableofcontents

\section{Критерий Эйзенштейна и лемма Гаусса}

\subsection{Критерий Эйзенштейна}

\begin{theorem}[Критерий Эйзенштейна]
Пусть $f \in \Z[x]$:
\[
f(x) = a_n x^n + a_{n-1} x^{n-1} + \dots + a_1 x + a_0,
\]
где $a_i \in \Z$.
Пусть $p$ --- простое число такое, что $p \mid a_i$ для всех $i = 0,\dots,n-1$, $p \nmid a_n$, $p^2 \nmid a_0$. Тогда многочлен $f(x)$ неприводим над $\Z$ (и, следовательно, над $\Q$).
\end{theorem}

\begin{proof}
Предположим противное: $f = gh$, где $g,h \in \Z[x]$ --- нетривиальные многочлены.
Рассмотрим редукцию по модулю $p$. В кольце $\F_p[x]$ имеем
\[
\overline f(x) = \overline{a_n} x^n.
\]
Так как $\F_p[x]$ --- область уникального разложения, то

\[
\overline f(x) = \overline g(x) \overline h(x),
\]
где
\[
\overline g(x) = b_m x^m, \qquad \overline h(x) = c_l x^\ell,
\]
где $m+\ell = n$ и $b_m,c_l \in \F_p^*$. Также сами многочлены $g, h \in \Z[x]$ имеют вид:
\[
 g = \sum_{i=0}^m b_ix_i, \:\: h=\sum_{i=0}^l c_ix_i,
\]
а $\overline g, \overline h \in \F_p[x]$:
\[
\overline g =  b_mx_m, \:\:\overline h=c_lx_l.
\]
Это возможно лишь в том случае, если все коэффициенты $g,h$, кроме старших, делятся на $p$. В частности,
$p \mid b_0$ и $p \mid c_0$, где $b_0,c_0$ --- свободные члены $g,h$.
Но тогда
\[
a_0 = b_0 c_0 \quad \Rightarrow \quad p^2 \mid a_0,
\]
что противоречит условию.
\end{proof}

\subsection{Лемма Гаусса}

\begin{definition}
Многочлен $f \in \Z[x]$ называется \emph{примитивным}, если наибольший общий делитель всех его коэффициентов равен $1$.
\end{definition}

\begin{lemma}[Лемма Гаусса]
Многочлен $f \in \Z[x]$ неприводим над $\Z$ тогда и только тогда, когда он неприводим над $\Q$.
\end{lemma}

\begin{proof}
Одно направление очевидно. Докажем обратное.
Пусть $f$ приводим над $\Q$, то есть
\[
f = gh, \qquad g,h \in \Q[x],
\]
где $\deg g, \deg h > 0$.
Существуют такие $a,b \in \Q^*$, что многочлены
\[
g_1 = a g, \qquad h_1 = b h
\]
имеют целые коэффициенты и являются примитивными. Тогда
\[
f_1 = ab f(x) = ag(x)\cdot bh(x) = g_1(x) h_1(x).
\]
Пусть простое $p \mid ab$. Тогда $p$ делит все коэффициенты $f_1$, и редукция $0=\overline f_1 = \overline{g}\overline{h_1}(mod \:p)$.
Следовательно,
\[
\overline g_1 \cdot \overline h_1 = 0 \quad \text{в } \F_p[x].
\]
Так как в $\F_p[x]$ здесь нет делителей нуля, то $\overline g_1 = 0$ или $\overline h_1 = 0$, что означает, что $p$ делит все коэффициенты $g_1$ или $h_1$. Это противоречит примитивности этих многочленов.
Следовательно, $ab = \pm 1$, и $f$ приводим над $\Z$.
\end{proof}

\section{Применение критерия Эйзенштейна и формальная производная}

\begin{proposition}
Пусть $p$ --- простое число. Тогда многочлен
\[
f(x) = x^{p-1} + x^{p-2} + \dots + x + 1
\]
неприводим над $\Q$.
\end{proposition}

\begin{proof}
Рассмотрим $f(x+1)$. Имеем
\[
f(x) = \frac{x^p - 1}{x-1},
\]
откуда
\[
f(x+1) = \frac{(x+1)^p - 1}{x}.
\]
По биному Ньютона
\[
(x+1)^p = x^p + \binom{p}{1}x^{p-1} + \dots + \binom{p}{p-1}x + 1, тогда 
\]
\[
f(x+1) = x^{p-1} + \binom{p}{1}x^{p-2} + \dots + \binom{p}{p-1}.
\]

Все биномиальные коэффициенты, кроме коэффициента при старшей степени, делятся на $p$, а свободный член после деления на $x$ не делится на $p^2$.
По критерию Эйзенштейна $f(x+1)$ неприводим, следовательно, неприводим и $f(x)$.
\end{proof}

\subsection{Формальная производная}

\begin{definition}
Пусть $K$ --- поле и
\[
f(x) = a_n x^n + \dots + a_1 x + a_0 \in K[x].
\]
\emph{Формальной производной} многочлена $f$ называется многочлен:
\[
f'(x) = n a_n x^{n-1} + \dots + 2 a_2 x + a_1.
\]
\end{definition}

\begin{proposition}
Для любых $f,g \in K[x]$ выполняются свойства:
\begin{enumerate}[label=(\roman*)]
\item $(f+g)' = f' + g'$,\
\item $(fg)' = f'g + fg'.$
\end{enumerate}
\end{proposition}

\begin{lemma}
Пусть $\alpha$ --- корень $f \in K[x]$. Тогда $\alpha$ является кратным корнем тогда и только тогда, когда $f'(\alpha)=0$.
\end{lemma}

\section{Корни из единицы. Круговые поля.}

\subsection{Корни из единицы}

Рассмотрим многочлен $x^n - 1$ над $\CC$, где
\[
x^n - 1 = \prod_{k=1}^n(x - e^{2\pi i k / n}), \qquad k=0,1,\dots,n-1.
\]

Его корни имеют вид
\[
\zeta_n^k = e^{2\pi i k / n}.
\]

\begin{definition}
Множество
\[
U_n = \{ \zeta_n^k \mid 0 \le k < n \}
\]
называется \emph{группой корней степени $n$ из единицы} ($\sqrt[n]{1}$).
Элемент $\zeta \in U_n$ называется \emph{примитивным корнем $\sqrt[n]{1}$}, если $\ord(\zeta)=n$.
\end{definition}

\subsection{Круговые поля}

\begin{definition}
Пусть $K$ --- поле, то \emph{$n$-тым круговым полем над $K$} называется поле разложения многочлена $x^n-1$ над $K$.
Обозначается $K^{(n)}$.
\end{definition}

\section{Структура группы корней из единицы}

\begin{theorem}
Пусть $K$ --- поле и $char \: K = p \ge 0$.
\begin{enumerate}[label=(\roman*)]
\item Если $p \nmid n$, то группа $U_n$ циклическая порядка $n$.
\item Если $p \mid n$ и $n = p^e m$, где $p \nmid m$, то $U_n = U_m$.
\end{enumerate}
\end{theorem}

\begin{proof}
Если $p \nmid n$, то $(x^n-1)' = n x^{n-1}$ не обращается в ноль на корнях, следовательно, все корни простые и $|U_n|=n$.
Далее доказывается цикличность стандартным групповым аргументом, используя разложение $n$ на простые множители и построение элемента порядка $n$.
В случае $p \mid n$ имеем
\[
x^n - 1 = (x^m - 1)^{p^e},
\]
откуда утверждение.
\end{proof}

% ==================== ii ====================

\section{Круговые многочлены}

Пусть $n \ge 1$, характеристика поля $K$ не делит $n$,
и $\zeta$ — примитивный корень степени $n$ из единицы.

\begin{definition}
\emph{$n$-тым круговым многочленом} называется многочлен
\[
\Phi_n(x)
=
\prod_{\substack{1 \le k \le n \\ (k,n)=1}} (x - \zeta^k).
\]
\end{definition}

\begin{proposition}
Выполняются следующие свойства:
\begin{enumerate}[label=(\roman*)]
  \item $\deg \Phi_n = \varphi(n)$;
  \item
  \[
  x^n - 1 = \prod_{d \mid n} \Phi_d(x);
  \]
  \item $\Phi_n(x)$ имеет коэффициенты в простом подполе $K$
  (в частности, $\Phi_n(x) \in \mathbb{Z}[x]$ при $K=\mathbb{Q}$).
\end{enumerate}
\end{proposition}

\begin{proof}
Корни многочлена $x^n-1$ — все корни степени $n$ из единицы.
Каждый такой корень имеет некоторый порядок $d \mid n$,
и является примитивным корнем степени $d$.
Группируя корни по порядкам, получаем разложение.
Степень $\Phi_n$ равна числу примитивных корней, то есть $\varphi(n)$.
\end{proof}

\begin{lemma}
Элемент $\zeta$ является примитивным корнем степени $n$ из единицы
тогда и только тогда, когда $\Phi_n(\zeta)=0$.
\end{lemma}

\begin{theorem}
Круговой многочлен $\Phi_n(x)$ неприводим над $\mathbb{Q}$.
\end{theorem}

\section{Круговые многочлены над конечными полями}

Пусть $p$ — простое число, $p \nmid n$.

\subsection{Круговое поле над $\mathbb{F}_p$}

Пусть $\zeta$ — примитивный корень степени $n$ из единицы
в некотором расширении поля $\mathbb{F}_p$.

\begin{proposition}
Пусть $d$ — наименьшее натуральное число такое, что
\[
p^d \equiv 1 \pmod n.
\]
Тогда $\zeta \in \mathbb{F}_{p^d}$, и поле разложения многочлена
$x^n-1$ над $\mathbb{F}_p$ изоморфно $\mathbb{F}_{p^d}$.
\end{proposition}

\begin{proof}
Если $\zeta \in \mathbb{F}_{p^k}$, то порядок элемента $\zeta$
делит $p^k-1$, следовательно $n \mid (p^k-1)$.
Минимальность $d$ означает, что $d$ есть порядок $p$
в группе $(\mathbb{Z}/n\mathbb{Z})^\times$.
Поэтому $\zeta \in \mathbb{F}_{p^d}$, и меньшего поля не существует.
\end{proof}

\subsection{Разложение $\Phi_n(x)$ над $\mathbb{F}_p$}

\begin{theorem}
Пусть $d$ — порядок $p$ по модулю $n$.
Тогда круговой многочлен $\Phi_n(x)$ над $\mathbb{F}_p$
раскладывается в произведение
\[
\Phi_n(x) = f_1(x) f_2(x) \cdots f_r(x),
\]
где:
\begin{enumerate}[label=(\roman*)]
  \item все $f_i(x)$ неприводимы над $\mathbb{F}_p$;
  \item $\deg f_i = d$ для всех $i$;
  \item $r = \varphi(n)/d$.
\end{enumerate}
\end{theorem}

\begin{proof}
Минимальный многочлен элемента $\zeta$ над $\mathbb{F}_p$
имеет степень $d$, так как
\[
\mathbb{F}_p(\zeta) = \mathbb{F}_{p^d}.
\]
Его корни имеют вид
\[
\zeta, \zeta^p, \zeta^{p^2}, \dots, \zeta^{p^{d-1}},
\]
и являются примитивными корнями степени $n$.
Следовательно, минимальный многочлен делит $\Phi_n(x)$.
Все примитивные корни разбиваются на такие орбиты,
откуда и следует утверждение.
\end{proof}

\begin{remark}
Разложение $\Phi_n(x)$ описывается действием автоморфизма Фробениуса
\[
\sigma(\alpha) = \alpha^p
\]
на множестве примитивных корней степени $n$ из единицы.
\end{remark}


\section*{Дополнительные замечания}

\begin{remark}[Интерпретация результатов]
Содержание лекции можно суммировать следующим образом:
структура разложения кругового многочлена $\Phi_n(x)$
над конечным полем $\mathbb{F}_p$ полностью определяется
порядком $p$ по модулю $n$.
\end{remark}

\begin{remark}[Связь с конечными полями]
Каждое конечное поле характеристики $p$
изоморфно $\mathbb{F}_{p^d}$,
а его мультипликативная группа циклическая.
Корни из единицы и круговые многочлены
дают удобный язык для описания таких расширений.
\end{remark}

\begin{remark}[Действие Фробениуса]
Автоморфизм Фробениуса
\[
\alpha \mapsto \alpha^p
\]
описывает сопряжения корней и объясняет,
почему все неприводимые множители $\Phi_n(x)$
над $\mathbb{F}_p$ имеют одинаковую степень.
\end{remark}

\end{document}