\documentclass[12pt]{article}
\usepackage[T2A]{fontenc}
\usepackage[utf8]{inputenc}
\usepackage[russian]{babel}
\usepackage{amsmath,amssymb,mathtools,amsthm}
\usepackage{geometry}
\geometry{a4paper, margin=2.3cm}

\newcommand{\Z}{\mathbb{Z}}
\newcommand{\Q}{\mathbb{Q}}
\newcommand{\F}{\mathbb{F}}
\newcommand{\Gal}{\mathrm{Gal}}
\newcommand{\Tr}{\mathrm{Tr}}
\newcommand{\Nm}{\mathrm{N}}
\newcommand{\OO}{\mathcal{O}}
\newcommand{\disc}{\mathrm{disc}}
\newcommand{\ord}{\mathrm{ord}}
\theoremstyle{plain}
\newtheorem{theorem}{Теорема}[section]
\newtheorem{lemma}[theorem]{Лемма}
\newtheorem{proposition}[theorem]{Утверждение}
\newtheorem{corollary}[theorem]{Следствие}

\theoremstyle{definition}
\newtheorem{definition}[theorem]{Определение}
\newtheorem{example}[theorem]{Пример}

\theoremstyle{remark}
\newtheorem{remark}[theorem]{Замечание}

\begin{document}

\begin{center}
{\Large \textbf{5.3. Квадратичные поля и их кольца целых}}\\[2mm]
\end{center}

\section*{1. Квадратичные расширения над $\Q$}

\begin{definition}
Пусть $F/\Q$ --- конечное расширение полей. Его \emph{степенью} называется число
\[
[F:\Q]=\dim_{\Q} F.
\]
\end{definition}

\begin{definition}
Поле $F$ называется \emph{квадратичным} над $\Q$, если $[F:\Q]=2$.
\end{definition}

\begin{proposition}
Если $F/\Q$ квадратично, то существует элемент $\alpha\in F\setminus \Q$ такой, что
\[
F=\Q(\alpha),
\]
и минимальный многочлен $\alpha$ над $\Q$ имеет степень $2$.
\end{proposition}

\begin{proposition}
Пусть $d\in\Z$ не является квадратом в $\Z$. Тогда
\[
F=\Q(\sqrt d)
\]
является квадратичным полем и
\[
[F:\Q]=2,\qquad \{1,\sqrt d\}\ \text{--- базис $F$ над $\Q$}.
\]
\end{proposition}

\begin{remark}
Обычно выбирают $d$ \emph{квадратсвободным} (squarefree), чтобы представление $\Q(\sqrt d)$ было ``нормальным''.
\end{remark}

\section*{2. Группа Галуа квадратичного поля, след и норма}

Рассмотрим квадратичное поле $F=\Q(\sqrt d)$, где $d\in\Z$ не квадрат.

\begin{proposition}
\[
\Gal(F/\Q)=\{ \mathrm{id},\sigma\},\qquad \sigma(\sqrt d)=-\sqrt d.
\]
\end{proposition}

\begin{proof}
Пусть $\tau\in\Gal(F/\Q)$. Тогда $\tau$ фиксирует $\Q$ и переводит корни минимального многочлена
$x^2-d$ в корни того же многочлена. Следовательно,
\[
\tau(\sqrt d)\in\{\sqrt d,-\sqrt d\}.
\]
Это даёт ровно два автоморфизма: тождественный и сопряжение $\sigma$.
\end{proof}

\begin{definition}
Для $\alpha\in F$ определим \emph{сопряжение} $\alpha'=\sigma(\alpha)$, а также \emph{след} и \emph{норму}:
\[
\Tr_{F/\Q}(\alpha)=\alpha+\alpha',\qquad \Nm_{F/\Q}(\alpha)=\alpha\alpha'.
\]
\end{definition}

Если $\alpha=r+s\sqrt d$ (где $r,s\in\Q$), то
\[
\alpha'=r-s\sqrt d,\qquad \Tr(\alpha)=2r,\qquad \Nm(\alpha)=r^2-ds^2.
\]

\section*{3. Кольцо целых элементов квадратичного поля}

\begin{definition}
Элемент $\alpha\in F$ называется \emph{целым (алгебраически целым)} над $\Z$, если он является корнем некоторого
унитарного многочлена $f(x)\in\Z[x]$.
Множество всех целых элементов поля $F$ обозначается $\OO_F$ и называется \emph{кольцом целых} поля $F$.
\end{definition}

\begin{lemma}\label{lem:int-tr-nm}
Пусть $F=\Q(\sqrt d)$ и $\alpha\in F$. Тогда $\alpha\in\OO_F$ тогда и только тогда, когда
\[
\Tr(\alpha)\in\Z \quad\text{и}\quad \Nm(\alpha)\in\Z.
\]
\end{lemma}

\begin{proof}
($\Rightarrow$) Если $\alpha$ целый, то он удовлетворяет унитарному многочлену степени $2$:
\[
x^2 - \Tr(\alpha)x + \Nm(\alpha)=0,
\]
причём коэффициенты этого многочлена лежат в $\Z$. Значит $\Tr(\alpha),\Nm(\alpha)\in\Z$.

($\Leftarrow$) Пусть $\Tr(\alpha),\Nm(\alpha)\in\Z$. Тогда $\alpha$ является корнем многочлена
\[
x^2 - \Tr(\alpha)x + \Nm(\alpha)\in\Z[x],
\]
который унитарен, значит $\alpha$ целый.
\end{proof}

\begin{theorem}[описание $\OO_F$]\label{thm:ring-of-integers}
Пусть $F=\Q(\sqrt d)$, где $d$ квадратсвободно. Тогда
\[
\OO_F=
\begin{cases}
\Z[\sqrt d], & d\equiv 2,3 \pmod 4,\\[1mm]
\Z\!\left[\dfrac{1+\sqrt d}{2}\right], & d\equiv 1 \pmod 4.
\end{cases}
\]
\end{theorem}

\begin{proof}
Пусть $\alpha=r+s\sqrt d\in F$ с $r,s\in\Q$. По лемме \ref{lem:int-tr-nm} нужно и достаточно:
\[
\Tr(\alpha)=2r\in\Z,\qquad \Nm(\alpha)=r^2-ds^2\in\Z.
\]
Из $2r\in\Z$ получаем $r=\dfrac{m}{2}$ для некоторого $m\in\Z$.

Пусть также $s=\dfrac{n}{2}$ для некоторого $n\in\Z$ (это будет следовать из условия на норму).
Тогда
\[
\Nm(\alpha)=\left(\frac{m}{2}\right)^2 - d\left(\frac{n}{2}\right)^2
=\frac{m^2-dn^2}{4}\in\Z,
\]
то есть
\[
m^2-dn^2\equiv 0\pmod 4.
\]
Рассмотрим случаи по $d\bmod 4$.

\medskip
\noindent\textbf{Случай 1: $d\equiv 2,3\pmod 4$.}
Если $n$ нечётно, то $n^2\equiv 1\pmod 4$, и тогда
\[
m^2-dn^2\equiv m^2-d\pmod 4.
\]
Но при $d\equiv 2,3\pmod 4$ число $d$ не сравнимо с квадратом по модулю $4$ так, чтобы получилось $0$ для любого $m$,
откуда следует, что $n$ должен быть чётным, то есть $s\in\Z$.
Тогда и $r\in\Z$ (иначе нарушится целочисленность нормы).
Значит $\alpha\in\Z[\sqrt d]$, и получаем $\OO_F=\Z[\sqrt d]$.

\medskip
\noindent\textbf{Случай 2: $d\equiv 1\pmod 4$.}
Тогда условие $m^2-dn^2\equiv 0\pmod 4$ эквивалентно
\[
m^2-n^2\equiv 0\pmod 4 \quad\Longleftrightarrow\quad m\equiv n\pmod 2.
\]
Следовательно,
\[
\alpha=\frac{m+n\sqrt d}{2},\qquad m\equiv n\pmod 2.
\]
Положим $\omega=\dfrac{1+\sqrt d}{2}$. Тогда любой такой $\alpha$ можно записать как
\[
\alpha=a+b\omega,\qquad a,b\in\Z,
\]
и наоборот, каждый $a+b\omega$ целый. Значит $\OO_F=\Z[\omega]$.
\end{proof}

\section*{4. Дискриминант квадратичного поля}

\begin{definition}
Пусть $F/\Q$ --- расширение степени $n$, и $\alpha_1,\dots,\alpha_n\in\OO_F$.
Определим \emph{дискриминант} набора $(\alpha_1,\dots,\alpha_n)$:
\[
\Delta(\alpha_1,\dots,\alpha_n)=\det\bigl(\Tr(\alpha_i\alpha_j)\bigr)_{i,j=1}^n.
\]
Дискриминант поля (точнее, дискриминант кольца целых) обозначают $\disc(F)$.
\end{definition}

\begin{proposition}
Пусть $F=\Q(\sqrt d)$, $d$ квадратсвободно. Тогда
\[
\disc(F)=
\begin{cases}
4d, & d\equiv 2,3\pmod 4,\\
d, & d\equiv 1\pmod 4.
\end{cases}
\]
\end{proposition}

\begin{proof}
Если $d\equiv 2,3\pmod 4$, то базис $\OO_F$ равен $(1,\sqrt d)$.
Считаем матрицу следов:
\[
\Tr(1\cdot 1)=2,\quad \Tr(1\cdot \sqrt d)=\Tr(\sqrt d)=0,\quad \Tr(\sqrt d\cdot \sqrt d)=\Tr(d)=2d.
\]
Значит
\[
\Delta(1,\sqrt d)=\det\begin{pmatrix}2&0\\0&2d\end{pmatrix}=4d.
\]

Если $d\equiv 1\pmod 4$, то базис $\OO_F$ равен $(1,\omega)$, где $\omega=\dfrac{1+\sqrt d}{2}$.
Тогда $\omega'=\dfrac{1-\sqrt d}{2}$, и
\[
\Tr(1)=2,\qquad \Tr(\omega)=\omega+\omega'=1,\qquad \Tr(\omega^2)=\omega^2+(\omega')^2.
\]
Заметим:
\[
\omega^2=\frac{(1+\sqrt d)^2}{4}=\frac{1+2\sqrt d+d}{4},\qquad
(\omega')^2=\frac{1-2\sqrt d+d}{4},
\]
поэтому
\[
\Tr(\omega^2)=\frac{1+d}{2}.
\]
Следовательно,
\[
\Delta(1,\omega)=\det\begin{pmatrix}2&1\\1&\frac{1+d}{2}\end{pmatrix}
=2\cdot\frac{1+d}{2}-1=d.
\]
\end{proof}

\section*{5. Разложение простого идеала $(p)$ в $\OO_F$}

Пусть $F=\Q(\sqrt d)$, $d$ квадратсвободно, $\OO_F$ --- кольцо целых.

\begin{theorem}[квадратичный закон разложения простого]\label{thm:splitting}
Пусть $p$ --- нечётное простое, $p\nmid \disc(F)$.
Тогда в $\OO_F$ возможны ровно три случая:
\begin{enumerate}
\item \textbf{Расщепление:} если $\left(\frac{d}{p}\right)=1$, то
\[
(p)=\mathfrak{p}\,\mathfrak{p}',\qquad \mathfrak{p}\neq \mathfrak{p}',\qquad
\Nm(\mathfrak{p})=\Nm(\mathfrak{p}')=p.
\]
\item \textbf{Инертность:} если $\left(\frac{d}{p}\right)=-1$, то $(p)$ прост в $\OO_F$ и
\[
(p)\ \text{остаётся простым},\qquad \Nm((p))=p^2.
\]
\item \textbf{Разветвление:} если $p\mid \disc(F)$, то
\[
(p)=\mathfrak{p}^2,\qquad \Nm(\mathfrak{p})=p.
\]
\end{enumerate}
\end{theorem}

\begin{remark}
Случай $p\mid \disc(F)$ --- это именно \emph{разветвление}. Для нечётного $p$ это эквивалентно $p\mid d$.
\end{remark}

\section*{6. Частный разбор: конструкция идеалов при $\left(\frac{d}{p}\right)=1$}

Пусть $p$ --- нечётное простое, $p\nmid d$ и $\left(\frac{d}{p}\right)=1$.
Тогда существует $a\in\Z$ такое, что
\[
a^2\equiv d\pmod p.
\]
Рассмотрим идеалы
\[
\mathfrak{p}=(p,\,a+\sqrt d),\qquad \mathfrak{p}'=(p,\,a-\sqrt d).
\]

\begin{proposition}
Имеем включение
\[
\mathfrak{p}\mathfrak{p}'\subseteq (p).
\]
\end{proposition}

\begin{proof}
Действительно, в произведении $\mathfrak{p}\mathfrak{p}'$ лежат элементы вида
\[
p\cdot p,\quad p(a-\sqrt d),\quad p(a+\sqrt d),\quad (a+\sqrt d)(a-\sqrt d)=a^2-d.
\]
Первые три очевидно кратны $p$, а для последнего имеем $a^2-d\equiv 0\pmod p$, значит $a^2-d\in (p)$.
\end{proof}

\begin{proposition}
Идеалы $\mathfrak{p}\neq\mathfrak{p}'$ и
\[
(p)=\mathfrak{p}\mathfrak{p}'.
\]
\end{proposition}

\begin{remark}
На картинке у преподавателя это записано как ``строим два разных простых идеала над $p$''.
\end{remark}

\section*{7. Замечания про $p=2$}

Для $p=2$ разложение зависит от $d\bmod 8$ и вида $\OO_F$.
Обычно отдельно разбирают случаи $d\equiv 1\pmod 8$, $d\equiv 5\pmod 8$ и т.д.,
потому что дискриминант может содержать степень двойки.

\bigskip
\noindent\textbf{Итого по страницам:}
\begin{itemize}
\item определили квадратичное поле $F=\Q(\sqrt d)$;
\item описали $\Gal(F/\Q)$ и сопряжение;
\item ввели $\Tr$ и $\Nm$;
\item через $\Tr,\Nm$ описали $\OO_F$;
\item вычислили $\disc(F)$;
\item сформулировали правило разложения $(p)$ по символу Лежандра $\left(\frac{d}{p}\right)$;
\item показали конструкцию идеалов $\mathfrak{p}=(p,a+\sqrt d)$ при $\left(\frac{d}{p}\right)=1$.
\end{itemize}


\subsection*{Разложение простого идеала $(p)$ в квадратичном поле}

Пусть $F=\Q(\sqrt d)$, $d$ --- квадратсвободное целое, $D=\mathcal O_F$.
Для $\alpha=r+s\sqrt d\in F$ (где $r,s\in\Q$) обозначим сопряжение
\[
\alpha' = r - s\sqrt d .
\]
Для идеала $P\subset D$ положим
\[
P'=\{\gamma'\mid \gamma\in P\}.
\]
Тогда $P'$ тоже идеал в $D$ (сопряжённый к $P$).

\medskip

\noindent\textbf{Напоминание про типы разложения.}
Пусть
\[
(p)=P_1^{e_1}\cdots P_g^{e_g},\qquad N(P_i)=p^{f_i}.
\]
Тогда выполняется равенство $\sum_{i=1}^g e_i f_i = [F:\Q]=2$.
Отсюда возможны ровно три случая:
\[
\begin{array}{ll}
\text{(a) разветвление:} & (p)=P^2,\quad e=2,\ f=1,\ g=1;\\
\text{(b) разложение:} & (p)=PP',\quad e=1,\ f=1,\ g=2,\ P\neq P';\\
\text{(c) инертность:} & (p)=P,\quad e=1,\ f=2,\ g=1.
\end{array}
\]

\medskip

\begin{theorem}[Критерий разложения через дискриминант]
Пусть $p$ --- нечётное простое, $p\neq 2$. Тогда:
\[
\left(\frac{\delta_F}{p}\right)=
\begin{cases}
0 &\Longleftrightarrow\ p\mid \delta_F\ \Longrightarrow\ (p)=P^2,\\[2mm]
1 &\Longleftrightarrow\ p\nmid \delta_F\ \Longrightarrow\ (p)=PP',\ P\neq P',\\[2mm]
-1 &\Longleftrightarrow\ p\nmid \delta_F\ \Longrightarrow\ (p)\ \text{прост в }D.
\end{cases}
\]
\end{theorem}

\subsubsection*{Случай 1: $\left(\frac{\delta_F}{p}\right)=0$ (разветвление)}

Предположим $p\mid \delta_F$. Для квадратичного поля $F=\Q(\sqrt d)$ имеем
$\delta_F\in\{d,\,4d\}$, поэтому из $p\mid\delta_F$ следует $p\mid d$.

Положим
\[
P=(p,\sqrt d)=\{\alpha p+\beta \sqrt d\mid \alpha,\beta\in D\}.
\]
Тогда
\[
P^2=(p,\sqrt d)^2=(p^2,\;p\sqrt d,\;d).
\]
Так как $p\mid d$, то $d=p\cdot \frac{d}{p}$ и
\[
(p^2,\;p\sqrt d,\;d)\subset (p)\cdot (p,\sqrt d, d/p).
\]
Обозначим
\[
I=(p,\sqrt d, d/p).
\]
Поскольку $p\mid d$, число $d/p\in\Z\subset D$.
Кроме того,
\[
(p,\ d/p)=1 \quad\Rightarrow\quad \exists\,r,s\in\Z:\ rp+s\frac{d}{p}=1,
\]
значит $1\in I$, то есть $I=D$. Следовательно,
\[
P^2\subset (p)\cdot I=(p).
\]
С другой стороны, очевидно $(p)\subset P$, значит $(p)^2\subset P^2$,
а так как индекс $[D:P]=p$, то $(p)\neq P$ и единственная возможность при степени $2$:
\[
(p)=P^2.
\]

\subsubsection*{Случай 2: $\left(\frac{\delta_F}{p}\right)=1$ (разложение)}

Пусть $\left(\frac{\delta_F}{p}\right)=1$. Тогда $p\nmid\delta_F$ и
(для нечётного $p$) это эквивалентно $\left(\frac{d}{p}\right)=1$,
то есть существует $a\in\Z$ такое, что
\[
a^2\equiv d\pmod p.
\]
Определим идеалы
\[
P=(p,\ a+\sqrt d),\qquad P'=(p,\ a-\sqrt d).
\]
Тогда
\[
PP' \subset (p,\ a+\sqrt d)(p,\ a-\sqrt d)
\subset (p,\ a+\sqrt d,\ a-\sqrt d,\ \tfrac{a^2-d}{p}).
\]
Обозначим
\[
I=(p,\ a+\sqrt d,\ a-\sqrt d,\ \tfrac{a^2-d}{p}).
\]
Заметим, что
\[
2a=(a+\sqrt d)+(a-\sqrt d)\in I,\qquad
2\sqrt d=(a+\sqrt d)-(a-\sqrt d)\in I.
\]
Кроме того, по построению $\frac{a^2-d}{p}\in\Z\subset D$, поэтому
\[
(p,\ 2a,\ \tfrac{a^2-d}{p})=1 \ \Rightarrow\ 1\in I \ \Rightarrow\ I=D.
\]
Следовательно,
\[
PP'\subset (p)\cdot I=(p).
\]
Обратно, поскольку $p\in P$ и $p\in P'$, имеем $(p)\subset PP'$.
Итак,
\[
(p)=PP'.
\]
Покажем, что $P\neq P'$. Если бы $P=P'$, то из $a+\sqrt d\in P=(p,a-\sqrt d)$
следовало бы
\[
(a+\sqrt d)-(a-\sqrt d)=2\sqrt d\in P.
\]
Тогда $\sqrt d\in P$ (так как $p$ нечётное, $2$ обратимо по модулю $P$),
а значит $P\supset(p,\sqrt d)$, что в сочетании с $p\nmid d$ приводит к противоречию
(в этом случае $(p,\sqrt d)=D$). Следовательно, $P\neq P'$.

\subsubsection*{Случай 3: $\left(\frac{\delta_F}{p}\right)=-1$ (инертность)}

Пусть $\left(\frac{\delta_F}{p}\right)=-1$, то есть $p\nmid\delta_F$ и
$\left(\frac{d}{p}\right)=-1$.

Предположим противное: $(p)$ не прост в $D$. Тогда
\[
(p)=PP'
\]
для некоторых различных простых идеалов $P\neq P'$ и при этом обязательно
$f=1$, то есть
\[
|D/P|=p.
\]
Рассмотрим класс $\overline{\sqrt d}\in D/P$. Так как $D/P\simeq\F_p$,
существует $a\in\Z$ такое, что $\overline{\sqrt d}=\overline a$ в $D/P$,
то есть
\[
\sqrt d-a\in P \quad\Rightarrow\quad d-a^2\in P\cap\Z=(p).
\]
Значит $a^2\equiv d\pmod p$, то есть $\left(\frac{d}{p}\right)=1$ --- противоречие.
Следовательно, разложения нет и остаётся единственный вариант:
\[
(p)\ \text{прост в }D,\qquad f=2.
\]

\subsection*{Замечание про случай $p=2$}

Далее рассматривается $p=2$ отдельно. Возможны три ситуации:
\[
\begin{array}{ll}
\text{(i)} & 2\mid\delta_F \ \Longrightarrow\ (2)=P^2;\\
\text{(ii)} & 2\nmid\delta_F,\ d\equiv 1\pmod 8 \ \Longrightarrow\ (2)=PP',\ P\neq P';\\
\text{(iii)} & 2\nmid\delta_F,\ d\equiv 5\pmod 8 \ \Longrightarrow\ (2)\ \text{прост в }D.
\end{array}
\]

\subsection*{Критерий единицы (пометка на полях)}
Для $\alpha\in D$:
\[
\alpha\in D^\times \quad\Longleftrightarrow\quad N_{F/\Q}(\alpha)=\pm 1.
\]


\subsection*{Случай $d<0$}

Пусть $d<0$. Обозначим через $\mathcal D$ кольцо целых поля $\mathbb Q(\sqrt d)$.
Тогда группа единиц
\[
U_d = \mathcal D^{\times}
\]
конечна.

\begin{enumerate}
\item При $d=-1$:
\[
U_{-1} = \{\pm 1, \pm i\}, \qquad |U_{-1}| = 4.
\]

\item При $d=-3$:
\[
U_{-3} = \{\pm 1, \pm \omega, \pm \omega^2\},
\qquad
\omega = \frac{-1+\sqrt{-3}}{2},
\qquad
|U_{-3}| = 6.
\]

\item При $d<-3$:
\[
U_d = \{\pm 1\}.
\]
\end{enumerate}

\begin{proof}
Пусть $x+y\sqrt d \in \mathcal D$ — единица.
Тогда
\[
x^2 - d y^2 = \pm 1.
\]

Если $|d|>1$, то при $y\neq 0$ имеем
\[
x^2 + |d|y^2 \ge |d| > 1,
\]
что невозможно.
Следовательно, $y=0$, откуда $x=\pm1$.
\end{proof}


\subsection*{Циклотомические поля}

Пусть $m\ge 1$,
\[
\zeta_m = e^{2\pi i/m},
\qquad
F = \mathbb Q(\zeta_m).
\]

Тогда
\[
x^m - 1 = \prod_{d\mid m} \Phi_d(x),
\]
где $\Phi_m(x)$ — $m$-й циклотомический многочлен.

\[
\deg \Phi_m = \varphi(m).
\]

Следовательно,
\[
[F:\mathbb Q] = \varphi(m).
\]


\subsection*{Группа Галуа циклотомического поля}

Рассмотрим
\[
\mathrm{Gal}(\mathbb Q(\zeta_m)/\mathbb Q).
\]

Для $\sigma \in \mathrm{Gal}(\mathbb Q(\zeta_m)/\mathbb Q)$ имеем
\[
\sigma(\zeta_m)^m = 1,
\]
то есть
\[
\sigma(\zeta_m) = \zeta_m^a,
\qquad
(a,m)=1.
\]

Тем самым получаем отображение
\[
\mathrm{Gal}(\mathbb Q(\zeta_m)/\mathbb Q)
\longrightarrow
(\mathbb Z/m\mathbb Z)^{\times},
\qquad
\sigma \mapsto a.
\]

\begin{lemma}
Это отображение является изоморфизмом групп.
\end{lemma}

\begin{proof}
Инъективность следует из того, что автоморфизм однозначно задаётся образом $\zeta_m$.
Сюръективность следует из того, что для каждого $(a,m)=1$ отображение
\[
\zeta_m \mapsto \zeta_m^a
\]
задаёт автоморфизм поля.
\end{proof}

Следовательно,
\[
\mathrm{Gal}(\mathbb Q(\zeta_m)/\mathbb Q)
\simeq
(\mathbb Z/m\mathbb Z)^{\times}.
\]


\subsection*{Композиция автоморфизмов}

Пусть $\sigma_a,\sigma_b \in \mathrm{Gal}(\mathbb Q(\zeta_m)/\mathbb Q)$,
где
\[
\sigma_a(\zeta_m)=\zeta_m^a,
\qquad
\sigma_b(\zeta_m)=\zeta_m^b.
\]

Тогда
\[
\sigma_a\circ\sigma_b(\zeta_m)
= \sigma_a(\zeta_m^b)
= \zeta_m^{ab}.
\]

Отсюда
\[
\sigma_a\circ\sigma_b = \sigma_{ab}.
\]

Нейтральный элемент:
\[
\sigma_1 = \mathrm{id}.
\]

Обратный элемент:
\[
\sigma_a^{-1} = \sigma_{a^{-1}},
\qquad
aa^{-1}\equiv 1 \pmod m.
\]

Тем самым группа автоморфизмов изоморфна
\[
(\mathbb Z/m\mathbb Z)^{\times}.
\]

\subsection*{Циклотомические поля}

Пусть $m \ge 1$ и
\[
\zeta_m = e^{2\pi i / m}.
\]
Определим циклотомическое поле
\[
K = \mathbb{Q}(\zeta_m).
\]

\begin{theorem}
Поле $\mathbb{Q}(\zeta_m)$ является расширением Галуа над $\mathbb{Q}$, и
\[
\mathrm{Gal}(\mathbb{Q}(\zeta_m)/\mathbb{Q}) \simeq (\mathbb{Z}/m\mathbb{Z})^\times.
\]
\end{theorem}

\begin{proof}
Каждому $a \in (\mathbb{Z}/m\mathbb{Z})^\times$ сопоставим автоморфизм
\[
\sigma_a(\zeta_m) = \zeta_m^a.
\]
Так как $(a,m)=1$, отображение корректно и сохраняет минимальный многочлен $\zeta_m$.
Композиция автоморфизмов соответствует умножению по модулю $m$, следовательно,
\[
\mathrm{Gal}(\mathbb{Q}(\zeta_m)/\mathbb{Q}) \cong (\mathbb{Z}/m\mathbb{Z})^\times.
\]
\end{proof}

\begin{corollary}
\[
[\mathbb{Q}(\zeta_m):\mathbb{Q}] = \varphi(m).
\]
\end{corollary}


\subsection*{Циклотомический многочлен}

\begin{definition}
\emph{Циклотомическим многочленом} $\Phi_m(x)$ называется минимальный многочлен числа $\zeta_m$ над $\mathbb{Q}$.
\end{definition}

Из определения следует:
\[
x^m - 1 = \prod_{d \mid m} \Phi_d(x).
\]

\begin{theorem}
Многочлен $\Phi_m(x)$ имеет целые коэффициенты и неприводим над $\mathbb{Q}$.
\end{theorem}

\begin{proof}
Докажем индукцией по $m$.
Предположим, что все $\Phi_d(x) \in \mathbb{Z}[x]$ для $d<m$. Тогда из разложения
\[
\Phi_m(x) = \frac{x^m - 1}{\prod_{d \mid m,\, d<m} \Phi_d(x)}
\]
следует, что $\Phi_m(x) \in \mathbb{Z}[x]$.

Неприводимость следует из того, что все сопряжения $\zeta_m^a$, $(a,m)=1$, являются корнями $\Phi_m(x)$, и их число равно $\varphi(m)$.
\end{proof}


\subsection*{Дискриминант циклотомического поля}

Пусть $K=\mathbb{Q}(\zeta_m)$.

\begin{theorem}
Дискриминант поля $\mathbb{Q}(\zeta_m)$ равен
\[
\mathrm{Disc}(K) = (-1)^{\varphi(m)/2}
\frac{m^{\varphi(m)}}{\prod_{p \mid m} p^{\varphi(m)/(p-1)}}.
\]
\end{theorem}

\begin{proof}
Используется формула дискриминанта через произведение разностей сопряжённых корней:
\[
\Delta = \prod_{i<j} (\zeta_m^i - \zeta_m^j)^2.
\]
После перегруппировки и применения свойств корней из единицы получается указанная формула.
\end{proof}


\subsection*{Разложение простых в циклотомических полях}

Пусть $p$ — простое число.

\begin{theorem}
Если $p \nmid m$, то разложение идеала $(p)$ в $\mathbb{Z}[\zeta_m]$ определяется порядком $p$ по модулю $m$.
\end{theorem}

\begin{proof}
Пусть $f$ — наименьшее число, такое что
\[
p^f \equiv 1 \pmod m.
\]
Тогда минимальный многочлен $\zeta_m$ по модулю $p$ распадается на $\varphi(m)/f$ неприводимых множителей степени $f$.
Следовательно,
\[
(p) = \mathfrak{p}_1 \cdots \mathfrak{p}_{\varphi(m)/f},
\quad \deg \mathfrak{p}_i = f.
\]
\end{proof}

 \begin{corollary}
Простое $p$ полностью раскладывается в $\mathbb{Q}(\zeta_m)$ тогда и только тогда, когда
\[
p \equiv 1 \pmod m.
\]
 \end{corollary}



\subsection*{Связь с кольцами вычетов}

\begin{theorem}
Существует изоморфизм
\[
\mathrm{Gal}(\mathbb{Q}(\zeta_m)/\mathbb{Q}) \cong \mathrm{Aut}(\mu_m) \cong (\mathbb{Z}/m\mathbb{Z})^\times,
\]
где $\mu_m$ — группа корней $m$-й степени из единицы.
\end{theorem}



\subsection*{Заключание}

Циклотомические поля являются фундаментальным примером абелевых расширений поля $\mathbb{Q}$.
Они играют ключевую роль в:
\begin{itemize}
  \item теории Галуа,
  \item разложении простых в числовых полях,
  \item доказательстве теоремы Кронекера–Вебера,
  \item арифметике круговых полей.
\end{itemize}

\end{document}

