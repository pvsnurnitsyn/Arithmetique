\documentclass[a4paper, 12pt]{article}
\usepackage[left=20mm, right=10mm, top=20mm, bottom=20mm, parindent=1.25cm]{geometry}
\usepackage{setspace}
\setstretch{1.5}
\usepackage[utf8]{inputenc}
\usepackage[russian]{babel}
\usepackage{amsmath}
\usepackage{amsfonts}
\usepackage{graphicx}
\usepackage{float} 
\usepackage{tabto} 

\title{A-I Лекция 9}
\author{Енина Анна, 491 группа}
\date{Москва, 11 ноября 2025}

% Переопределяем \maketitle
\makeatletter
\renewcommand{\maketitle}{
     \thispagestyle{empty}
    \begin{center}
        {\LARGE\bfseries \@title}
    \end{center}
    \vspace{1cm}
    \vfill % Заполняет пространство сверху
    \begin{center}
        {\LARGE\bfseries Тригонометрические суммы. Уравнения над конечными полями}
    \end{center}
    
    \vfill % Заполняет пространство, чтобы контент был сверху
    
    \vspace{2cm} % Отступ перед автором и датой
    
    \noindent
    \hfill % Сдвигает автора вправо
    \@author
    
    \vspace{0.5cm} % Отступ между автором и датой
    
    \begin{center}
        \@date
    \end{center}
}
\makeatother

\begin{document}

\maketitle

\newpage
\pagenumbering{arabic}



Рассмотрим уравнения над конечным полем $\mathbb{F}_p$. Для начала изучим простое уравнение:

\[
x^2 + y^2 \equiv 1 \pmod{p}
\]

Количество решений этого уравнения в поле $\mathbb{F}_p$ обозначим как $N_p(x^2 + y^2 = 1)$.


Известно, что для уравнения $x^2 = a$ в конечном поле выполняется:

\[
N_p(x^2 = a) = 1 + \left(\frac{a}{p}\right)
\]

где $\left(\frac{a}{p}\right)$ -- символ Лежандра.


Рассмотрим теперь более общее уравнение:

\[
x^2 + y^2 = 1
\]

Количество решений можно выразить через сумму символов Лежандра:

\[
N_p(x^2 + y^2 = 1) = \sum_{\substack{a,b \in \mathbb{F}_p \\ a + b = 1 }} N_p(x^2 = a) \cdot N_p(y^2 = b) 
\]

Используя свойства характеров конечных полей, получаем:

\[
N_p(x^2 + y^2 = 1) = \sum_{a+b=1} \left(1 + \left(\frac{a}{p}\right)\right) \cdot \left(1 + \left(\frac{b}{p}\right)\right) \]

\[
= \sum_{a+b=1} 1 + \sum_{a} \left(\frac{a}{p}\right) + \sum_{b} \left(\frac{b}{p}\right) + \sum_{a+b=1} \left(\frac{a}{p}\right)\left(\frac{b}{p}\right) \]
В этой сумме второе и третье слагаемые равны нулю, а количество пар $a, b$, таких, что $a+b=1$, равно $p$.


\textbf{Опр.} Пусть $\chi$, $\lambda$ - мультипликативные  характеры. Тогда $J(\chi, \lambda) = \sum_{a+b=1} \chi(a) \lambda(b)$ - \textbf{матрица Якоби}.

\textbf{Теорема.} (cвойства матрицы Якоби)
Пусть $\chi, \lambda$ - не главные характеры, $\chi_0$ - главный, тогда выполняются следующие утверждения:
\\ 1. $J(\chi_0, \chi_0) = p$
\\ 2.  $J(\chi, \chi_0) = 0$
\\ 3. $J(\chi, \lambda^{-1}) = -\chi(-1) $
\\ 4. Если $\chi \lambda \ne \chi_0 $, то $J(\chi,\lambda) = \frac{G(\chi)*G(\lambda)}{G(\chi\lambda)}$
\\
\textbf{Следствие.} 
Пусть $\chi, \lambda$ такие, что $\chi \lambda \ne \chi_0 $, тогда верно: $|J(\chi, \lambda)| = \sqrt{p}$

\item
\textbf{Пример:} $N_p(x^2+y^2=1) = \begin{cases}
   p-1 &\text{,  $p \equiv 1 (4)$}\\
   p+1 &\text{, $p \equiv 3 (4)$}
 \end{cases}$


\subsection*{}

Теперь рассмотрим общий случай:
$x^n+y^n=1$ и пусть $p\equiv 1 (m)$

\ Тогда
\[
N_p(x^n + y^n = 1) = \sum_{a+b=1} N_p(x^n = a)N_p(y^n=b)\]

Используем $N(x^n = a) = \sum_{\chi^n=\chi_0} (\chi(a))$, и то, что $n$ делит $p-1$, тогда при $\chi_1$ - порождающий главный характер

из этого следует, что \[
N_p(x^n + y^n = 1) =  \sum_{j=0}^{n-1} \sum_{i=0}^{n-1} J(\chi^i, \chi^j)\]

\textbf{Теорема.} 

Для уравнения $x^n + y^n = 1$ в конечном поле $\mathbb{F}_p$:
\[
N_p(x^n + y^n = 1) = p + 1 + O_n(\sqrt{p})
\]

{Введем Обозначение:}

Символ Ландау:
\[
f(n) = o(g(n)) \quad \text{означает} \quad \exists C: |f(n)| < C|g(n)|
\]

\subsection*{Свойства тригонометрических сумм}

Сумма характеров:
\[
\sum_{\chi=1}^p e^{2\pi i \frac{a\chi}{p}} = 
\begin{cases} 
p, & a \equiv 0 \pmod{p} \\
0, & a \not\equiv 0 \pmod{p}
\end{cases}
\]

{Общая постановка задачи:}

Пусть $F(x_1, \ldots, x_n) \in \mathbb{F}_p[x_1, \ldots, x_n]$ — многочлен над конечным полем. 
Требуется найти количество решений:
\[
N_p(F = 0) - ?
\]

Для квадратичных форм:
\[
F(x_1, \ldots, x_n)\approx O(p)
\]

{Метод тригонометрических сумм:}

Используем формулу:
\[
\sum_{x_1, \ldots, x_n} \sum_y e^{2\pi i \frac{y}{p} F(x_1, \ldots, x_n)} = 
\begin{cases} 
p, & F(x_1, \ldots, x_n) \equiv 0 \pmod{p} \\
0, & F(x_1, \ldots, x_n) \not\equiv 0 \pmod{p}
\end{cases}
\]

Отсюда получаем формулу для количества решений:
\[
N_p(F = 0) = \frac{1}{p} \sum_{x_1, \ldots, x_n} \sum_y e^{2\pi i \frac{y}{p} F(x_1, \ldots, x_n)}
\]

{Асимптотическое поведение:}

\textbf{Теорема.} (без д-ва) Для многочлена $F$ общего положения:
\[
N_p(F = 0) = p^{n-1} + O(p^{n-1 - \frac{1}{2}})
\]


\textbf{Теорема(Хассе-Вейла).} 

Рассмотрим многочлен $F \in \mathbb{F}_p[x, y]$. Количество решений уравнения $F = 0$ оценивается как:
\[
N_p(F = 0) = p + 1 + O(\sqrt{p}) (2g\sqrt{p})
\]


\textbf{Лемма.} 
Пусть $d = (r, p-1)$, $z \neq 0$. Тогда:
\[
N_p(x^r = z) = \sum_{s=0}^{d-1} \chi_s(z)
\]
где $\chi_0, \chi_1, \ldots, \chi_{d-1}$ — характеры, и $\chi^d = \chi_0$.


{Д-во:}
Пусть $\mathbb{F}^*_p = <\eta>$
Рассмотрим замену переменных:
\[
x^r = z, \quad z = \eta^k, \quad x=\eta^n
\]
тогда и только тогда, когда $rn\equiv k(p-1)$

из этого следует:
\[
N_p(x^r = z) = 
\begin{cases}
d, & d \mid k \\
0, & d  \nmid  k \\
 \end{cases}
\]
С другой стороны:

Характер порядка $d$ определяются как:
\[
\chi_s(z) = e^{2\pi i \frac{ks}{d}} \quad \text{при } z = \eta^k, \quad 0 \leq s \leq d-1
\]
\[
\sum_{s=0}^{d-1} \chi_s(z) = 
\begin{cases}
d, & d \mid k \\
0, & d  \nmid  k \\
 \end{cases}
\]



\section*{Уравнения над конечными полями}

\textbf{Теорема.} 

Рассмотрим линейную форму над конечным полем $\mathbb{F}_p$:
\[
F(x_1, \ldots, x_n) = a_1 x_1 + \cdots + a_n x_n
\]
где $a_i \in \mathbb{F}_p^*$ — ненулевые коэффициенты.

Обозначим количество решений уравнения $F = 0$ как:
\[
N_p = N_p(F = 0)
\]

Имеет место оценка:
\[
|N_p - p^{n-1}| \leq C(p-1)p^{\frac{n}{2}-1}
\]
\\
Док-во:
\\

Выразим количество решений через тригонометрические суммы:
\[
N_p(F) = p^{n-1} + \frac{1}{p} \sum_{y \in \mathbb{F}_p^*} \sum_{x_1, \ldots, x_r \in \mathbb{F}_p} e^{2\pi i \frac{y}{p}(a_1 x_1 + \cdots + a_r x_r)}
\]

Или в более компактной форме:
\[
N_p(F) = p^{n-1} + \frac{1}{p} \sum_{y \in \mathbb{F}_p^*} \prod_{i=1}^n \sum_{x_i \in \mathbb{F}_p} e^{2\pi i \frac{y a_i x_i ^{r_i} }{p}}
\]


Для каждого $x \in \mathbb{F}_p$:
\[
\sum_{x \in \mathbb{F}_p} e^{2\pi i \frac{a x^r}{p}} = \sum_{z} N_p(x ^ r = z) e^{2\pi i \frac{a z}{p}} 
= 1 + \sum_{z \neq 0} \sum_{s=0}^{d-1} \chi_s(z) e^{2\pi i \frac{a z}{p}}
\]
\[
= 1 + \sum_{z=0} \chi_0(z) e^{2\pi i \frac{a z}{p}} + \sum_{z \neq 0} \sum_{s=1}^{d-1} \chi_s(z) e^{2\pi i \frac{a z}{p}} 
= \sum_{s=1}^{d-1} \G_a(\chi_s)
\]


Используя свойства характеров и сумм Гаусса, получаем:
\[
|N_p - p^{n-1}| \leq \frac{1}{p} \sum_{y \neq 0} \left| \prod_{i=1}^n \sum_{j=1}^{d-1} G_a(\chi_{ij}) \right| 
= \frac{1}{p} (p-1) C p^{\frac{n}{2}}
\]

При условии:
\[
\prod_{i=1}^n |d_i| = 1
\]




\section*{Количество решений квадратичных уравнений над конечными полями}

\subsection*{Случай квадратичной формы}
\textbf{Теорема.} 
Для уравнения суммы квадратов в конечном поле $\mathbb{F}_p$:
\[
N_p(x_1^2 + \ldots + x_n^2 = 1) = 
\begin{cases}
p^{n-1} + (-1)^{\frac{n-1}{2}\frac{p-1}{2}} p^{\frac{n-1}{2}}, & n \equiv 1 \pmod{2} \\
p^{n+1} + (-1)^{\frac{n}{2}\frac{p-1}{2}} p^{\frac{n}{2} - 1}, & n \equiv 0 \pmod{2}
\end{cases}
\]

 поле $\mathbb{F}_q$, где $q = p^s$ — степень простого числа.

\textbf{Лемма.} 
Пусть $m \in \mathbb{Z}$ и определим сумму:
\[
S(m) = \sum_{x \in \mathbb{F}_q^*} \chi
\] в $\mathbb{F}_q$


Тогда:
\[
S(m) = 
\begin{cases}
- 1, & {q-1}  | m  (p) \\
0, & ${q-1} \nmid m  (p)$
\end{cases}
\]
\\
Док-во:

рассмотрим случай $q-1  | m (p)$
\[ \forall   x \in \mathbb{F}_q^* : x^{q-1} = 1 
\]

Для характера $\chi$ и $x \in \mathbb{F}_q^*$:
\[
\chi^m = 1 \quad \Rightarrow \quad S(m) = (q - 1) \cdot 1 = - 1
\]

Если существует $\alpha \in \mathbb{F}_q^*$ такой, что $\alpha^m \neq 1$, то:
\[
S(m) = \sum (ax)^m = a^m \zeta(m)
\]
что влечёт $\zeta(m) = 0$.

\textbf{Следствие.}

\begin{consequence}
Пусть $\Phi(x_1, \ldots, x_n) \in \mathbb{F}_q[x_1, \ldots, x_n]$ -- многочлен от $s$ переменных над конечным полем $\mathbb{F}_q$, причём $\deg \Phi \leq n \cdot (q-1)$.

Тогда сумма значений многочлена по всем точкам пространства:
\[
\sum_{x_1, \ldots, x_n \in \mathbb{F}_q} \Phi(x_1, \ldots, x_n) = 0
\]
в $\mathbb{F}_q$.
\end{consequence}
\\
Д-во:

Введём:
\[
\Phi = (x_1^{k_1}, \cdots, x_s^{k_s})
\]

Рассмотрим сумму мономов:
\[
\sum_{x_1, \ldots, x_s \in \mathbb{F}_q} x_1^{n_1} \cdots x_s^{n_s} = \left(\sum_{x_1 \in \mathbb{F}_q} x_1^{k_1}\right) \cdots \left(\sum_{x_s \in \mathbb{F}_q} x_s^{k_s}\right)
\]

\[
k_1 + \cdots + k_s \leq  n(q-1)
\]
и $S \leq  n$.

Если существует $k_i < d - 1$, то для суммы:
\[
\sum_{x \in \mathbb{F}_q} x^{k_i} = 0
\]
при условии, что $k_i$ не делится на $q$.




\textbf{Теорема(Варинга).}
Пусть $F \in \mathbb{F}_q[x_1,\ldots, x_n]$ -- многочлен от n переменных над конечным полем $\mathbb{F}_q$, где $q = p^s$, и пусть $\deg F = r < n$.

Тогда $p$ делит $N_q(F)$
\\
Док-во:
\[
\Phi(x_1,\ldots, x_s) = 1 - F(x_1, \ldots, x_n)^{q-1}
\]
и
\[
\deg \Phi = r(q-1) < n(q-1)
\]

Из того, что:
\[
\sum_{x_1,\ldots, x_n \in \mathbb{F}_q} \Phi(x_1,\ldots, x_n) = N_q(F) \cdot 1 \Rightarrow p \mid N_q(F)
\]

\textbf{Теорема(Шавали).}
Пусть $F \in \mathbb{F}_q[x_1, \ldots,x_n]$ -- многочлен от n переменных над конечным полем $\mathbb{F}_q$, и $\deg F = r < n$. 

Тогда уравнение $F = 0$ имеет нетривиальное решение в $\mathbb{F}_q^\infty$.
\\
Док-во:
 $F(0, \ldots, 0) = 0$, то:
\[
N_q(F) \geq 1 \Rightarrow N_q(F) > p
\]

\end{document}

