\documentclass[12pt]{article}

\usepackage[utf8]{inputenc}
\usepackage[T2A]{fontenc}
\usepackage[russian]{babel}
\usepackage{amsmath,amssymb,amsthm}
\usepackage{geometry}
\usepackage{xcolor}

\geometry{margin=2.5cm}

\begin{document}

\begin{center}
{\LARGE \textbf{Метризованные поля. $p$-адические числа}}
\end{center}

\vspace{1em}

Пусть $p$ — простое число. Для рационального числа $x=\dfrac{a}{b}$ определена функция порядка
\[
\nu_p(x)=\nu,
\]
где $\nu$ — показатель степени $p$, с которой $p$ входит в разложение дроби $a/b$, то есть
\[
x = p^{\nu}\frac{a'}{b'}, \qquad p\nmid a',\; p\nmid b'.
\]

На основе этого определена функция
\[
\varphi_p(x)=
\begin{cases}
0, & x=0,\\
\rho^{\nu_p(x)}, & x\neq 0,
\end{cases}
\]
где $\rho$ — фиксированное действительное число, $0<\rho<1$.

Функция $\varphi_p$ обладает свойствами абсолютного значения:
\[
\varphi_p(xy)=\varphi_p(x)\varphi_p(y),
\]
\[
\varphi_p(x+y)\le \max\{\varphi_p(x),\varphi_p(y)\},
\]
\[
\varphi_p(x)\ge 0,
\]
для всех $x,y\in\mathbb{Q}$.

\medskip

Целые $p$-адические числа $ \mathbb{Z}_p $ определяются как множество последовательностей
\[
(x_n)_{n\ge 0},
\]
где $x_n\in\mathbb{Z}/p^n\mathbb{Z}$ и выполнено условие согласованности:
\[
x_n \equiv x_{n-1} \pmod{p^{n-1}}.
\]

Эквивалентно:
\[
\nu_p(x_n-x_{n-1})\ge n,
\quad\text{или}\quad
\varphi_p(x_n-x_{n-1})\le \rho^n.
\]

На $\mathbb{Z}_p$ определяются операции сложения и умножения, и это кольцо без делителей нуля. Его поле частных — поле $p$-адических чисел $\mathbb{Q}_p$.

Всякое ненулевое $p$-адическое число единственным образом представляется в виде
\[
x=p^{\nu_p(x)}u,
\]
где $u$ — обратимый элемент в $\mathbb{Z}_p$, который имеет разложение
\[
u=a_0+a_1p+a_2p^2+\dots,
\quad a_0\neq 0,\quad 0\le a_i\le p-1.
\]

\medskip

Абсолютное значение на поле $K$ — это отображение
\[
\varphi:K\to\mathbb{R}_{\ge 0},
\]
такое что:
\[
\varphi(x)=0 \iff x=0,
\quad
\varphi(xy)=\varphi(x)\varphi(y),
\quad
\varphi(x+y)\le \varphi(x)+\varphi(y).
\]

Соответствующая метрика:
\[
d(x,y)=\varphi(x-y).
\]

Пару $(K,\varphi)$ называют \emph{метризованным полем}.

\medskip

Примеры:
\begin{itemize}
\item обычный модуль на $\mathbb{R}$;
\item модуль комплексного числа;
\item тривиальная метрика: $\varphi(x)=1$ при $x\neq 0$;
\item $p$-адическая метрика.
\end{itemize}

\medskip

Свойства:
\[
\varphi(-x)=\varphi(x),
\quad
\varphi(x-y)\ge \bigl|\varphi(x)-\varphi(y)\bigr|,
\quad
\varphi\!\left(\frac{x}{y}\right)=\frac{\varphi(x)}{\varphi(y)}.
\]

\medskip

Последовательность $(x_n)$ сходится к $x$, если
\[
\lim_{n\to\infty}\varphi(x_n-x)=0.
\]

Последовательность называется \emph{фундаментальной (Коши)}, если
\[
\varphi(x_n-x_m)\xrightarrow[n,m\to\infty]{}0.
\]

Метризованное поле называется \emph{полным}, если всякая фундаментальная последовательность имеет предел в этом поле.

\medskip

Поле $\mathbb{R}$ полно относительно обычного модуля. Поле $\mathbb{Q}_p$ полно относительно $p$-адической метрики.

\medskip

Пусть $K_0\subset K$ — расширение полей с метриками $\varphi_0,\varphi$. Расширение называется метризованным, если ограничение $\varphi$ на $K_0$ совпадает с $\varphi_0$.

\medskip

\textbf{Теорема (о пополнении).}
Для любого метризованного поля $(K,\varphi)$ существует полное метризованное поле $(\overline K,\overline\varphi)$ такое, что:
\begin{itemize}
\item $K$ плотно в $\overline K$;
\item $\overline\varphi|_K=\varphi$;
\item пополнение единственно с точностью до топологического изоморфизма.
\end{itemize}

\[
\triangleright
\text{Доказательство аналогично построению } \mathbb{R} \text{ как пополнения } \mathbb{Q}.
\quad\blacksquare
\]

\medskip

Метрики $\varphi_1,\varphi_2$ называются \emph{эквивалентными}, если они задают одну и ту же топологию.

\medskip

\textbf{Теорема.}
Для метризованного поля $(K,\varphi_1,\varphi_2)$ эквивалентны:
\begin{enumerate}
\item $\varphi_1$ и $\varphi_2$ эквивалентны;
\item сходящиеся последовательности совпадают;
\item $\varphi_1(x)<1 \iff \varphi_2(x)<1$;
\item существует $\alpha>0$ такое, что
\[
\varphi_1(x)=\varphi_2(x)^{\alpha}.
\]
\end{enumerate}

\[
\triangleright
\text{Основной шаг: }(3)\Rightarrow(4)\text{ через логарифмы и плотность }\mathbb{Q}.
\quad\blacksquare
\]

\medskip

Следовательно, все метрики вида $\rho^{\nu_p(x)}$ эквивалентны. Обычно берут $\rho=p^{-1}$:
\[
|x|_p=p^{-\nu_p(x)}.
\]

\medskip

\textbf{Теорема Островского.}
Всякая нетривиальная абсолютная величина на $\mathbb{Q}$ эквивалентна либо обычному модулю, либо $p$-адическому модулю для некоторого простого $p$.

\medskip

\textbf{Следствие.}
Все пополнения поля $\mathbb{Q}$ — это либо $\mathbb{R}$, либо поля $\mathbb{Q}_p$.

\medskip

Метрика называется \emph{неархимедовой}, если
\[
\varphi(x+y)\le \max\{\varphi(x),\varphi(y)\}.
\]

$p$-адические метрики — неархимедовы. Для них верно:
\[
\varphi(z+1)\le \max\{\varphi(z),1\}.
\]

\medskip

Формула произведения:
\[
|x|_\infty \prod_p |x|_p = 1,
\quad x\in\mathbb{Q}^\times.
\]

\medskip

\end{document}
