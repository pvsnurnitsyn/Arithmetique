\documentclass[11pt]{article}
\usepackage[utf8]{inputenc}
\usepackage[russian]{babel}
\usepackage{amsmath}
\usepackage{amssymb}
\usepackage{amsthm}
\usepackage{algorithm}
\usepackage{algorithmic}
\usepackage{fullpage}
\usepackage{cancel}

% Окружения теорем (названия на русском)
\newtheorem{theorem}{Теорема}
\newtheorem{lemma}{Лемма}
\newtheorem{proposition}{Утверждение}
\newtheorem{corollary}{Следствие}
\newtheorem{fact}{Факт}
\newtheorem{definition}{Определение}
\newtheorem{remark}{Замечание}

% Обозначения
\newcommand{\F}{\mathbb{F}}
\newcommand{\R}{\mathbb{R}}
\newcommand{\N}{\mathbb{N}}
\newcommand{\E}{\mathbb{E}}
\newcommand{\Z}{\mathbb{Z}}
\newcommand{\wt}{\mathrm{wt}}
\newcommand{\Spec}{\mathrm{Spec}}
\newcommand{\Inv}{\mathrm{Inv}}
\newcommand{\codim}{\mathrm{codim}}
\newcommand{\sgn}{\mathrm{sgn}}

\begin{document}

\section{P-адические числа}

Вводный пример:
Рассмотрим:
\begin{equation}
    x^2 \equiv 2 \ (mod \ 7^n), \ n \in \Z_{\geq1}
\end{equation}
Пусть сначала:
\begin{equation}
    ] \ n = 1: \ x^2 \equiv 2 \ (mod \ 7) \newline
    \Rightarrow \ x = \pm \ 3 \ (7)
\end{equation}
Пусть теперь:
\begin{equation}
    ] \ n = 2: \ x^2 \equiv 2 \ (mod \ 49) \newline
\end{equation}
Ищем решение вида: 
\begin{equation}
    x_1= x_0 + 7 t; x_0  = 3 \Rightarrow \ (3 + 7t)^2 = 2 \ (7^2)
\end{equation}
\begin{equation}
    9 + 6 \cdot 7 t + 7^2t^2 = 2 \ (7^2) \ \Rightarrow
\end{equation}
\begin{equation}
    7 + 6 \cdot 7 t + 7^2t^2 = 0 \ (7^2) \ \Rightarrow
\end{equation}
\begin{equation}
    1 + 6t + 7 t^2 = 0 \ (7) \ \Rightarrow
\end{equation}
\begin{equation}
    6t \equiv -1 \ (7) \ \Rightarrow
\end{equation}
\begin{equation}
    t_1 = 1 \ \Rightarrow
\end{equation}
\begin{equation}
    x_1 = x_0 + 7t = 3 + 7 = 10 \ (7^2) 
\end{equation}
Продолжим:
\begin{equation}
    ] \ n = 3: \ x^2 \equiv 2 \ (mod \ 7^3) \newline
\end{equation}
Ищем:
\begin{equation}
    x_2= x_1 + 7^2 t_2 \ (7^3)
\end{equation}
И т.д. \newline
Обратим внимание, что:
\begin{equation}
    x_n \equiv x_{n-1} \ (7^n)
\end{equation}
И что:
\begin{equation}
    x_n^2 \equiv 2 \ (7^{n+1})
\end{equation}

\begin{definition}
{$\forall p-$}простого, ${\forall n \in \Z_{>0}}$ $\exists \ a: p^a | n, \ p^{a+1}$ не делит $n$, тогда $a = \nu_p(n)$, где $\nu_p(\cdot)$ - функция показатель.
\end{definition}
\begin{definition}
Если $p-$простое, то последовательность $(x_n) = \{x_0, x_1, x_2, ...\}, \ x_n \in \Z \ \forall n$ и $x_n \equiv x_{n-1} \ (p^n), n \geq 1$ называется согласованной. Более того, две согласованных последовательности $(x_n), (x_n^{'})$ называются эквивалентыми, если $x_n \equiv x_n^{'} \ (p^{n+1}) \ \forall n$.
\end{definition}
\begin{definition}
Множество классов эквивалентности назовем целыми $p-$адическими числами. Множество целых $p-$адических чисел будем обозначать $\Z_p$.
\end{definition}
В примере выше:
$(x_0, x_1, ...) = \alpha, \ \alpha^2=2$. По сути: $\alpha$ это $7-$адический корень из $2$.
\begin{definition}
Последовательность $(x_n)$ называется канонической, если $\forall n: 0\leq x_n < p^{n+1}$ и $\overline{x_n} \equiv x_n \ (p^{n+1})$ и тогда $\overline{(x_n)} \sim (x_n)$.
\end{definition}
\begin{lemma}
Различные канонические последовательности определяют различные $p-$адические целые числа.
\end{lemma}
\begin{proof}
Пусть $(x_n) \sim \alpha$, - каноническая последовательность.
Тогда $x_{n+1} = x_n + a_{n+1}p^{n+1} \newline$
\begin{equation}
0 \leq x_{n+1} < p^{n+2}\newline
\end{equation}
\begin{equation}
0 \leq x_n < p^{n+1}\newline
\end{equation}
\begin{equation}
\Rightarrow 0 \leq a_{n+1} < p\newline
\end{equation}
\begin{equation}
x_n = a_0 + a_1p + a_2p^2 + ... + a_np^n.
\end{equation}
\end{proof}
\begin{lemma}
$\Z_p$ имеет мощьность континуум.
\end{lemma}
\begin{proof}
Доказательство аналогично доказательству аналогичного утверждения для вещественных чисел в математическом анализе.
\end{proof}
Введем операции кольца (определим их как покоординатное сложение/умножение):
\begin{equation}
\alpha \sim (x_n), \beta \sim (y_n) 
\end{equation}
\begin{equation}
\alpha + \beta \sim (x_n + y_n)
\end{equation}
\begin{equation}
\alpha \cdot \beta \sim (x_n \cdot y_n)
\end{equation}
\begin{lemma}
Предлагается доказать, что $\alpha + \beta$ и $\alpha \cdot \beta$ корректно введены.
\end{lemma}
\begin{lemma}
$\Z_p -$коммутативное кольцо с единицей без делителей нуля.
\end{lemma}
\begin{definition}
Пусть $\alpha, \beta \in \Z_p$. Будем обозначать $\alpha |\beta$, если $\exists \gamma\in \Z_p: \beta = \alpha \gamma$.
\end{definition}
\begin{lemma}
$\alpha = (x_n) \in \Z_p -$обратимый $\Longleftrightarrow x_0 \not \equiv 0 \ (p)$
\end{lemma}
\begin{proof}
$\newline$
Необходимость:
$\newline \alpha -$ единичный (обратимиый) $\Rightarrow \exists \beta \in (y_n): \alpha\beta=1 \Longleftrightarrow \forall n \ x_ny_n \equiv 1 \ (p^{n+1}); x_0y_0 \equiv 1 \ (p) \Rightarrow x_0 \not \equiv 0 \ \forall n.\newline$
Достаточночть:
$\newline x_0 \not \equiv 0 \ (p);$ так как 
$x_n \equiv x_{n-1} \equiv ... \equiv x_0 \ (p) \ \Rightarrow x_n \not \equiv 0 \ (p) \ \Rightarrow x_n$ обратим по $p$. 
\end{proof}
\begin{theorem}
$\forall \alpha \in \Z_p / \{0\}: \alpha = p^m\epsilon,$ причем такое разложение единственно, где $ \epsilon $ является $p-$адическим обратимым числом, то есть $\epsilon \in \Z_p^*-$ множество обратимых элементов.
\end{theorem}
\begin{proof}
$\newline ] \ \alpha \not \equiv e \Longleftrightarrow x_0 \not \equiv 0 \ (p)$
$\alpha \not \equiv 0 \Rightarrow \exists m: x_m \not \equiv 0 \ (p^{m+1})$
$x_{m+s} \equiv x_{m-1} \equiv 0 \ (p^m) \ \forall s \Rightarrow$
$y_s \equiv \frac{x_{m+s}}{p_m} - $целые числа $\in \Z$. Согласованность очевидна.
$\newline y_0 = \equiv \frac{x_m}{p_m} \not \equiv 0 \ (p)$
$(y_s) \sim \epsilon$
$p^{m}y_s= x_{m+s} \equiv x_s \ (p^{s+1})$
$\alpha = p^m\epsilon \newline$
Докажем единственность:
$\newline \alpha = p^k\mu$
$\mu \sim (\Z_s)$
$\Rightarrow p^m\epsilon = p^k\mu \Rightarrow p^my_s = p^kz_s \ (p^{s+1})$
$y_0, z_0 \not \equiv 0 \ (p) \Rightarrow$
$s = m \Rightarrow p^my_m \equiv p^kz_m \ (p^{m+1})$
Так как это выполняется для всех $m$, то $k=m$.
$p^my_{s+m} \equiv p^mz_{s+m} \ (p^{s+1})$
$\Rightarrow y_{s+1} \equiv z_{s+1} \Rightarrow \epsilon = \mu$
\end{proof}
$\forall \alpha \in \Z_p \Rightarrow \alpha = p^m\epsilon \Rightarrow$
$\nu_p(\alpha) = m \Rightarrow$
$\nu_p(\alpha \beta) = \nu_p(\alpha) + \nu_p(\beta)$
Делаем вывод: подтягиваются все свойства показательной функции.
\begin{theorem}
$\alpha \equiv \beta \ (\gamma) \Longleftrightarrow \gamma | (\alpha - \beta) $в $\Z_p \Longleftrightarrow p^n | (\alpha - \beta)$. Считаем, что $\gamma = p^n\epsilon$
\end{theorem}
Поезность ввода $p-$адических чисел:
$\newline \romannumeral 1. \ \forall \ \alpha \in \Z_p: \ \exists \ a \in \Z: \ \alpha \equiv a \ (p^n)$. Тогда $a \equiv b \ (p^n) \Longleftrightarrow a \equiv b \ (p^n)$. В первом случае сравнение происходит в кольце целых $p-$адических чисел $\Z_p$, а во втором - в поле характеристики $p^n: \F_{p^n}$. То есть можно убрать "хвост" и получить обычное сравнение в кольце вычетов.
% \section{Конечные поля}
% \begin{equation}
%     \Z /p \Z = \F_p = GF(p)
% \end{equation}
% \begin{equation}
% f,g \in \F_p[x]
% \end{equation}
% \begin{equation}
%     f \in \F_p[x] \text{ - неприводим}
% \end{equation}
% \begin{equation}
%     g |f \Rightarrow g\in \F_p or f= \alpha g
% \end{equation}
% \begin{equation}11
% gcd(f,g) = d = d(x)
% \end{equation}


% \begin{equation}
%     I: \forall f \in \F_p[x] \; f=c \prod f_i^{m_i} 
% \end{equation}

% $f_i$ - неприводимый унитарный

% % всё ок
% \begin{equation}
%     \F_p[x] - \text{кольцо главных идеалов} \; \forall \text{идеала } I \exists \text{неприводимый } f : I=(f)
% \end{equation}

% \begin{theorem}
% $f \in \F_p[x]$ - неприводимый. $F_p[x] / f$ - поле из $p^n$ элементов, $n=deg(f)$. Тогда $\F_p[x] /f, \; \Z/(p) = \F_p$
% \end{theorem}
% \begin{proof}
% $a(x) \in \F_p[x]$
% $\exists ! r(x) \in \F_p[x] : a \equiv r(f) \leftrightarrow f | a-r$
% $\exists ! $ доказательство
% Пусть $\exists r_1, r_2$
% $r_1 \equiv a(f), r_e \equiv a(f) \Rightarrow $ вы стёрли.........


% $r(x) = a_1x^{n-1} + ... + a_0$
% $a_1 \in \F_p$
% есть $p^n$ штук вот таких многочленов
% $| \F_p(x) / (f)| = p^n$
% Обозначим $O=(f)$
% \begin{equation}
%     E = \{ f \in \F_p[x] : g \equiv 1(f)
% \end{equation}
% $A \neq O$
% $a(x) \in A \Leftrightarrow f \; x \; a \Leftrightarrow (f,a) =(1) = \F_p[x] \Leftrightarrow \exists u,v \in \F_p[x]$
% $fu+av=1 \Rightarrow av=1(f)$
% то есть $V= [v(x)]$ класс в $\F_p[x]$

% Докажем единсвтенность

% $AW=E, W \neq V$
% $\exists w(x), w \cancel{\equiv} v(а)$
% $aw=1(f); av=1(f)$
% $a(w-v) = 0(f)$
% $fxa \Rightarrow f | w-v$
% \end{proof}

% Отсюда $q=p^n, \F_q=\F_p[x]/(f)$. $def(f) = r, f - $ неприводим

% \begin{theorem}
%     \begin{equation}
%         \forall n \ge 1 \exists f \in \F_p[x]
%     \end{equation}
%     неприводимый. $deg(f) = 1$
% \end{theorem}

%     Введём $N(g) = Ng=p(deg(g))$ для $g \in \F_p[x]$
%     Эта функция мультипликативна. $N(fg)=Nf \; Ng$
%     Аналог дзета функции. 
%     $\zeta (s) = \prod_p^*(1 - \frac{1}{(Nf)^s})^{-1}$
%     $\prod^*$ - произвдение по всем неприводимым унитарным многочленам из кольца
%     Область сходимости  $\zeta (s) = \sum_{n=1}^{\inf}\frac{1}{n^3} = \prod_{p - \text{простые}} (1 - \frac{1}{p^3})^{-1}$ при $Re(s) > 1$

%     \begin{equation}
%         \zeta = \prod_p^*(1 + \sum_{m=1}^{\inf} \frac{1}{(Nf)^{ms}})^{-1} = 
%         1 + \sum_{g}^{*} \frac{1}{(Nf)^{s}}
%     \end{equation}
%     \begin{equation}
%         \sum^* - \text{сумма по унитарным } g \in \F_p[x]
%     \end{equation}
%     \begin{equation}
%         = 1 \sum_{n=1}^{\inf} \sum_{g, deg(g) = n} = \frac{1}{(Ng)^s}
%     \end{equation}
%     \begin{equation}
%         = 1 + \sum_{n=1}^{\inf} p^n \frac{1}{p^{ns}}
%     \end{equation}
%     \begin{equation}
%         = (1-\frac{p}{p^s})^{-1}
%     \end{equation}
% Количество неприводимых унитарных мночленов где $deg(g) = n$ - $\nu (n) $

% \begin{equation}
%     \prod_{n=1}^{\inf} (1 - \frac{1}{p^{ns}})^{- \nu(n)} = 
%     \prod_f^* = (1 - \frac{1}{(Nf)^s})^{-1}
% \end{equation}
% прологорифмируем:
% \begin{equation}
%     \sum_{n=1}^{\inf} (-\nu(n) log(1-\frac{1}{p^{ns}} = -log(1 - \frac{p}{p^s})
%     \end{equation}
%     \begin{equation}
%     log(1-\tau) = \sum_{m=1}^{\inf} \frac{1}{m} \tau^m
% \end{equation}

% У доказательства есть бонус.
% \begin{equation}
%     \sum_{n=1}^{\inf} \nu(n) \sum_{l=1}^{\inf} \frac{1}{l} \frac{1}{p^{lns}} 
%     = \sum_{m=1}^{\inf} \frac{1}{m}\frac{p^m}{p^{ms}} 
% \end{equation}

% \begin{equation}
%     \sum_{n=1}^{\inf} \nu(n) \sum_{l=1}^{\inf} \frac{1}{l} \frac{1}{p^{lns}} 
%     = \sum_{n} \sum_l \frac{\nu(n)}{l}\frac{1}{p^{ms}} = 
%     \sum_{m=1}^{\inf} \sum _{n |m} \frac{\nu(n)}{m / n}\frac{1}{p^{ms}} = 
% \end{equation}

% \begin{equation}
%      = (\sum _{n |m} \nu(n) n)\frac{1}{m} \frac{1}{p^{ms}} = 
% \end{equation}
% \begin{equation}
%     \sum _{n |m} \nu(n) n = p^m
% \end{equation}
% \begin{equation}
%     \Rightarrow \nu(n) = \frac{1}{n} \sum _{d |m} \mu (d) p^{n/d} \cancel{=} 0 
% \end{equation}

% \begin{theorem}
%     \begin{equation}
%         \forall z \in \F_q^*, q - p^n, \F_q = \F_p[x]/(f)
%     \end{equation}
%     Тогда
%     \begin{equation}
%         z^{q-1} -1 = 0
%     \end{equation}
% \end{theorem}
% \begin{proof}
%     Пусть
%     \begin{equation}
%         g \in \F_p[x], Fxf
%     \end{equation}
%     \begin{equation}
%         \prod_r gr = \prod_r r(f)
%     \end{equation}
%     \begin{equation}
%         (g^{q-1} - 1) \prod r \equiv 0(f)
%     \end{equation}
%     Следовательно
%     \begin{equation}
%         \prod_{z \in \F_q} (x -z) = x^q - x
%     \end{equation}
% \end{proof}

% \begin{lemma}
%     \begin{equation}
%         f | x^q - x, deg(f) = d \Rightarrow
%     \end{equation}
%     $f$ имеет $d$ различных корней
% \end{lemma}
% \begin{proof}
%     \begin{equation}
%         x^q-x = f(x) g(x)
%     \end{equation}
%     f(x) - d корней, g(x) -  q-d корней

%     если $<d $ корней у f $\Rightarrow d+(q^1d) = 1$ у $x^q - x$ 
% \end{proof}
% \begin{theorem}
%     мультипликативная группа $\F_q^*$ циличская то содержит $\phi(q-1)$
% \end{theorem}
% \begin{proof}
%     \begin{equation}
%         m | q-1; \psi(m) - \text{число элементов поля}
%     \end{equation}
%     Если $\psi(m) > 0$, $\alpha \in \F^*_q -$ порядок $m$.
%     $1, \alpha, ..., \alpha^{m-1}$ - различны корни $x^m - 1$
%     это все корни $x^m - 1$
%     \begin{equation}
%         \forall \beta \in \F_q^* - \text{порядок }  | m
%     \end{equation}
%     \begin{equation}
%         \beta = \alpha^s, r\le s \le m-1
%     \end{equation}
%     если $(s,m) =d, \alpha^s$ - порядок $m/d$ \\
%     Было упражнение что 
%     \begin{equation}
%         \alpha^s \text{ порядок } m \Leftrightarrow (s,m)=1
%     \end{equation}
%     Таким образом если $\psi(m) > 0$ то $\psi(m) = \phi(m)$
%     \begin{equation}
%         \sum_{m | q-1} \psi(m) = q-1
%     \end{equation}
%     Мы знаем что
%     \begin{equation}
%         \sum_{m | q-1} \phi(m) = q-1
%     \end{equation}
%     Тогда
%     \begin{equation}
%         \sum_{m | q-1} (\phi(m) - \psi(m)) = 0 \Rightarrow 
%         \psi(q-1)=\phi(q-1) 
%     \end{equation}
% \end{proof}

\end{document}