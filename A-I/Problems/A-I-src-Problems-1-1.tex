\noindent\textbf{Упражнения и задачи}

\begin{enumerate}[topsep=0pt]
    \item Докажите свойства делимости: 
    \begin{itemize}[noitemsep,topsep=0pt]
        \item $a | a$, $a \neq 0$;
        \item $a|b, b|a \implies a=\pm b$;
        \item $a|b, b|c \implies a|c$;
        \item $a|b, a|c \implies a| b\pm c$.
    \end{itemize}
    \item Алгоритм Евклида: Пусть $a,b \in\mathbb{Z} \setminus \{0\}, a>b$, определим последовательность $b>r_1>r_2>\dots > r_n$ следующим образом: $a=bq_0+r_1$, $b=r_1q_1+r_2$, $r_1=r_2q_2+r_3$, \dots $r_n=r_{n-1} q_{n-1}+r_n$. Докажите, что $\exists n: r_{n-1}=r_nq_n$ и $r_n=(a,b)$. (Сначала докажите , что $(a,b)=(b,r_1)$).
    \item Докажите, что $\sqrt{2}$ --- иррациональное число, т.е. что не $\exists$ рационального $r=a/b$ ($a,b \in \mathbb{Z}$) такого, что $r^2=2$.
    \item Пусть $\alpha \in \mathbb{R}$, $b\in\mathbb{Z}_+$, докажите, что $\left[\dfrac{[\alpha]}{b}\right]=\dfrac{\alpha}{b}$.
    \item Пусть $(a,b)=1$, докажите, что $(a+b,a-b)=1$ или $2$.
    \item Пусть $a,b,c \in \mathbb{Z}$, докажите что уравнение $ax+by=c$ разрешимо в целых числах $\iff$ $d=(a,b)|c$. Докажите также, что если $x_0,y_0$ --- решение этого уравнения, то все решения имеют вид $x=x_0+t\dfrac{b}{d}$, $y=y_0-t\dfrac{b}{d}$, где  $t \in \mathbb{Z}$.
    \item Докажите следующие свойства:
    \begin{itemize}[noitemsep,topsep=0pt]
        \item $\nu_p([a,b])=\max(\nu_p(a),\nu_p(b))$;
        \item $\nu_p(a+b)\geqslant \min(\nu_p(a),\nu_p(b))$, причем $\nu_p(a+b)= \min(\nu_p(a),\nu_p(b))$, если $\nu_p(a) \neq \nu_p(b)$;
        \item $(a,b)[a,b]=ab$;
        \item $(a+b,[a,b])=(a,b)$.
    \end{itemize}
    \item Докажите, что $\nu_p(n!)= \left[\frac{n}{p}\right] + \left[\frac{n}{p^2}\right] + \left[\frac{n}{p^3}\right] + \dots $
    \item Докажите, что существует бесконечно много простых вида $4k-1$, $k\in\mathbb{Z}$. %например, [IR], [Stein ent]
    \item Пусть $a,b,c,d \in \mathbb{Z}$, $(a,b)=1$, $(c,d)=1$. Докажите, что если $\dfrac{a}{b}+\dfrac{c}{d} \in \mathbb{Z}$, то $b=\pm d$.
    \item Пусть $n \in \mathbb{Z}$, $n>2$ Докажите, что числа
    $$
    \dfrac{1}{2}+\dfrac{1}{3}+\dots+\dfrac{1}{n};\ 
        \dfrac{1}{3}+\dfrac{1}{5}+\dots+\dfrac{1}{2n+1}
    $$
    не являются целыми.
    \item Пусть $f(n)$ --- мультипликативная функция. Докажите, что функции
    $$
        g(n)=\sum\limits_{d|n} f(d),\ 
        h(n)=\sum\limits_{d|n} \mu(\frac{n}{d})f(d)
    $$
    также мультипликативны.
    \item Докажите, что $\forall n \in \mathbb{Z}$
    $$
    \sum\limits_{d|n} \mu(\frac{n}{d}) \nu(d) = 1,\ 
        \sum\limits_{d|n} \mu(\frac{n}{d}) \sigma(d) = n.
    $$
    \item Докажите, что $\forall m,n \in \mathbb{Z}$
    \begin{itemize}[noitemsep,topsep=0pt]
        \item $\varphi(n)\varphi(m)=\varphi((n,m))\varphi([n,m])$;
        \item $\varphi(mn)\varphi((m,n))=(m,n)\varphi(m)\varphi(n)$.
    \end{itemize}
    \item Пусть $P,Q\in\mathbb{Z}_+$ --- нечетные, $(P,Q)=1$. Докажите, что 
    $$
    \sum_{0<x<\frac{Q}{2}}\left[\frac{P}{Q}x\right] + \sum_{0<y<\frac{P}{2}}\left[\frac{Q}{P}y\right] = \frac{P-1}{2}\frac{Q-1}{2}.
    $$
    (Используйте подсчет целых точек в некоторой ограниченной области на плоскости).
    
\end{enumerate}

\noindent\textbf{SageMath}

\begin{itemize}[topsep=0pt]

    \item Исследуйте основные теоретико-числовые функции в SageMath:
    \begin{itemize}[noitemsep,topsep=0pt]
        \item НОД, НОК: \texttt{gcd(), xgcd(), lcm()};
        \item разложение на множители: \texttt{factor(), valuation()};
        \item простые числа: \texttt{is\_prime(), next\_prime(), previous\_prime()};
        \item делители: \texttt{divisors(), prime\_divisors() };
        \item функции Эйлера и Мёбиуса, число и сумма делителей: \texttt{euler\_phi(), moebius(), sigma()};
        \item число простых чисел: \texttt{prime\_pi()} (постройте график этой функции).
    \end{itemize}
    
\end{itemize}

%\noindent\textbf{Темы для самостоятельного изучения}