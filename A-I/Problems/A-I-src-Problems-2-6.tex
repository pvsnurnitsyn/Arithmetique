\noindent\textbf{Упражнения и задачи}

\begin{enumerate}[topsep=0pt]
    \item Докажите, что $|\mathbb{P}^n(\mathbb{F}_q)|=q^n+q^{n-1}+\dots+1$.
    \item Докажите, что для $f=-y_0^2+y_1^2+y_2^2+y_3^2$ дзета-функция имеет вид $Z_f(u) = (1-u)^{-1}(1-qu)^{-2}(1-q^2u)^{-1}$, если $-1$ --- квадрат в $\mathbb{F}_q$, и $Z_f(u) = (1-u)^{-1}(1-qu)^{-1}(1+qu)^{-1}(1-q^2u)^{-1}$ в противном случае. %[IR, p153]
    \item Докажите, что проективная $n$-мерная гиперплоскость в $\mathbb{P}^n(\mathbb{F}_q)$ (т.е. гиперповерхность, заданная однородным многочленом степени 1) имеет столько же точек сколько $n-1$-мерное проективное пространство $\mathbb{P}^{n-1}(\mathbb{F}_q)$. %[IR Ex 10.4]
    \item Пусть $f(x_0,x_1,x_2)\in \mathbb{F}_q[x_0,x_1,x_2]$ --- однорлный многочлен $\deg f = n$. $h = \in \mathbb{F}_q[x_0,x_1,x_2]$ --- линейная форма, такая что не каждый её нуль является нулём $f$. Докажите, что в $\mathbb{P}^2(\mathbb{F}_q)$ у $f$ и $h$ может быть не более $n$ общих нулей (т.е. плоская проективная кривая пересекается с проективной прямой в не более чем $n$ точках). %[IR Ex 10.5]
    \item Пусть $\SL_n(\mathbb{F}_q)$ --- множество $n\times n$ матриц с элементами из поля $\mathbb{F}_q$ и определителем равным $1$. Покажите, что $\SL_n(\mathbb{F}_q)$ можно рассматривать как гиперповерхность в $\mathbb{A}^{n^2}(\mathbb{F}_q)$ и что число её точек равно $(q-1)^{-1} (q^n-1) (q^n-q) \dots (q^n-q^{n-1})$. %[IR Ex 10.6]
    \item Пусть $\frac{\partial}{\partial x_i}$ --- операторы формальных производных на $\mathbb{F}_q[x_0,\dots,x_n]$ (например, для $f(x)=a_0 x^n +\dots + a_{n-1}x+a_n$ по определению $\frac{\partial}{\partial x} f=a_0 x^{n-1} +\dots + a_{n-1}$ и пусть $f\in\mathbb{F}_q[x_0,\dots,x_n]$ --- однородный многочлен $\deg f = m$. Докажите, что:
    \begin{itemize}[topsep=0pt]
        \item $\sum_{i=0}^n x_i \frac{\partial}{\partial x_i} = mf$; %[IR Ex 10.7]
        \item если $(m,p)=1$ ($p=\chr \mathbb{F}_q$) и для $a=(a_0,\dots,a_n)$ при всех $i$ выполняется $\frac{\partial}{\partial x_i} f(a)=0$, то $f(a)=0$. (Такая точка $a$ называется особой точкой гиперповерхности $f=0$). %[IR Ex 10.8]
    \end{itemize}
    \item Пусть $q=p^n$, $(m,p)=1$. Докажите, что гиперповерхность $a_0 x_0^m + \dots + a_n x_n^m=0$ не имеет особых точек в $\mathbb{P}^n(\mathbb{F}_q)$. %[IR Ex 10.9]
    \item Пусть $q=p^n$, $p\neq 2$. Рассмотрим кривую $ax^2+bxy+cy^22=1$, $a,b,c \in \mathbb{F}_q$. Докажите, что если $d=b^2-4ac$  не является квадратом в $\mathbb{F}_q$, то не существует бесконечно удаленных точек на кривой в $\mathbb{P}^n(\mathbb{F}_q)$, а если $d$ --- квадрат, то существует одна или две бесконечно удаленные точки, в зависимости от обращения $d$ в ноль. При этом если $d=0$, то бесконечно удаленная точка является особой точкой заданной кривой. %[IR Ex 10.13]
    \item Выпишите дзета-функцию кривой $x_0 x_1 - x_2 x_3=0$ над $\mathbb{F}_p$. %[IR Ex 11.4]
    \item Выпишите дзета-функцию для $f = a_0 x_0^2 + \dots + a_n x_n^2$ над $\mathbb{F}_q$ при $\chr(\mathbb{F}_q) \neq 2$. %[IR Ex 11.5]
    \item Покажите, что на кривой $x_0^3+x_1^3+x_2^3 = 0$ в $\mathbb{P}^2(\mathbb{F}_4)$ лежит девять точек. Выпишите дзета-функцию этой кривой. %[IR Ex 11.6]
    \item Выпишите дзета-функцию кривой $y^2=x^3+x^2$ над $\mathbb{F}_p$. %[IR Ex 11.9]
    \item Пусть $q \equiv 1\,(3)$, $\alpha\in\mathbb{F}_q^*$. Покажите, что дзетв-функция кривой $y^2=x^3+\alpha$ над $\mathbb{F}_q$ имеет вид $Z(u) = (1+au+qu^2)(1-u)^{-1}(1-qu)^{-1}$, где $a\in\mathbb{Z}$, $|a|\leqslant 2\sqrt{q}$. %[IR Ex 11.10]
    \item Пусть $C_1$ --- кривая над $\mathbb{F}_p$ заданная $y^2=x^3-Dx$, $D\neq 0$. Покажите, что подстановка $x=\frac{1}{2}(u+v^2)$, $y=\frac{1}{2}v(u+v^2)$ переводит $C_1$ в кривую $C_2$ заданную уравнением $u^2-v^4=4D$. Докажите, что для любого расширения $\mathbb{F}_q/\mathbb{F}_p$ для числа точек справедливо $|C_1(\mathbb{F}_q)| > |C_2(\mathbb{F}_q)|$. %[IR Ex 11.11]
\end{enumerate}

\noindent\textbf{SageMath}
\begin{itemize}[topsep=0pt]
    \item Исследуйте основные функции SageMath связанные с количеством точек на кривых над конечными полями:
    \begin{itemize}[noitemsep,topsep=0pt]
        \item Для эллиптических и гиперэллиптических кривых: \texttt{cardinality()}.
     \end{itemize}
\end{itemize}

\noindent\textbf{Темы для самостоятельного изучения}
\begin{itemize}[topsep=0pt]
    \item $L$-функции Артина. Суперэллиптическое уравнение. [Степ], \S\S I.3--I.4.
\end{itemize}