\noindent\textbf{Упражнения и задачи}

\begin{enumerate}[topsep=0pt]
    \item Докажите, что
    \begin{itemize}[noitemsep,topsep=0pt]
        \item для $a\in\mathbb{Z}$ $a^l-1|a^m-1$ $\Leftrightarrow$ $l|m$
        \item в $\mathbb{F}_q[x]$ $x^l-1|x^m-1$ $\Leftrightarrow$ $l|m$
    \end{itemize}
    \item Завершите доказательсво теоремы о цикличности $\Gal(\mathbb{F}_{p^n}/\mathbb{F}_p)$: покажите, что $|\Gal(\mathbb{F}_{p^n}/\mathbb{F}_p)| \leqslant n$.
    \item Докажите, что характеристический многочлен $g_\alpha(x)$ и определитель $\det A_\alpha$ не зависят от выбора базиса расширения $L/K$.
    \item Докажите следующие свойства нормы и следа:
    \begin{enumerate}[noitemsep,topsep=0pt]
        \item $\N_{L/K}(a)=a^n$, $\Tr_{L/K}(a)=na$, $a\in K$;
        \item $\N_{L/K}(\alpha\beta)=\N_{L/K}(\alpha)\N_{L/K}(\beta)$, $\alpha, \beta \in L$;
        \item $\N_{L/K}(a\alpha)=a^n\N_{L/K}(\alpha)$, $\Tr_{L/K}(a\alpha)=a\Tr_{L/K}(\alpha)$, $a\in K$, $\alpha \in L$.
    \end{enumerate}
    \item Пусть $L/K$ --- расширение, $\alpha \in L$, $g_\alpha(x)$ --- характеристический многочлен $\alpha$, $M/K$ --- расширение, в котором $g_\alpha(x)$ полностью раскладывается на линейные множители: $g_\alpha(x) = (x-\alpha_1)\dots(x-\alpha_n)$. Докажите, что:
    $$
        \N_{L/K}(\alpha)=\alpha_1\dots \alpha_n,\ \Tr_{L/K}(\alpha) = \alpha_1 + \dots + \alpha_n.
    $$
    \item Пусть $L/K$, $M/L$ --- конечные расширения, $\gamma\in M$. Докажите, что $\N_{M/K}(\gamma)=\N_{L/K}(\N_{M/L}(\gamma))$, $\Tr_{M/K}(\gamma)=\Tr_{L/K}(\Tr_{M/L}(\gamma))$.
    \item Пусть $(\alpha_1, \dots, \alpha_n)$, $(\beta_1, \dots, \beta_n)$ --- два базиса расширения $L/K$. Докажите, что $\Delta(\alpha_1, \dots, \alpha_n)=\gamma^2\Delta(\beta_1, \dots, \beta_n)$ для некоторого $\gamma\in K^*$.
    %\item TODO 90
    \item Пусть $q=p^n$. Докажите, что $\forall a\in\mathbb{F}_p$ уравнение $\N_{\mathbb{F}_q/\mathbb{F}_p} (x) = a$ в $\mathbb{F}_q$ имеет $(p^n-1)/(p-1)$ решений, а также что для каждого $b\in\mathbb{F}_p$  уравнение $\Tr_{\mathbb{F}_q/\mathbb{F}_p} (x) = b$ имеет $p^{n-1}$ решений. %[Степ Задача I.2.7]
    \item Пусть $\mathbb{F}_{q^m}/\mathbb{F}_q$ --- расширение конечных полей, $\alpha \in \mathbb{F_{q^m}}$. Докажите, что сопряженные элементы $\alpha, \alpha^q, \alpha^{q^2}, \dots \alpha^{q^{m-1}}$ имеют один и тот же порядок в мультипликативной группе $\mathbb{F}_{q^m}^*$.  %[LN, p50 th2.18]
    \item Пусть $\mathbb{F}_{q^m}/\mathbb{F}_q$ --- расширение. Докажите, что $\forall c\in \mathbb{F}_q$
    $$
        \sum_{j=0}^{m-1} x^{q^j} - c = \prod_{\alpha\in \mathbb{F}_{q^m}, \Tr(\alpha)=c} (x-\alpha).
    $$ %[LN Ex 2.34]
    \item Пусть $\mathbb{F}_{q^m}/\mathbb{F}_q$ --- расширение, $L$ --- линейный оператор на $\mathbb{F}_{q^m}$ как на векторном пространстве над $\mathbb{F}_q$. Докажите, что $\exists ! \alpha_0, \dots, \alpha_{m-1}\in \mathbb{F}_{q^m}$: оператор $L$ имеет вид $L(\beta) = \alpha_0 \beta + \alpha_1 \beta^q + \dots + \alpha_{m-1}\beta^{q^{m-1}}$. %[LN Ex 2.36]
    \item Докажите, что число базисов $\mathbb{F}_{q^m}/\mathbb{F}_q$ равно $(q^m-1)(q^m-q)\dots (q^m-q^{m-1})$. %[LN Ex 2.37]
    \item Пусть $\mathbb{F}_{q^m}/\mathbb{F}_q$ --- расширение, $\alpha\in\mathbb{F}_{q^m}$. Докажите, что 
    $$
        \Delta(1,\alpha, \dots, \alpha^{m-1}) = \prod_{0\leqslant i < j \leqslant m-1} (\alpha^{q^i}-\alpha^{q^j})^2.
    $$ %[LN Ex 2.45]
    \item Пусть $\mathbb{F}_{q^m}/\mathbb{F}_q$ --- расширение, $\alpha \in \mathbb{F}_{q^m}$. Докажите, что $\Delta(1,\alpha, \dots, \alpha^{m-1})$ совпадает с дискриминантом характеристического многочлена $g_\alpha(x)$. %[LN Ex 2.46]
    %\item TODO [Степ упр I.2.7]
    %\item TODO [Степ упр I.2.9]
    %\item TODO [Степ упр I.2.14]
\end{enumerate}


\noindent\textbf{SageMath}
\begin{itemize}[topsep=0pt]
    \item Исследуйте основные функции SageMath связанные с автоморфизмами и расширениями конечных полей:
    \begin{itemize}[noitemsep,topsep=0pt]
        \item Группа автоморфизмов поля: \texttt{End()};
        \item Автоморфизм Фробениуса: \texttt{frobenius\_endomorphism()};
        \item Базисы расширений конечных полей;
        \item Характеристический многочлен \texttt{charpoly()};
        \item Норма, след, дискриминант.
     \end{itemize}
\end{itemize}

\noindent\textbf{Темы для самостоятельного изучения}
\begin{itemize}[topsep=0pt]
    \item Нормальный базис и теорема о нормальном базисе, [LN].
\end{itemize}    