\noindent\textbf{Упражнения и задачи}

\begin{enumerate}[topsep=0pt]

    \item Докажите, что различные канонические последовательности определяют различные целые $p$-адические числа. %[БШ]

    \item Докажите, что для целых $p$-адических чисел $\alpha$ и $\beta$ заданные в лекции операции $\alpha \beta$, $\alpha + \beta$ корректно определены (то есть результат не зависит от выбора последовательностей-представителей $\alpha \sim (x_n)$, $\beta \sim (y_n)$) и $\mathbb{Z}_p$ --- действительно коммутативное кольцо. %[БШ]

    \item Пусть $\alpha = \sum_{n=0}^{\infty} a_n p^n \in \mathbb{Z}_p$. Какой будет иметь вид разложение числа $-\alpha$? %[БШ, G]

    \item Докажите, что уравнение $x^2=2$ не имеет решений в $\mathbb{Q}_5$.

    \item Докажите, что $\forall \alpha \in \mathbb{Z}_p\ \exists a \in \mathbb{Z}:\ \alpha \equiv a\ \pmod{p^n}$. Для $a, b \in \mathbb{Z}$ $a \equiv b\ \pmod{p^n}$ как сравнение в $\mathbb{Z}_p$ $\Leftrightarrow$ $a \equiv b\ \pmod{p^n}$ как сравнение в $\mathbb{Z}$. %[БШ]

    %\item Пусть $p\neq 2$, $c$ --- квадратичный вычет $\Mod p$. Докажите, что существует два различных $p$-адических числа $\alpha, \beta \in \mathbb{Q}_p:\ \alpha^2 = \beta^2 = c$. %[БШ] то же самое что следующая

    \item Пусть $p\neq 2$, $(m,p)=1$. Сформулируйте и докажите необходимое и достаточное условие разрешимости уравнения $x^2=m$ в $\mathbb{Q}_p$. Сделайте вывод, что $\mathbb{Q}_p$ не является алгебраически замкнутым.

    \item Докажите, что если $\xi_n \rightarrow \xi$ в $\mathbb{Q}_p$ и $\xi\neq 0$, то $1 / \xi_n \rightarrow 1/\xi$ в $\mathbb{Q}_p$. %[БШ]

    \item Докажите $p$-адический аналог утверждения из анализа: из всякой ограниченной последовательности можно выделить сходящуюся подпоследовательность. %[БШ]

    \item Докажите $p$-адический критерий Коши: последовательность $(\xi_n)$ сходится $\Leftrightarrow$\\ ${\nu_p(\xi_m-\xi_n) \rightarrow \infty}$, при $m,n \rightarrow \infty$. %[БШ]

    \item Пусть последовательность $(x_n)$ определена как $x_n = 1+p+\dots +p^{n-1}$. Докажите, что в $\mathbb{Q}_p$ $x_n \rightarrow 1/(1-p)$, $n \rightarrow \infty$. %[БШ]

    %\item Пусть $c \in \mathbb{Z}$, $p \nmid c$. Докажите, что последовательность $(c^{p^n})$ сходится в $\mathbb{Q}_p$, при этом для $\gamma = \lim c^{p^n}$ имеем $\gamma \equiv c \pmod{p}$, $\gamma^{p-1}=1$. Используя этот результат, докажите что в $\mathbb{Q}_p[t]$ многочлен $t^{p-1}-1$ раскладывается на линейные множители.
    
    \item Докажите, что для $0 \neq \xi \in \mathbb{Q}_p \cap \mathbb{Q}$ представление $\xi = \sum_{n=0}^{\infty} a_n p^n$, $0 \leqslant a_n \leqslant p-1$ имеет периодические коэффициенты (начиная с некоторого номера $k_0$, т.е. $\exists m: \forall k \ge k_0\ a_{m+k} = a_{k}$). Обратно всякий такой ряд представляет рациональное число.
    
\end{enumerate}

\noindent\textbf{SageMath}

\begin{itemize}[topsep=0pt]
    \item Исследуйте основные функции SageMath связанные с арифметикой $p$-адических чисел. Определение кольца и поле $p$-адических чисел: \texttt{Zp(n)}, \texttt{Qp(n)}. Рассмотрите примеры уравнения $x^2=m$ (используйте функцию \texttt{sqrt()}).
        
\end{itemize}

\noindent\textbf{Темы для самостоятельного изучения}
\begin{itemize}[topsep=0pt]
    \item Элементы $p$-адического анализа: последовательности, ряды, дифференцирование, интегрирование. [Gouv], \S\S 5.1--5.4.
\end{itemize}

