\noindent\textbf{Упражнения и задачи}
\begin{enumerate}[topsep=0pt]
    \item Пусть $\mathcal{D}$ --- кольцо целых квадратичного поля $\mathbb{Q}(\sqrt{d})$. Докажите, что $\mathcal{D}=\mathbb{Z}[\sqrt{d}]$ при $d\equiv 2,3\, (4)$ и $\mathcal{D}=\mathbb{Z}\left[\frac{-1+\sqrt{d}}{2}\right]$ при $d\equiv 1\, (4)$. %[IR]
    \item Пусть $\zeta$ --- примитивные корень степени $m$ из единицы. Докажите, что \\ $\Delta(1,\zeta,\dots,\zeta^{\varphi(m)-1}) | m^{\varphi(m)}$.  %[IR]
    \item Докажите, что числовое поле нечетной степени не может содержать примитивных корней из единицы степени $n>2$ %[IR Ex 13.1]
    \item Пусть $K$ --- вещественное квадратичное поле (т.е. $K\subset \mathbb{R}$). Докажите, что если $\exists \alpha \in K$: $\N(\alpha)=-1$, то простые $p\equiv 3\,(4)$ не разветвляются. %[IR Ex 13.2]
    \item Пусть $K$ --- вещественное квадратичное поле такое что $\zeta_n\in K$ для некоторого $n\geqslant 3$. Докажите, что $\forall\alpha\in F^*$ $\N(\alpha) > 0$. %[IR Ex 13.3]
    \item Докажите, что квадратичное поле $K$ не может одновременно содержать $\sqrt{p}$, $\sqrt{q}$ для двух различных простых $p$, $q$. %[IR Ex 13.5]
    \item Пусть $K$ --- вещественное квадратичное поле. Докажите, что $\forall M > 0$ $\exists\beta\in\mathcal{D}_K$: $|1-\beta|>M$, а также что $\forall \varepsilon > 0$ $\exists\alpha\in\mathcal{D}_K$: $|1-\alpha|<\varepsilon$. %[IR Ex 13.7]
    \item Пусть $p>2$ --- простое, $\zeta=\zeta_p$, $K=\mathbb{Q}(\zeta_p)$. Докажите следующие свойства:
    \begin{itemize}[topsep=0pt,noitemsep]
        \item $\N_{K/\mathbb{Q}}(1+\zeta)=1$;
        \item $A=\prod_{(s/p)=1} (1+\zeta^s)\in\mathbb{Q}(\sqrt{p})$;
        \item если $p\equiv 1\, (4)$, то $A=\frac{1}{2}(t+u\sqrt{p})$, где $t\equiv u\,(2)$;
        \item $\left(\dfrac{t^2-pu^2}{4}\right)^{\frac{p-1}{2}}=1$; $t^2-pu^2=\pm 4$;
        \item $A>0$.
        %\item TODO (g)
    \end{itemize} %[IR Ex 13.9]
    \item Для каких $d$ квадратичное поле $\mathbb{Q}(\sqrt{d})$ имеет базис вида $\alpha, \alpha'$? %[IR Ex 13.10]
    \item Пусть $\zeta=e^{2\pi i / 5}$, $K=\mathbb{Q}(\zeta)$. Покажите, что $-(\zeta^3+\zeta^2)\in\mathcal{U}_{\mathcal{D}_K}$. %[IR Ex 13.11]
    \item Пусть $K=\mathbb{Q}(\zeta_p)$. Покажите, что $\frac{\sin(\pi j / p)}{\sin(\pi / p)} \in \mathcal{U}_{\mathcal{D}_K}$. %[IR Ex 13.12]
    \item Пусть $p\equiv 1\,(4)$, $K=\mathbb{Q}(\zeta_p)$. Докажите, что группа единиц $\mathcal{U}_{\mathcal{D}_K}$ бесконечна %[IR Ex 13.13]
    \item Докажите, что всякое квадратичное поле содержится в некотором круговом поле. %[IR Ex 13.19]
\end{enumerate}

%\noindent\textbf{SageMath}
%\begin{itemize}[topsep=0pt]
%    \item 
%\end{itemize}

\noindent\textbf{Темы для самостоятельного изучения}
\begin{itemize}[topsep=0pt]
    \item Арифметика кольца целых кругового поля, приложения к доказательству квадратичного закона взаимности, [IR], \S\S 13.2--13.3.
    \item Порядки числовых полей, [БШ]; [Cox].
\end{itemize}