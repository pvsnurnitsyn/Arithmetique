\noindent\textbf{Упражнения и задачи}

\begin{enumerate}[topsep=0pt]
    \item Пусть $G$ --- конечная абелева группа, $H$ --- собственная подгруппа, $g\in G$, $g \not\in H$. Докажите, что существует характер $\chi$ группы $G$ такой что $\chi(g)\neq 1$ и $\forall h\in H$ $\chi(h)=1$. %[LN Ex 5.1]
    \item Пусть $G$ --- конечная абелева группа, $\widehat{G}$ --- группа характеров, $H < G$ ---  подгруппа, $A < \widehat{G}$ --- аннигилятор $H$: $A=\{\chi\in\widehat{G}: \forall h\in H\ \chi(h)=1 \}$. Докажите, что $A\cong G/H$ и что $H\cong \widehat{G}/A$. %[LN Ex 5.2]
    \item Пусть $G = G_1\times \cdots \times G_k$ --- прямое произведение конечных абелевых групп (множество $k$-кортежей с операцией $(g_1,\dots,g_k) (h_1,\dots,h_k) = (g_1 h_1, \dots, g_k h_k)$). Докажите, что $\widehat{G}\cong \widehat{G_1}\times \cdots \times \widehat{G_k}$. %[LN Ex 5.4]
    \item Основная теорема о стуктуре конечных абелевых групп утверждает, что каждая такая группа изоморфна прямому произведению конечного числа циклических групп. Выведете из этой теоремы, что если $G$ --- конечная абелева группа, то $\widehat{G}\cong G$. %[LN Ex 5.5]
    \item Пусть $G$ -- конечная абелева группа, $m>0$ --- целое. Докажите, что $g\in G$ является $m$-ой степенью в $G$ $\iff$ $\forall$ характера порядка $m$ выполняется $\chi(g)=1$. %???
    \item Покажите, что для аддитивных характеров $\psi_a$, $\psi_b$ поля $\mathbb{F}_q$ выполняется $\psi_a \psi_b = \psi_{a+b}$, и что из этого следует изоморфизм аддитивной группы $\mathbb{F}_q$ группе аддитивных характеров поля. %[LN Ex 5.6]
    \item Докажите, что для аддитивного характера $\psi=\psi_1$ поля $\mathbb{F}_q/\mathbb{F}_p$ для всех $\alpha\in\mathbb{F}_q$, $j\in\mathbb{Z}_+$ справедливо $\psi_1(\alpha^{p^j})=\psi_1(\alpha)$. %[LN Ex 5.7]
    \item Пусть $\chi'$ --- мультипликативный характер $\mathbb{F}_{q^s}$ порядка $m$, $\chi$ --- ограничение $\chi'$ на $\mathbb{F}_q$. Докажите, что $\chi$ --- мультипликативный характер $\mathbb{F}_q$ порядка $m/(m,(q^s-1)/(q-1))$. %[LN Ex 5.8]
    \item Пусть $\chi$ --- мультипликативный характер $\mathbb{F}_q$ порядка, $\chi'$ --- продолжение $\chi$ на $\mathbb{F}_{q^s}$. Докажите, что $\chi'(a)=\chi(q)^s$ $\forall a \in\mathbb{F}_q^{*}$. %[LN Ex 5.10]
    \item Пусть $p\neq 2$, $ab\not\equiv 0\,(p)$, $\chi$ --- квадратичный характер $\mathbb{F}_p^*$. Докажите, что
    \begin{itemize}[topsep=0pt]
        \item $G(\chi,\psi_a)G(\chi,\psi_b) = \left(\dfrac{-ab}{p}\right) p$; %[БШ стр 26 упр 5]
        \item $\sum_a G(\chi,\psi_a)=0$. %[БШ стр 26 упр 6]
    \end{itemize}
    \item Пусть $p>2$, $G = \sum_{x=0}^{p-1} e^{2\pi i x^2/p}$ --- сумма Гаусса для квадратичного характера, $A = (a_{st})_{0\leqslant s,t \leqslant p-1}$ --- $p\times p$ матрица с элементами $a_{st}=e^{2\pi i st/p}$. Докажите, что:
    \begin{itemize}[topsep=0pt]
        \item если $\lambda_0,\dots,\lambda_{p-1}$ – характеристические числа матрицы $A$, то $\sum_{k=0}^{p-1} \lambda_k = G$;
        \item характеристический многочлен матрицы $A^2$ имеет вид: $(t-p)^{(p+1)/2} (t+p)^{(p-1)/2}$;
        \item для определителя матрицы $A$ справедливо $\det A = i^{p(p-1)/2}p^{p/2}$.
    \end{itemize} %???
    \item Пусть $q=p^n, p>2$. Определим аналог символа Лежандра для $\mathbb{F}_q$: $\left(\frac{\alpha}{q}\right)=1$, если $\alpha$ --- квадрат в $\mathbb{F}_q$; $\left(\frac{\alpha}{q}\right)=1$, если $\alpha$ не является квадратом в $\mathbb{F}_q$; $\left(\frac{0}{q}\right)=0$. Докажите следующие свойства этого символа: 
    \begin{itemize} [topsep=0pt]
        \item $\left(\frac{\alpha\beta}{q}\right) = \left(\frac{\alpha}{q}\right)\left(\frac{\beta}{q}\right)$, $\alpha, \beta \in \mathbb{F}_q$; 
        \item $\sum_{\alpha\in\mathbb{F}_q} \left(\frac{\alpha}{q}\right) = 0$; 
        \item $\left(\frac{\alpha}{q}\right)=\left(\frac{\N_{\mathbb{F}_q/\mathbb{F}_p}(\alpha)}{p}\right)$ --- обычный символ Лежандра $\pmod p$.
    \end{itemize} %[Степ Задача I.2.9]
    \item Докажите свойства обобщенных сумм Гаусса для конечного подя $\mathbb{F}_q/\mathbb{F}_p$:
    \begin{itemize}[topsep=0pt]
        \item $G(\chi,\psi_{ab})=\chi(a) G(\chi, \psi_b)$, $a\in\mathbb{F}_q^*$, $b\in \mathbb{F}_q$;
        \item $G(\chi,\bar{\psi}) = \chi(-1) G(\chi,\psi)$;
        \item $G(\bar{\chi},\psi) = \chi(-1) \overline{G(\chi,\psi)}$;
        \item $G(\chi,\psi)G(\bar{\chi},\psi)=\chi(-1) q$, $\chi\neq\chi_0$, $\psi\neq\psi_0$;
        \item $G(\chi^p,\psi_b)=G(\chi,\psi_{\sigma(b)})$, $b\in\mathbb{F}_q$, $\sigma$ --- автоморфизм Фробениуса.
    \end{itemize} %[LN Th 5.12]
    \item Пусть $f:\mathbb{F}_q \rightarrow \mathbb{C}$, $\hat{f} = \frac{1}{q}\sum_{t\in\mathbb{F}_q} f(t)\overline{\psi(st)}$ --- конечное преобрахование Фурье. Докажите, что $f(t)=\sum_{s\in\mathbb{F}_q} \hat{f}(s) \psi(st)$. %[IR Ex 10.23]
\end{enumerate}

\noindent\textbf{SageMath}
\begin{itemize}[topsep=0pt]
    \item Исследуйте основные функции SageMath связанные с группой характеров конечных абелевых групп:
    \begin{itemize}[noitemsep,topsep=0pt]
        \item \texttt{character\_table()}.
     \end{itemize}
\end{itemize}

\noindent\textbf{Темы для самостоятельного изучения}
\begin{itemize}[topsep=0pt]
    \item Докажательство квадратичного закон взаимности через суммы Гаусса. [IR], \S 7.3; [LN], \S 5.2.
\end{itemize}
