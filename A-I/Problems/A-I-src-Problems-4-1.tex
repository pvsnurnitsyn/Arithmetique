\noindent\textbf{Упражнения и задачи}
\begin{enumerate}[topsep=0pt]

    \item Пусть $R$ --- евклидово кольцо (с нормой $N(\cdot)$). Докажите, что $u$ --- единица $R$ $\iff$ $N(u)=1$.
    
    \item Пусть $R$ --- кольцо. Докажите следующие утверждения (свойства делимости на языке идеалов):
    \begin{itemize}[noitemsep,topsep=0pt]
        \item $a|b$ $\Leftrightarrow$ $(b) \subseteq (a)$;
        \item $u$ --- единица $\Leftrightarrow$ $(u)=R$;
        \item $a,b$ --- ассоциированы $\Leftrightarrow$ $(a)=(b)$;
        \item $p$ --- простой элемент $\Leftrightarrow$ $ab \in (p)$ $\Rightarrow$ $a \in (p)$ или $b \in (p)$;
        \item $p$ --- неприводимый элемент $\Leftrightarrow$ $(p) \subseteq (a)$ $\Rightarrow$ $(a) = R$ или $(a) = (p)$.
    \end{itemize}

    \item Пусть $R$ --- кольцо главных идеалов, $a,b \in R$ $d$ --- НОД $a,b$. Докажите, что $\exists d \in R: (d)=(a,b)$. 
    
    \item Пусть $R$ --- кольцо главных идеалов. Докажите, что $p\in R$ --- неприводимый элемент $\iff$ $p$ --- простой.
    
    \item Докажите свойство показателя в кольце главных идеалов: если $p$ --- неприводимый элемент, $a,b \in R^*$, то $\nu_p(ab) = \nu_p(a)+\nu_p(b)$.
    
    \item Докажите теорему об однозначности разложения на простые множители в кольцах главных идеалов.
    
    \item Пусть $\pi\in\mathbb{Z}[i]$ --- простой элемент, $\nu_\pi(\alpha)$ --- соответствующий показатель, $|\alpha|_\pi = (\N p)^{-\nu_\pi(\alpha)}$ --- метрика заданная на $\mathbb{Z}[i]$ и $\mathbb{Q}(i)$. Опишите ограничение этой метрики на $\mathbb{Q}$. %[Gouv]
    
    \item Докажите, что $\mathbb{Z}[\omega]$ --- евклидово кольцо. Найдите единицы $\mathbb{Z}[\omega]$. %[IR]
    
    \item Докажите, что для функций определенных в лекции выполняется $d(n_1)=d_1(n)-d_3(n)$. %[DSV]
    
    \item Докажите оценку для числа представлений в виде суммы двух квадратов: $r_2(n) = \mathcal{O}_\varepsilon (n^\varepsilon)$. %[DSV]
  
\end{enumerate}

\noindent\textbf{SageMath}
\begin{itemize}[topsep=0pt]
    \item Рассмотрите примеры арифметики кольца гауссовых чисел: \texttt{ZZ[I]}, исследуйте базовые функции такие как \texttt{gcd(), xgcd(), factor(), ...}.
    \item Исследуйте функции для нахождения разложений целых чисел в виде суммы двух и четырёх квадратов, например, библиотека \texttt{sum\_of\_squares}.
\end{itemize}

\noindent\textbf{Темы для самостоятельного изучения}
\begin{itemize}[topsep=0pt]
    \item Арифметика кольца чисел Эйзенштейна $\mathbb{Z}[\omega]$, [IR], \S\S 9.1--9.2.
    \item Алгебра кватернионов, число представлений суммой четырёх квадратов, [DSV], \S\S 2.5--2.6.
\end{itemize}