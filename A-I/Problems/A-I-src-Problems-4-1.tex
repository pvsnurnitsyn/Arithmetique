
\noindent\textbf{Упражнения и задачи}
\begin{enumerate}[topsep=0pt]

    \item Пусть $R$ --- евклидово кольцо (с нормой $N(\cdot)$). Докажите, что $u$ --- единица $R$ $\Leftrightarrow$ $N(u)=1$.
    
    \item Пусть $R$ --- кольцо. Докажите следующие утверждения (свойства делимости на языке идеалов):
    \begin{itemize}[noitemsep,topsep=0pt]
        \item $a|b$ $\Leftrightarrow$ $(b) \subseteq (a)$;
        \item $u$ --- единица $\Leftrightarrow$ $(u)=R$;
        \item $a,b$ --- ассоциированы $\Leftrightarrow$ $(a)=(b)$;
        \item $p$ --- простой элемент $\Leftrightarrow$ $ab \in (p)$ $\Rightarrow$ $a \in (p)$ или $b \in (p)$;
        \item $p$ --- неприводимый элемент $\Leftrightarrow$ $(p) \subseteq (a)$ $\Rightarrow$ $(a) = R$ или $(a) = (p)$.
    \end{itemize}

    \item Пусть $R$ --- кольцо главных идеалов, $a,b \in R$ $d$ --- НОД $a,b$. Докажите, что $\exists d \in R: (d)=(a,b)$. 
    
    \item %стр ...
    \item %стр - + теорема
    \item %стр --..
    \item %стр --...
    \item %стр ---..
    
  
\end{enumerate}

%\noindent\textbf{SageMath}


%\noindent\textbf{Темы для самостоятельного изучения}
