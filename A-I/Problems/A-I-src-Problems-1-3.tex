\noindent\textbf{Упражнения и задачи}

\begin{enumerate}[topsep=0pt]
    \item Путь $p$ --- простое, докажите, что $p|\binom{p}{k}$ для $1 \leqslant k < p$.
    \item Путь $p>2$ --- простое, $l\geqslant 2$. Докажите, что $\forall a \in \mathbb{Z}$ $(1+ap)^{p^{l-2}} \equiv 1+ap^{l-1}\ (p^l)$.
    \item Пусть $p>2$ --- простое, $g$ --- первообразный корень $\Mod p^n$. Докажите, что тогда $g$ --- первообразный корень $\Mod p$. %[IR, p48, Ex3]
    \item Пусть $p$ --- простое, $p \equiv 1 (\Mod 4)$. Докажите что $g$ --- первообразный корень $(\Mod p)$ $\Leftrightarrow$ $-g$ --- первообразный корень $(\Mod p)$. %[IR, p48, Ex4]
    \item Пусть p --- простое, $p \equiv 3 (\Mod 4)$. Докажите что $g$ --- первообразный корень $(\Mod p)$ $\Leftrightarrow$ $-g$ имеет порядок $(p-1)/2$. %[IR, p48, Ex5]
    %\item Докажите, что $3$ --- первообразный корень простого числа вида $p=2^{2^n}+1$. %[IR, p48, Ex6]. TODO перенести в следующий листок
    \item Пусть $p>2$ --- простое. Докажите, что $g$ --- первообразный корень $\Mod p$ $\Leftrightarrow$\\ $g^{(p-1)/q}\not\equiv 1 (p)$ для всех простых делителей $q\ |\ p-1$. %[IR, p48, Ex8]
    \item Пусть $p>2$ --- простое. Докажите, что ${\prod\limits_{g}}'g \equiv (-1)^{\varphi(p-1)}\ (p)$, где $\prod'$ --- произведение по всем $0\leqslant g \leqslant p-1$, $g$ --- первообразный корень $\Mod p$. %[IR, p48, Ex9]
    \item Пусть $g$ --- первообразный корень $\Mod p$, $d|(p-1)$. Докажите, что $g^{(p-1)/d}$ имеет порядок $d$, а также что $a$ является $d$-ой степенью $\Leftrightarrow$ $a \equiv g^{kd} (p)$ для некоторого $k$. %[IR, p48, Ex21]
    \item Пусть $G$ --- конечная циклическая группа порядка $n$, $g$ --- образующая $G$. Докажите, что все образующие имеют вид $g^k$, $(k,n)=1$. %[IR, p48, Ex13]
    \item Пусть $G$ --- конечная абелева группа, $a,b$ – элементы порядков $m,n$ соответственно. Докажите, что если $(m,n)=1$ то порядок элемента $ab$ равен $mn$. %[IR, p48, Ex14]

\end{enumerate}

\noindent\textbf{SageMath}
\begin{itemize}[topsep=0pt]

    \item Исследуйте основные классы и функции SageMath релевантные материалу лекции:
    \begin{itemize}[noitemsep,topsep=0pt]
        %\item Корни многочлена: \texttt{roots()};
        \item Первообразные корни: \texttt{primitive\_root(), is\_primitive\_root()};
        \item Образующие группы единиц: \texttt{unit\_gens()};
        \item Порядок элемента в кольце вычетов: \texttt{multiplicative\_order()};
        \item Индекс и дискретный логарифм в кольце вычетов: \texttt{log()};
        \item Абелевы группы \texttt{AbelianGroup()}, образующие и порядки \texttt{gens(), gens\_orders()}.
    
    \end{itemize}

    \item Пусть $a$ --- наименьшее положительное число являющееся первообразным корнем $\Mod p$. Постройте частотную таблицу для $a$, что можно заметить?
    \item Пусть $a\neq -1$ и не является полным квадратом. Постройте примеры последовательностей простых, для которых $a$ является первообразным корнем (согласно гипотезе Артина таких простых бесконечно много, также можно оценить плотность их распределения).
    
\end{itemize}

\noindent\textbf{Темы для самостоятельного изучения}
\begin{itemize}[topsep=0pt]
    \item Структура группы единиц $U(\mathbb{Z}/2^l\mathbb{Z})$ ([IR, глава 4], [Вин, глава 6]).
    \item Критерии разрешимости сравнения $x^n\equiv a (\Mod n)$ ([IR, глава 4]).
    \item Основы криптографии с открытым ключем: протокол Диффи--Хеллмана и RSA, [Stein-ent], глава 3.
\end{itemize}

