\documentclass[a4paper, 12pt]{article}

\usepackage[T1,T2A]{fontenc}
\usepackage[utf8]{inputenc}
\usepackage[english, russian]{babel}
\usepackage{fullpage}
\usepackage{amssymb}
\usepackage{amsmath}
\usepackage{enumitem}
\usepackage{indentfirst}
\usepackage{titlesec}
\usepackage{hyperref}

\hypersetup{
     colorlinks,
     citecolor=black,
     filecolor=black,
     linkcolor=black,
     urlcolor=black
 }

\setcounter{tocdepth}{2}
\setcounter{secnumdepth}{2}

%reduce spaces before eqs
\expandafter\def\expandafter\normalsize\expandafter{%
    \normalsize%
    \setlength\abovedisplayskip{2pt}%
    \setlength\belowdisplayskip{2pt}%
    \setlength\abovedisplayshortskip{-2pt}%
    \setlength\belowdisplayshortskip{2pt}%
}

%declare and redeclare math operators

\let\iff\relax
\DeclareMathOperator{\iff}{\Leftrightarrow} 
\let\implies\relax
\DeclareMathOperator{\implies}{\Rightarrow} 

\DeclareMathOperator{\ZZ}{\mathbb{Z}}
\DeclareMathOperator{\QQ}{\mathbb{Q}}
\DeclareMathOperator{\RR}{\mathbb{R}}
\DeclareMathOperator{\CC}{\mathbb{C}}
\DeclareMathOperator{\HH}{\mathbb{H}}
\DeclareMathOperator{\PP}{\mathbb{P}}
\let\AA\relax
\DeclareMathOperator{\AA}{\mathbb{A}}
\DeclareMathOperator{\FF}{\mathbb{F}}

\DeclareMathOperator{\GL2}{GL_2}
\DeclareMathOperator{\SL2}{SL_2}

\let\Re\relax
\DeclareMathOperator{\Re}{Re} 
\let\Im\relax
\DeclareMathOperator{\Im}{Im} 
\DeclareMathOperator{\Res}{Res} 
%\DeclareMathOperator{\dd}{d} 

\let\div\relax
\DeclareMathOperator{\div}{div} 

%\DeclareMathOperator{\ord}{ord}
%\DeclareMathOperator{\chr}{char}
%\DeclareMathOperator{\Mod}{mod}
%\DeclareMathOperator{\Gal}{Gal}

\newcommand*\smat[4]{\left(\begin{smallmatrix}#1&#2\\#3&#4\end{smallmatrix}\right)}


\title{Спецкурс Арифметика III:\\ Римановы поверхности и алгебраические кривые}
%\author{П.В. Снурницын}
\date{}

\begin{document}

\maketitle

\tableofcontents

\section{О курсе}

Осенний семестр, 15--16 недель

Курс состоит из двух частей. В первой части (лекции 1--10) излагается теория римановых поверхностей с уклоном в приложения для современной теории чисел. Развивается необходимый для доказательства теоремы Римана--Роха аппарат, сама теорема Римана-Роха доказывается используя двойственность Серра. Общие результаты сопровождаются примерами для случаев римановой сферы, комплексного тора (эллиптической кривой), а также римановых поверхностей, возникающих при изучении пространств модулярных форм (модулярных кривых). В частности затрагиваются вопросы компактификации модулярных кривых, а также оценка размерности пространств модулярных форм для конгруэнц-подгрупп (эти вопросы поднимались в конце курса Арифметика II: Решётки и формы).

Вторая часть (лекции 11--15) посвящена общему случаю алгебраических функциональных полей (над алгебраически замкнутым полем). Теорема Римана--Роха доказывается в более общем случае. Вводится понятие дзета-функции алгебраической кривой. Рассматривается идея доказательства теоремы Хассе--Вейля об оценке числа точек алгебраической кривой над конечным полем.

По сути курс является введением в алгебраическую и арифметическую геометрию (по крайней мере в случай плоских алгебраических кривых). Изложение через римановы поверхности и случай над полем $\CC$ позволяет развить аналитическую и геометрическую интуицию стояющую за понятиями дифференциалов, дивизоров, линейных пространств и теоремой Римана-Роха, и используя эту интуицию перейти к более общему случаю алгебраических функциональных полей.

\section{Содержание лекций и примеры задач}

%Тема 1
\subsection{Римановы поверхности: определения и примеры}

{\bf Аннотация:}
Повторение определений и необходимых фактов из топологии: топологические пространства, связность, хаусдорфовость, род, число Эйлера. Определения комплексных карт, атласов, структуры двумерного многообразия и римановой поверхности. Примеры: комплексная плоскость, риманова сфера, комплексный тор, афинные кривые, проективная прямая, проективная плоскость, проективные кривые.

{\bf Источники:} \cite{Mir}, глава I.

{\bf Примеры задач:}
\begin{enumerate}[noitemsep,topsep=0pt]
    \item Пусть $X$ --- топологическое пространство, $\phi:U\rightarrow V$ --- комплексная карта на $X$, $\psi: V \rightarrow W$ --- взаимно-однозначное голоморфное отображение на подмножествах $\mathbb{C}$. Докажите, что $\psi \circ \phi: U \rightarrow W$ также является комплексной картой на $X$ и что $\psi \circ \phi$ совместима с любой картой на $X$, с которой совместима $\phi$. %[Mir I.1.B]
    \item Докажите, что определенная в лекции эквивалентность комплексных атласов действительно является отношением эквивалентности. %[Mir I.1.H]
    \item Проверьте, что отображение $\PP^1 \rightarrow S^2$ проективной прямой над $\CC$ в сферу единичного радиуса в $\RR^3$, заданное 
    $$
    [z:w] \mapsto (2\Re(w\overline{z}), 2\Im(w\overline{z}), |w|^2 - |z|^2) / |w|^2 + |z|^2,
    $$ 
    является гомеоморфизмом. (Поэтому проективная прямая является компактной римановой поверхностью рода 0). %[Mir I.2.C]
    \item Докажите, что группа сложения точек комплексного тора $X$ является делимой, то есть, что $\forall p\in X$ $\forall n\in \mathbb{Z}_+$ $\exists q \in X: n\cdot q = p$. %[Mir I.2.F]
    \item Пусть многочлен от двух переменных имеет вид $f(z,w)=w^2 - h(z)$. 
    \begin{itemize}[noitemsep,topsep=0pt]
        \item Докажите, что $f(z,w)$ является неприводимым $\iff$ $h(z)$ не является точным квадратом.
        \item Докажите, что $f(z,w)$ является невырожденным $\iff$ все корни $h(z)$ различны. 
    \end{itemize} %[Mir I.2.G]   
    \item Пусть $f(z,w)$ --- многочленом второй степени. Афинная кривая $X$ заданная $f(z,w) = 0$ называется афинной коникой.  
    \begin{itemize}[noitemsep,topsep=0pt]
        \item Докажите, что если $f(z,w)$ --- вырожденный, то $f(z,w)$ раскладывается в произведение двух линейных множителей. Что в этом случае можно сказать про $X$?
        \item Приведите примеры гладких афинных коник.
    \end{itemize}%[Mir I.2.H]
    \item Пусть $F(x,y,z)$ --- однородный многочлен степени $d$. Докажите, что $F$ невырожденный $\iff$ в каждой афинной карте $F$ задаёт гладкую афинную кривую. %[Mir Lemma 3.5 p15]
    \item Докажите, что любые две проективные прямые в $\PP^2$ пересекаются в единственной точке. %[Mir I.3.E]
\end{enumerate}


%Тема 2
\subsection{Функции на римановых поверхностях}

{\bf Аннотация:} Голоморфные и мероморыне функции, ряды Лорана, порядки нулей и полюсов. Повторение необходимых результатов из комплексного анализа от одной переменной. Кольцо голоморфных функций и поле мероморфных функций.
Мероморфные функции на римановой сфере, равенство числа нулей и полюсов с учетом кратностей. Мероморфные функции на торе, проективной прямой и гладкой алгебраической кривой (приложение теоремы Гильберта о нулях).

{\bf Источники:} \cite{Mir} \S II.1--II.2.

{\bf Примеры задач:}
\begin{enumerate}[noitemsep,topsep=0pt]
    \item Пусть $f$ --- комплексно-значная функция определенная на римановой сфере $\CC_\infty$ в окрестности $\infty$. Докажите, что $f$ голоморфна в $\infty$ $\iff$ $f(1/z)$ голоморфна в $0$. %[Mir II.1.A (Exmp 1.6)]
    \item Пусть $p(z,w), q(z,w) \in \CC[z,w]$ --- однородные многочлены одинаковой степени, $q(z_0,w_0) \neq 0$. Покажите, что $f([z:w]) = p(z,w)/q(z,w)$ --- корректно определенная на $\PP^1$ функция голоморфная функция в окрестности $[z_0:w_0]$. %[Mir II.1.A (Exmp 1.7)]
    \item Пусть $X \subset \PP^2$ --- проективная кривая заданная невырожденным многочленом $F(x,y,z)=0$, $F(x,y,z), G(x,y,z) \in \CC[x,y,z]$ --- однородные многочлены одинаковой степени, причем $H$ не равен тождественно нулю на $X$. Покажите, что $G(x, y, z) / H(x, y, z)$ --- мероморфная функция на $X$. %[Mir II.1.B (Exmp 1.22)]
    \item Пусть $U$ --- окрестность точки $p \in X$ $f,g \in \mathcal{M}(U)$. Докажите следующие свойства порядка $\nu_p$: 
    \begin{itemize}[noitemsep,topsep=0pt]
        \item $\nu_p(fg) = \nu_p(f) + \nu_p(g)$;
        \item $\nu_p(1/f) = -\nu_p(f)$, $\nu_p(f/g) = \nu_p(f) - \nu_p(g)$;
        \item $\nu_p(f\pm g) \geqslant \min(\nu_p(f), \nu_p(g))$.
    \end{itemize} %[Mir II.1.F (Lemma 1.29)]
    \item Докажите, что ряд определяющий тета-функцию $\theta_\tau(z) = \sum_{n\in\ZZ} e^{\pi i (n^2 \tau + 2nz)}$ сходится абсолютно и равномерно на компактных подмножествах $\CC$. %[Mir II.2.B]
    \item Докажите, что $z_0$ --- нуль $\theta_\tau(z)$ $\iff$ $\forall m,n\in\ZZ$ точка $z_0+m+n\tau$ является нулём $\theta_\tau(z)$. %[Mir II.2.E]
    \item Докажите, что $(1/2)+(\tau/2) + m + n\tau, m,n\in\ZZ$ --- единственные нули $\theta_\tau$, причём эти нули простые.  %[Mir II.2.F]
    \item Пусть $L$ --- решётка в $\CC$, $X=\CC/L$ --- комплексный тор, $\pi: \CC \rightarrow X$ --- естественная проекция, и пусть заданы два набора $\{p_i\}_{i=1}^d, \{q_i\}_{i=1}^d \subset X$. Покажите, что существуют два набора $\{x_i\}_{i=1}^d, \{y_i\}_{i=1}^d \subset \CC$: $\pi(x_i) = p_i, \pi(y_i)=q_i$, $\sum_i x_i = \sum_i y_i$ $\iff$ $\sum_i p_i = \sum_i q_i$, где последнее суммирование выполняется в групповом законе тора.  %[Mir II.2.G]
    
\end{enumerate}


%Тема 3
\subsection{Теорема Гильберта о нулях}

{\bf Аннотация:} Повторение определений и необходимых фактов из алгебры: идеалы, модули, кольца главных идеалов, максимальный идеал, радикальный идеал. Идеал множества точек кривой. Неприводимые компоненты алгебраических множеств. Лемма Зарисского, теорема Гильберта о базисе, теорема Гильберта о нулях.

{\bf Источники:} \cite{Fult} глава 1; \cite{Mir} \S III.5.

{\bf Примеры задач:}
\begin{enumerate}[noitemsep,topsep=0pt]
    
    \item Пусть $k$ --- произвольное поле. Докажите, что алгебраические подмножества $\AA^1(k)$ исчерпываются конечными подмножествами и самим $\AA^1(k)$. %[Fult 1.8]
    \item Пусть $k$ --- произвольное поле. Докажите следующие свойства алгебраических множеств в $\AA^n(k)$ и их идеалов: 
    \begin{itemize}[noitemsep,topsep=0pt]
        \item $X \subset Y \implies I(X) \supset I(Y)$;
        \item $I(\{a_1, \dots, a_n\}) = (x_1-a_1, \dots, x_n-a_n)$, $I(\emptyset) = k[x_1, \dots x_n]$, $I(\AA^n(k)) = (0)$ (при не конечном $k$);
        \item $I(V(S)) \supset S$, $V(I(X)) \supset X$, где $S \subset k[x_1, \dots x_n]$, $X \subset \AA^n(k)$;
        \item $V(I(V(S))) = V(S)$, $I(V(I(X))) = I(X)$, где $S$ и $X$ как выше;
        \item $\forall X \subset \AA^n(k)$ $I(X)$ --- радикальный идеал;
        \item $V = W \iff I(V)=I(W)$, где $V,W \subset \AA^n(k)$ --- алгебраические. 
        \item $V(I) = V(\sqrt{I})$, $\sqrt{I} \subset I(V(I))$, где $I$ --- идеал в $k[x_1, \dots x_n]$.
    \end{itemize}
    %[Fult свойства идеалов из раздела 1.3 + задача 1.16]
    \item Докажите, что $I(\{a_1, \dots, a_n\}) = (x_1-a_1, \dots, x_n-a_n)$ является максимальным идеалом в $k[x_1, \dots x_n]$. %[Fult 1.21]
    \item Докажите, что $I=(x^2+1) \subset \RR[x]$ --- радикальный идеал, но при этом $I$ не является идеалом никакого множества $\AA^1(\RR)$. %[Fult 1.19]
    \item Докажите, что $V(y-x^2)\subset \AA^2(\CC)$ неприводимо и $I(V(y-x^2)) = (y-x^2)$. %[Fult 1.25(a)]
    \item Разложите $V(y^4-x^2, y^4-x^2y^2+xy^2-x^3) \AA^2(\CC)$ на неприводимые компоненты. %[Fult 1.25(b)]
    %\item TODO [Fult 1.31]
    %\item TODO [Fult 1.33]
\end{enumerate}


%Тема 4
\subsection{Отображения римановых поверхностей}

{\bf Аннотация:} Голоморфные отображения римановых поверхностей, индуцирование гомоморфизмов колец голоморфных и полей мероморфных функций. Изоморфизм римановой сферы и проективной прямой. Мероморфные функции как голоморфные отображения на риманову сферу. Теорема о локальной нормальная форме, кратность отображения. Степень отображения. Теорема о сумме порядков мероморфных функций. Формула Гурвица. Точки ветвления для афинных и проективных кривых. Мероморфные функции на торе как отношения тета-функций. Изоморфизмы комплексных торов

{\bf Источники:} \cite{Mir} \S\S II.3--II.4, \S III.1

{\bf Примеры задач:}
\begin{enumerate}[noitemsep,topsep=0pt]
    \item Докажите следующие свойства голоморфных отображений:
    \begin{itemize}[noitemsep,topsep=0pt]
        \item Если $F:X\rightarrow Y$, $G:Y\rightarrow Z$ --- голоморфные отображения, то $G\circ F: X \rightarrow Z$ --- голоморфное отображение;
        \item Если $F:X\rightarrow Y$ --- голоморфное отображение, $g$ --- голоморфная функция, определенная на открытом подмножестве $W\subset Y$, то $g\circ F$ --- голоморфная функция, определенная на $F^{-1}(W)$;
        \item Если $F:X\rightarrow Y$ --- голоморфное отображение, $g$ --- мероморфная функция, определенная на открытом подмножестве $W\subset Y$, то $g\circ F$ --- мероморфная функция, определенная на $F^{-1}(W)$.
    \end{itemize} %[Mir, II.3.B (Lemma 3.5)]
    \item Покажите, что при изоморфизме между комплексной проективной прямой $\PP^1$ и римановой сферой $\CC_{\infty}$ точки $[z:1]$ соответствуют конечным точкам $z\in\CC$, а точка $[1:0]$ соответствует $\infty$. %[Mir, II.3.С]
    \item Пусть $f(z,w), g(z,w)\in\CC[z,w]$ --- ненулевые, однородные многочлены одинаковой степени, не имеющие общих множителей. Докажите, что отображение $F:\PP^1\rightarrow\PP^1:[z:w]\mapsto [f(z,w):g(z,w)]$ корректно определено и голоморфно. Что можно сказать про случай, когда $f$ и $g$ имеют общие множители? %[Mir, II.3.F]
    \item Пусть $A = \smat{a}{b}{c}{d} \in \GL2(\CC)$, докажите следующие свойства:
    \begin{itemize}[noitemsep,topsep=0pt]
        \item $F_A:\PP^1\rightarrow\PP^1:[z:w]\mapsto[az+b:cz+d]$ --- автоморфизм $\PP^1$, $F_{AB}=F_A\circ F_B$;
        \item При отождествлении $\PP^1$ с $\CC_\infty$ отображение $F_A$ соответствует преобразованию $z\mapsto (az+b)/(cz+d)$.
    \end{itemize} %[Mir, II.3.G-H]
    \item Пусть $X$ --- компактная риманова поверхность, $f$ --- мероморфная непостоянная функция на $X$. Докажите что $f$ имеет хотя бы один нуль и хотя бы один полюс. %[Mir, II.3.I]
    \item Обозначим через $L=L(\omega_1, \omega_2)\subset\CC$ решётку на комплексной плоскости с базисом $\omega_1, \omega_2 \in \CC$. Докажите следующие свойства:
    \begin{itemize}[noitemsep,topsep=0pt]
        \item Пусть $L\subseteq L'$, докажите, что естественная проекция $\CC/L\rightarrow\CC/L'$ голоморфно, и что голоморфное отображение $\CC/L'\rightarrow\CC/L$ существует $\iff$ $L=L'$;
        \item Пусть $L$ --- решётка в $\CC$, $\alpha \in \CC^*$. Покажите, что $\alpha L$ --- также решётка, и что отображение $\phi: \CC/L \rightarrow \CC/(\alpha L): z+L \mapsto \alpha z + \alpha L$ --- корректно определенное биголоморфное отображение.
        \item Покажите, что всякий тор $\CC/L$ изоморфен тору вида $\CC/L(1,\tau)$, $\tau \in \HH$.
    \end{itemize}% [Mir, II.3.K]
    %\item Равенство числа нулей и полюсов на торе с учетом кратностей. %[Mir, Lemma 3.14]
    \item Пусть $f$ --- непостоянная мероморфная функция на торе $X=\CC/L$. Докажите, что $\sum_p \nu_p(f) = 0$. %[Mir, II.4.B]
    \item Пусть $F:X\rightarrow Y$, $G:Y\rightarrow Z$ --- два непостоянных голоморфных отображения, $f$ --- мероморфная функция на $Y$, $p\in X$. Докажите, что $e_p(F\circ G) = e_p(F) e_p(G)$, $\nu_p(f\circ F) = e_p(F) \nu_{F(p)}(f)$. %[Mir, II.4.C]
    %\item TODO [Mir, II.4.G]
    \item Докажите, что всякая прямая в $\PP^2$ невырождена и изоморфна $\PP_1$. %[Mir, Lemma III.1.1]
    \item Докажите, что в $\PP^2$ всякая гладка кривая второго порядка (коника) изоморфна кривой вида $x^2+y^2+z^2=0$. (В частности в $\PP^2$ все гладкие коники изоморфны между собой). %[Mir, Cor III.1.4]
\end{enumerate}

%Тема 5
\subsection{Группы, действующие на римановых поверхностях}

{\bf Аннотация:} Действие группы, орбита, стабилизатор. Фактор-пространство как риманова поверхность, степень отображения факторизации. Теорема Гурвица о действии конечной группы. Действие полной модулярной группы и её конгруэнц подгрупп $\Gamma$ на верхней комплексной полуплоскости $\HH$. Структура римановой поверхности на $Y(\Gamma) = \HH/\Gamma$. Эллиптические и параболические точки. Компактификация $Y(\Gamma)$ в $X(\Gamma)$, род $X(\Gamma)$.

{\bf Источники:} \cite{Mir} \S III.3; \cite{DS} \S 2.3.1.

{\bf Примеры задач:}
\begin{enumerate}[noitemsep,topsep=0pt]
    \item Пусть $G$ --- конечная группа действующая на множестве $X$, $p\in X$. Докажите, что $|G\cdot p| |G_p| = |G|$. %[Mir III.3.A]
    \item Пусть $K$ --- ядро действия $G$ на $X$. Докажите, что $K$ --- нормальная подгруппа $G$, и что ядро действия $G/K$ на $X$ тривиально, а орбиты совпадают с орбитами действия $G$. %[Mir III.3.B]
    \item Пусть $G$ --- конечная подгруппа мультипликативной группы $\CC^*$ порядка $n$. Покажите что $G=\{e^{2\pi i/k}: 0\leqslant k \leqslant n\}$. %[Mir III.3.E]
    \item Покажите, что группа действий на римановой сферы $\CC_\infty$ порожденная двумя элементами $z\mapsto e^{2\pi i /r}$ и $z\mapsto 1/z$ есть диэдральная группа порядка $2r$. Докажите также, что действие этой группы голоморфно и эффективно. Определите точки ветвления и их индексы ветвления. %[Mir III.3.H]
    \item Докажите, что кривая определенная уравнением $xy^3+yz^3+zx^3=0$ (кривая Клейна) является гладкой проективной кривой. Покажите, что на этой кривой достигается граница теоремы Гурвица. %[Mir III.3.K]
    \item Пусть $\pi:\HH \rightarrow Y(\Gamma)=\HH/\Gamma: z\mapsto \Gamma z$ --- естественная проекция. Докажите, что для открытых множеств $U_1, U_2 \subset \HH$ справедливо $\pi(U_1)\cap\pi(U_2)=\emptyset$ $\iff$ $\Gamma U_1\cap U_2 = \emptyset$. %[DS Ex 2.1.2 (equiv (2.1))]
    \item Пусть $z_1, z_2\in\HH$. Докажите, что существуют окрестности $U_1$ и $U_2$ точек $z_1$ и $z_2$ обладающие следующим свойством: $\forall \gamma \in \SL2(\ZZ)\ \gamma(U_1)\cap U_2\neq\emptyset \implies \gamma(z_1)=z_2$. %[DS Prop 2.1.1]
    
\end{enumerate}

%Тема 6
\subsection{Дифференциальные формы и дивизоры}

{\bf Аннотация:} Голоморфные и мероморфные дифференциальные формы на римановых поверхностях. Интегрирование дифференциальных форм. Вычеты, теорема о сумме вычетов. Дивизоры функций, степень дивизора, главные дивизоры.
Дивизоры дифференциальных форм, канонические дивизор. Степень канонического дивизора на компактной римановой поверхности. Линейная эквивалентность дивизоров. Свойства дивизоров на римановой сфере и торе. Теорема Абеля для тора. Понятие степени гладкой проективной кривой, теорема Безу.

{\bf Источники:} \cite{Mir} глава IV, \S\S V.1--V.2.

{\bf Примеры задач:}
\begin{enumerate}[noitemsep,topsep=0pt]
    
    \item Пусть на римановой сфере $X=\CC_\infty$ заданы две карты с локальными координатами $z$ и $w=1/z$ и пусть $\omega\in \mathcal{M}^{(1)}(X)$. Докажите, что если $\omega=f(z)dz$ (в локальной координате $z$), то $f$ --- рациональная функция от $z$. Докажите также, что $\Omega^1(X)=\{0\}$. Какие точки являются нулями и полюсами форм $dz, dz/z$. %[Mir IV.1.A]
    \item Пусть $L$ --- решётка в $\CC$, $X=\CC/L$ --- тор, $\pi:\CC\rightarrow X$ --- естественная проекция. Покажите, что для формы $dz$ в каждой карте $X$ локальная формула корректно определена и задаёт голоморфную 1-форму на $X$, и что эта форма не имеет нулей. %[Mir IV.1.B]
    \item Пусть $X$ --- гладкая плоская афинная кривая заданная уравнением $f(u,v)=0$. Покажите, что $du$, $dv$ --- корректно определенные голоморфные 1-формы на $X$, также как и $p(u,v)du$, $p(u,v)dv$ для любого $p(u,v)\in\CC[u,v]$. Покажите что если $r(u,v)$ --- рациональная функция, то $r(u,v)du$, $r(u,v)dv$ --- корректно определенные мероморфные 1-формы. %[Mir IV.1.C]
    \item Пусть $X$ --- риманова поверхность определенная уравнением $y^2=h(x)$, где $h\in\CC[x]$, $\deg h=2g+1,2g+2$ (то есть $X$ --- гиперэллиптическая кривая, поверхность рода $g$). Покажите, что $dx/y$ --- голоморфная 1-форма при $g\geqslant 1$. Покажите также, что если $p(x)\in\CC[x]$, $\deg (p) \leqslant g-1$, то $p(x)dx/y$ --- голоморфная 1-форма. %[Mir IV.1.G]
    \item Пусть $X$ --- гиперэллиптическая кривая $y^2=x^5-x$, тогда $x,y \in \mathcal{M}(X)$. Определите $\div(x), \div(y)$. %[Mir V.1.A]
    \item Пусть $X=\CC/L$ --- тор. Покажите, что форма $dz$ --- корректно определённая голоморфная 1-форма всюду отличная от нуля. Что в этом случае можно сказать о главных и канонических дивизорах? %[Mir V.1.C]
    \item Пусть $X$ --- плоская проективная кривая $y^2 z=x^3-x z^2$. Определите дивизоры пересечений $X$ с прямыми $x=0$, $y=0$, $z=0$. %[Mir V.1.H]
    \item Докажите следующие свойства дивизоров на римановой сфере $X=\CC_\infty$
    \begin{itemize}[noitemsep,topsep=0pt]
        \item $D_1\sim D_2$ $\iff$ $\deg(D_1)=\deg(D_2)$; %[Mir V.2.A (Cor 2.6)]
        \item Если $\deg(D) \geqslant 0$, то $D\sim D_0$, $D_0\geqslant 0$. %[Mir V.2.B (Cor 2.7)]
    \end{itemize}
\end{enumerate}

%Тема 7
\subsection{Пространства функций дивизоров и линейные системы}

{\bf Аннотация:} Линейное пространство мероморфных функций $L(D)$ и полная линейная система $|D|$. Базовые свойства линейных пространств и линейных систем. Линейное пространство мероморфных форм $L^{(1)}(D)$, изоморфизм между простнствами $L$ и $L^{(1)}$. Линейные пространства $L(D)$ для случаев римановой сферы и тора. Оценка размерности $L(D)$. Голоморфные отображения римановых поверхностей в проективные пространства. Базовые точки линейных систем. Обратные образы (пуллбэки) дивизоров и форм. Гиперплоскостные дивизоры.

{\bf Источники:} \cite{Mir} \S\S V.3--V.4.

{\bf Примеры задач:}
\begin{enumerate}[noitemsep,topsep=0pt]
    \item Пусть $X$ --- компактна, $D$ --- дивизор на $X$, $\deg D = 0$. Докажите, что если $D \sim 0$, то $\dim L(D) = 1$, и что если $D \not\sim 0$, то $L(D) = \{0\}$. %[Mir V.3.C]
    \item Пусть $X$ --- компактна рода $g$, $\mathcal{M}(X)\neq \CC$, докажите, что если $\deg D < 2-2g$, то $L^{(1)}(D)=0$. %[Mir V.3.D]
    \item Пусть $X=\CC/L$ --- тор, $L=\ZZ+\ZZ \tau$, $\Im \tau >0$, $\pi:\CC\rightarrow X$ --- естественная проекция, $p_0=\pi(0)$. Докажите следующие свойства:
    \begin{itemize}[noitemsep,topsep=0pt]
        \item Пусть $n\in\ZZ$, $h\in L(n p_0)$. Тогда $\Res_{p_0}(h\,dz)=0$.
        \item Пусть $z$ --- локальная координата в окрестности $p_0$, $h(z)=\sum_{i=-n}^\infty c_i z^i$ --- разложение в ряд Лорана функции $h\in L(n p_0)$. Тогда если $\forall i\leqslant 0$ $c_i=0$, то $h$ тождественно равна $0$.
        \item Пусть $f\in L(2 p_0)$, тогда $\forall x \in X$ $f(x)=f(-x)$.
        \item $\exists! f\in L(2 p_0)$ такая что разложение в ряд Лорана имеет вид: $f(z)=z^{-2} + a_2z^2 + a_4 z^4 + \dots$.
        \item Пусть $g\in L(3 p_0)$, тогда $\forall x \in X$ $g(x)=-g(-x)$.
        \item $\exists! g\in L(3 p_0)$ такая что разложение в ряд Лорана имеет вид: $g(z)=z^{-3} + b_1 z + b_1 z^3 + \dots$.
        \item $\exists A,B\in\CC: g^2=f^3+Af+B$, где $f\in L(2 p_0), g\in L(3 p_0)$ определены как выше. При этом многочлен $w^3+A w + B$ не имеет кратных корней.
    \end{itemize}%[Mir V.3.F, some of the statements]
    \item Докажите, что $\forall f,g \in \mathcal{M}(X)$ $\exists$ дивизор $D$: $f,g\in L(D)$ %[Mir V.3.G]
    \item Пусть $X$ --- компакнтая, и пусть $D>0$ --- дивизор такой, что $\dim L(D)=1+\deg(D)$. Докажите, что $\exists p\in X$: $\dim L(p)=2$, и что $X$ изоморфна римановой сфере $\CC_\infty$. %[Mir V.3.H]
    \item Докажите, что на римановой сфере полная линейная система дивизор неотрицательной степени не содержит базовых точек. %[Mir V.4.B (Ex 4.10)]
    \item Докажите, что на комплексном торе полная линейная система дивизора степени $\geqslant 2$ не содержит базовых точек. %[Mir V.4.B (Ex 4.11)]
    \item Пусть $X$ --- кривая в $\PP^3$ определенная уравнениями $xw=yz$, $xz=y^2$, $yw=z^2$ (скрученная кубика). Используя степень гиперплоскостного дивизора $\div(x)$ докажите, что степень кривой $X$ равна $3$. Определите также $\div (y)$. %[Mir V.2.F] 
\end{enumerate}

%Тема 8
\subsection{Алгебраические кривые, слабая аппроксимация}

{\bf Аннотация:} Понятие алгебраической кривой. Примеры римановых поверхностей являющихся алгебраическими кривыми. Функции с заданными порядками в точке и функции с заданными отрезками рядов Лорана. Теорема о слабой аппроксимации. Конечная порождённость поля рациональных функций алгебраической кривой. Степень поля функций и степень кривой.

{\bf Источники:} \cite{Mir}, \S VI.1.

{\bf Примеры задач:}
\begin{enumerate}[noitemsep,topsep=0pt]
    \item Докажите, что следующие римановы поверхности являются алгебраическими кривыми:
    \begin{itemize}[noitemsep,topsep=0pt]
        \item риманова сфера $\CC_\infty$;
        \item комплексный тор $\CC/L$;
        \item гиперэллиптическая кривая;
        \item гладкая проективная кривая.
    \end{itemize}%[Mir VI.1.C--F]
    \item Пусть $X$ --- алгебраическая кривая. Используя компактность $X$ докажите, что в $\mathcal{M}(X)$ существует конечное число глобальных мероморфных функций отделяющих точки и касательные. %[Mir VI.1.I]
    \item Пусть $X$ --- компактная риманова поверхность. Докажите, что если $\forall p_1,\dots,p_n \in X$ $\forall m_1, \dots, m_n \in\ZZ$ $\exists f\in \mathcal{M}(X):$ $\nu_{p_i}(f)=m_i$, то $X$ --- алгебраическая кривая. %[Mir VI.1.J]
    \item Пусть $G$ --- конечная группа, действующая эффективно на алгебраической кривой $X$.
    \begin{itemize}[noitemsep,topsep=0pt]
        \item Покажите, что можно задать действие $G$ на $\mathcal{M}(X)$.
        \item Докажите, что $\mathcal{M}(X/G) = \mathcal{M}(X)^G$.
        \item Докажите, что $X/G$ является алгебраической кривой.
    \end{itemize}%[Mir VI.1.L]
    \item Докажите следующие утверждения:
    \begin{itemize}[noitemsep,topsep=0pt]
        \item $\mathcal{M}(\CC_\infty)$ порождается локальной координатой $z$.
        \item $\mathcal{M}(\CC/L)$ порождается отношениями тета-функций.
        \item Если $X$ --- гиперэллиптическая кривая $y^2=h(x)$, то $\mathcal{M}(X)$ порождается $x$ и $y$.
        \item Если $X$ --- гладкая проективная кривая, то $\mathcal{M}(X)$ --- поле рациональных функций.
    \end{itemize} %[Mir VI.1.K]
\end{enumerate}

%Тема 9
\subsection{Сильная аппроксимация, теорема Римана--Роха}

{\bf Аннотация:} Дивизоры отрезков рядов Лорана. Задача Миттаг--Леффлера. Пространство $H^1(D)$. Теорема Римана--Роха, двойственность Серра. Замечание про три определения рода. Замечание про язык аделей.

{\bf Источники:} \cite{Mir} \S\S VI.2--VI.3

{\bf Примеры задач:}
\begin{enumerate}[noitemsep,topsep=0pt]   
    \item Пусть $f$ --- мероморфная функция, $D$ --- дивизор. Докажите, что определенный в лекции оператор умножения $\mu_f^D:\mathcal{T}[D](X)\rightarrow\mathcal{T}[D-\div(f)](X)$ является изоморфизмом с обратным отображением $\mu_{1/f}^{D-\div(f)}$. %[Mir VI.2.A]
    \item Пусть $D$ --- дивизор, $f,g$ --- глобальные мероморфные функции на $X$, $\alpha_D:\mathcal{M}(X)\rightarrow \mathcal{T}[D](X)$ --- отображение, определенное в лекции. Докажите, что $\mu_f^D (\alpha_D(g))=\alpha_{D-\div(f)}(fg)$. %[Mir VI.2.B]
    \item Докажите, что $D_1 \leqslant D_2$ $\implies$ $\alpha_{D_2}=t_{D_2}^{D_1} \circ \alpha_{D_1}$ %[Mir VI.2.C]
    \item Пусть $X=\CC_\infty$ --- риманова сфера. Докажите, что $H^1(0)=0$ явным образом используя прообраз $\alpha_0$. %[Mir VI.2.E]
    \item Пусть $X=\CC/L$ --- комплексный тор, $p$ --- тождественный элемент группового закона на $X$, $z$ --- локальная координата в окрестности $p$, $Z=z^{-1}\cdot p \in \mathcal{T}[0](X)$. Докажите, что $Z$ не лежит в прообразе $\alpha_0$ (то есть $H^1(0)\neq 0$) %[Mir VI.2.H]
    \item Пусть $f\in \mathcal{M}(X)$, $\omega\in L^{(1)}(-D)$. Докажите, что $f\omega \in L^{(1)}(-D-\div(f))$, и что $\Res_\omega \circ \mu_f^D=\Res_{f\omega}$ в $\mathcal{T}[D+\div(f)](X)$. %[Mir VI.3.A]
    \item Докажите, что если $D$ --- положительный дивизор, $\deg D \geqslant g+1$, то в $L(D)$ существует по крайней мере одна непостоянная функция. %[Mir VI.3.B]
    \item Пусть $X$ --- алгебраическая кривая, $K$ --- канонический дивизор, $D$ --- дивизор степени $\deg D > 0$. Докажите, что $H^{1}(K+D)=0$. %[Mir VI.3.D]
    \item Докажите, что если $g\geqslant 2$, $m\geqslant 2$, то $\dim L(mK)=(g-1)(2m-1)$. %[Mir VI.3.G]
\end{enumerate}

%Тема 10
\subsection{Некотороые приложения теоремы Римана--Роха}

{\bf Аннотация:} Первые приложения теоремы Римана Роха: всякая алгебраическая кривая является проективной, кривые рода 0 изоморфны римановой сфере, кривые рода 1 изоморфны комплексным торам, кривые рода 2 изоморфны гиперэллиптическим кривым. Теорема Клиффорда. Существование мероморфных 1-форм. Автоморфные формы и мероморфные 1-формы на модулярной кривой, размерность пространств модулярных форм конгруэнц подгрупп (обзорно).

{\bf Источники:} \cite{Mir} \S VII.1 ; \cite{DS} глава 3.

{\bf Примеры задач:}
\begin{enumerate}[noitemsep,topsep=0pt]
    
    \item Пусть $X$ --- алгебраическая кривая, $D$ --- дивизор степени $\deg D >0$. Докажите, что $\dim L(D) = 1+\deg D$ $\iff$ $g(X)=0$. %[Mir VII.1.A]
    \item Пусть $X$ --- алгебраическая кривая рода $g(X)=g \geqslant 2$, $D$ --- дивизор степени $\deg D >0$. Докажите, что если $\deg D \leqslant 2g-3$, то $\dim L(D) \leqslant g$. %[Mir VII.1.B]
    \item Пусть $X$ --- компактная Риманова поверхность, $D_1$, $D_2$ --- дивизоры. Докажите, что $\dim L(D_1) + \dim L(D_2) \leqslant \dim L(\min (D_1,D_2))+\dim L(\max (D_1,D_2))$. %[Mir Lemma VII.1.11]
    \item Пусть $X$ --- алгебраическая кривая рода $g(X)=g$, $K$ --- канонический дивизор, $D$ --- дивизор такой, что $\dim L(D) \geqslant 1$ и $\dim L(K-D) \geqslant 1$. Докажите, что $\dim L(D)+\dim L(K-D) \leqslant 1+g$. %[Mir Lemma VII.1.12]
    \item Пусть $X$ --- алгебраическая кривая $g(X)\geqslant 1$, $K$ --- канонический дивизор, $D$ --- дивизор, такой что пространства $L(D)$, $L(K-D)$ ненулевые и $2\dim L(D)\leqslant \deg D + 2 $. Докажите, что тогда $D$ --- либо главный, либо канонический. %[Mir VII.1.D]
    \item Покажите, что разложение в ряд Лорана модулярного инварианта имеет вид: $j(z)=1/q+\sum_{n=0}^\infty a_n q^n$, $a_n\in\ZZ$, $q=e^{2\pi i z}$. %[DS 3.2.2]
    \item Докажите, что $\forall k\in\ZZ$ если $f\in\mathcal{A}_k(\Gamma)$, $f\neq 0$, то $\mathcal{A}_k(\Gamma)=\mathcal{M}(X(\Gamma))\cdot f$. %[DS 3.2.4]
    \item Докажите, что $\mathcal{S}_2(\Gamma) \cong \Omega^1(X(\Gamma))$.  %[DS 3.3.6]
\end{enumerate}

%Тема 12
\subsection{Нормирования функциональных полей, слабая аппроксимация}

{\bf Аннотация: TBD}

{\bf Источники:} \cite{Stich}; \cite{Step}

{\bf Примеры задач:}
\begin{enumerate}[noitemsep,topsep=0pt]
    
    \item TBD
\end{enumerate}

%Тема 13
\subsection{Дивизоры, линейные пространства, теорема Римана}

{\bf Аннотация: TBD}

{\bf Источники:} \cite{Stich}; \cite{Step}

{\bf Примеры задач:}
\begin{enumerate}[noitemsep,topsep=0pt]
    
    \item TBD
\end{enumerate}

%Тема 14
\subsection{Теорема Римана--Роха, сильная аппроксимация}

{\bf Аннотация: TBD}

{\bf Источники:} \cite{Stich}; \cite{Step}

{\bf Примеры задач:}
\begin{enumerate}[noitemsep,topsep=0pt]
    
    \item TBD
\end{enumerate}

%Тема 15
\subsection{Дзета функция алгебраической кривой}

{\bf Аннотация: TBD}

{\bf Источники:} \cite{Step}; \cite{Stich}

{\bf Примеры задач:}
\begin{enumerate}[noitemsep,topsep=0pt]
    
    \item TBD
\end{enumerate}

%Тема 16
\subsection{Теорема Хассе--Вейля}

{\bf Аннотация: TBD}

{\bf Источники:} \cite{Step}; \cite{Stich}

{\bf Примеры задач:}
\begin{enumerate}[noitemsep,topsep=0pt]
    
    \item TBD
\end{enumerate}


\addcontentsline{toc}{section}{Список литературы}
\begin{thebibliography}{9}

\bibitem[Mir]{Mir}
R. Miranda, Algebraic Curves and Riemann Surfaces, AMS, 1995.

\bibitem[Stich]{Stich}
H. Stichtenoth, Algebraic Function Fields and Codes, 2nd edition, 2009.

\bibitem[Степ]{Step}
С. А. Степанов, Арифметика алгебраических кривых, Наука, 1991.

\bibitem[Fult]{Fult}
W. Fulton, Algebraic Curves: An Introduction to Algebraic Geometry, 3rd edition, 2008.

\bibitem[DS]{DS}
F. Diamond, J. Shurman, A First Course in Modular Forms, Springer, 2005.


%\bibitem{texbook}
%Donald E. Knuth (1986) \emph{The \TeX{} Book}, Addison-Wesley Professional.

%\bibitem[lamp]{lamport94}
%Leslie Lamport (1994) \emph{\LaTeX: a document preparation system}, Addison
%Wesley, Massachusetts, 2nd ed.

\end{thebibliography}

\end{document}