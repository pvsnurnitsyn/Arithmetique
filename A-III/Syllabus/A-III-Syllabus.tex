\documentclass[a4paper, 12pt]{article}

\usepackage[T1,T2A]{fontenc}
\usepackage[utf8]{inputenc}
\usepackage[english, russian]{babel}
\usepackage{fullpage}
\usepackage{amssymb}
\usepackage{amsmath}
\usepackage{enumitem}
\usepackage{indentfirst}
\usepackage{titlesec}
\usepackage{hyperref}
\usepackage{tabularx}
\usepackage{longtable}
\usepackage{multirow}
\usepackage{makecell}
\usepackage{ifthen}

\titleformat*{\section}{\large\bfseries\scshape}

\hypersetup{
     colorlinks,
     citecolor=black,
     filecolor=black,
     linkcolor=black,
     urlcolor=black
 }

\setcounter{tocdepth}{2}
\setcounter{secnumdepth}{2}

%reduce spaces before eqs
\expandafter\def\expandafter\normalsize\expandafter{%
    \normalsize%
    \setlength\abovedisplayskip{2pt}%
    \setlength\belowdisplayskip{2pt}%
    \setlength\abovedisplayshortskip{-2pt}%
    \setlength\belowdisplayshortskip{2pt}%
}

%declare and redeclare math operators

\let\iff\relax
\DeclareMathOperator{\iff}{\Leftrightarrow} 
\let\implies\relax
\DeclareMathOperator{\implies}{\Rightarrow} 

\DeclareMathOperator{\ZZ}{\mathbb{Z}}
\DeclareMathOperator{\QQ}{\mathbb{Q}}
\DeclareMathOperator{\RR}{\mathbb{R}}
\DeclareMathOperator{\CC}{\mathbb{C}}
\DeclareMathOperator{\HH}{\mathbb{H}}
\DeclareMathOperator{\PP}{\mathbb{P}}
\let\AA\relax
\DeclareMathOperator{\AA}{\mathbb{A}}
\DeclareMathOperator{\FF}{\mathbb{F}}

\DeclareMathOperator{\GL2}{GL_2}
\DeclareMathOperator{\SL2}{SL_2}

\let\Re\relax
\DeclareMathOperator{\Re}{Re} 
\let\Im\relax
\DeclareMathOperator{\Im}{Im} 
\DeclareMathOperator{\Res}{Res} 
%\DeclareMathOperator{\dd}{d} 

\let\div\relax
\DeclareMathOperator{\div}{div} 
\DeclareMathOperator{\Div}{Div} 
\DeclareMathOperator{\Cl}{Cl} 

%\DeclareMathOperator{\ord}{ord}
\DeclareMathOperator{\chr}{char}
%\DeclareMathOperator{\Mod}{mod}
\DeclareMathOperator{\Gal}{Gal}
\DeclareMathOperator{\Aut}{Aut}

\newcommand*\smat[4]{\left(\begin{smallmatrix}#1&#2\\#3&#4\end{smallmatrix}\right)}

%\title{Спецкурс Арифметика III:\\ Римановы поверхности и алгебраические кривые}
%\author{П.В. Снурницын}
%\date{}

\begin{document}

\newboolean{isbachelor}
\setboolean{isbachelor}{false}
%\setboolean{isbachelor}{true}

\begin{titlepage}

    \centering

    Федеральное государственное бюджетное образовательное учреждение\\
    высшего образования \\
    Московский государственный университет имени М.В. Ломоносова\\
    Факультет вычислительной математики и кибернетики

    \vspace{24pt}

    \begin{flushright}
        \textbf{
        УТВЕРЖДАЮ\\
        декан факультета вычислительной\\
        математики и кибернетики\\
        \vspace{12pt}
        \underline{\hspace{2.5cm}} /И.А. Соколов/\\
        \flqq \underline{\hspace{1cm}}\frqq \underline{\hspace{3cm}} 2025 г.}
    \end{flushright}

    \vspace{24pt}

    \textbf{\textsc{Рабочая программа дисциплины (модуля)}}

    \vspace{24pt}

    Наименование дисциплины (модуля):\\
    \textbf{\underline{\ifthenelse{\boolean{isbachelor}}{Введение в алгебраические кривые}{Римановы поверхности и алгебраические кривые}}}

    \vspace{16pt}

    Уровень высшего образования:\\
    \textbf{\ifthenelse{\boolean{isbachelor}}{бакалавриат}{магистратура}}

    \vspace{16pt}

    Направление подготовки / специальность:\\
    \textbf{\ifthenelse{\boolean{isbachelor}}{01.03.02 \flqq Прикладная математика и информатика\frqq}{01.04.02 \flqq Прикладная математика и информатика\frqq}}

    \vspace{16pt}

    Направленность (профиль) ОПОП:\\
    \textbf{дисциплина относится к вариативной части программы \ifthenelse{\boolean{isbachelor}}{\flqq Математические методы обработки информации и принятия решений\frqq}{\flqq Информационная безопасность компьютерных систем\frqq}}

    \vspace{16pt}

    Форма обучения:\\
    \textbf{очная}

    \vspace{48pt}

    \begin{flushright}

        Рабочая программа рассмотрена и утверждена\\
        на заседании  Ученого совета факультета ВМК\\
        (протокол №\underline{\hspace{0.5cm}} от \underline{\hspace{3cm}})

    \end{flushright}

    \vspace*{\fill}

    Москва 2025

\end{titlepage}

Рабочая программа дисциплины (модуля) разработана в соответствии с самостоятельно разрабатываемым образовательным стандартом МГУ имени М.В. Ломоносова для реализуемых основных профессиональных образовательных программ высшего образования по направлению подготовки \ifthenelse{\boolean{isbachelor}}{01.03.02 \flqq Прикладная математика и информатика\frqq}{01.04.02 \flqq Прикладная математика и информатика\frqq}.

\newpage

\tableofcontents

\newpage

\section{Место дисциплины (модуля) в структуре ОПОП ВО}

Настоящая дисциплина включена в учебный план по направлению \ifthenelse{\boolean{isbachelor}}{01.03.02 \flqq Прикладная математика и информатика\frqq}{01.04.02 \flqq Прикладная математика и информатика\frqq}, профиль \ifthenelse{\boolean{isbachelor}}{\flqq Математические методы обработки информации и принятия решений\frqq}{\flqq Информационная безопасность компьютерных систем\frqq} и входит в базовую часть программы.
Дисциплина является кафедральным (вариативным)  курсом и изучается по выбору студента.
Дисциплина рассчитана на студентов, знакомых с основными понятиями и результатами алгебры, теории чисел, действительного и комплексного анализа, а также владеющих основами языка программирования Python.

\section{Цели и задачи дисциплины}

Курс является введением в алгебраическую и арифметическую геометрию, в частности в случай плоских алгебраических кривых. В первой части курса вводится понятие римановой поверхности, излагаются основные результаты теории алгебраических кривых над полем комплексных чисел, доказывается теорема Римана--Роха и рассматриваются некоторые её приложения. Вторая часть курса посвящена обобщению на произвольное базовое поле, включая случай конечных полей. Основным результатов второй части является рассмотрение доказательства теоремы Хассе–Вейля об оценке числа точек алгебраической кривой над конечным полем.

\section{Результаты обучения по дисциплине (модулю)}

Компетенции выпускников, частично формируемые при реализации дисциплины (модуля):

{\bf Содержание и код компетенции}
\begin{itemize}[noitemsep,topsep=0pt]
    \item {\bf ОПК-1.} Способность формулировать и решать актуальные задачи в области фундаментальной и прикладной математики.
    
    \item {\bf ОПК-4.} Способность комбинировать и адаптировать современные ин\-фор\-ма\-ци\-онно-комму\-ни\-ка\-ци\-онные технологии для решения задач в области профессиональной деятельности с учетом требований информационной безопасности.
    
    \item {\bf ПК-2.} Способность в рамках задачи, поставленной специалистом более высокой квалификации, проводить научные исследования и (или) осуществлять разработки в области прикладной математики и информатики с получением научного и (или) научно-практического результата;
    
    \item {\bf СПК-ВТЧП-1М.} Способность формулировать и решать задачи в области теории чисел и её приложений, используя современные ин\-фор\-ма\-ци\-онно-комму\-ни\-ка\-ци\-онные технологии.
\end{itemize}

{\bf Индикатор (показатель) достижения компетенции}
\begin{itemize}[noitemsep,topsep=0pt]
    \item {\bf СПК-ВТЧПРП-1М.1.} Владение основными понятиями и методами теории римановых поверхностей.
    \item {\bf СПК-ВТЧПРП-1М.2.} Владение основными понятиями и результатами из области функциональных алгебраических полей.
    
    \item {\bf СПК-ВТЧП-1М.5.} Владение системой компьютерной алгебры SageMath для решения задач теории чисел и алгебры.
    \item {\bf СПК-ВТЧП-1М.6.} Понимание приложений теории чисел для задач криптографии.
    \item {\bf СПК-ВТЧП-1М.7.} Понимание приложений теории чисел для задач теории кодирования.
\end{itemize}

{\bf Планируемые результаты обучения по 
дисциплине, сопряженные с индикаторами   
достижения компетенций}
\begin{itemize}[noitemsep,topsep=0pt]
    \item {\bf Знать}
    \begin{itemize}[noitemsep,topsep=0pt]
        \item Основные понятия, определения и результаты теории римановых поверхностей;
        \item Основные понятия, определения и результаты теории алгебраических функциональных полей;
        \item Основные направления приложений теории чисел и алгебраической геометрии в криптографии и теории кодирования.
    \end{itemize}
    \item {\bf Уметь}
    \begin{itemize}[noitemsep,topsep=0pt]
        \item Решать задачи теории чисел, используя  методы алгебраической и арифметической геометрии;
        \item Применять системы компьютерной алгебры и символьных вычислений для решения задач алгебры и теории чисел;
        \item Применять методы теории чисел к формализации постановок прикладных задач, включая криптографию и теорию кодирования.
    \end{itemize}
    \item {\bf Владеть}
    \begin{itemize}[noitemsep,topsep=0pt]
        \item Навыками работы в системе компьютерной алгебры SageMath
    \end{itemize}
\end{itemize}

\section{Формат обучения и объём дисциплины (модуля)}

Формат обучения: занятия проводятся с использованием меловой или маркерной доски, интерактивные материалы демонстрируются с помощью ноутбука и проектора.

Объем дисциплины (модуля) составляет 120 академических часов, в том числе 60 академических часов, отведенных на контактную работу обучающихся с преподавателем, 60 академических часов на самостоятельную работу обучающихся.

\section{Содержание дисциплины (модуля), структурированное по темам (разделам) с указанием отведённого на них количества академических часов и видов учебных занятий}

\subsection{Структура дисциплины (модуля) по темам (разделам) с указанием отведенного на них количества академических часов и виды учебных занятий (в строгом соответствии с учебным планом)}

\noindent
\begin{longtable}{ | >{\raggedright}p{5cm} | p{1.5cm}| p{1.5cm} | p{1.5cm} | p{1.5cm} | p{2cm} | } 
    \hline
     & \multicolumn{3}{c|}{\textbf{\makecell[c]{Номинальные трудо-\\затраты обучающегося,\\ академические часы}}} & & \\
    \hline
    \textbf{Наименование разделов и тем дисциплины (модуля), Форма промежуточной аттестации по дисциплине (модулю)} & Кон\-такт\-ная работа, занятия лекционного типа & Кон\-такт\-ная работа, занятия семинарского типа & Самос\-тоятель\-ная работа обучающегося &  \textbf{Всего академических часов} & \textbf{Форма текущего контроля успеваемости* (наименование)} \\
    \hline
    \hline
    \multicolumn{6}{|l|}{\textit{Раздел 1. Римановы поверхности}} \\ \hline
    Тема 1. Римановы поверхности: определения и примеры & 2 & 2 & 4 & 8 & \\ \hline
    Тема 2. Функции на римановых поверхностях & 2 & 2 & 4 & 8 & \\ \hline
    Тема 3. Теорема Гильберта о нулях & 2 & 2 & 4 & 8 & \\ \hline
    Тема 4. Отображения римановых поверхностей & 2 & 2 & 4 & 8 & \\ \hline
    Тема 5. Группы, действующие на римановых поверхностях & 2 & 2 & 4 & 8 & \\ \hline
    Тема 6. Дифференциальные формы и дивизоры & 2 & 2 & 4 & 8 & \\ \hline
    Тема 7. Пространства функций дивизоров и линейные системы & 2 & 2 & 4 & 8 & \\ \hline
    Тема 8. Алгебраические кривые, слабая аппроксимация & 2 & 2 & 4 & 8 & \\ \hline
    Тема 9. Сильная аппроксимация, теорема Римана-Роха & 2 & 2 & 4 & 8 & \\ \hline
    Тема 10. Некоторые приложения теоремы Римана-Роха & 2 & 2 & 4 & 8 & \\ \hline
    \multicolumn{6}{|l|}{\textit{Раздел 2. Алгебраические функциональные поля}} \\ \hline
    Тема 11. Нормирования функциональных полей & 2 & 2 & 4 & 8 & \\ \hline
    Тема 12. Дивизоры и линейные пространства & 2 & 2 & 4 & 8 & \\ \hline
    Тема 13. Теорема Римана-Роха для функциональных полей & 2 & 2 & 4 & 8 & \\ \hline
    Тема 14. Дзета функция алгебраической кривой & 2 & 2 & 4 & 8 & \\ \hline
    Тема 15. Теорема Хассе-Вейля & 2 & 2 & 4 & 8 & \\ \hline
    \multicolumn{5}{|l|}{\textbf{Итоговая аттестация}} & \textbf{зачет} \\ \hline
    \textbf{Итого, академические часы}  & 30 & 30 & 60 & 120 & \\ \hline
\end{longtable}

\subsection{Содержание разделов (тем) дисциплины}

Курс состоит из двух частей. В первой части (лекции 1--10) излагается теория римановых поверхностей с уклоном в приложения для современной теории чисел. Развивается необходимый для доказательства теоремы Римана--Роха аппарат, сама теорема Римана-Роха доказывается используя двойственность Серра. Общие результаты сопровождаются примерами для случаев римановой сферы, комплексного тора (эллиптической кривой), а также римановых поверхностей, возникающих при изучении пространств модулярных форм (модулярных кривых). В частности затрагиваются вопросы компактификации модулярных кривых, а также оценка размерности пространств модулярных форм для конгруэнц-подгрупп.

Вторая часть (лекции 11--15) посвящена общему случаю алгебраических функциональных полей (над алгебраически замкнутым полем). Теорема Римана--Роха доказывается в более общем случае. Вводится понятие дзета-функции алгебраической кривой. Рассматривается идея доказательства теоремы Хассе--Вейля об оценке числа точек алгебраической кривой над конечным полем.

\vspace{8pt}

\noindent
\begin{longtable}{ | >{\raggedright}p{5cm} | p{10cm} | } 
    \hline
    \textbf{Наименование разделов (тем) дисциплины} & \textbf{Содержание разделов (тем) дисциплин} \\
    \hline
    \hline
    \multicolumn{2}{|l|}{\textit{Раздел 1. Римановы поверхности}}\\ \hline
    Тема 1. Римановы поверхности: определения и примеры &
    Повторение определений и необходимых фактов из топологии: топологические пространства, связность, хаусдорфовость, род, число Эйлера. Определения комплексных карт, атласов, структуры двумерного многообразия и римановой поверхности. Примеры: комплексная плоскость, риманова сфера, комплексный тор, афинные кривые, проективная прямая, проективная плоскость, проективные кривые.
    \textit{Источники: \cite{Mir}, глава I.} \\ \hline
    Тема 2. Функции на римановых поверхностях & Голоморфные и мероморыне функции, ряды Лорана, порядки нулей и полюсов. Повторение необходимых результатов из комплексного анализа от одной переменной. Кольцо голоморфных функций и поле мероморфных функций. Мероморфные функции на римановой сфере, равенство числа нулей и полюсов с учетом кратностей. Мероморфные функции на торе, проективной прямой и гладкой алгебраической кривой (приложение теоремы Гильберта о нулях).
    \textit{Источники: \cite{Mir} \S II.1--II.2.} \\ \hline
    Тема 3. Теорема Гильберта о нулях & Повторение определений и необходимых фактов из алгебры: идеалы, модули, кольца главных идеалов, максимальный идеал, радикальный идеал. Идеал множества точек кривой. Неприводимые компоненты алгебраических множеств. Лемма Зарисского, теорема Гильберта о базисе, теорема Гильберта о нулях.
    \textit{Источники: \cite{Fult} глава 1; \cite{Mir} \S III.5.} \\ \hline
    Тема 4. Отображения римановых поверхностей & Голоморфные отображения римановых поверхностей, индуцирование гомоморфизмов колец голоморфных и полей мероморфных функций. Изоморфизм римановой сферы и проективной прямой. Мероморфные функции как голоморфные отображения на риманову сферу. Теорема о локальной нормальная форме, кратность отображения. Степень отображения. Теорема о сумме порядков мероморфных функций. Формула Гурвица. Точки ветвления для афинных и проективных кривых. Мероморфные функции на торе как отношения тета-функций. Изоморфизмы комплексных торов.
    \textit{Источники: \cite{Mir} \S\S II.3--II.4, \S III.1.} \\ \hline
    Тема 5. Группы, действующие на римановых поверхностях & Действие группы, орбита, стабилизатор. Фактор-пространство как риманова поверхность, степень отображения факторизации. Теорема Гурвица о действии конечной группы. Действие полной модулярной группы и её конгруэнц подгрупп $\Gamma$ на верхней комплексной полуплоскости $\HH$. Структура римановой поверхности на $Y(\Gamma) = \HH/\Gamma$. Эллиптические и параболические точки. Компактификация $Y(\Gamma)$ в $X(\Gamma)$, род $X(\Gamma)$.
    \textit{Источники: \cite{Mir} \S III.3; \cite{DS} \S 2.3.1.}\\ \hline
    Тема 6. Дифференциальные формы и дивизоры & Голоморфные и мероморфные дифференциальные формы на римановых поверхностях. Интегрирование дифференциальных форм. Вычеты, теорема о сумме вычетов. Дивизоры функций, степень дивизора, главные дивизоры. Дивизоры дифференциальных форм, канонические дивизор. Степень канонического дивизора на компактной римановой поверхности. Линейная эквивалентность дивизоров. Свойства дивизоров на римановой сфере и торе. Теорема Абеля для тора. Понятие степени гладкой проективной кривой, теорема Безу.
    \textit{Источники: \cite{Mir} глава IV, \S\S V.1--V.2.}\\ \hline
    Тема 7. Пространства функций дивизоров и линейные системы & Линейное пространство мероморфных функций $L(D)$ и полная линейная система $|D|$. Базовые свойства линейных пространств и линейных систем. Линейное пространство мероморфных форм $L^{(1)}(D)$, изоморфизм между простнствами $L$ и $L^{(1)}$. Линейные пространства $L(D)$ для случаев римановой сферы и тора. Оценка размерности $L(D)$. Голоморфные отображения римановых поверхностей в проективные пространства. Базовые точки линейных систем. Обратные образы (пуллбэки) дивизоров и форм. Гиперплоскостные дивизоры.
    \textit{Источники: \cite{Mir} \S\S V.3--V.4.}\\ \hline
    Тема 8. Алгебраические кривые, слабая аппроксимация & {\bf Аннотация:} Понятие алгебраической кривой. Примеры римановых поверхностей являющихся алгебраическими кривыми. Функции с заданными порядками в точке и функции с заданными отрезками рядов Лорана. Теорема о слабой аппроксимации. Конечная порождённость поля рациональных функций алгебраической кривой. Степень поля функций и степень кривой.
    \textit{Источники: \cite{Mir}, \S VI.1.}\\ \hline
    Тема 9. Сильная аппроксимация, теорема Римана-Роха & Дивизоры отрезков рядов Лорана. Задача Миттаг--Леффлера. Пространство $H^1(D)$. Теорема Римана--Роха, двойственность Серра. Замечание про три определения рода. Замечание про язык аделей.
    \textit{Источники: \cite{Mir} \S\S VI.2--VI.3.}\\ \hline
    Тема 10. Некоторые приложения теоремы Римана-Роха & Первые приложения теоремы Римана Роха: алгебраические кривые являются проективными, кривые рода 0 изоморфны римановой сфере, кривые рода 1 изоморфны комплексным торам, кривые рода 2 изоморфны гиперэллиптическим кривым. Теорема Клиффорда. Существование мероморфных 1-форм. Автоморфные формы и мероморфные 1-формы на модулярной кривой, размерность пространств модулярных форм конгруэнц подгрупп (обзорно).
    \textit{Источники: \cite{Mir} \S VII.1 ; \cite{DS} глава 3.}\\ \hline
    \multicolumn{2}{|l|}{\textit{Раздел 2. Алгебраические функциональные поля}}\\ \hline
    Тема 11. Нормирования функциональных полей & Алгебраические функциональные поля. Дискретные нормирования алгебраических функциональных полей и их свойства. Точки функциональных полей, степень точек. Теорема о слабой аппроксимации Поле рациональных функций.
    \textit{Источники: \cite{Stich} \S\S 1.1--1.3; \cite{Step} \S IV.2.1-IV.2.2.}\\ \hline
    Тема 12. Дивизоры и линейные пространства & Дивизоры, группа классов дивизоров. Линейные пространства $L(D)$. Теорема о равенстве числа нулей и полюсов. Алгебраический род. Теорема Римана.
    \textit{Источники: \cite{Stich} \S 1.4; \cite{Step} \S IV.2.3, \S IV.3.1.}\\ \hline
    Тема 13. Теорема Римана-Роха для функциональных полей & Кольцо аделей (распределений). Дифференциалы Вейля, канонический класс. Теорема Римана--Роха, двойственность Серра. Теорема о сильной аппроксимации. Теорема Вейершрасса о пропусках.
    \textit{Источники: \cite{Stich} \S\S 1.5--1.6; \cite{Step} \S IV.3.2--IV.3.4.}\\ \hline
    Тема 14. Дзета функция алгебраической кривой & Рациональные точки алгебраических кривых, рациональные дивизоры. Конечность числа классов дивизоров нулевой степени. Дзета функция алгебраической кривой. Рациональность дзета-функции, произведение Эйлера и функциональное уравнение.
    \textit{Источники: \cite{Step} \S\S V.1.1--V.1.3; \cite{Stich} \S 5.1; \cite{VNTs} \S 3.1.}\\ \hline
    Тема 15. Теорема Хассе-Вейля & Связь дзета функции с числом точек на кривой над конечным полем. Оценка числа точек на кривой. Теорема Вейля (аналог гипотезы Римана для нулей дзета функции алгебраической кривой). Оценка Хассе--Вейля. Обзор некоторых других приложений (оценки тригонометрических сумм, гипотеза Рамануджана).
    \textit{Источники: \cite{Step} \S\S V.1.3-V.2; \cite{Stich}  \S 5.2; \cite{VNTs} \S 3.1.}\\ \hline
\end{longtable}


\subsection{Примеры задач для семинаров и самостоятельной работы \label{problems}}

\subsubsection{Тема 1. Римановы поверхности: определения и примеры}

\begin{enumerate}[noitemsep,topsep=0pt]
    \item Пусть $X$ --- топологическое пространство, $\phi:U\rightarrow V$ --- комплексная карта на $X$, $\psi: V \rightarrow W$ --- взаимно-однозначное голоморфное отображение на подмножествах $\mathbb{C}$. Докажите, что $\psi \circ \phi: U \rightarrow W$ также является комплексной картой на $X$ и что $\psi \circ \phi$ совместима с любой картой на $X$, с которой совместима $\phi$. %[Mir I.1.B]
    \item Докажите, что определенная в лекции эквивалентность комплексных атласов действительно является отношением эквивалентности. %[Mir I.1.H]
    \item Проверьте, что отображение $\PP^1 \rightarrow S^2$ проективной прямой над $\CC$ в сферу единичного радиуса в $\RR^3$, заданное 
    $$
    [z:w] \mapsto (2\Re(w\overline{z}), 2\Im(w\overline{z}), |w|^2 - |z|^2) / |w|^2 + |z|^2,
    $$ 
    является гомеоморфизмом. (Поэтому проективная прямая является компактной римановой поверхностью рода 0). %[Mir I.2.C]
    \item Докажите, что группа сложения точек комплексного тора $X$ является делимой, то есть, что $\forall p\in X$ $\forall n\in \mathbb{Z}_+$ $\exists q \in X: n\cdot q = p$. %[Mir I.2.F]
    \item Пусть многочлен от двух переменных имеет вид $f(z,w)=w^2 - h(z)$. 
    \begin{itemize}[noitemsep,topsep=0pt]
        \item Докажите, что $f(z,w)$ является неприводимым $\iff$ $h(z)$ не является точным квадратом.
        \item Докажите, что $f(z,w)$ является невырожденным $\iff$ все корни $h(z)$ различны. 
    \end{itemize} %[Mir I.2.G]   
    \item Пусть $f(z,w)$ --- многочленом второй степени. Афинная кривая $X$ заданная $f(z,w) = 0$ называется афинной коникой.  
    \begin{itemize}[noitemsep,topsep=0pt]
        \item Докажите, что если $f(z,w)$ --- вырожденный, то $f(z,w)$ раскладывается в произведение двух линейных множителей. Что в этом случае можно сказать про $X$?
        \item Приведите примеры гладких афинных коник.
    \end{itemize}%[Mir I.2.H]
    \item Пусть $F(x,y,z)$ --- однородный многочлен степени $d$. Докажите, что $F$ невырожденный $\iff$ в каждой афинной карте $F$ задаёт гладкую афинную кривую. %[Mir Lemma 3.5 p15]
    \item Докажите, что любые две проективные прямые в $\PP^2$ пересекаются в единственной точке. %[Mir I.3.E]
\end{enumerate}

\subsubsection{Тема 2. Функции на римановых поверхностях}

\begin{enumerate}[noitemsep,topsep=0pt]
    \item Пусть $f$ --- комплексно-значная функция определенная на римановой сфере $\CC_\infty$ в окрестности $\infty$. Докажите, что $f$ голоморфна в $\infty$ $\iff$ $f(1/z)$ голоморфна в $0$. %[Mir II.1.A (Exmp 1.6)]
    \item Пусть $p(z,w), q(z,w) \in \CC[z,w]$ --- однородные многочлены одинаковой степени, $q(z_0,w_0) \neq 0$. Покажите, что $f([z:w]) = p(z,w)/q(z,w)$ --- корректно определенная на $\PP^1$ функция голоморфная функция в окрестности $[z_0:w_0]$. %[Mir II.1.A (Exmp 1.7)]
    \item Пусть $X \subset \PP^2$ --- проективная кривая заданная невырожденным многочленом $F(x,y,z)=0$, $F(x,y,z), G(x,y,z) \in \CC[x,y,z]$ --- однородные многочлены одинаковой степени, причем $H$ не равен тождественно нулю на $X$. Покажите, что $G(x, y, z) / H(x, y, z)$ --- мероморфная функция на $X$. %[Mir II.1.B (Exmp 1.22)]
    \item Пусть $U$ --- окрестность точки $p \in X$ $f,g \in \mathcal{M}(U)$. Докажите следующие свойства порядка $\nu_p$: 
    \begin{itemize}[noitemsep,topsep=0pt]
        \item $\nu_p(fg) = \nu_p(f) + \nu_p(g)$;
        \item $\nu_p(1/f) = -\nu_p(f)$, $\nu_p(f/g) = \nu_p(f) - \nu_p(g)$;
        \item $\nu_p(f\pm g) \geqslant \min(\nu_p(f), \nu_p(g))$.
    \end{itemize} %[Mir II.1.F (Lemma 1.29)]
    \item Докажите, что ряд определяющий тета-функцию $\theta_\tau(z) = \sum_{n\in\ZZ} e^{\pi i (n^2 \tau + 2nz)}$ сходится абсолютно и равномерно на компактных подмножествах $\CC$. %[Mir II.2.B]
    \item Докажите, что $z_0$ --- нуль $\theta_\tau(z)$ $\iff$ $\forall m,n\in\ZZ$ точка $z_0+m+n\tau$ является нулём $\theta_\tau(z)$. %[Mir II.2.E]
    \item Докажите, что $(1/2)+(\tau/2) + m + n\tau, m,n\in\ZZ$ --- единственные нули $\theta_\tau$, причём эти нули простые.  %[Mir II.2.F]
    \item Пусть $L$ --- решётка в $\CC$, $X=\CC/L$ --- комплексный тор, $\pi: \CC \rightarrow X$ --- естественная проекция, и пусть заданы два набора $\{p_i\}_{i=1}^d, \{q_i\}_{i=1}^d \subset X$. Покажите, что существуют два набора $\{x_i\}_{i=1}^d, \{y_i\}_{i=1}^d \subset \CC$: $\pi(x_i) = p_i, \pi(y_i)=q_i$, $\sum_i x_i = \sum_i y_i$ $\iff$ $\sum_i p_i = \sum_i q_i$, где последнее суммирование выполняется в групповом законе тора.  %[Mir II.2.G]
    
\end{enumerate}

\subsubsection{Тема 3. Теорема Гильберта о нулях}

\begin{enumerate}[noitemsep,topsep=0pt]
    \item Пусть $k$ --- произвольное поле. Докажите, что алгебраические подмножества $\AA^1(k)$ исчерпываются конечными подмножествами и самим $\AA^1(k)$. %[Fult 1.8]
    \item Пусть $k$ --- произвольное поле. Докажите следующие свойства алгебраических множеств в $\AA^n(k)$ и их идеалов: 
    \begin{itemize}[noitemsep,topsep=0pt]
        \item $X \subset Y \implies I(X) \supset I(Y)$;
        \item $I(\{a_1, \dots, a_n\}) = (x_1-a_1, \dots, x_n-a_n)$, $I(\emptyset) = k[x_1, \dots x_n]$, $I(\AA^n(k)) = (0)$ (при не конечном $k$);
        \item $I(V(S)) \supset S$, $V(I(X)) \supset X$, где $S \subset k[x_1, \dots x_n]$, $X \subset \AA^n(k)$;
        \item $V(I(V(S))) = V(S)$, $I(V(I(X))) = I(X)$, где $S$ и $X$ как выше;
        \item $\forall X \subset \AA^n(k)$ $I(X)$ --- радикальный идеал;
        \item $V = W \iff I(V)=I(W)$, где $V,W \subset \AA^n(k)$ --- алгебраические. 
        \item $V(I) = V(\sqrt{I})$, $\sqrt{I} \subset I(V(I))$, где $I$ --- идеал в $k[x_1, \dots x_n]$.
    \end{itemize}
    %[Fult свойства идеалов из раздела 1.3 + задача 1.16]
    \item Докажите, что $I(\{a_1, \dots, a_n\}) = (x_1-a_1, \dots, x_n-a_n)$ является максимальным идеалом в $k[x_1, \dots x_n]$. %[Fult 1.21]
    \item Докажите, что $I=(x^2+1) \subset \RR[x]$ --- радикальный идеал, но при этом $I$ не является идеалом никакого множества $\AA^1(\RR)$. %[Fult 1.19]
    \item Докажите, что $V(y-x^2)\subset \AA^2(\CC)$ неприводимо и $I(V(y-x^2)) = (y-x^2)$. %[Fult 1.25(a)]
    \item Разложите $V(y^4-x^2, y^4-x^2y^2+xy^2-x^3) \AA^2(\CC)$ на неприводимые компоненты. %[Fult 1.25(b)]
    %\item TODO [Fult 1.31]
    %\item TODO [Fult 1.33]
\end{enumerate}

\subsubsection{Тема 4. Отображения римановых поверхностей}

\begin{enumerate}[noitemsep,topsep=0pt]
    \item Докажите следующие свойства голоморфных отображений:
    \begin{itemize}[noitemsep,topsep=0pt]
        \item Если $F:X\rightarrow Y$, $G:Y\rightarrow Z$ --- голоморфные отображения, то $G\circ F: X \rightarrow Z$ --- голоморфное отображение;
        \item Если $F:X\rightarrow Y$ --- голоморфное отображение, $g$ --- голоморфная функция, определенная на открытом подмножестве $W\subset Y$, то $g\circ F$ --- голоморфная функция, определенная на $F^{-1}(W)$;
        \item Если $F:X\rightarrow Y$ --- голоморфное отображение, $g$ --- мероморфная функция, определенная на открытом подмножестве $W\subset Y$, то $g\circ F$ --- мероморфная функция, определенная на $F^{-1}(W)$.
    \end{itemize} %[Mir, II.3.B (Lemma 3.5)]
    \item Покажите, что при изоморфизме между комплексной проективной прямой $\PP^1$ и римановой сферой $\CC_{\infty}$ точки $[z:1]$ соответствуют конечным точкам $z\in\CC$, а точка $[1:0]$ соответствует $\infty$. %[Mir, II.3.С]
    \item Пусть $f(z,w), g(z,w)\in\CC[z,w]$ --- ненулевые, однородные многочлены одинаковой степени, не имеющие общих множителей. Докажите, что отображение $F:\PP^1\rightarrow\PP^1:[z:w]\mapsto [f(z,w):g(z,w)]$ корректно определено и голоморфно. Что можно сказать про случай, когда $f$ и $g$ имеют общие множители? %[Mir, II.3.F]
    \item Пусть $A = \smat{a}{b}{c}{d} \in \GL2(\CC)$, докажите следующие свойства:
    \begin{itemize}[noitemsep,topsep=0pt]
        \item $F_A:\PP^1\rightarrow\PP^1:[z:w]\mapsto[az+b:cz+d]$ --- автоморфизм $\PP^1$, $F_{AB}=F_A\circ F_B$;
        \item При отождествлении $\PP^1$ с $\CC_\infty$ отображение $F_A$ соответствует преобразованию $z\mapsto (az+b)/(cz+d)$.
    \end{itemize} %[Mir, II.3.G-H]
    \item Пусть $X$ --- компактная риманова поверхность, $f$ --- мероморфная непостоянная функция на $X$. Докажите что $f$ имеет хотя бы один нуль и хотя бы один полюс. %[Mir, II.3.I]
    \item Обозначим через $L=L(\omega_1, \omega_2)\subset\CC$ решётку на комплексной плоскости с базисом $\omega_1, \omega_2 \in \CC$. Докажите следующие свойства:
    \begin{itemize}[noitemsep,topsep=0pt]
        \item Пусть $L\subseteq L'$, докажите, что естественная проекция $\CC/L\rightarrow\CC/L'$ голоморфно, и что голоморфное отображение $\CC/L'\rightarrow\CC/L$ существует $\iff$ $L=L'$;
        \item Пусть $L$ --- решётка в $\CC$, $\alpha \in \CC^*$. Покажите, что $\alpha L$ --- также решётка, и что отображение $\phi: \CC/L \rightarrow \CC/(\alpha L): z+L \mapsto \alpha z + \alpha L$ --- корректно определенное биголоморфное отображение.
        \item Покажите, что всякий тор $\CC/L$ изоморфен тору вида $\CC/L(1,\tau)$, $\tau \in \HH$.
    \end{itemize}% [Mir, II.3.K]
    %\item Равенство числа нулей и полюсов на торе с учетом кратностей. %[Mir, Lemma 3.14]
    \item Пусть $f$ --- непостоянная мероморфная функция на торе $X=\CC/L$. Докажите, что $\sum_p \nu_p(f) = 0$. %[Mir, II.4.B]
    \item Пусть $F:X\rightarrow Y$, $G:Y\rightarrow Z$ --- два непостоянных голоморфных отображения, $f$ --- мероморфная функция на $Y$, $p\in X$. Докажите, что $e_p(F\circ G) = e_p(F) e_p(G)$, $\nu_p(f\circ F) = e_p(F) \nu_{F(p)}(f)$. %[Mir, II.4.C]
    %\item TODO [Mir, II.4.G]
    \item Докажите, что всякая прямая в $\PP^2$ невырождена и изоморфна $\PP_1$. %[Mir, Lemma III.1.1]
    \item Докажите, что в $\PP^2$ всякая гладка кривая второго порядка (коника) изоморфна кривой вида $x^2+y^2+z^2=0$. (В частности в $\PP^2$ все гладкие коники изоморфны между собой). %[Mir, Cor III.1.4]
\end{enumerate}

\subsubsection{Тема 5. Группы, действующие на римановых поверхностях}

\begin{enumerate}[noitemsep,topsep=0pt]
    \item Пусть $G$ --- конечная группа действующая на множестве $X$, $p\in X$. Докажите, что $|G\cdot p| |G_p| = |G|$. %[Mir III.3.A]
    \item Пусть $K$ --- ядро действия $G$ на $X$. Докажите, что $K$ --- нормальная подгруппа $G$, и что ядро действия $G/K$ на $X$ тривиально, а орбиты совпадают с орбитами действия $G$. %[Mir III.3.B]
    \item Пусть $G$ --- конечная подгруппа мультипликативной группы $\CC^*$ порядка $n$. Покажите что $G=\{e^{2\pi i/k}: 0\leqslant k \leqslant n\}$. %[Mir III.3.E]
    \item Покажите, что группа действий на римановой сферы $\CC_\infty$ порожденная двумя элементами $z\mapsto e^{2\pi i /r}$ и $z\mapsto 1/z$ есть диэдральная группа порядка $2r$. Докажите также, что действие этой группы голоморфно и эффективно. Определите точки ветвления и их индексы ветвления. %[Mir III.3.H]
    \item Докажите, что кривая определенная уравнением $xy^3+yz^3+zx^3=0$ (кривая Клейна) является гладкой проективной кривой. Покажите, что на этой кривой достигается граница теоремы Гурвица. %[Mir III.3.K]
    \item Пусть $\pi:\HH \rightarrow Y(\Gamma)=\HH/\Gamma: z\mapsto \Gamma z$ --- естественная проекция. Докажите, что для открытых множеств $U_1, U_2 \subset \HH$ справедливо $\pi(U_1)\cap\pi(U_2)=\emptyset$ $\iff$ $\Gamma U_1\cap U_2 = \emptyset$. %[DS Ex 2.1.2 (equiv (2.1))]
    \item Пусть $z_1, z_2\in\HH$. Докажите, что существуют окрестности $U_1$ и $U_2$ точек $z_1$ и $z_2$ обладающие следующим свойством: $\forall \gamma \in \SL2(\ZZ)\ \gamma(U_1)\cap U_2\neq\emptyset \implies \gamma(z_1)=z_2$. %[DS Prop 2.1.1]
    
\end{enumerate}

\subsubsection{Тема 6. Дифференциальные формы и дивизоры}

\begin{enumerate}[noitemsep,topsep=0pt]    
    \item Пусть на римановой сфере $X=\CC_\infty$ заданы две карты с локальными координатами $z$ и $w=1/z$ и пусть $\omega\in \mathcal{M}^{(1)}(X)$. Докажите, что если $\omega=f(z)dz$ (в локальной координате $z$), то $f$ --- рациональная функция от $z$. Докажите также, что $\Omega^1(X)=\{0\}$. Какие точки являются нулями и полюсами форм $dz, dz/z$. %[Mir IV.1.A]
    \item Пусть $L$ --- решётка в $\CC$, $X=\CC/L$ --- тор, $\pi:\CC\rightarrow X$ --- естественная проекция. Покажите, что для формы $dz$ в каждой карте $X$ локальная формула корректно определена и задаёт голоморфную 1-форму на $X$, и что эта форма не имеет нулей. %[Mir IV.1.B]
    \item Пусть $X$ --- гладкая плоская афинная кривая заданная уравнением $f(u,v)=0$. Покажите, что $du$, $dv$ --- корректно определенные голоморфные 1-формы на $X$, также как и $p(u,v)du$, $p(u,v)dv$ для любого $p(u,v)\in\CC[u,v]$. Покажите что если $r(u,v)$ --- рациональная функция, то $r(u,v)du$, $r(u,v)dv$ --- корректно определенные мероморфные 1-формы. %[Mir IV.1.C]
    \item Пусть $X$ --- риманова поверхность определенная уравнением $y^2=h(x)$, где $h\in\CC[x]$, $\deg h=2g+1,2g+2$ (то есть $X$ --- гиперэллиптическая кривая, поверхность рода $g$). Покажите, что $dx/y$ --- голоморфная 1-форма при $g\geqslant 1$. Покажите также, что если $p(x)\in\CC[x]$, $\deg (p) \leqslant g-1$, то $p(x)dx/y$ --- голоморфная 1-форма. %[Mir IV.1.G]
    \item Пусть $X$ --- гиперэллиптическая кривая $y^2=x^5-x$, тогда $x,y \in \mathcal{M}(X)$. Определите $\div(x), \div(y)$. %[Mir V.1.A]
    \item Пусть $X=\CC/L$ --- тор. Покажите, что форма $dz$ --- корректно определённая голоморфная 1-форма всюду отличная от нуля. Что в этом случае можно сказать о главных и канонических дивизорах? %[Mir V.1.C]
    \item Пусть $X$ --- плоская проективная кривая $y^2 z=x^3-x z^2$. Определите дивизоры пересечений $X$ с прямыми $x=0$, $y=0$, $z=0$. %[Mir V.1.H]
    \item Докажите следующие свойства дивизоров на римановой сфере $X=\CC_\infty$
    \begin{itemize}[noitemsep,topsep=0pt]
        \item $D_1\sim D_2$ $\iff$ $\deg(D_1)=\deg(D_2)$; %[Mir V.2.A (Cor 2.6)]
        \item Если $\deg(D) \geqslant 0$, то $D\sim D_0$, $D_0\geqslant 0$. %[Mir V.2.B (Cor 2.7)]
    \end{itemize}
\end{enumerate}

\subsubsection{Тема 7. Пространства функций дивизоров и линейные системы}

\begin{enumerate}[noitemsep,topsep=0pt]
    \item Пусть $X$ --- компактна, $D$ --- дивизор на $X$, $\deg D = 0$. Докажите, что если $D \sim 0$, то $\dim L(D) = 1$, и что если $D \not\sim 0$, то $L(D) = \{0\}$. %[Mir V.3.C]
    \item Пусть $X$ --- компактна рода $g$, $\mathcal{M}(X)\neq \CC$, докажите, что если $\deg D < 2-2g$, то $L^{(1)}(D)=0$. %[Mir V.3.D]
    \item Пусть $X=\CC/L$ --- тор, $L=\ZZ+\ZZ \tau$, $\Im \tau >0$, $\pi:\CC\rightarrow X$ --- естественная проекция, $p_0=\pi(0)$. Докажите следующие свойства:
    \begin{itemize}[noitemsep,topsep=0pt]
        \item Пусть $n\in\ZZ$, $h\in L(n p_0)$. Тогда $\Res_{p_0}(h\,dz)=0$.
        \item Пусть $z$ --- локальная координата в окрестности $p_0$, $h(z)=\sum_{i=-n}^\infty c_i z^i$ --- разложение в ряд Лорана функции $h\in L(n p_0)$. Тогда если $\forall i\leqslant 0$ $c_i=0$, то $h$ тождественно равна $0$.
        \item Пусть $f\in L(2 p_0)$, тогда $\forall x \in X$ $f(x)=f(-x)$.
        \item $\exists! f\in L(2 p_0)$ такая что разложение в ряд Лорана имеет вид: $f(z)=z^{-2} + a_2z^2 + a_4 z^4 + \dots$.
        \item Пусть $g\in L(3 p_0)$, тогда $\forall x \in X$ $g(x)=-g(-x)$.
        \item $\exists! g\in L(3 p_0)$ такая что разложение в ряд Лорана имеет вид: $g(z)=z^{-3} + b_1 z + b_1 z^3 + \dots$.
        \item $\exists A,B\in\CC: g^2=f^3+Af+B$, где $f\in L(2 p_0), g\in L(3 p_0)$ определены как выше. При этом многочлен $w^3+A w + B$ не имеет кратных корней.
    \end{itemize}%[Mir V.3.F, some of the statements]
    \item Докажите, что $\forall f,g \in \mathcal{M}(X)$ $\exists$ дивизор $D$: $f,g\in L(D)$ %[Mir V.3.G]
    \item Пусть $X$ --- компакнтая, и пусть $D>0$ --- дивизор такой, что $\dim L(D)=1+\deg(D)$. Докажите, что $\exists p\in X$: $\dim L(p)=2$, и что $X$ изоморфна римановой сфере $\CC_\infty$. %[Mir V.3.H]
    \item Докажите, что на римановой сфере полная линейная система дивизор неотрицательной степени не содержит базовых точек. %[Mir V.4.B (Ex 4.10)]
    \item Докажите, что на комплексном торе полная линейная система дивизора степени $\geqslant 2$ не содержит базовых точек. %[Mir V.4.B (Ex 4.11)]
    \item Пусть $X$ --- кривая в $\PP^3$ определенная уравнениями $xw=yz$, $xz=y^2$, $yw=z^2$ (скрученная кубика). Используя степень гиперплоскостного дивизора $\div(x)$ докажите, что степень кривой $X$ равна $3$. Определите также $\div (y)$. %[Mir V.2.F] 
\end{enumerate}

\subsubsection{Тема 8. Алгебраические кривые, слабая аппроксимация}

\begin{enumerate}[noitemsep,topsep=0pt]
    \item Докажите, что следующие римановы поверхности являются алгебраическими кривыми:
    \begin{itemize}[noitemsep,topsep=0pt]
        \item риманова сфера $\CC_\infty$;
        \item комплексный тор $\CC/L$;
        \item гиперэллиптическая кривая;
        \item гладкая проективная кривая.
    \end{itemize}%[Mir VI.1.C--F]
    \item Пусть $X$ --- алгебраическая кривая. Используя компактность $X$ докажите, что в $\mathcal{M}(X)$ существует конечное число глобальных мероморфных функций отделяющих точки и касательные. %[Mir VI.1.I]
    \item Пусть $X$ --- компактная риманова поверхность. Докажите, что если $\forall p_1,\dots,p_n \in X$ $\forall m_1, \dots, m_n \in\ZZ$ $\exists f\in \mathcal{M}(X):$ $\nu_{p_i}(f)=m_i$, то $X$ --- алгебраическая кривая. %[Mir VI.1.J]
    \item Пусть $G$ --- конечная группа, действующая эффективно на алгебраической кривой $X$.
    \begin{itemize}[noitemsep,topsep=0pt]
        \item Покажите, что можно задать действие $G$ на $\mathcal{M}(X)$.
        \item Докажите, что $\mathcal{M}(X/G) = \mathcal{M}(X)^G$.
        \item Докажите, что $X/G$ является алгебраической кривой.
    \end{itemize}%[Mir VI.1.L]
    \item Докажите следующие утверждения:
    \begin{itemize}[noitemsep,topsep=0pt]
        \item $\mathcal{M}(\CC_\infty)$ порождается локальной координатой $z$.
        \item $\mathcal{M}(\CC/L)$ порождается отношениями тета-функций.
        \item Если $X$ --- гиперэллиптическая кривая $y^2=h(x)$, то $\mathcal{M}(X)$ порождается $x$ и $y$.
        \item Если $X$ --- гладкая проективная кривая, то $\mathcal{M}(X)$ --- поле рациональных функций.
    \end{itemize} %[Mir VI.1.K]
\end{enumerate}

\subsubsection{Тема 9. Сильная аппроксимация, теорема Римана--Роха}

\begin{enumerate}[noitemsep,topsep=0pt]   
    \item Пусть $f$ --- мероморфная функция, $D$ --- дивизор. Докажите, что определенный в лекции оператор умножения $\mu_f^D:\mathcal{T}[D](X)\rightarrow\mathcal{T}[D-\div(f)](X)$ является изоморфизмом с обратным отображением $\mu_{1/f}^{D-\div(f)}$. %[Mir VI.2.A]
    \item Пусть $D$ --- дивизор, $f,g$ --- глобальные мероморфные функции на $X$, $\alpha_D:\mathcal{M}(X)\rightarrow \mathcal{T}[D](X)$ --- отображение, определенное в лекции. Докажите, что $\mu_f^D (\alpha_D(g))=\alpha_{D-\div(f)}(fg)$. %[Mir VI.2.B]
    \item Докажите, что $D_1 \leqslant D_2$ $\implies$ $\alpha_{D_2}=t_{D_2}^{D_1} \circ \alpha_{D_1}$ %[Mir VI.2.C]
    \item Пусть $X=\CC_\infty$ --- риманова сфера. Докажите, что $H^1(0)=0$ явным образом используя прообраз $\alpha_0$. %[Mir VI.2.E]
    \item Пусть $X=\CC/L$ --- комплексный тор, $p$ --- тождественный элемент группового закона на $X$, $z$ --- локальная координата в окрестности $p$, $Z=z^{-1}\cdot p \in \mathcal{T}[0](X)$. Докажите, что $Z$ не лежит в прообразе $\alpha_0$ (то есть $H^1(0)\neq 0$) %[Mir VI.2.H]
    \item Пусть $f\in \mathcal{M}(X)$, $\omega\in L^{(1)}(-D)$. Докажите, что $f\omega \in L^{(1)}(-D-\div(f))$, и что $\Res_\omega \circ \mu_f^D=\Res_{f\omega}$ в $\mathcal{T}[D+\div(f)](X)$. %[Mir VI.3.A]
    \item Докажите, что если $D$ --- положительный дивизор, $\deg D \geqslant g+1$, то в $L(D)$ существует по крайней мере одна непостоянная функция. %[Mir VI.3.B]
    \item Пусть $X$ --- алгебраическая кривая, $K$ --- канонический дивизор, $D$ --- дивизор степени $\deg D > 0$. Докажите, что $H^{1}(K+D)=0$. %[Mir VI.3.D]
    \item Докажите, что если $g\geqslant 2$, $m\geqslant 2$, то $\dim L(mK)=(g-1)(2m-1)$. %[Mir VI.3.G]
\end{enumerate}

\subsubsection{Тема 10. Некоторые приложения теоремы Римана--Роха}

\begin{enumerate}[noitemsep,topsep=0pt]    
    \item Пусть $X$ --- алгебраическая кривая, $D$ --- дивизор степени $\deg D >0$. Докажите, что $\dim L(D) = 1+\deg D$ $\iff$ $g(X)=0$. %[Mir VII.1.A]
    \item Пусть $X$ --- алгебраическая кривая рода $g(X)=g \geqslant 2$, $D$ --- дивизор степени $\deg D >0$. Докажите, что если $\deg D \leqslant 2g-3$, то $\dim L(D) \leqslant g$. %[Mir VII.1.B]
    \item Пусть $X$ --- компактная Риманова поверхность, $D_1$, $D_2$ --- дивизоры. Докажите, что $\dim L(D_1) + \dim L(D_2) \leqslant \dim L(\min (D_1,D_2))+\dim L(\max (D_1,D_2))$. %[Mir Lemma VII.1.11]
    \item Пусть $X$ --- алгебраическая кривая рода $g(X)=g$, $K$ --- канонический дивизор, $D$ --- дивизор такой, что $\dim L(D) \geqslant 1$ и $\dim L(K-D) \geqslant 1$. Докажите, что $\dim L(D)+\dim L(K-D) \leqslant 1+g$. %[Mir Lemma VII.1.12]
    \item Пусть $X$ --- алгебраическая кривая $g(X)\geqslant 1$, $K$ --- канонический дивизор, $D$ --- дивизор, такой что пространства $L(D)$, $L(K-D)$ ненулевые и $2\dim L(D)\leqslant \deg D + 2 $. Докажите, что тогда $D$ --- либо главный, либо канонический. %[Mir VII.1.D]
    \item Покажите, что разложение в ряд Лорана модулярного инварианта имеет вид: $j(z)=1/q+\sum_{n=0}^\infty a_n q^n$, $a_n\in\ZZ$, $q=e^{2\pi i z}$. %[DS 3.2.2]
    \item Докажите, что $\forall k\in\ZZ$ если $f\in\mathcal{A}_k(\Gamma)$, $f\neq 0$, то $\mathcal{A}_k(\Gamma)=\mathcal{M}(X(\Gamma))\cdot f$. %[DS 3.2.4]
    \item Докажите, что $\mathcal{S}_2(\Gamma) \cong \Omega^1(X(\Gamma))$.  %[DS 3.3.6]
\end{enumerate}

\subsubsection{Тема 11. Нормирования функциональных полей}

\begin{enumerate}[noitemsep,topsep=0pt]
    %\item TODO \cite{Stich} Exercise 1.7
    \item Пусть $K(x)/K$ --- поле рациональных функций, $z\in K(x)\setminus K$, $z=f(x)/g(x)$, где $f,g\in K[x]$ взаимно просты, $\deg z=\max(\deg f,\deg g)$. Докажите, что
    \begin{itemize}[noitemsep,topsep=0pt]
        \item $[K(x):K(z)]=\deg z$;
        \item $K(x)=K(z)$ $\iff$ $z=(ax+b)/(cx+d)$, $a,b,c,d \in K$, $ad-bc \neq 0$.
    \end{itemize} %\cite{Stich} Exercise 1.1
    \item Пусть $\Aut(L/M)$ обозначает группу автоморфизмов расширения $L/M$ и пусть $K(x)/K$ --- поле рациональных функций. Докажите, что
    \begin{itemize}[noitemsep,topsep=0pt]
        \item $\forall\, \sigma\in \Aut(K(x)/K)$ $\exists\, a,b,c,d\in K$: $ad-bc\neq 0$ $\sigma(x) = (ax+b)/(cx+d)$;
        \item $\forall\, a,b,c,d\in K$: $ad-bc\neq 0$ $\exists\, \sigma\in \Aut(K(x)/K)$: $\sigma(x) = (ax+b)/(cx+d)$;
        \item $\Aut(K(x)/K) \cong \GL2(K)/K^*$.
    \end{itemize} %\cite{Stich} Exercise 1.2
    \item Пусть $L^G=\{z\in L: \sigma(z)=z\ \forall\,\sigma\in G\}$ обозначает неподвижное поле подгруппы $G<\Aut(L/M)$, $L/L^G$ является конечным расширением и $[L:L^G]=|G|$. Пусть $G<\Aut(K(x)/K)$ --- конечная подгруппа, $u=\prod_{\sigma\in G} \sigma(x)$, $v=\sum_{\sigma\in G} \sigma(x)$. Докажите, что
    \begin{itemize}[noitemsep,topsep=0pt]
        \item либо $v\in K$, либо $K(v)=K(x)^G$;
        \item либо $u\in K$, либо $K(u)=K(x)^G$.
     \end{itemize} %\cite{Stich} Exercise 1.3
    \item Пусть $K(x)/K$ --- поле рациональных функций. Докажите $\forall\,z\in K(x)$ существует единственное представление вида $z=\sum_{i=1}^r\sum_{j=1}^{k_i}\frac{c_{ij}(x)}{p_i(x)^j} + h(x)$, где
    \begin{itemize}[noitemsep,topsep=0pt]
        \item $p_1(x), \dots, p_r(x)\in K[x]$ --- различны и неприводимы,
        \item $k_i\geqslant 1$,
        \item $c_{ij}(x)\in K[x]$, $\deg(c_{ij}(x))<\deg(p_i(x))$,
        \item $c_{ik_i}\neq 0$,
        \item $h(x)\in K[x]$.
    \end{itemize} %\cite{Stich} Exercise 1.5
    \item Пусть $K(x)/K$ --- поле рациональных функций, $P_\infty=\{f(x)/g(x): f,g \in K[x], \deg(f)<\deg(g)\}$ --- точка $K(x)$. Докажите, что $\deg P_\infty = 1$, $t=1/x$ ---  соответствующий простой элемент, и $\nu_\infty = \deg(g) - \deg(f)$ --- соответствующее нормирование. %\cite{Stich} Proposition 1.2.1 (c)
\end{enumerate}

\subsubsection{Тема 12. Дивизоры и линейные пространства}

\begin{enumerate}[noitemsep,topsep=0pt]
    \item Пусть $F$ --- функциональное поле, $D\in \Div(F)$. Докажите, что
    \begin{itemize}[noitemsep,topsep=0pt]
        \item $x\in L(D)$ $\iff$ $\nu_P(x) \geqslant -\nu_P(D)$ $\forall P \in \mathcal{P}_F$.
        \item $L(D)\neq \{0\}$ $\iff$ $\exists\,D' \in \Div(F)$, $D'\geqslant 0$: $D'\sim D$.
    \end{itemize} %\cite{Stich} Remark 1.4.5
    \item Докажите, что оценка $l(D)\leqslant \deg D + 1$ справедлива $\forall\,D\in \Div(F)$, $\deg D \geqslant 0$. %\cite{Stich} оценка $l(D)$ на стр. 21
    \item Пусть $F=K(x)$ --- поле рациональных функций. найдите базисы следующих линейных пространств: $L(rP_\infty), L(rP_\alpha), L(P_{p(x)})$. %\cite{Stich} Exercise 1.4
    \item Пусть $g(F)>0$, $D\in\Div(F)$, $l(D)>0$. Докажите, что $l(D)=\deg D + 1$ $\iff$ $D$ --- главный. %\cite{Stich} Exercise 1.8
    \item Пусть $F/K$ --- функциональное поле. Докажите, что следующие условия эквиваленты:
    \begin{itemize}[noitemsep,topsep=0pt]
        \item $g(F)=0$;
        \item $\exists\, D\in \Div(F)$: $\deg D = 2$, $l(D)=3$;
        \item $\exists\, D\in \Div(F)$: $\deg D \geqslant 1$, $l(D) > \deg D$;
        \item $\exists\, D\in \Div(F)$: $\deg D \geqslant 1$, $l(D) = \deg D + 1$;
        \item при условии, что $\chr(K)$ $\neq 2$: $\exists\,x,y\in F$: $F=K(x,y)$, $y^2=ax^2+b$ ($a,b\in K^*$).
    \end{itemize} %\cite{Stich} Exercise 1.9
    \item Пусть $\RR(x)$ --- поле рациональных функций над $\RR$. Докажите следующие утверждения:
    \begin{itemize}[noitemsep,topsep=0pt]
        \item Многочлен $f(T)=T^2+(x^2+1)\in \RR(x)[T]$ неприводим над $\RR(x)$. Пусть $F=\RR(x,y)$, $y^2+x^2+1=0$.
        \item $\RR$ --- поле констант для $F$, $g(F)=0$.
        \item $F/\RR$ не является полем рациональных функций.
        \item Все точки $F$ имеют степень $2$.
    \end{itemize} %\cite{Stich} Exercise 1.10
    \item Пусть $E$ --- проективная эллиптическая кривая над полем $F$: $z y^2=x^3+a z x^2 + b z^3$, $4a^3+27b^2\neq 0$, $P, P' \in E$.
    \begin{itemize}[noitemsep,topsep=0pt]
        \item Докажите, что сопоставление $E\rightarrow \Cl^0(E): P\mapsto C_P=[P-P']$ задаёт взаимно-однозначное отображение между $E$ и $\Cl^0(E)$.%\cite{Step} Задача IV.3.15
        \item Докажите, что $C_P+C_Q+C_R=0$ $\iff$ точки $P,Q,R$ лежат на одной проективной прямой. %\cite{Step} Задача IV.3.16
        \item Опишите закон сложения классов $C_P,C_Q$ в группе $\Cl^0(E)$ в терминах точек $P,Q$ кривой $E$. %\cite{Step} Задача IV.3.17
        \item Покажите, что группа $\Cl^0(E)$ имеет ровно четыре элемента второго порядка. Какие точки кривой $E$ соответствуют этим элементам? %\cite{Step} Задача IV.3.18
    \end{itemize} 
\end{enumerate}

\subsubsection{Тема 13. Теорема Римана--Роха для функциональных полей}

\begin{enumerate}[noitemsep,topsep=0pt]
    \item Пусть $F=K(x)$ --- поле рациональных функций, $\mathcal{A}_F$ --- кольцо аделей (распределений). Докажите, что $\mathcal{A}_F=\mathcal{A}_F(0)+F$. %\cite{Stich} Exercise 1.6
    \item Пусть $\chr(K)\neq 2$, $F=K(x,y)$, $y^2=f(x)$, $f\in K[x]$, $\deg f = 2m+1\geqslant 3$. Докажите, что
    \begin{itemize}[noitemsep,topsep=0pt]
        \item $K$ --- поле констант для $F$.
        \item $\exists!$ точка $P\in\mathcal{P}_F$ являющаяся полюсом $x$ и единственным полюсом $y$.
        \item $\forall\, r\in\ZZ_{+}$, $0\leqslant s <r-m$ $1,x,x^2,\dots,x^r,y,xy,\dots,x^s y \in L(2r P)$.
        \item $g(F)\leqslant m$.
    \end{itemize} %\cite{Stich} Exercise 1.11
    \item Пусть $F=\FF_3(x)$ --- поле рациональных функций над конечным полем $\FF_3$. Докажите, что
    \begin{itemize}[noitemsep,topsep=0pt]
        \item $f(T)=T^2+x^4-x^2+1 \in \FF_3(x)[T]$ неприводим.
        \item если $F=\FF_3(x,y)$, $y^2+x^4-x^2+1=0$, $K$ --- поле констант $F$, то $|K|=9$, $F=K(x)$.
    \end{itemize} %\cite{Stich} Exercise 1.12
    \item Пусть $F/K$ --- функциональное поле, $g=g(F)$, $\exists\, P\in \mathcal{P}_F$: $\deg P = 1$. Докажите, что $\exists\, x,y\in F$: $[F:K(x)]=[F:K(y)=2g+1]$ и $F=K[x,y]$. %\cite{Stich} Exercise 1.13
    \item Пусть $F/K$ --- функциональное поле, $D\in\Div(F)$, $\omega\in\Omega_F$ --- ненулевой дифференциал Вейля, $W=(\omega)$. Докажите, что отображение $s:L(W-D) \times \mathcal{A}_F/(\mathcal{A}_F(D)+F) \rightarrow K: (x,\alpha) \mapsto \omega(x\alpha)$ задаёт невырожденное спаривание. (Невырожденное спаривание --- это билинейное отображение векторных пространств $s:V\times U \rightarrow K$ такое, что $\forall\,v\in V\setminus\{0\}$ $\exists\,u\in U: s(v,u)\neq 0$ и $\forall\,u\in U\setminus\{0\}$ $\exists\,v\in V: s(v,u)\neq 0$.) %\cite{Stich} Exercise 1.14
    \item Пусть $K$ --- алгебраически замкнутое поле, $F/K$ --- функциональное поле $g(F)=g$. Докажите, что $\forall\, d\in\ZZ_{+}$, $d\geqslant g$ $\exists\, D\in Div(F)$ такой, что $\deg D = d$ и $l(D)=\deg D +1 - g$. %\cite{Stich} Exercise 1.15
    \item Пусть $D\in\Div(F)$. Докажите, что
    \begin{itemize}[noitemsep,topsep=0pt]
        \item $i(D)\leqslant \max(0,2g(F)-1-\deg D)$;
        \item если $i(D)>0$, то $\forall\, D'\in\Div(F)$ $l(D-D')\leqslant i(D')$.
    \end{itemize} %\cite{Stich} Exercise 1.16
    \item Пусть $C\in\Div(F)$, $|D|$ --- линейная система. Класс дивизора $[C]\in\Cl(F)$ называется примитивным, если $\neg\exists\,B\in\Div(F)$: $B>0$ $\land$ $\forall\,A\in[C]$ $B\leqslant A$. Докажите, что
    \begin{itemize}[noitemsep,topsep=0pt]
        \item $\forall\, D\in\Div(F)$ $\deg D \geqslant 2g(F)$ класс $[D]$ --- примитивный;
        \item если $g(F)\geqslant 1$, то канонический класс является примитивным;
        \item если $g(F)\geqslant 1$, $K\in\Div(F)$ --- канонический, $P\in\mathcal{P}_F$ --- точка степени $1$, то класс $[K+P]$ не может быть примитивным. 
    \end{itemize} %\cite{Stich} Exercise 1.17
\end{enumerate}

\subsubsection{Тема 14. Дзета функция алгебраической кривой}

\begin{enumerate}[noitemsep,topsep=0pt]
    \item Пусть $X=\PP^1(\overline{\FF}_q)$ --- проективная прямая. Покажите, что
    \begin{itemize}[noitemsep,topsep=0pt]
        \item $Z_X(t)=(1-t)^{-1}(1-qt)^{-1}$;
        \item $Z_X(1/(qt))=qt^2 Z_X(t)$.
    \end{itemize} %\cite{Step} Задача V.1.1
    \item Пусть $X\subset\PP^2(\FF_q)$ --- проективная кривая $x^2+y^2-z^2=0$. Покажите, что $Z_X(t)=(1-t)^{-1}(1-qt)^{-1}$. %\cite{Step} Задача V.1.2
    \item Пусть $\chr(\FF_q)\neq 2$, $X\subset\AA^2(\FF_q)$ --- афинная кривая, $N_{q^s}$ --- число $\FF_{q^s}$- рациональных точек этой кривой, $Z_X(t)=\exp\left(\sum_{s=1}^\infty \frac{N_{q^s}}{s} t^s\right)$ --- дзета функция. Докажите следующие утверждения:
    \begin{itemize}[noitemsep,topsep=0pt]
        \item $Z_{\AA^1}(t)=(1-qt)^{-1}$;
        \item $Z_{\AA^n}(t)=(1-q^nt)^{-1}$;
        \item для $X$: $x^2+y^2=1$ $Z_X(t)=(1-t)(1-qt)^{-1}$ при $q\equiv 1\, (4)$ и $Z_X(t)=(1+t)(1-qt)^{-1}$ при $q\equiv 3\, (4)$;
        \item для $X$: $y^2=x^3$ $Z_X(t)=(1-qt)^{-1}$;
        \item для $X$: $y^2=x^3+x^2$ $Z_X(t)=(1-t)(1-qt)^{-1}$.
    \end{itemize} %\cite{Step} Задача V.1.3
    \item Пусть $X\subset\PP^2(\FF_q)$ --- эллиптическая кривая. Докажите, что 
    \begin{itemize}[noitemsep,topsep=0pt]
        \item $Z_X(t) = (1+\alpha t + qt^2)(1-t)^{-1}(1-qt)^{-1}$;
        \item $Z_X(1/(qt)) = Z_X(t)$.
    \end{itemize} %\cite{Step} Задача V.1.4 а-б
    \item Докажите по определению, что $Z_{\PP^n}(t)=(1-t)^{-1} (1-qt)^{-1} \dots (1-q^n t)^{-1}$. %\cite{Step} Задача V.1.6
    \item Пусть $X\subset \PP^n$ --- многообразие размерности $r$, $Z_X(t)=\prod_{i=1}^a (1-\omega_i t) / \prod_{j=1}^b (1-\omega'_j t)$. Докажите, что функциональное уравнение для $Z_X(t)$ равносильно выполнению соотношений $\omega_i \omega_{a-i+1}=q^r$, $\omega'_j \omega'_{b-j+1}=q^r$. %\cite{Step} Задача V.1.7
    \item Пусть $X\subset \AA^n(\FF_q)$ --- афинная гиперповерхность, $N_{q^s}$ --- число $\FF_{q^s}$ рациональных точек, $Z_X(t)=\exp\left(\sum_{s=1}^\infty \frac{N_{q^s}}{s} t^s\right) = 1+\sum_{m=1}^\infty a_m t^m$. Докажите, что 
    \begin{itemize}[noitemsep,topsep=0pt]
        \item $\forall m$ $a_m\in\ZZ_{\geqslant 0}$;
        \item $\forall m$ $a_m\leqslant q^{mn}$.
    \end{itemize} %\cite{Step} Задача V.1.10
\end{enumerate}

\subsubsection{Тема 15. Теорема Хассе--Вейля}

\begin{enumerate}[noitemsep,topsep=0pt]
    \item Пусть $X$ --- кривая над $\FF_q$, $g(X)=g$. Докажите, что $N_{q},N_{q^2},\dots, N_{q^g}$ однозначно определяют $N_{q^s}$ для $s\geqslant g+1$. %\cite{Step} Задача V.2.1
    \item Пусть $f,g\in\FF_q[x]$, $N_{q^s}$ --- число решений системы уравнений $y^n=f(x)$, $z^q-z=g(x)$ в элементах $x,y,z\in\FF_{q^s}$, $\chi$, $\psi$ обозначают мультипликативный и аддитивный характеры поля $\FF_q$. Докажите, что $N_{q^s}=\sum_{\chi^n=\epsilon}\sum_{\psi}\sum_{x\in\FF_{q^s}} \chi_s(f(x))\psi_s(g(x))$. %\cite{Step} Задача V.2.2
    \item Пусть $f,g\in\FF_q[x]$, $\deg f = m$, $\deg g = l$, $(m,n)=1$, $(l,q)=1$. Докажите, что уравнения $y^n=f(x)$, $z^q-z=g(x)$ определяют абсолютную кривую в афинном пространстве $\AA^3(\overline{\FF}_q)$. %\cite{Step} Задача V.2.3
    \item Пусть $f,g\in\FF_q[x]$, $\deg f = l$, $\deg g = n$, $f=f_1^{k_1}\dots f_r^{k_r}$ --- разложение на неприводимые множители, $m=\deg(f_1\dots f_r)$, $\chi$ --- нетривиальный мультипликативный характер порядка $N$, $\psi$ --- нетривиальный аддитивный характер  поля $\FF_q$. Докажите, что если $(l,N)=1$, $(n,q)=1$, то $\left|\sum_{x\in\FF_{q^s}} \chi_s(f(x)) \psi_s(g(x))\right| \leqslant (m+n-1) q^{s/2}$. %\cite{Step} Задача V.2.4
\end{enumerate}

\section{Фонд оценочных средств (ФОС, оценочные и методические материалы) для оценивания результатов обучения по дисциплине (модулю)}

\subsection{Типовые контрольные задания или иные материалы для проведения текущего контроля успеваемости}

Вопросы к зачёту:

\noindent
\begin{longtable}{ | p{0.5cm} | p{11.5cm} | p{2cm} | } 
    \hline
    \bf № & \textbf{Вопрос} & \textbf{Раздел и тема дисциплины} \\
    \hline
    \hline
    \endhead
    1 & Комплексные карты и атласы, структура римановой поверхности, примеры римановых поверхностей & 1 \\ \hline
2 & Голоморфные и мероморыне функции, ряды Лорана, порядки нулей и полюсов & 2 \\ \hline
3 & Мероморфные функции на римановой сфере, торе, проективной прямой и гладкой алгебраической кривой & 2 \\ \hline
4 & Идеал множества точек кривой и его свойства, неприводимые компоненты алгебраических множеств & 3 \\ \hline
5 & Теорема Гильберта о базисе, теорема Гильберта о нулях & 3 \\ \hline
6 & Голоморфные отображения римановых поверхностей, изоморфизм римановой сферы и проективной прямой & 4 \\ \hline
7 & Теорема о локальной нормальная форме, кратность отображения, cтепень отображения. Теорема о сумме порядков мероморфных функций & 4 \\ \hline
8 & Формула Гурвица. Точки ветвления для афинных и проективных кривых & 4 \\ \hline
9 & Действие группы, фактор-пространство как риманова поверхность. Теорема Гурвица о действии конечной группы & 5 \\ \hline
10 & Дифференциальные формы на римановых поверхностях, интегрирование дифференциальных форм, теорема о сумме вычетов & 6 \\ \hline
11 & Дивизоры функций и дифференциальных форм, степень канонического дивизора на компактной римановой поверхности, линейная эквивалентность дивизоров & 6 \\ \hline
12 & Теорема Абеля, степень гладкой проективной кривой, теорема Безу & 6 \\ \hline
13 & Линейные пространства и полные линейные системы, базовые свойства & 7 \\ \hline
14 & Оценка размерности линейных пространств. Базовые точки линейных систем & 7 \\ \hline
15 & Функции с заданными порядками в точке и функции с заданными отрезками рядов Лорана, теорема о слабой аппроксимации & 8 \\ \hline
16 & Конечная порождённость поля рациональных функций алгебраической кривой, степень поля функций и степень кривой & 8 \\ \hline
17 & Дивизоры отрезков рядов Лорана. Задача Миттаг-Леффлера & 9 \\ \hline
18 & Теорема Римана-Роха, двойственность Серра & 9 \\ \hline
19 & Дискретные нормирования алгебраических функциональных полей и их свойства & 11 \\ \hline
20 & Точки функциональных полей, теорема о слабой аппроксимации & 11 \\ \hline
21 & Группа классов дивизоров, линейные пространства. Теорема о равенстве числа нулей и полюсов & 12 \\ \hline
22 & Алгебраический род, теорема Римана & 12 \\ \hline
23 & Теорема Римана-Роха, Теорема о сильной аппроксимации & 13 \\ \hline
24 & Дзета функция алгебраической кривой, её рациональность & 14 \\ \hline
24 & Рациональные точки алгебраических кривых, оценка Хассе-Вейля & 15 \\ \hline

\end{longtable}

Примеры контрольных задач приведены в разделе \ref{problems}.

\section{Ресурсное обеспечение}

\subsection{Перечень основной и дополнительной литературы}

\begin{thebibliography}{Stich}

\bibitem[Mir]{Mir}
R. Miranda, Algebraic Curves and Riemann Surfaces, AMS, 1995.

\bibitem[Stich]{Stich}
H. Stichtenoth, Algebraic Function Fields and Codes, 2nd edition, Springer, 2009.

\bibitem[Степ]{Step}
С.А. Степанов, Арифметика алгебраических кривых, Наука, 1991.

\bibitem[Fult]{Fult}
W. Fulton, Algebraic Curves: An Introduction to Algebraic Geometry, 3rd edition, AMS, 2008.

\bibitem[DS]{DS}
F. Diamond, J. Shurman, A First Course in Modular Forms, Springer, 2005.

\bibitem[ВНЦ]{VNTs}
С.Г. Влэдуц, Д.Ю. Ногин, М.А. Цфасман, Алгеброгеометрические коды. Основные понятия, МЦНМО, 2003.

\bibitem[Шаф]{Sch}
И.Р. Шафаревич, Основы алгебраической геометрии, 3-е изд., МЦНМО, 2007.

\bibitem[FK]{FK}
H.M. Farkas, I. Kra, Riemann Surfaces, 2nd edition, Springer, 1992.

\bibitem[КЛП]{KLP}
М.Э. Казарян, С.К. Ландо, В.В. Прасолов, Алгебраические кривые. По направлению к пространствам модулей, МЦНМО, 2019.

\end{thebibliography}

\subsection{Перечень лицензионного программного обеспечения, в том числе отечественного производства}

При реализации дисциплины может быть использовано следующее программное обеспечение:
\begin{itemize}
    \item Операционная система Linux (Свободно-распространяемое ПО) / MacOS / Windows;
    \item Язык программирования Python и система компьютерной алгебры SageMath. Свободно-распространяемое ПО;
    \item Среда разработки Jupyter / VS Code / Vim (Emacs), ... Свободно-распространяемое ПО;
    \item Издательская система LaTeX. Свободно-распространяемое ПО.
\end{itemize}

\subsection{Перечень профессиональных баз данных и информационных справочных систем}

\begin{enumerate}
    \item http://www.edu.ru --- портал Министерства образования и науки РФ;
    \item http://www.ict.edu.ru --- система федеральных образовательных порталов «ИКТ в образовании»;
    \item http://www.openet.ru --- Российский портал открытого образования;
    \item http://www.mon.gov.ru  --- Министерство образования и науки Российской Федерации;
    \item http://www.fasi.gov.ru --- Федеральное агентство по науке и инновациям.
\end{enumerate}

\subsection{Перечень ресурсов информационно-телекоммуникационной сети «Интернет»}

\begin{enumerate}
    \item Math-Net.Ru [Электронный ресурс] : общероссийский математический портал / Математический институт им. В. А. Стеклова РАН ; Российская академия наук, Отделение математических наук. - М. : [б. и.], 2010. - Загл. с титул. экрана. - Б. Ц. URL: http://www.mathnet.ru;
    \item Университетская библиотека Online [Электронный ресурс] : электронная библиотечная система / ООО "Директ-Медиа" . - М. : [б. и.], 2001. - Загл. с титул. экрана. - Б. ц.  URL: www.biblioclub.ru;
    \item Универсальные базы данных EastView [Электронный ресурс] : информационный ресурс / EastViewInformationServices. - М. : [б. и.], 2012. - Загл. с титул. экрана. - Б. Ц. URL: www.ebiblioteka.ru;
    \item Научная электронная библиотека eLIBRARY.RU [Электронный ресурс] : информационный портал / ООО "РУНЭБ" ; Санкт-Петербургский государственный университет. - М. : [б. и.], 2005. - Загл. с титул. экрана. - Б. Ц. URL: www.eLibrary.ru.
\end{enumerate}

\subsection{Описание материально-технического обеспечения}

Образовательная организация, ответственная за реализацию данной Программы, располагает соответствующей материально-технической базой, включая современную вычислительную технику, объединенную в локальную вычислительную сеть, имеющую выход в Интернет. Используются специализированные компьютерные классы, оснащенные современным оборудованием. Материальная база соответствует действующим санитарно-техническим нормам и обеспечивает проведение всех видов занятий (лекционных, практических, семинарских, лабораторных, дисциплинарной и междисциплинарной подготовки) и научно-исследовательской работы обучающихся, предусмотренных учебным планом.

\section{Методические рекомендации по организации изучения дисциплины}

\subsection{Формы и методы преподавания дисциплины}

\begin{itemize}
    \item Используемые формы и методы обучения: лекции и семинары, самостоятельная работа студентов.
    \item В процессе преподавания дисциплины преподаватель использует как классические формы и методы обучения (лекции и практические занятия), так и активные методы обучения. 
    \item При проведении лекционных занятий преподаватель использует аудиовизуальные, компьютерные и мультимедийные средства обучения, а также демонстрационные и наглядно-иллюстрационные (в том числе раздаточные) материалы.
    \item Семинарские (практические) занятия по данной дисциплине проводятся с использованием компьютерного и мультимедийного оборудования, при необходимости - с привлечением полезных Интернет-ресурсов и пакетов прикладных программ. 
\end{itemize}

\subsection{Методические рекомендации преподавателю}

Перед началом изучения дисциплины преподаватель должен ознакомить студентов с видами учебной и самостоятельной работы, перечнем литературы и интернет-ресурсов, формами текущей и промежуточной аттестации, с критериями оценки качества знаний для итоговой оценки по дисциплине. 
При проведении лекций, преподаватель:
\begin{itemize}[noitemsep,topsep=0pt]
    \item формулирует тему и цель занятия;
    \item излагает основные теоретические положения;
    \item сопровождает теоретические положения наглядными примерами (численные результаты и частные случаи);
    \item в конце занятия дает вопросы для самостоятельного изучения.
\end{itemize}

Во время выполнения заданий в учебной аудитории студент может консультироваться с преподавателем, определять наиболее эффективные методы решения поставленных задач. Если какая-то часть задания остается не выполненной, студент может продолжить её выполнение во время внеаудиторной самостоятельной работы.

Перед выполнением внеаудиторной самостоятельной работы преподаватель проводит инструктаж (консультацию) с определением цели задания, его содержания, сроков выполнения, основных требований к результатам работы, критериев оценки, форм контроля и перечня источников и литературы.

Для оценки полученных знаний и освоения учебного материала по каждому разделу и в целом по дисциплине преподаватель использует формы текущего, промежуточного и итогового контроля знаний обучающихся.

\vspace{8pt}
{\bf Для семинарских занятий}

Подготовка к проведению занятий проводится регулярно. Организация преподавателем семинарских занятий должна удовлетворять следующим требования: количество занятий должно соответствовать учебному плану программы, содержание планов должно соответствовать программе, план занятий должен содержать перечень рассматриваемых вопросов.

Во время семинарских занятий используются словесные методы обучения, как беседа и дискуссия, что позволяет вовлекать в учебный процесс всех слушателей и стимулирует творческий потенциал обучающихся. 

При подготовке семинарскому занятию преподавателю необходимо знать план его проведения, продумать формулировки и содержание учебных вопросов, выносимых на обсуждение. 

В начале занятия преподаватель должен раскрыть теоретическую и практическую значимость темы занятия, определить порядок его проведения, время на обсуждение каждого учебного вопроса. В ходе занятия следует дать возможность выступить всем желающим и предложить выступить тем слушателям, которые проявляют пассивность.

Целесообразно, в ходе обсуждения учебных вопросов, задавать выступающим и аудитории дополнительные и уточняющие вопросы с целью выяснения их позиций по существу обсуждаемых проблем, а также поощрять выступление с места в виде кратких дополнений. На занятиях проводится отработка практических умений под контролем преподавателя

\vspace{8pt}
{\bf Для практических занятий}

Подготовка преподавателя к проведению практического занятия начинается с изучения исходной документации и заканчивается оформлением плана проведения занятия.

На основе изучения исходной документации у преподавателя должно сложиться представление о целях и задачах практического занятия и о том объеме работ, который должен выполнить каждый обучающийся. Далее можно приступить к разработке содержания практического занятия. Для этого преподавателю (даже если он сам читает лекции по этому курсу) целесообразно вновь просмотреть содержание лекции с точки зрения предстоящего практического занятия. Необходимо выделить понятия, положения, закономерности, которые следует еще раз проиллюстрировать на конкретных задачах и упражнениях. Таким образом, производится отбор содержания, подлежащего усвоению.

Важнейшим элементом практического занятия является учебная задача (проблема), предлагаемая для решения. Преподаватель, подбирая примеры (задачи и логические задания) для практического занятия, должен представлять дидактическую цель: привитие каких навыков и умений применительно к каждой задаче установить, каких усилий от обучающихся она потребует, в чем должно проявиться творчество студентов при решении данной задачи.

Преподаватель должен проводить занятие так, чтобы на всем его протяжении студенты были заняты напряженной творческой работой, поисками правильных и точных решений, чтобы каждый получил возможность раскрыться, проявить свои способности. Поэтому при планировании занятия и разработке индивидуальных заданий преподавателю важно учитывать подготовку и интересы каждого студента. Педагог в этом случае выступает в роли консультанта,  способного вовремя оказать необходимую помощь, не подавляя самостоятельности и инициативы студента.

\subsection{Методические рекомендации студентам по организации самостоятельной работы}

Приступая к изучению новой учебной дисциплины, студенты должны ознакомиться с учебной программой, учебной, научной и методической литературой, имеющейся в библиотеке университета, встретиться с преподавателем, ведущим дисциплину, получить в библиотеке рекомендованные учебники и учебно-методические пособия, осуществить запись на соответствующий курс в среде электронного обучения университета.

Глубина усвоения дисциплины зависит от активной и систематической работы студента на лекциях и практических занятиях, а также в ходе самостоятельной работы, по изучению рекомендованной литературы. 

На лекциях важно сосредоточить внимание на ее содержании. Это поможет лучше воспринимать учебный материал и уяснить взаимосвязь проблем по всей дисциплине. Основное содержание лекции целесообразнее записывать в тетради в виде ключевых фраз, понятий, тезисов, обобщений, схем, опорных выводов. Необходимо обращать внимание на термины, формулировки, раскрывающие содержание тех или иных явлений и процессов, научные выводы и практические рекомендации. Желательно оставлять в конспектах поля, на которых делать пометки из рекомендованной литературы, дополняющей материал прослушанной лекции, а также подчеркивающие особую важность тех или иных теоретических положений. С целью уяснения теоретических положений, разрешения спорных ситуаций необходимо задавать преподавателю уточняющие вопросы. Для закрепления содержания лекции в памяти, необходимо во время самостоятельной работы внимательно прочесть свой конспект и дополнить его записями из учебников и рекомендованной литературы. Конспектирование читаемых лекций и их последующая доработка способствует более глубокому усвоению знаний, и поэтому являются важной формой учебной деятельности студентов.

\vspace{8pt}
{\bf Методические указания для обучающихся по подготовке к семинарским занятиям}

Для того чтобы семинарские занятия приносили максимальную пользу, необходимо помнить, что упражнение и решение задач проводятся по вычитанному на лекциях материалу и связаны, как правило, с детальным разбором отдельных вопросов лекционного курса. Следует подчеркнуть, что только после усвоения лекционного материала с определенной точки зрения (а именно с той, с которой он излагается на лекциях) он будет закрепляться на семинарских занятиях как в результате обсуждения и анализа лекционного материала, так и с помощью решения проблемных ситуаций, задач.

При этих условиях студент не только хорошо усвоит материал, но и научится применять его на практике, а также получит дополнительный стимул (и это очень важно) для активной проработки лекции.

При самостоятельном решении задач нужно обосновывать каждый этап решения, исходя из теоретических положений курса. Если студент видит несколько путей решения проблемы (задачи), то нужно сравнить их и выбрать самый рациональный. Полезно до начала вычислений составить краткий план решения проблемы (задачи). Решение проблемных задач или примеров следует излагать подробно, вычисления располагать в строгом порядке, отделяя вспомогательные вычисления от основных. Решения при необходимости нужно сопровождать комментариями, схемами, чертежами и рисунками. 

Следует помнить, что решение каждой учебной задачи должно доводиться до  окончательного логического ответа, которого требует условие, и по возможности с выводом. Полученный ответ следует проверить способами, вытекающими из существа данной задачи. Полезно также (если возможно) решать несколькими способами и сравнить полученные результаты. Решение задач данного типа нужно продолжать до приобретения твердых навыков в их решении. 

При подготовке к семинарским занятиям следует использовать основную литературу из представленного списка, а также руководствоваться приведенными указаниями и рекомендациями. Для наиболее глубокого освоения дисциплины рекомендуется изучать литературу, обозначенную как «дополнительная» в представленном списке.

\vspace{8pt}
{\bf Методические указания для обучающихся по подготовке к практическим занятиям}

Целью практических занятий по данной дисциплине является закрепление теоретических знаний, полученных при изучении дисциплины. 

При подготовке к практическому занятию целесообразно выполнить следующие рекомендации: изучить основную литературу; ознакомиться с дополнительной литературой, новыми публикациями в периодических изданиях: журналах, газетах и т. д.; при необходимости доработать конспект лекций. При этом учесть рекомендации преподавателя и требования учебной программы.

При выполнении практических занятий основным методом обучения является самостоятельная работа студента под управлением преподавателя. На них пополняются теоретические знания студентов, их умение творчески мыслить, анализировать, обобщать изученный материал, проверяется отношение студентов к будущей профессиональной деятельности.

Оценка выполненной работы осуществляется преподавателем комплексно: по результатам выполнения заданий, устному сообщению и оформлению работы. После подведения итогов занятия студент обязан устранить недостатки, отмеченные преподавателем при оценке его работы.

\vspace{8pt}
{\bf Методические указания для самостоятельной работы обучающихся}

Прочное усвоение и долговременное закрепление учебного материала невозможно без продуманной самостоятельной работы. Такая работа требует от студента значительных усилий, творчества и высокой организованности. В ходе самостоятельной работы студенты выполняют следующие задачи: дорабатывают лекции, изучают рекомендованную литературу, готовятся к практическим занятиям, к коллоквиуму, контрольным работам по отдельным темам дисциплины. При этом эффективность учебной деятельности студента во многом зависит от того, как он распорядился выделенным для самостоятельной работы бюджетом времени.

Результатом самостоятельной работы является прочное усвоение материалов по предмету согласно программы дисциплины. В итоге этой работы формируются профессиональные умения и компетенции, развивается творческий подход к решению возникших в ходе учебной деятельности проблемных задач, появляется самостоятельности мышления.


\end{document}