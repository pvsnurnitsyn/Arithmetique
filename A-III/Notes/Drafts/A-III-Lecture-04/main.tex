\documentclass[a4paper,12pt]{article}
\usepackage[utf8]{inputenc}
\usepackage[russian]{babel}
\usepackage{amsmath,amsfonts,amssymb}
\usepackage{graphicx}
\usepackage{float}
\usepackage{booktabs}
\usepackage{geometry}

\usepackage{amsmath}   % можно, но не обязательно
\usepackage{tikz}
\usepackage{tikz-cd}

\geometry{margin=2cm}

\begin{document}

% ------------------- ТИТУЛЬНЫЙ ЛИСТ -------------------
\begin{titlepage}
\centering
\textbf{\Large Московский государственный университет имени М.В. Ломоносова}\\[0.5cm]
\textbf{\large Факультет вычислительной математики и кибернетики}\\[2cm]

\vspace{4cm}
\textbf{\Huge Лекция №4. Отображение Римановых поверхностей. }\\[0.5cm]


\vfill
Москва, 2025
\end{titlepage}

% ------------------- 1. ВВЕДЕНИЕ -------------------
\section{Введение}
Было рассмотрено:

\textbf{Определение 4.1}
$X \: -$ \textbf{РП} $-$ связные хаусдорфовы топологические пространства со счетной базой, на которых задана комплексная структура атласом $A: \left\{ \phi_\alpha: U_\alpha \to V_\alpha  \right\}$, где $U_\alpha \subset X \: -$ открытое множество пространства РП, $V_\alpha \subset \mathbb{C} \: -$ открытое множество на комплексной плоскости, $\phi_\alpha \: -$ гомеоморфизм.

\textbf{Определение 4.2}
\textbf{Функция склейки:} $T_{\alpha_\beta} = \phi_\alpha \circ {\phi_\alpha}^{-1}$ $-$ голоморфны на $U_\alpha \cap U_\beta$.

\textbf{Определение 4.3}
\textbf{Комплекснозначная функция на РП} $- \:$ $f: X \to \mathbb{C} - \:$ голоморфна в точке $p \in X$, если $\exists$ карта $\phi: U \to V, p \in U, f \circ \phi^{-1}$ голоморфна.

\textbf{Определение 4.4}
Функция $f: X \to \mathbb{C} - \: -$ \textbf{мероморфна}, если она либо голоморфна, либо имеет устранимую особую точку, либо полюс.

\textbf{Определение 4.5}
Пусть $X, Y -$ РП, отображение$F:X \to Y$ называется голоморфным в точке $p \in X \rightleftharpoons \exists$ карта $\phi_1: U_1 \to V_1, U_1 \subset X, p \in U_1$ и $\exists$ карта $\phi_2: U_2 \to V_2, U_2 \subset Y, F(p) \in U_2$ такие что $\phi_2 \circ F \circ \phi_1^{-1}$ голоморфна в точке $\phi_1(p)$.



\textbf{Свойство 4.1}
$F - $ голоморфна на $W \subset X - $ открытое множество $\rightleftharpoons F$ голоморфно в $p \in W \: \forall p \in W$. $F: X \to Y -$ голоморф. отображение, если голоморф. $\forall p \in X$. 

\textbf{Лемма 4.1}
\begin{enumerate}
    \item $F$ голоформф. в $p \Leftrightarrow$ вместо "$\exists \:$" можно поставить квантер "$\forall$"
    \item $F$ голоморфна на $W \Leftrightarrow$ $\exists \left\{ \varphi_1^{\left( i \right)}: u_1^{\left( i \right)} \to v_1^{\left( i \right)} \right\}, \left\{ \varphi_2^{\left( j \right)}: u_2^{\left( j \right)} \to v_2^{\left( j \right)} \right\} - \:$ набор карт на $X Y$, такие что $W\subset \bigcup_{i}^{}u_1^{\left( i \right)}, \: F(w)\subset \bigcup_{i}^{}u_2^{\left( j \right)}, \: \varphi_2^{\left( j \right)}\circ F\circ (\varphi_1^{\left( i \right)})^{-1} - \:$ голоморф. в области определения.
\end{enumerate}

\textbf{Лемма 4.2}
\begin{enumerate}
    \item $F: X \to Y, G Y \to Z - \:$ голоморф. $\Rightarrow G \circ F: X\to Z - \:$ голоморф.
    \item $F: X \to Y- \:$ голоморф., $g: Y \to \mathbb{C} - \:$ голоморф. на $W \subset Y \Rightarrow g \circ F - \:$ голоморф. функция на $F^{-1}(W)$.
    \item $F: X \to Y- \:$ голоморф., $g - \:$ мераморф. на $W \in Y F(X) \nsubseteq \left\{ \text{полюсам }g \right\} \Rightarrow g \circ F - \:$ мераморф.
\end{enumerate}

\vspace{1\baselineskip}
Пусть $o_x, \mu_x$ полюс/поле голоморф./мераморф. функций.
\textbf{Замечание} Свойство 2.3 эквивалентно: $g \in O_{W,Y} \to g \circ F \in O_{F^{-1}(W), X}$ то есть $F^{*}(g) = g \circ F$

Есть $F: X \to Y$ можем построить $F^{*}: O_{W,Y} \to O_{F^{-1}(W), X} $

Аналогично можно построить отображение мераформфных функций.

\textbf{Определение 4.6}
$f: x \to Y - \:$\textbf{ изоморфизм }РП $\rightleftharpoons  f - \:$ голоморф. вз. однознач. отображение $f^{-1}: Y \to X$ голоморф. Если $Y=X$ то $f$ называется \textbf{автоморфизм.}

\vspace{1\baselineskip}
\textbf{Теорема 4.1} $\mathbb{С}_{\infty} \: \: \mathbb{P}' =  \mathbb{P}' (\mathbb{C})$ эти поверхности как РП изоморфны.

\textbf{Доказательство:}
$F: \mathbb{P}' \to \mathbb{С}_{\infty}:\left\{ z: w \right\} = \left( 2Rez \bar{w}/, 2Im(z\bar{w})  \right), \frac{\left| z \right|^2 - \left| w \right|^2}{\left| z \right|^2 + \left| w \right|^2} $

\vspace{1\baselineskip}
\textbf{Аналоги следствия из ТФКП}
\vspace{1\baselineskip}

\textbf{Лемма 4.3}
$F:X \to Y - \:$ непостоянное, голоморф. отображение, то $F - \:$ открытое отображение.


\textbf{Лемма 4.4} 
$F:X \to Y - \:$ голоморф., инъективное отображение, то $F:X \to F(X) - \:$ изоморфизм.


\textbf{Лемма 4.5}
$F,G:X \to Y - \:$ голоморф., $S \subset X$ множество имеющее предельную точку и $\forall p \in S F(p)=G(p)$. Тогда $F=G$.


\textbf{Лемма 4.6}
$F:X \to Y - \:$ непостоянное, голоморф. отображение $X - \:$ компакт $\Rightarrow F - \:$ сюръекция на $Y - \:$ компакт.

\textbf{Лемма 4.7}
$F:X \to Y - \:$ непостоянное, голоморф. $\forall y \in Y, \: F^{-1}(y) - \:$ дискретное множество. Если $X - \:$ компакт $\Rightarrow F^{-1}(y) - \:$ конечно

\textbf{Доказательство:}
$w = g(z, 2\psi(F(\varphi^{-1}(z))))$ голоморф. Пусть $y \in Y, x \in F^{-1}(Y)$. Карты выбираем с центром в $x,y \Rightarrow \varphi(x) =0, \psi(y) = 0$. Тогда прообразу $x \in F^{-1}(y)$ будет соответствовать $z: g(z)= 0$ множества нудей голоморф. функций дискретно.

\vspace{1\baselineskip}
\textbf{Замечание:} $F:X \to \mathbb{C} = Y - \:$ голоморф. функция, то она голоморфна, как отображение. 
Пусть $f:X \to \mathbb{C}$ мераморф., рассмотрим ее как функцию $F: x \to \mathbb{С}_{\infty} =  \mathbb{С} \cup \left\{ \infty  \right\} $

$$
F(x)=\begin{cases}
f(x),& x - \text{не полюс} f\\
\infty,& x - \text{полюс} f
\end{cases}
$$

Тогда $F - \:$  голоморф. отображение РП.
Таким образом мераморф. функциям $f:X \to \mathbb{C} \longleftarrow $ голоморф. отображение $F: x \to \mathbb{С}_{\infty}$

Для комплексного тора $\mathbb{C}\L, L - \:$ решетка.

\vspace{1\baselineskip}
\textbf{Теорема 4.2}
Пусть $\forall f \in \mu_{\mathbb{C} / L}$ — непостоян.

Тогда
\[
 \sum_{p}^{} \nu_p(f)=0
\]

\textbf{\text{Доказательство.}}

Пусть
\[
L = L_\omega = L(1,\omega),
\qquad
\omega \in \mathbb{H}
\quad (\operatorname{Im}z > 0).
\]
Утверждение равносильно с условием кратности одно число нулей и полюсов.
Предположим, что число нудей и полюсов различны:
$(p_i), i=1,...,n - \:$ нули с повторениями.
$(q_j), j=1,...,n - \:$ полюса с повторениями.

Пусть $n < m $, то есть $n \neq m $, если вдруг иначе заменяем $f$ на $\frac{1}{f}$.
$$\exists p_{n+1},...,p{m}: \sum_{i=1}^{n}p_i+\sum_{i=n+1}^{m}pi= \sum_{j=1}^{m}q_j $$

$p_i,\; q_j \in \mathbb{C}/L.$

$\pi : \mathbb{C} \to \mathbb{C}/L :
\; z \mapsto z \bmod L.$

Поднимим $p_i,\; q_j$.
до
\[
x_i \in \pi^{-1}(p_i),
\qquad
y_j \in \pi^{-1}(q_j).
\]

Тогда
\[
\sum_{i=1}^{m} x_i = \sum_{j=1}^{m} y_j
\qquad
(\text{в } \mathbb{C}).
\]

(возможно, так как $L$ — решётка)

Напомним:
\[
\Theta_\omega(z)
=
\sum_{\ell \in \mathbb{Z}}
e^{\pi i( \ell^2lz + l^2w)}.
\]

\[
\Theta_\omega^{(x)}(z)
=
\Theta_\omega\!\left(
z - \frac{1}{2} - \frac{\omega}{2} - x
\right).
\]

Т:
Если $x_i,\; y_j$ :
\[
\sum_i x_i - \sum_j y_j \in \mathbb{Z},
\]
то
\[
\prod_i \Theta_\omega^{(x_i)} \Big/ \prod_j \Theta_\omega^{(y_j)}
\in \mu_{\mathbb{C}/L}.
\]

$x_i$ — нули,
\qquad
$y_j$ — полюса.

\[
\sum x_i - \sum y_j = 0 \in \mathbb{Z}.
\]

\[
\Rightarrow\quad
R = \prod \Theta_\omega^{(x_i)} \Big/ \prod \Theta_\omega^{(y_j)},
\qquad
R = R(z).
\]

\[
R \in \mu_{\mathbb{C}/L)}.
\]

Рассмотрим
\[
g = \frac{R}{f}.
\]

$g$ не имеет полюсов, т.е. может иметь только нули.

Нули: $p_{n+1},\ldots,p_m$,
и т.е. $g$ — голоморф.

$\mathbb{C}/L$ — компакт,
\[
\Rightarrow\quad g = \text{const} = 0.
\]

\[
\Rightarrow\quad R \equiv 0,
\qquad
\text{но } R \not\equiv \text{const}\; (\text{противоречие})
\]

\[
\Rightarrow\quad m = n.
\]

\[
\Rightarrow\quad \sum \nu_p(f) = 0.
\]

\vspace{1\baselineskip}
\textbf{Теорема (о локальной нормальной форме).}

$F : X \to Y$ — непотоянное голоморф отображение,
$p \in X$.

$F$ определено в $p$.

\[
\exists\ ! m \in \mathbb{Z}_{> 1} :
\]

\[
\forall\, \varphi_2 : u_2 \to v_2 :
\varphi_2(F(p)) = 0,
\]

\[
\exists\, \varphi_1 : u_1 \to v_1 :
\varphi_1(p) = 0 :
\]

\[
\varphi_2\!\left(
F\bigl(\varphi_1^{-1}(z)\bigr)
\right)
=
z^m.
\]

\textbf{Доказательство.}

Пусть
\[
\varphi_2 : u_2 \to v_2 :
\varphi_2(F(p)) = 0.
\]

\[
\forall\, \psi : U \to V
\text{ — карта на } X,
\quad
\psi(p) = 0.
\]

т.к.
\[
\varphi_2\!\left(
F(\psi^{-1}(z))
\right)
\text{ — голоморфна.}
\]
\[
T(w)
=
\varphi_2\!\left(
F\bigl(\psi_1^{-1}(z)\bigr)
\right)
=
\sum_{i \ge m} c_i w^i.
\]

где
\[
m \ge 0,
\quad \text{если } m \ge 1,
\quad \text{то } T(0) = 0.
\]

\[
\Rightarrow\quad
T(w) = w^m\,\xi(w),
\qquad
\xi \text{ — голоморфна в окр},
\quad
w = 0,
\quad
\xi(0) \ne 0.
\]

\[
\Rightarrow\quad
\exists\, R(w) :
\ \xi(w) = R(w)^m,
\qquad
R \text{ — голоморфна}.
\]

\[
T(w) = (w R(w))^m
=
(\eta(w))^m,
\qquad
\eta \text{ — голоморфна}.
\]

\[
\eta' = R + w R',
\qquad
\eta'(0) = R(0) \ne 0.
\]

\[
\Rightarrow\quad
\eta \text{ — будет. в окр. } 0.
\]

Тогда \[ \varphi_1 = \eta \circ \psi - \text{карта}\]
\[
\varphi_2( F(\varphi_1^{-1}(z)))
=
\varphi_2(F( \psi^{-1}(\eta_1^{-1}(z))))
=
T\bigl(\eta^{-1}(z)\bigr)
=
T(w).
\]

\[
T(w) = \eta(w)^m
\;\Rightarrow\;
\text{локально } z \mapsto z^m.
\]

\textbf{Определение 4.7}
Число $m$ называется кратности $F$ в точке $p$,
обозн. $m_p(F)$.

\textbf{Лемма 4.7}
Пусть $F : X \to Y$ — голоморф.отображение в окрестности $p \in X$, $\varphi, \psi$ — карты.

\[
w = h(z)= \psi(F(\varphi^{-1}(z)))
\]

Тогда
\[
m_p(F) = 1 + \nu_{z_0}\bigl(h'(z)\bigr),
\qquad
h' = \frac{dh}{dz}.
\]

\[
h(z)
=
h(z_0)
+
\sum_{i > m_p(F)} c_i (z - z_0)^i.
\]

\vspace{1\baselineskip}
\textbf{Замечание}
$w - w_0 = h(z) - h(z_0)$.


\textbf{Лемма 4.8}
$F : X \to Y$ — голоморфно.

\{$\forall\, p \in X$ :
$F$ — опр. в $p$,
$m_p(F) \ge 2$\} дичкретное множество.

\textbf{Доказательство}

Следует из того что соответвует нулям $h(z)$
$
h \text{ — голоморфна},
\;\Rightarrow\;
\text{нулей дискретно}.
$

\vspace{1\baselineskip}
\textbf{Определение 4.8}
$p \in X$ — называется точкой ветвления 
$F : X \to Y$,
если
$
m_p(F) \ge 2.
$

\textbf{Лемма 4.9}
Пусть $f : X \to \mathbb{C}$ — мераморф.
$F : X \to \mathbb{C}_\infty$ — соответствующее голоморфное отображение.
\[
1)\; p \in X \text{ — нуль } f
\;\Rightarrow\;
m_p(f) = \nu_p(f).
\]

\[
2)\; p \in X \text{ — полюс }
\;\Rightarrow\;
m_p(f) = -\nu_p(f).
\]

\[
3)\; p \text{ — не нуль и не полюс }
\;\Rightarrow\;
m_p(f) = \nu_p\!\bigl(f - f(p)\bigr).
\]

Для кривых(без док.)

\vspace{1\baselineskip}
\textbf{Теорема 4.3}
Пусть $X$ :
$f(x,y)=0$ —
гладкая аффинная кривая,
$p$ — точка ветвления 
\[
\Leftrightarrow\quad
\frac{\partial f}{\partial y}(p)\neq 0,
\qquad
\pi : X \to \mathbb{C},
\ (x,y)\mapsto x.
\]

\vspace{1\baselineskip}
\textbf{Теорема 4.4}
$X$ :
$F(x,y,z)=0$ — гладкая проективная кривая,
$\pi : X \to \mathbb{P}'$ — проекция
на $y=0$, т.е.
\[
\pi : (x,y,z)\mapsto (x:z).
\]

$p\in X$ — точка ветвления
$
\Leftrightarrow\quad
\frac{\partial f}{\partial y}(p)=0.
$

\textbf{Лемма 4.10}
$X, Y$ — компактные РП.

$F : X \to Y$ — голоморф., непостоян.

\[
\forall\, y \in Y \quad \text{определим }
d_y(F) = \sum_{p \in F^{-1}(y)} m_p(F).
\]

Тогда
$d_y(F)$ не зависит от $y$.

$
\forall\, y \in Y \quad d_y(F) = d.
$

\textbf{Доказательство(идея)}
\begin{itemize}
    \item $F^{-1}(y) = \{x_1,\dots,x_n\}$ —
    конечное,
    т.к. $Y$ — компакт.
    
    $\forall$ окр. $x_i$
    $\exists$ лок. координата $z_i, (z_i = \varphi(x),\ x \in \text{окр. } x_i),$
    \[
    F : z_i \mapsto z_i^{m_i}.
    \]
    \item $w = z^m \;-\; \text{рассмотрим}, f : D \to D, D = \{\, z \in \mathbb{C} : |z| \le 1 \,\}.$

\[
z = 0 \;-\; \text{точка ветвл.},
\qquad
f^{-1}(0) = \{0\}.
\]

\[
m_0(f) = m.
\]

\[
\forall\, w \in D \setminus \{0\},
\qquad
f^{-1}(w) = \{ \sqrt[m]{w} \}.
\]

\[
|f^{-1}(w)| = m,
\qquad
\forall\, p \in f^{-1}(w),
\quad
m_p(f) = 1.
\]

\[
d_w(f) =
\begin{cases}
m, & w = 0, \\[4pt]
\sum_{i=1}^{m} 1 = m, & w \neq 0.
\end{cases}
\]

\[
\text{т.е. } f : D \to D,\quad z \mapsto z^m \text{верно}.
\]
\item $
\text{Из компактности и связности:}
$

\[
y \mapsto d_y(f) = m_i,
\quad
\text{— лок. постоянна} \Rightarrow\quad m_i = m = \text{const}.
\]
\end{itemize}

\textbf{Определение 4.9}
$F : X \to Y$ — голоморфное, непостоян. отображение,
$X, Y$ — компакты. $d(F) = d_y(F)$ называется степенью отображения $d(F) = \deg(F)$.

\textbf{Лемма 4.11}
$F : X \to Y$ — -  — 
\[
F \text{ — изомарфизм } \Longleftrightarrow \deg F = 1.
\]

\vspace{1\baselineskip}
\textbf{Теорема 4.5}
$X$ — компактная РП,
$\exists\, f \in \mu_X$ :
$f$ имеет единственный полюс $p \in X, \nu_p(f) = -1.$
Тогда $X$ изоморфно $\mathbb{C}_{\infty}$.


\textbf{Доказательство.}

Пусть
\[
F : X \to \mathbb{C}_\infty.
\]

\[
m_p(F) = 1,
\qquad
F^{-1}(\infty) = \{p\},
\quad
\text{и т.к. единственный полюс}.
\]

\[
\deg F = d_\infty(F) = 1
\;\Longleftrightarrow\;
F \text{ — изом.}
\]

\vspace{1\baselineskip}
\textbf{Теорема 4.6}
$X$ — компактная РП,
$f \in \mu_X \Rightarrow\quad
\sum_{p \in X} \nu_p(f) = 0.$

\textbf{Доказательство}

$F : X \to \mathbb{C}_\infty$ —
соответствующее $f$ голоморф. отображение. Если 
$(x_i) \; - \; \text{нули } f, (y_j) \; - \; \text{полюса } f.$

\[
F : x_i \mapsto 0,
\qquad
y_j \mapsto \infty.
\]

\[
\sum m_{x_i}(F)
=
\deg F
=
\sum m_{y_j}(F),
\]

\[
\text{т.е.}\quad
\nu_{x_i}(f)
=
-\nu_{y_j}(f).
\]

\[
\Rightarrow\quad
0 = d - d
=
\sum \nu_{x_i}(f)
+
\sum \nu_{y_j}(f).
\]

\vspace{1\baselineskip}
\textbf{Теорема 4.7}
$\forall\, f \in \mu_{\mathbb{C}/L}$,
$f$ — такого вида $f = \prod \Theta_\omega^{(x_i)} \Big/ \prod \Theta_\omega^{(y_j)}
\in \mu_{\mathbb{C}/L}.$

\[
\Theta_\omega^{(x)}(z)
=
\Theta_\omega\!\left(
z - \frac{1}{2} - \frac{\omega}{2} - x
\right),
\]

\[
\Theta_\omega(z)
=
\sum_{\ell \in \mathbb{Z}}
e^{\pi i (2lz + l^2w)}.
\]

\textbf{Доказательство}
\[
\sum \nu_p(f) = 0
\;\Longleftrightarrow\;
\text{число нулей совп. с числом полюсов}.
\]

$(p_i)_{i=1}^n$ — нули (с повторениями).

$(q_j)_{j=1}^n$ — полюса (с повторениями).

Предположим, что $\sum p_i \ne \sum q_j$

Дополним до равенства 
$\exists\, p_0,\ q_0 \in \mathbb{C}/L.$

\[
\sum_{i=0}^{n} p_i = \sum_{j=0}^{n} q_j.
\]

Поднимим $p_i,\ q_j$ в $\mathbb{C}$ относительно $\pi : \mathbb{C} \to \mathbb{C}/L, x_i \in \pi^{-1}(p_i),
\quad
y_j \in \pi^{-1}(q_j).
$
\[
\sum_{i=0}^{n} x_i = \sum_{j=0}^{n} y_j.
\]

Рассмотрим
\[
R
=
\prod_{i=0}^{n} \Theta_\omega^{(x_i)}
\Big/
\prod_{j=0}^{n} \Theta_\omega^{(y_j)}.
\]

\[
g = \frac{R}{f},
\qquad
g \text{ по построению один нуль } p_0, \text{ по построению один полюс } q_0,
\]
и они кратности 1.
Соответсвующее голоморф. отображение$
G : X \to \mathbb{C}_\infty.
$

\[
\deg g = 1
\;\Rightarrow\;
X = \mathbb{C}/L
\text{ — изоморф.} \mathbb{C}_{\infty}.
\]

Но
\[
g(x) = 1,
\qquad
g( \mathbb{C}_{\infty}) = 0. 
\]

\[
\Rightarrow\;
\sum_{i=1}^{n} p_i
=
\sum_{i=1}^{n} q_i.
\]

(Аналогично)

\[
\exists\, x_i,\ y_i :
\qquad
\sum_{i=1}^{n} x_i
=
\sum_{i=1}^{n} y_i.
\]

\[
R_1
=
\prod \Theta_\omega^{(x_i)}
\Big/
\prod \Theta_\omega^{(y_i)}.
\]

\[
g_1 = \frac{R_1}{f},
\qquad
g_1 \text{ не имеет нулей и полюсов}.
\]

\[
\Rightarrow\;
g_1 = \mathrm{const}
\;\Rightarrow\;
f = c\,R_1.
\]

\vspace{1\baselineskip}
\textbf{Теорема 4.8}
(формула Гурвица)

\[
F : X \to Y
\ \text{голоморфное непостоян. отображение},
\quad
X, Y \text{ — компактн. РП}.
\]

\[
2g(x) - 2
=
\deg F \,(2g(y) - 2)
+
\sum_{p \in X} (k_p(F) - 1).
\]

\vspace{1\baselineskip}
\textbf{Теорема 4.9}

$X = \mathbb{C}/L,\quad Y = \mathbb{C}/M.$

$F : X \to Y$ — голоморфное.

Пусть $F$ имеет вид
\[
G(z) = \gamma z + a.
\]

\[
\gamma : \gamma L \subset M.
\quad
a = 0
\;\Longleftrightarrow\;
F \text{ — гомом. групп},
\ (F(0)=0).
\]

\[
\deg F = [\,M : \gamma L\,].
\]

\[
F \text{ — изом. }
\;\Longleftrightarrow\;
\gamma L = M.
\]

\vspace{1\baselineskip}
\textbf{Теорема 4.10}

$X = \mathbb{C}/L,
\quad
F : X \to X$ — автоморфизм.$
\Longleftrightarrow\quad
\text{одно из}
$

\[
1)\; L \text{ — квадр. реш.},\quad
\gamma \in \sqrt[4]{1}.
\]

\[
2)\; L \text{ гексоганальная реш.},\quad
\gamma \in \sqrt[6]{1}.
\]

\[
3)\; L \text{ — не квадратная, не гексогональная реш.},\quad
\gamma = \pm 1.
\]

Или$
Aut(X)
$

\[
1)\; Aut(X) = \mathbb{Z}/4 \mathbb{Z}.
\]

\[
2)\; Aut(X) = \mathbb{Z}/6 \mathbb{Z}.
\]

\[
3)\; Aut(X) = \mathbb{Z}/2 \mathbb{Z}.
\]

\textbf{Следствие:}
$L$ — квадратная, $M$ — гексагональная, то $X/L,\quad X/M \text{ — не изоморф.}$

\vspace{1\baselineskip}
\textbf{Теорема 4.11}

\[
L_{\omega_1} = L(1,\omega_1),
\qquad
L_{\omega_2} = L(1,\omega_2).
\]

\[
\omega_1,\omega_2 \in \mathbb{H}.
\]

\[
X_{\omega_1} \text{изоморф.} X_{\omega_2}
\;\Leftrightarrow\;
\exists\, g \in SL_2(\mathbb{Z}) :
\omega_1 = \frac{a\omega_2 + b}{c\omega_2 + d}.
\]

т.е.
$
\text{классы изоморфизмов торов}
\;\Leftrightarrow\;
\text{точки фактор пространств}.
$



\end{document}

