% !TEX TS-program = pdflatex
% !TEX encoding = UTF-8 Unicode

\documentclass[12pt]{article} 

\usepackage[utf8]{inputenc} 
\usepackage{amsmath,amssymb, amsthm}
\usepackage[main=russian,english]{babel}
\usepackage{pgfplots}
\usepackage{float}

\newtheorem{theorem}{Теорема}
\newtheorem{lemma}{Лемма}
\newtheorem{definition}{Определение}
\newtheorem{corollary}{Следствие}
\newtheorem{remark}{Замечание}
\newtheorem{proposition}{Предложение}

\renewcommand*{\proofname}{Доказательство.}

\title{Лекция 1}
\author{Павел Владимирович Снурницын}

\begin{document}
\maketitle

\section{Введение}

Курс <<Арифметика-III: Римановы поверхности и алгебраические кривые>> является продолжением следующих двух курсов:
\begin{enumerate}
	\item <<Арифметика-I: Основы алгебраической теории чисел>>. Курс представляет собой введение в теорию чисел, начиная от делимости, сравнений, квадратичных вычетов и заканчивая полями алгебраических чисел и полями $p$-аддических чисел;
	\item <<Арифметика II: Решётки и формы>>. В рамках курса рассматривались рациональные евклидовы решетки и квадратичные и модулярные формы.
\end{enumerate}

Несмотря на то, что данный курс является продолжением этой серии, большая часть материала не связана с содержанием предыдущих курсов. 

Чтобы создать общую картину курса и мотивировать слушателей, предложим несколько утверждений и определений.

Для начала рассмотрим \textbf{решетку} $L \subset \mathbb{C},$ то есть множество $$L = \{z = n\omega_1 + m\omega_2 \colon n, m \in \mathbb{Z}, \omega_1, \omega_2 \in \mathbb{C}\}.$$
\newpage
\begin{figure}[h]
	\centering
	\begin{tikzpicture}
		\begin{axis}[
			grid = major,
			grid style = dashed,
			axis lines=left, 
			y=1cm,
			x=1cm,
			xmin=0, xmax=6,
			ymin=0, ymax=6,
			xtick={0,1,2,3,4,5}, 
			ytick={1,2,3,4,5},
			xlabel = $Re(z)$,
			ylabel = $Im(z)$
			]
		\end{axis}
		\foreach \x in {0,...,5} {
			\foreach \y in {0,...,5} {
				\draw[fill=blue] (\x, \y) circle(2pt) ;
			}
		}
	\end{tikzpicture}
	\caption{Решетка для $\omega_1 = 1$ и $\omega_2 = i$}
\end{figure}

Фактор-пространство по этой решетке $\mathbb{C} / L$ определяется так называемым \textbf{фундаментальным параллелепипедом} $P = \{z = x + iy\colon 0 \leq x \leq 1, 0 \leq y \leq 1\}$, так как любое комплексное число $z \in \mathbb{C}$ представимо в виде суммы: $z = \omega + p,$ где $\omega \in L$, а $p \in P.$ 

Теперь <<склеим>> противолежащие стороны нашего фундаментального параллелепипеда. <<Склеив>> первую пару сторон, увидим полый круговой цилиндр. Окружности на концах полого цилиндра будут второй парой противолежащих сторон. <<Склеив>> вторую пару, получим \textbf{тор}:

\begin{figure}[H]
	\centering
	\begin{tikzpicture}
		\begin{axis}[
			axis equal image,
			hide axis,
			z buffer = sort,
			view = {122}{30}
			]
			\addplot3[
			surf,
			samples = 40,
			samples y = 80,
			domain = 0:2*pi,
			domain y = 0:2*pi,
			colormap/blackwhite,
			](
			{(3+sin(deg(\x)))*cos(deg(\y))},
			{(3+sin(deg(\x)))*sin(deg(\y))},
			{cos(deg(\x))}
			);
		\end{axis}
	\end{tikzpicture}
	\caption{Тор, полученный из фундаментального параллелепипеда}
\end{figure}

Забегая вперед, полученная нами поверхность имеет \textbf{топологический род 1}.

Теперь рассмотрим следующую группу целочисленных невырожденных матриц: $$\Gamma = SL_2(\mathbb{Z}) = \{\bigl(\begin{smallmatrix}a & b\\ c & d\end{smallmatrix} \bigr)\colon ad - bc = 1; a,b,c,d \in \mathbb{Z}\}.$$
Эта группа действует в верхней комплексной полуплоскости: $$\mathbb{H} = \{z \in \mathbb{C} \colon Im(z) > 0.\}$$
Опишем её действие: $$\gamma \cdot z = \gamma z = \bigl(\begin{smallmatrix}a & b\\ c & d\end{smallmatrix} \bigr) z  = \bigl(\begin{smallmatrix}az + b \\ cz + d\end{smallmatrix} \bigr).$$
\textbf{Модулярной функцией} называется функция, обладающая <<хорошими>> свойствами симметрии относительно данного действия, например $$f(\gamma z) = (cz + d)^k f(z),$$ где $k$ -- некоторое целое число. 

Рассмотрим разбиение $\mathbb{H}$ на фактор-множества при действии $\Gamma\colon$ $$\mathbb{H} / \Gamma \sim D.$$

Множество $D$ называется \textbf{фундаментальной областью}. Она задает поверхность, имеющую \textbf{топологический род 0}. Примечательно, что если рассмотреть подгруппу $\Gamma,$ то есть некоторую группу $\Gamma',$ то задаваемая ей поверхность будет иметь \textbf{топологический род} $g \geq 0.$ Род $g$ имеет важную роль в исследовании модулярных функций.

Далее рассмотрим конечное поле $\mathbb{F}_q$ из $q = p^n$ элементов, где $p$ -- простое число. Пусть $f(x,y) \in \mathbb{F}_q[x,y]$ -- многочлен от двух переменных с коэффициентами из этого поля. Данный многочлен задает некоторую кривую. Обозначим через $N_m$ число точек на этой кривой, то есть множество решений уравнения $$f(x,y) = 0,$$ принадлежащих \textbf{проективному пространству} $\mathbb{P}^2(\mathbb{F}_q)$ (формально определим позднее). Для $N_m$ верна теорема:

\begin{theorem}[Хассе-Вейль]
	$$| N_m - (q^n + 1) | \leq 2gq^{n/2},$$ где $g$ -- алгебраический род кривой $f = 0.$
\end{theorem}

Данный курс имеет целью раскрыть данную теорию и показать, как эти вопросы связаны через алгебраическую геометрию и теорию Римановых поверхностей.

\section{Римановы поверхности (РП). Введение.}

Определим основные объекты и понятия.

\begin{definition}
	Пусть $X$ -- множество. Говорят, что на $X$ задана структура топологического пространства, если $\forall x \in X$ задано непустое множество подмножеств $X,$ обозначаемое $\mathcal{U}(x) \neq \emptyset,$ и
	\begin{enumerate}
		\item $\forall ~U \in \mathcal{U}(x)\colon x \in U;$
		\item $U,$ $V \in \mathcal{U}(x) \implies U \cap V \in \mathcal{U}(x);$
		\item $U \in \mathcal{U}(x),$ $ V \subset X,$ $x \in V \text{ и } U \subset V \implies V \in \mathcal{U}(x);$
		\item $U \in \mathcal{U}(x),$ $\dot{U} = \{z \in U\colon U \in \mathcal{U}(z)\} \in \mathcal{U}(x).$
	\end{enumerate}
\end{definition}

Существует более полезное определение:

\begin{definition}
	Топологическим пространством называется пара $(X, T)$, где $T$ - множество подмножеств $X$, называемое системой открытых множеств, и удовлетворяющее:
	\begin{enumerate}
		\item $\emptyset, X \subset T$;
		\item $\forall~ (U_\alpha) \in T\colon$ $\bigcup_\alpha U_\alpha \in T;$
		\item $\forall~ (U_i)_{i = 1}^n,$ $U_i\in T,$ $\bigcap_{i = 1}^{n} U_i \in T$
	\end{enumerate}
\end{definition}

\underline{Примеры:} 
\begin{enumerate}
	\item Множество $\mathbb{C}$ является топологическим пространством с системой открытых множеств в виде окрестностей $|z| < \varepsilon$;
	\item $\mathbb{R}^2$ является топологическим пространством с системой открытых множеств в виде окрестностей $\sqrt{x^2 + y^2} < \varepsilon$.
\end{enumerate}

\begin{definition}
	Пусть $X$, $Y$ -- топологические пространства. Отображение $f\colon X \to Y$ называется непрерывным, если $\forall x \in X\colon$ $\forall~ V \in \mathcal{U}(f(x))$ (в топологии $Y$) $\exists~ U \in \mathcal{U}(x)$ (в топологии $X$) такое, что $f(U) \subset V.$
\end{definition}

\begin{definition}
	Отображение $f\colon X \to Y$ называется открытым, если для любого открытого множества $U$ множество $f(U)$ открытое (в топологии $Y$).
\end{definition}

\begin{definition}
	Если $f\colon X \to Y$ -- непрерывно и биективно, то $f$ называется гомеоморфизмом.
\end{definition}

\begin{definition}
	Пусть $X$ -- топологическое пространство, $U \subset X$ -- открытое множество, $V \subset \mathbb{C}$ -- открытое множество. Гомеоморфизм $\phi: U \to V$ называется комплексной картой. Если для некоторой $p \in U\colon \phi(p) = 0,$ то $p$ называется центром карты.
\end{definition}

Комплексную карту можно представлять себе как некоторую <<локальную координату>>.


\underline{Примеры:}
\begin{enumerate}
	\item Пусть $X = \mathbb{R}^2$, $U \subset X$ -- открытое подмножество. Тогда отображение $\phi_U(x, y) = x + iy$ будет комплексной картой. 
	\item Можно предложить другое отображение: $$\phi_U(x,y) = \frac{x}{1 + \sqrt{x^2 + y^2}} + i\frac{y}{1 + \sqrt{x^2 + y^2}},$$ которое также будет картой.
\end{enumerate}

\begin{definition}
	Две карты на $X$: $\phi_1\colon U_1 \to V_1$ и $\phi_2\colon U_2 \to V_2$ называются совместимыми, если либо $U_1 \cap U_2 = \emptyset$, либо $T = \phi_2 \circ \phi_1^{-1},$ $T\colon \phi_1(U_1 \cap U_2) \to \phi_2(U_1 \cap U_2)$ является голоморфной, то есть комплексно дифференцируемой функцией в окрестности любой точки рассматриваемой области.
\end{definition}

Функция $T$ называется функцией склейки. Она отвечает за совпадение карт на пересечении областей определения.

Заметим, что если $\psi$ -- карта, задающая отображение из подмножества $\mathbb{C}$ в подмножество $\mathbb{C},$ то композиция $\phi \circ \psi$ по сути осуществляет замену координат.

\begin{lemma}
	Пусть $T$ -- функция склейки. Тогда её производная $T' = \frac{dT}{dz} \neq 0$ в области определения $T$.
\end{lemma}
\begin{proof}
	Так как $T$ -- голоморфная, то существует обратная функция $S = T^{-1}$. При этом $S \circ T$ является тождественной функцией, поэтому для любого $z$ из области определения $T$ верно $S\circ T (z) = z$, но тогда $$(S\circ T (z))' = S'(T(z)) \cdot T'(z) = 1,$$ поэтому $T'(z)$ не может равняться нулю.
\end{proof}

\begin{corollary}
	Пусть $\phi, \psi$ -- две карты, $z_0 = \phi(p),$ $w_0=\psi(p),$ где $p$ -- точка, в которой обе карты определены. Тогда ряд Тейлора для функции склейки имеет вид: $$z = T(w) = z_0 + \sum_{n = 1}^{\infty}a_n (w - w_0)^n, a_1 \neq 0.$$
\end{corollary}

Последнее неравенство является следствием леммы.

\underline{Упражнение.} Докажите, что не совместимы две карты: $$\phi_U(x,y) = x + iy,$$ $$\phi_U(x,y) = \frac{x}{1 + \sqrt{x^2 + y^2}} + i\frac{y}{1 + \sqrt{x^2 + y^2}}.$$

\begin{definition}
	Комплексным атласом $\mathcal{A}$ на $X$ называется набор $$\mathcal{A} = \{\phi_\alpha\colon U_\alpha \to V_\alpha\},$$
	для которого 
	\begin{enumerate}
		\item $\forall~ \alpha, \beta\colon$ $\phi_\alpha$ и $\phi_\beta$ совместимы;
		\item $\bigcup_\alpha U_\alpha = X.$
	\end{enumerate}
\end{definition}

Далее будем обозначать $T_{\alpha \beta} = \phi_\beta \circ \phi^{-1}_\alpha,$ то есть соответствующую функцию склейки.

\underline{Примеры}
\begin{enumerate}
	\item Сфера в $\mathbb{R}^3, $ которую обозначают $$S^2 = \{(x_1, x_2, x_3) \in \mathbb{R}^3\colon x_1^2 + x_2^2 + x_3^2 = 1\}.$$
	Рассмотрим $\mathbb{R}^2$ как плоскость, вложенную в $\mathbb{R}^3,$ то есть $\mathbb{R}^2 = \{(x_1, x_2, 0)\}$ и одновременно как $\mathbb{C}.$ Определим проекцию из точки $(0,0,1)$ (полюса сферы):
	$$\phi(x_1, x_2, x_3) = \frac{x_1}{1 - x_3} + i \frac{x_2}{1 - x_3}.$$ Данное отображение определено везде, кроме самого полюса. Аналитически можно описать прообраз: $$\phi^{-1}(z) = \bigl( \frac{2Re(z)}{|z|^2+1}, \frac{2Im(z)}{|z|^2+1}, \frac{|z|^2-1}{|z|^2+1}\bigr).$$
\end{enumerate}

\begin{figure}[h]
	\centering
	\begin{tikzpicture}
		% Оси
		\draw[->] (-1.5,0) -- (2,0);
		\draw[->] (0,-1.5) -- (0,2);
		
		\draw (0,0) circle (1);
		
		\draw[dashed] (0,1) -- (1.5,0);
		
		\fill (1.5,0) circle (1pt) node[below right] {$P_A$};
		\fill (12/13, 5/13) circle (1pt) node[above right] {A};
	\end{tikzpicture}
	\caption{Проекция точки $A$ сферы на плоскость}
\end{figure}

Проекция из $(0,0,-1)$ имеет вид $$\psi(x_1, x_2, x_3) = \frac{x_1}{1 + x_3} - i\frac{x_2}{1 + x_3}.$$

Тогда функция склейки для $\phi$ и $\psi$ равна $T(z) = \frac{1}{z}.$ 

Множество $\{\phi, \psi\}$ задает комплексный атлас на $S^2.$

\begin{definition}
	Два атласа $\mathcal{A}_1$ и $\mathcal{A}_2$ называются эквивалентными, если $\forall~ \phi_\alpha \in \mathcal{A}_1$, $\forall~ \phi_\beta \in \mathcal{A}_2$ карты $\phi_\alpha$ и $\phi_\beta$ совместимы.
\end{definition}

\begin{definition}
	Класс эквивалентности комплексного атласа называется комплексной структурой.
\end{definition}

\begin{definition}
	Топологическое пространство $X$ называется Хаусдорфовым, если $\forall ~x, y \in X,$ $ x \neq y\colon$ $\exists~ U, V$ -- окрестности $x$ и $y$, соответственно, такие, что $U \cap V = \emptyset.$
\end{definition}

\begin{definition}
	Топологическое пространство $X$ называется пространством со счетной базой, если в системе открытых множеств $T$ можно выделить счетное число подмножеств, которое порождает $T$.
\end{definition}

\begin{definition}
	Пространство $X$ называется связным, если $\nexists$ открытые непустые подмножества $U_1, U_2$ множества $X$, для которых $X = U_1 \cup U_2$ и $U_1 \cap U_2 = \emptyset.$
\end{definition}

\begin{definition}
	$X$ называется Римановой поверхностью (РП), если
	\begin{enumerate}
		\item $X$ -- связное Хаусдорфово топологическое пространство со счетной базой;
		\item на $X$ задана комплексная структура.
	\end{enumerate}
\end{definition}

\underline{Примеры}
\begin{enumerate}
	\item $X = S^2$ с картами $\{\phi, \psi\}$ является Римановой поверхностью. Будем называть её сферой Римана (Римановой сферой). Иногда про сферу Римана говорят, что она описывает множество $\mathbb{C}~ \cup~ \{\infty\} = \mathbb{C}_{\infty},$ где бесконечно удаленным точкам соответствуют полюса сферы.
\end{enumerate}

\begin{proposition}
	Как задать Риманову поверхность на множестве $X$? Для этого нужно:
	\begin{enumerate}
		\item счетное покрытие $(U_\alpha)\colon \bigcup_\alpha U_\alpha = X;$
		\item $\forall~\alpha $ определим биекцию $\phi_\alpha\colon U_\alpha \to V_\alpha \subseteq \mathbb{C},$ где $V_\alpha$ -- открытые множества, а для биекций верно:
		\begin{enumerate}
			\item $\forall~ \alpha, \beta\colon$ $\phi_\alpha(U_\alpha \cap U_\beta)$ -- открытое в $V_\alpha$;
			\item $\forall~ \alpha, \beta\colon$ $\phi_\alpha, \phi_\beta$ -- совместимые;
			\item топология на $X$ связна и Хаусдорфова.
		\end{enumerate}
		
	\end{enumerate}
\end{proposition}

Теперь определим проективное пространство, которое было упомянуто во введении:

\begin{definition}
	Если $\mathbb{C}^{n+1}$ -- комплексное векторное пространство, то можно определить отношение эквивалентности ненулевых векторов: $$(z_0, z_1, ..., z_n) \sim (z_0', z_1', ..., z_n'),$$ если $\exists$ $\lambda \neq 0 \in \mathbb{C}\colon$ $\forall~i \in \overline{0,n}$ верно $z_i = \lambda z_i'.$ Тогда проективное пространство определяется, как $$\mathbb{P}^n(\mathbb{C}^n) = \mathbb{C}^{n+1} / \sim.$$
\end{definition}

Через $\mathbb{C}^{n+1}$ будем обозначать аффинное пространство соответствующей размерности: $\mathbb{C}^{n+1} = \mathbb{A}^{n+1}(\mathbb{C}).$

Пока мы работаем только с комплексными координатами, будем обозначать $\mathbb{P}^n = \mathbb{P}^n (\mathbb{C}).$ Элементы $\mathbb{P}^n$ обычно обозначают через представителей классов эквивалентности: $[z_0\colon z_1\colon ... \colon z_n] = [\lambda z_0\colon \lambda z_1\colon ... \colon\lambda  z_n].$

Далее покажем, что на $\mathbb{P}^1$ можно определить структуру Римановой поверхности. 

Пусть
$$U_0 = \{[z_0\colon z_1]\colon z_0 \neq 0\},$$
$$U_1 = \{[z_0\colon z_1]\colon z_1 \neq 0\}.$$

$\{U_0, U_1\}$ образуют покрытие $\mathbb{P}^1,$ то есть $\mathbb{P}^1 = U_0 \cup U_1.$ 

Зададим отображения:
$$\phi_0\colon U_0 \to \mathbb{C},$$
$$\phi_1\colon U_1 \to \mathbb{C}$$
следующего вида: 
$$\phi_0([z_0\colon z_1]) = \frac{z_1}{z_0}, $$
$$\phi_1([z_0\colon z_1]) = \frac{z_0}{z_1}.$$

При этом $\phi(U_0 \cap U_1) = \mathbb{C}^* = \mathbb{C}\setminus\{0\}$ -- открытое в обычной топологии $\mathbb{C}.$ 

Далее для функции склейки:
$$T = (\phi_1 \circ \phi_0^{-1})(z) = \frac{1}{z}.$$ На $\mathbb{C}^*$ данная функция голоморфна, поэтому $\phi_0$ и $\phi_1$ совместимы. Получаем, что $\{\phi_0, \phi_1\}$ -- атлас. 

Заметим, что $U_0$ и $U_1$ связны. Их пересечение $U_0 \cap U_1$ тоже связно, поэтому $\mathbb{P}^1 = U_1 \cup U_2$ связно.

Докажем Хаусдорфовость. Пусть $p,q \in \mathbb{P}^1.$ Если $p,q \in U_i, i \in \{0,1\}$ то отделимость следует из отделимости $\mathbb{C}.$ Пусть $p \in U_0 \setminus U_1$ и $q \in U_1 \setminus U_0.$ Тогда $p = [1\colon 0]$ и $q = [0\colon 1].$ Рассмотрим $D = \{|z|<1, z \in \mathbb{C}\}.$ Имеем $p \in \phi_0^{-1}(D)$ и $q \in \phi_1^{-1}(D)$, а также $\phi_0^{-1}(D) \cap \phi_1^{-1}(D) = \emptyset,$ так как $T(D) = \{|z| > 1\}.$ 

Заметим, что для замкнутого диска $\overline{D} = \{|z| \leq 1\}\colon$ $$\mathbb{P}^1 = \phi_0^{-1}(\overline{D}) \cup \phi_1^{-1}(\overline{D}).$$ Рассмотрим еще одно топологическое понятие:
\begin{definition}
	$X$ называется компактом, если из любого покрытия $(U_\alpha)$ можно выделить конечное подпокрытие.
\end{definition}

Тогда $\mathbb{P}^1$ является компактом.

\begin{remark}
	Для римановых поверхностей существует более общее понятие $n$-мерных комплексных многообразий: пусть задано топологическое пространство $X$, $n$-мерные комплексные карты $\phi\colon U \to V,$ где $U \subset X,$ а $V$ -- открытое подмножество $\mathbb{C}^n.$ Тогда также $\mathbb{P}^n$ -- $n$-мерное комплексное многообразие, а карты являются отображениями вида: 
	$$\phi_i = \{[z_0\colon,...,\colon z_n] \colon z_i \neq 0\} \to (\frac{z_0}{z_i}, ..., \hat{i},..., \frac{z_n}{z_i}).$$
\end{remark}

Вернемся к понятию решетки, то есть множества вида $$L = \mathbb{Z}\omega_1 + \mathbb{Z}\omega_2,$$ где $\omega_1, \omega_2$ -- линейно независимые комплексные числа. Основной параллелепипед решетки -- это множество $$P = \{\lambda_1\omega_1 + \lambda_2\omega_2\colon 0 \leq \lambda_i \leq 1\}.$$

Рассмотрим фактор-пространство $X = \mathbb{C}/L.$
По сути, так как любое комплексное число $z$ представляется в виде $z = p + \omega,$ где $p \in P, \omega \in L,$ то класс эквивалентности по этому фактору имеет вид $[z] = p = (z \mod L)$ или в обозначениях естественной проекции -- $\pi(z).$ Используя эту проекцию, можно задать естественную топологию на $X$ -- топологию фактор-пространств, то есть $U \subset X$ -- открытое тогда и только тогда, когда $\pi^{-1}(U)$ открыто в $\mathbb{C}.$ При этом $\pi$ непрерывно в этой топологии, а $X$ связно,  так как связно $\mathbb{C}.$

\begin{lemma} 
	Верны следующие утверждения:

	\begin{enumerate}
		\item Для любого открытого множества $U \subset X$ существует открытое $V \subset \mathbb{C}$, $\pi(U) = V;$
		\item $\pi$ -- открытое отображение (переводит открытые множества в открытые).
	\end{enumerate}
\end{lemma}

Доказательство оставляется читателю в качестве упражнения.

Так как $\pi\colon P \to X,$ то есть это сюръекция и $P$ -- компактное множество, то $X$ также является компактным множеством.

Далее решетка $L$ является дискретным множеством, то есть $\exists$ $\varepsilon > 0\colon$ $\forall~\omega \in L \setminus \{0\}$ верно $|w| > 2\varepsilon.$ Возьмем $D_{z_0} = \{|z - z_0| < \varepsilon\}.$ Любые $z_1, z_2 \in D_{z_0}$ удовлетворяют $z_1 \neq z_2 \mod L.$ Теперь карты можно определить, как $$\forall~z_0 \in \mathbb{C}, \phi_{z_0}\colon \pi(D_{z_0}) \to D_{z_0}.$$ По сути эти карты -- прообразы ограничения (сужения) $\pi$ на $D_{z_0},$ то есть $\phi_{z_0} = (\pi | _{D_{z_0}})^{-1}.$ Геометрически это означает, что либо $z_0$ полностью лежит с окрестностью внутри параллелепипеда, либо её окрестность <<раскинута>> по разным <<частям>> параллелепипеда.

Рассмотрим вопрос совместимости $\phi_1 = \phi_{z_1}$ и $\phi_2 = \phi_{z_2}.$ Пусть $U = \pi(D_{z_1}) \cap \pi(D_{z_2}) \neq \emptyset.$ Для функции склейки $$T(z) = \phi_2(\phi_1^{-1}(z)) = \phi_2(\pi(z))$$ можно записать $$\pi(T(z)) = \pi(z).$$ Получаем, что $z$ и $T(z)$ лежат в одном классе решетки, но тогда $T(z) - z = \omega = \omega(z).$ Функция $\omega(z)$ есть отображение $\phi_1(U) \to L.$ Можно заметить, что $\omega$ непрерывна, так как непрерывны $\phi$, $T.$ При этом $L$ -- дискретное множество, поэтому $\omega(z)$ локально постоянна, то есть на $\phi_1(U)$ верно $T(z) = z + \omega,$ где $\omega$ -- константа. Из этого следует, что $T$ -- это просто <<сдвиг>> на $\omega$ и тогда данная функция голоморфна.
\newpage

\section{Род и число Эйлера}

Неформально говоря, род поверхности -- это число её <<дырок>> или <<ручек>>.

\begin{figure}[h]
	\centering
	\begin{tikzpicture}[thick, scale=1]
		
		% --- Род 0: Сфера ---
		\begin{scope}[xshift=0cm]
			\draw (0,0) circle (1);                          
			\draw (-1,0) arc (180:360:1 and 0.3);             
			\draw[dashed, thin] (1,0) arc (0:180:1 and 0.3);  
			\node at (0,-1.5) {$g=0$};
		\end{scope}
		
		% --- Род 1: Тор ---
		\begin{scope}[xshift=3.5cm]
			\draw (0,0) ellipse (1.4 and 0.8);                
			\draw (-0.7,0.1) to[bend right=40] (0.7,0.1);
			\draw (-0.55,0.05) to[bend left=40] (0.55,0.05);
			\node at (0,-1.5) {$g=1$};
		\end{scope}
		
		% --- Род 2: Крендель (Биторий) ---
		\begin{scope}[xshift=8cm]
			\draw plot [smooth cycle, tension=0.8] coordinates {
				(-1.8,0) (-1,0.8) (0,0.4) (1,0.8) (1.8,0) 
				(1,-0.8) (0,-0.4) (-1,-0.8)
			};
			\draw (-1.3,0.1) to[bend right=45] (-0.5,0.1);
			\draw (-1.15,0.05) to[bend left=45] (-0.65,0.05);
			\draw (0.5,0.1) to[bend right=45] (1.3,0.1);
			\draw (0.65,0.05) to[bend left=45] (1.15,0.05);
			\node at (0,-1.5) {$g=2$};
		\end{scope}
	\end{tikzpicture}
	\caption{Поверхности различных родов $g$}
\end{figure}

\begin{definition}
	Пусть $X$ -- это комплексное двумерное многообразие. Триангуляцией $X$ называется разбиение $X = \bigcup_{i=1}^n T_i\colon$ $T_i$ -- замкнуто, гомеоморфно треугольнику и $T_i \cap T_j \in$ $\{\emptyset, \text{вершина}, \text{ребро}\} \forall i \neq j.$
\end{definition}

\begin{definition}
	Пусть $T$ -- триангуляция $X.$ Эйлеровой характеристикой называется $\chi(T) = v - e + t,$ где $v,e,t$ -- число, соответственно, вершин, рёбер и треугольников в триангуляции.
\end{definition}

\begin{lemma}[без доказательства] 
	$\chi(T)$ не зависит от $T,$ то есть $\chi = \chi(X).$
\end{lemma}

\begin{definition}
	Топологическим родом $X$ назовем число $$g(X) = \frac{1}{2}(2 - \chi(X)).$$
\end{definition}

Наше изначальное понимание топологического рода было связано со следующей теоремой:

\begin{theorem}[без доказательства] Всякое комплексное ориентируемое многообразие гомеоморфно сфере с $g$ ручками, где $g$ -- топологический род.
\end{theorem}

\begin{figure}[H]
	\centering
	\begin{tikzpicture}[thick, scale=0.7]
		
		% Макрос для рисования одной "дырки"
		\def\hole{
			\draw (-0.4, 0.15) to[bend right=50] (0.4, 0.15);
			\draw (-0.5, 0.05) to[bend left=50] (0.5, 0.05);
		}
		
		% --- 1. Левый блок (Дырки 1 и 2) ---
		
		\draw (3, 0.4)                         
		to[out=180, in=0] (2, 0.9)         
		to[out=180, in=0] (1, 0.4)         
		to[out=180, in=0] (0, 0.9)         
		arc (90:270:0.9)                   
		to[out=0, in=180] (1, -0.4)
		to[out=0, in=180] (2, -0.9)
		to[out=0, in=180] (3, -0.4);
		
		% --- 2. Правый блок (Дырка n) ---
		\draw (4, 0.4)                         
		to[out=0, in=180] (5, 0.9)         
		arc (90:-90:0.9)                   
		to[out=180, in=0] (4, -0.4);       
		
		% --- 3. Разрыв (Пунктир) ---
		\draw[dashed] (3, 0.4) -- (4, 0.4);
		\draw[dashed] (3, -0.4) -- (4, -0.4);
		\node at (3.5, 0) {\Large $\dots$};
		
		% --- 4. Отрисовка дырок и номеров ---
		% Координаты центров: 0, 2 и 5
		\foreach \x/\n in {0/1, 2/2, 5/n} {
			\begin{scope}[xshift=\x cm]
				\hole
				\node at (0, -1.2) {$\n$};
			\end{scope}
		}
		
	\end{tikzpicture}
	\caption{Вид $X$ при $g=n$}
\end{figure}

\underline{Примеры:}
\begin{enumerate}
	\item Разобьем сферу на <<треугольные>> поверхности следующим образом:
	
	\begin{figure}[H]
		\centering
		\begin{tikzpicture}[thick, scale=2]
			
			% 1. Контурная окружность (Синяя)
			\draw[blue] (0,0) circle (1);
			
			% 2. Экватор (Красная)
			\draw[red] (-1,0) arc (180:360:1 and 0.4);            
			\draw[red, dashed, thin] (1,0) arc (0:180:1 and 0.4);
			
			% 3. Меридиан (Зеленая)
			\draw[green!60!black] (0,1) arc (90:270:0.4 and 1);               
			\draw[green!60!black, dashed, thin] (0,-1) arc (-90:90:0.4 and 1); 
		\end{tikzpicture}
		\caption{Разбиение сферы на триангуляции}
	\end{figure}
	
	Здесь триангуляции -- это поверхности, ограниченные тремя дугами разного цвета. Посчитаем Эйлерову характеристику: $\chi = 6 - 12 + 8 = 2.$ Тогда род $g = 0.$
	
	\item В случае тора достаточно вспомнить, что тор можно <<вытянуть>> в прямоугольник и далее разбить его на треугольники, учитывая, что противолежащие стороны прямоугольника отождествляются на торе. Подсчет топологического рода далее производится аналогично сфере.
\end{enumerate}

\begin{theorem}[то, что нам нужно из топологии]
	Если $X$ -- Риманова поверхность, то $X$ гомеоморфно обобщенному тору рода $g \geq 0.$
\end{theorem}

\section{Аффинные кривые}

Рассмотрим некоторые понятия алгебраической геометрии.

\begin{definition}
	 Пусть $f \in \mathbb{C}[z,w]$ (многочлен от двух переменных). Множество $$X_f = \{(z,w) \in \mathbb{C}^2\colon f(z,w) = 0\}$$ называется аффинной кривой над $\mathbb{C}.$
\end{definition}

\begin{definition}
	$X_f$ или $f$ называется невырожденной (неособенной) кривой в точке $p \in X_f,$ если $$\frac{\delta f}{\delta z} (p) \neq 0$$ или $$\frac{\delta f}{\delta w} (p) \neq 0.$$
	Если $X_f$ невырождена в каждой точке $p \in X_f$, то $X_f$ называют гладкой аффинной кривой.
\end{definition}

\begin{theorem}[без доказательства]
	Если $f$ -- неприводимый многочлен, то $X_f$ -- Риманова поверхность.
\end{theorem}

Основная сложность доказательства теоремы выше состоит в доказательстве связности.

\underline{Пример.} Рассмотрим многочлен $f = w^2 - h(z).$ Если $f$ -- неприводимый, то решения уравнения $w^2 = h(z)$ задают Риманову поверхность. Утверждается, что $f$ неприводим над $\mathbb{C}$ тогда и только тогда, когда $h$ не является полным квадратом.

В частности, при $\deg h = 3$ получим \textbf{эллиптическую кривую}. При $\deg h > 3$ получим \textbf{гиперэллиптическую кривую}.

\section{Проективные кривые}

\begin{definition}
	Пусть $F(x,y,z)$ -- однородный многочлен степени $d$ ($F \in \mathbb{C}[x,y,z]_d$), то есть $F(\lambda x,\lambda y,\lambda z) = \lambda^d F(x,y,z).$ Множество $$X_F = \{[x\colon y \colon z] \in \mathbb{P}^2\colon F(x,y,z) = 0\}$$ будем называть проективной кривой.
\end{definition}

Аналогично аффинным кривым определяются невырожденности $X_F$ и $F$, а также верна теорема:

\begin{theorem}[без доказательства]
	Если $F$ -- невырожденный многочлен, то $X_F$ -- компактная Риманова поверхность.
\end{theorem}

Кроме того, верна следующая теорема:

\begin{theorem}[без доказательства]
	Если $F$ -- невырожденный многочлен, а также однородный многочлен степени $d$, то $$g(X_F) = \frac{(d-1)(d-2)}{2}.$$
\end{theorem}

\underline{Пример}. Эллиптическая кривая имеет степень 3, откуда по теореме выше её род $g$ равен 1.

\end{document}
