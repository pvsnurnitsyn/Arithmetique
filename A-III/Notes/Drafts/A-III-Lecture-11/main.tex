\documentclass[a4paper,14pt]{article}

%%% Работа с русским языком
\usepackage{cmap}					% поиск в PDF
\usepackage{mathtext} 				% русские буквы в формулах
\usepackage[T2A]{fontenc}			% кодировка
\usepackage[utf8]{inputenc}			% кодировка исходного текста
\usepackage[english,russian]{babel}	% локализация и переносы
\usepackage{indentfirst} %Абзацный отступ первого после заголовка абзаца
\frenchspacing
\usepackage{misccorr}%Точки после номеров заголовков 


\renewcommand{\epsilon}{\ensuremath{\varepsilon}}
\renewcommand{\phi}{\ensuremath{\varphi}}
\renewcommand{\kappa}{\ensuremath{\varkappa}}
\renewcommand{\le}{\ensuremath{\leqslant}}
\renewcommand{\leq}{\ensuremath{\leqslant}}
\renewcommand{\ge}{\ensuremath{\geqslant}}
\renewcommand{\geq}{\ensuremath{\geqslant}}
\renewcommand{\emptyset}{\varnothing}

%%% Дополнительная работа с математикой
\usepackage{amsmath,amsfonts,amssymb,amsthm,mathtools} % AMS
\usepackage{icomma} % "Умная" запятая: $0,2$ --- число, $0, 2$ --- перечисление
\usepackage{mathrsfs}

%% Номера формул
%\mathtoolsset{showonlyrefs=true} % Показывать номера только у тех формул, на которые есть \eqref{} в тексте.
%\usepackage{leqno} % Нумереация формул слева

%% Свои команды
\DeclareMathOperator{\sgn}{\mathop{sgn}}
\newcommand{\nullcounter}{
\setcounter{theorem}{0}
\setcounter{lemma}{0}
\setcounter{proposition}{0}
\setcounter{definition}{0}
\setcounter{corollary}{0}
\setcounter{task}{0}
\setcounter{example}{0}
\setcounter{remark}{0}
\setcounter{equation}{0}
} % Обнуление всех счётчиков типа теорема

%% Перенос знаков в формулах (по Львовскому)
\newcommand*{\hm}[1]{#1\nobreak\discretionary{}
{\hbox{$\mathsurround=0pt #1$}}{}}

%%% Работа с картинками
\usepackage{graphicx}  % Для вставки рисунков
\graphicspath{{images/}{images2/}}  % папки с картинками
\setlength\fboxsep{3pt} % Отступ рамки \fbox{} от рисунка
\setlength\fboxrule{3pt} % Толщина линий рамки \fbox{}
\usepackage{wrapfig} % Обтекание рисунков текстом

%%% Работа с таблицами
\usepackage{array,tabularx,tabulary,booktabs} % Дополнительная работа с таблицами
\usepackage{longtable}  % Длинные таблицы
\usepackage{multirow} % Слияние строк в таблице

%%% Теоремы
\theoremstyle{plain} %Стиль курсивом 
%\newtheorem{theorem}{Теорема}
%\newtheorem{lemma}{Лемма}
\newtheorem{proposition}{Предложение}
%-------------------------------------------------
\theoremstyle{definition} %Стиль без курсива
%\newtheorem{definition}{Определение}
\newtheorem{corollary}{Следствие}
\newtheorem{task}{№}
%\newtheorem{example}{Пример}
\newtheorem{axiom}{Аксиома}

\theoremstyle{remark}
%\newtheorem{remark}{Замечание}
%\newtheorem*{remark*}{Замечание}
\newtheorem*{solution}{Решение}

%%% Программирование
\usepackage{etoolbox} % логические операторы

%%% Страница
\usepackage{extsizes} % Возможность сделать 14-й шрифт
\usepackage{geometry} % Простой способ задавать поля
	\geometry{top=25mm}
	\geometry{bottom=25mm}
	\geometry{left=25mm}
	\geometry{right=20mm}
 %
\usepackage{fancyhdr} % Колонтитулы
 \pagestyle{fancy}
	\renewcommand{\headrulewidth}{1pt}  % Толщина линейки, отчеркивающей верхний колонтитул
	\lfoot{Лектор: Снурницын П.\,В.}
\rfoot{Спецкурс A-III <<Римановы поверхности>>}
	\rhead{НЕКОТОРЫЕ ПРИЛОЖЕНИЯ ТЕОРЕМЫ РИМАНА-РОХА}
%	\chead{}
 	\lhead{ЛЕКЦИЯ №11}
%	\cfoot{Нижний в центре} % По умолчанию здесь номер страницы

\usepackage{setspace} % Интерлиньяж
%\onehalfspacing % Интерлиньяж 1.5
%\doublespacing % Интерлиньяж 2
%\singlespacing % Интерлиньяж 1

\usepackage{statmath}

\usepackage{lastpage} % Узнать, сколько всего страниц в документе.

\usepackage{soul} % Модификаторы начертания

\usepackage{hyperref}
\usepackage[usenames,dvipsnames,svgnames,x11names, table,rgb]{xcolor}


\hypersetup{				% Гиперссылки
    unicode=true,           % русские буквы в раздела PDF
    pdftitle={Заголовок},   % Заголовок
    pdfauthor={Автор},      % Автор
    pdfsubject={Тема},      % Тема
    pdfcreator={Создатель}, % Создатель
    pdfproducer={Производитель}, % Производитель
    pdfkeywords={keyword1} {key2} {key3}, % Ключевые слова
    colorlinks=true,       	% false: ссылки в рамках; true: цветные ссылки
    linkcolor=black,          % внутренние ссылки
    citecolor=black,        % на библиографию
    filecolor=magenta,      % на файлы
    urlcolor=blue           % на URL
}

\usepackage{csquotes} % Еще инструменты для ссылок

%\usepackage[style=authoryear,maxcitenames=2,backend=biber,sorting=nty]{biblatex}

\usepackage{multicol} % Несколько колонок

\usepackage{tikz} % Работа с графикой
\usepackage{pgfplots}
\usepackage{pgfplotstable}

%GeoGebra
\usepackage{pgf}
\pgfplotsset{compat=1.15}
\usepackage{mathrsfs}
\usetikzlibrary{arrows}
\usetikzlibrary{patterns}

\usepackage {lscape}
\usepackage{multicol}

\usepackage{tikzsymbols}
\usetikzlibrary {arrows.meta}
\usetikzlibrary{graphs}
\usetikzlibrary{graphs.standard}
\usetikzlibrary{patterns}

\usepackage{asymptote}
%\usepackage{tcolorbox} 
\usepackage{rotating}

\newcommand{\answer}[1]{\begin{flushright}\textit{Ответы:} \texttt{#1}\end{flushright}}

\usepackage[utf8]{inputenc}
\usepackage[most]{tcolorbox}
\usepackage{lipsum}
% counters
\newcounter{theorem}
\newcounter{lemma}
\newcounter{difinition}
\newcounter{mark}
%\counterwithin{theorem}{section}
%\counterwithin{lemma}{section}

% names for the structures
\newcommand\theoname{Théorème}
\newcommand\lemmname{Lemme}
\newcommand\difname{difinition}

\usepackage{amsmath}
\usepackage{tikz}


\makeatletter
% environment for theorems
\newtcolorbox{theorem}[1][]{
	breakable,
	enhanced,
	colback=White!100,
	colframe=Coral3!70,
	top=\baselineskip,
	enlarge top by=\topsep,
	overlay unbroken and first={
		\node[thick,draw=Coral3!80!black,fill=Coral1!10,rounded corners] at (frame.north) %
		{\strut{}\if#1\@empty\relax\relax\else~(#1)\fi};
		\vspace{1mm}
	}
}

% environment for property
\newtcolorbox{property}[1][]{
	breakable,
	enhanced,
	colback=White!100,
	colframe=LightBlue3!100!black,
	top=\baselineskip,
	enlarge top by=\topsep,
	overlay unbroken and first={
		\node[thick,draw=LightBlue3!85!black,fill=LightBlue2!25,rounded corners] at (frame.north) %
		{\strut{}\if#1\@empty\relax\relax\else~(#1)\fi};
		\vspace{1mm}
	}
}
% environment for lemas
\newtcolorbox{difinition}[1][]{
	breakable,
	enhanced,
	colback=White!100,
	colframe=OliveDrab4!60,
	top=\baselineskip,
	enlarge top by=\topsep,
	overlay unbroken and first={
		\node[thick,draw=OliveDrab3!80!black,fill=OliveDrab1!10,rounded corners] at (frame.north) %
		{\strut{}\if#1\@empty\relax\relax\else~(#1)\fi};
		\vspace{1mm}
	}
}

\newtcolorbox{remark}[1][]{
	breakable,
	enhanced,
	colback=White!100,
	colframe=Goldenrod3!60,
	top=\baselineskip,
	enlarge top by=\topsep,
	overlay unbroken and first={
		\node[thick,draw=Goldenrod3!90,fill=Goldenrod1!10,rounded corners] at (frame.north) %
		{\strut{}\if#1\@empty\relax\relax\else~(#1)\fi};
		\vspace{1mm}
	}
}

\newtcolorbox{example}[1][]{
	breakable,
	enhanced,
	colback=White!100,
	colframe=MediumPurple2!60,
	top=\baselineskip,
	enlarge top by=\topsep,
	overlay unbroken and first={
		\node[thick,draw=MediumPurple2!90,fill=MediumPurple1!10,rounded corners] at (frame.north) %
		{\strut{}\if#1\@empty\relax\relax\else~(#1)\fi};
		\vspace{1mm}
	}
}

\makeatother%включение преамбулы


\begin{document} %начало документа

%\maketitle
\begin{center}


\begin{Huge}

\textbf{Некоторые приложения теоремы Римана-Роха}

%\vspace{5mm}


\vspace{5mm}

\large{ Конспект лекции\\
	по спецкурсу A-III <<Римановы поверхности>>\\
	лектор: Снурницын П.\,В.\\
	конспект подготовила: Писаренкова~Е.\,Д., 601 группа\\
	осень 2025}

\end{Huge}\ \\

%\textcolor{blue}{\textit{\normalsize{в разработке}}}
\end{center}

\thispagestyle{empty}
%\tableofcontents
\newpage
%\include{S0} % вводные слова
\newpage
\section*{Некоторые приложения теоремы Римана-Роха}

Напомним прежде формулировку основного утверждения.

\begin{theorem}[\bf Теорема Римана-Роха]
	Пусть $X$ --- АК (алгебраическая кривая), т.\,е. компактная РП (риманова поверхность), которая разделяет точки и касательные.
	Для любого дивизора $D \in \operatorname{Div} X$ на этой римановой поверхности выполняется равенство
	\begin{equation*}
		%\label{eq1}
		\dim L(D) = \deg D - g(X) + 1 + \dim L^{(1)}(-D).
	\end{equation*}
\end{theorem}
Отметим, что
\begin{equation}
	\label{eq2}
	\dim L^{(1)}(-D) = \dim  L(K-D) = \dim H^1 (D),
\end{equation}
где $K=(\omega)$ --- канонический дивизор. Первое равенство в \eqref{eq2} возможно вследствие изоморфизма пространств, второе возникает при~доказательстве теоремы Римана-Роха. 

\subsection*{Классификация компактных РП (АК)}
Величина $g(X)$ --- род поверхности $X$ ($g\geqslant0$ и с точки зрения топологии означает, что всякая двумерная поверхность гомеоморфна сфере с~некоторым количеством <<ручек>> равным $g$). 

\textbf{1.} $g=0$

Например, сфера --- поверхность рода 0.

\begin{property}[\bf Лемма]
	Если поверхность $X$ --- компактна и $\exists\, p\in X$ такая, что $\dim L(p)>1$, то $X$ изоморфна сфере Римана $\mathbb{C}_{\infty}\equiv \mathbb{C}\cup\{\infty\}$.
\end{property}
\begin{proof}
	$\dim L(p)>1$ означает, что найдется непостоянная функция $f \in L(p)\backslash\mathbb{C}$ (она же непостоянная мероморфная функция из $\mathcal{M}(X)\backslash\mathbb{C}$), со свойством, что точка $p$ --- единственный полюс. Но тогда соответствующее голоморфное отображение
	%(всякой мероморфной функции можно поставить с соответствие голоморфное отображение, таким образом, что)
	 $F: X \to \mathbb{C}_{\infty}$, действующее по закону
%	 \begin{equation*}
	 %	\label{eq3}
	 	\begin{align*}
	 	 &x \mapsto	f(x), \quad x \text{\,--- не полюс},\\
	 	 &p \mapsto	\infty,
	 	\end{align*}
%	 \end{equation*}
	имеет единственный полюс простого порядка, т.\,е. кратность точки $p$ равна 1, и $p$ --- единственный прообраз бесконечно удаленной точки, а значит $\deg F = m_p(f)=1$. Тогда по доказанному ранее отображение $F$ задает изоморфизм римановой поверхности и сферы, т.\,е. $X\simeq \mathbb{C}_{\infty}$.
\end{proof}

\begin{remark}[\bf Следствие]
	Если $X$ --- компактна и рода $g(X)\geqslant 1$, то для любой точки $p \in X$ пространство $L(p)=\mathbb{C}$.
\end{remark}

\begin{theorem}[\bf Теорема]
	Если $X$ --- АК рода $g(X) =0 $, то $X\simeq \mathbb{C}_{\infty}$.
\end{theorem}
\begin{proof}
	Пусть $p\in X$, $K=(\omega)$ --- канонический дивизор степени $\deg K= 2g-2 = -2$. Но тогда и $\deg (K-p)<0$ (она будет $-2$, либо $-3$ в случае, когда точка $p$ бесконечно удаленная). Отрицательная степень означает, что
	$\dim L(K-p)=0$. В итоге по теореме Римана-Роха имеем
	\begin{equation*}
		\dim L(p) = \deg p + 1 - g + \dim L(K-p) = 1+1-0+0 =2.
	\end{equation*}
	То есть существуют непостоянные функции в пространстве $L(p)$ и по доказанной лемме выше $X\simeq \mathbb{C}_{\infty}$.
\end{proof}

\textbf{2.} $g=1$

\begin{theorem}[\bf Теорема]
	Если $X$ --- АК рода $g(X)=1$, то $X$ изоморфна гладкой проективной кривой степени 3 и вкладывается в пространство $\mathbb{P}^2$.
\end{theorem}
\begin{proof}
	Ранее было показано, что если существует дивизор $D$ степени $\,deg D \geqslant 2g+1 = 3$, то он задает голоморфное вложение $$\varphi_{P} : X \mapsto \mathbb{P}^2,$$ причем $\,deg \varphi_P(X)=3.$ Ранее мы доказали, что степень такого отображения совпадает со степенью кривой в обычном смысле.
\end{proof}
\noindent Таким образом, получаем, что компактные поверхности рода 1, обладающие свойством Римана-Роха суть гладкие проективные кривые степени 3, известные также как эллиптические кривые.

\begin{theorem}[\bf Теорема]
	Если $X$ --- АК рода $g(X)=1$, то поверхность $X$ изоморфна комплексному тору $\mathbb{C}/L$.
\end{theorem}
\begin{proof}
	Доказательство проводится на основе теоремы Римана-Роха (см. \cite{B1}).
\end{proof}

\textbf{3.} $g=2$.

В этом случае приведем только такой результат.
\begin{theorem}[\bf Теорема]
	Если $X$ --- АК рода $g(X)=2$, то $X$ --- гиперэллиптическая кривая.
\end{theorem}

\bigskip
Когда мы ввели понятие РП, то приводили различные примеры от простых --- сферы торов, до более сложных --- показывали, что если есть уравнение в виде некоторого многочлена равного нулю, в частности, в виде однородного многочлена равного нулю на проективной плоскости, то это уравнение задает РП. Задать карты задача простая, а вот доказать связность уже сложнее. Используя некоторые <<хитрости>> связность возможно вывести из теоремы Римана-Роха.

\begin{theorem}[\bf Теорема]
	Гладкая проективная кривая $X$ связна в комплексной топологии.
\end{theorem}

\begin{proof}
	Допустим, что мы оперируем полем рациональных функций, то есть у нас есть множество $\mathcal{M}(X)$, которое разделяет точки и касательные поверхности $X$.
	
	Пусть $X$ не связно, то есть $X=X_1 \cup X_2$, где $X_1$, $X_2$ --- непустые, неперескающиеся, замкнутые множества.
	Зафиксируем точку $p \in X_1$ и рассмотрим дивизор $D=(g+1)p$. 
	
	По теореме Римана-Роха $\,\dim L(D)\geqslant \deg D +1-g = 2$, но тогда существует непостоянная $f\in L(D)\subset \mathcal{M}(X)$, для которой $p$ --- единственный полюс. Но тогда $f$ голоморфна на $X_2$, значит $f$ постоянна на $X_2$, следовательно $f$ постоянна на $X$. Получили противоречие.
\end{proof}

%---------------------------------------------
\subsection*{Теоремы Абеля и Якоби}
\begin{example}[ПрПримеры]
	1. Пусть $X=\mathbb{C}_{\infty}$, дивизор $D \in \operatorname{PDiv}(X)$, т.е. $D=(f)$, тогда и только тогда, когда $\deg D =0$.
	
	2. Пусть $X=\mathbb{C}/L$, дивизор $D\in \operatorname{PDiv}(X)$ тогда и только тогда, когда $\deg D =0$ и $A(D)=0$, где $A$ --- отображение Абеля (-Якоби). 
	
	Отображение $A: \operatorname{Div(\mathbb{C}/L)} \mapsto \mathbb{C}/L$.
\end{example}

Перейдем к более общему утверждению.

Пусть $X$ --- АК рода $g(X)=g$ и $\Omega^1(X)\subset \mathcal{M}^{(1)}(x)$ --- пространство мероморфных 1-форм. Отображение $A: X \mapsto \Omega^1 (X)^*$, при этом $A(p)(\omega)=\int\limits_{\gamma_p}\omega$, где $\gamma_p$ --- путь из фиксированной начальной точки $p_0$ в точку $p$. (Однако в этом случае результат определен с точностью до интегралов по замкнутым путям, т.\,е. будет зависеть от выбранного пути, поэтому корректно определить отображение так $A:\;X\mapsto \operatorname{Jac}(X)= \Omega^1(X)^*/\Lambda.$)

\begin{property}[\bf Лемма]
	Отображение  $A$ не зависит от выбранной точки $p_0$.
\end{property}
%------------------------------------------------

\begin{theorem}[\bf Теорема Абеля]
	Пусть $X$ --- АК, дивизор $D=(f)$ тогда и только тогда, когда $\deg D =0$ и образ $D$ при отображении Абеля равен нулю, т.\,е. $A(D)=0$.
\end{theorem}
\begin{proof}
	Доказательство этого утверждения можно увидеть в \cite{B1}, глава $VIII$.
\end{proof}

\subsection*{Теорема Клиффорда}
В примерах, которые мы рассмотрели ранее, дивизор $\,D\,$ такой, что пространство $H^1(D)=0$ (или же $L(K-D)=0$).
Однако в некоторых приложениях возникает потребность анализировать пространство Римана-Роха и для других дивизоров, у которых $H^1(D)\neq0$. 
\begin{difinition}[\bf Определение]
 Дивизор $D \in \operatorname{Div}X: \,H^1(D)\neq 0$ называется \textit{специальным} (или $\dim L(D)\geqslant1$, $\dim L(K-D)\geqslant 1$).
\end{difinition}
В литературе величину $\dim H^1(D)=i(D)$ называют \textit{индекс специальности}.

\begin{theorem}[\bf Теорема Клиффорда]
	Если $D$ --- специальный дивизор, то 
	
	1) $\dim L(D)+\dim L(K-D) \leqslant g+1$
	
	2) $2\dim L(D) \leqslant \deg D+2$
\end{theorem}
\begin{proof}
	Доказательство см. в \cite{B1}.
\end{proof}

\subsection*{Существование мероморфных 1-форм}
Ранее мы решали задачу Миттаг-Леффлера о том, что существует мероморфная функция с заданными свойствами в заданных точках. Рассмотрим теперь вопрос о том, что будет для мероморфных 1-форм.

\begin{theorem}[\bf Теорема]
	Если $X$---АК и $p_1,\ldots, p_n \in X$. Есть $n$ чисел $r_1,\ldots,r_n \in \mathbb{C}$ таких, что $\sum r_i =0$. Тогда существует $\omega \in \mathcal{M}^{(1)}(X)$ такая, что 
	$p_i$ --- единственные простые полюса (т.\,е. $\nu_{p_i}(\omega)=-1$) и $\operatorname{Res}_{p_i} \omega = r_i$.
\end{theorem}
\begin{proof}
	Следует из теоремы Римана-Роха.
\end{proof}


\subsection*{Размерность пространств модулярных форм}
Речь пойдет о приложении к модулярным формам, которые возникали на спецкурсе A-II <<Решетки и формы>>. 

Напомним некоторые сведения.

Рассмотрим 
$$
SL_2(\mathbb{Z}) = \left\{ \begin{pmatrix}
	a&b\\
	c&d
\end{pmatrix}, ad-bc=1 \right\}.
$$
Она действует на $\mathbb{H}=\{z\in\mathbb{C}: Imz>0\}$ дробно-линейными преобразованиями
$$
gz = \dfrac{az+b}{cz+d}.
$$
Элемент $\begin{pmatrix}
	1&0\\
	0&1
\end{pmatrix}=I$ тождественный. 
Элемент $\begin{pmatrix}
	-1&0\\
	0&-1
\end{pmatrix}=-I$ действует тривиально. Тогда $SL_2(Z)/\{\pm I\}=\Gamma(1)$ --- полная модулярная группа.

Рассмотрим несколько ее подгрупп.

\noindent $\circ$ Главные конгруэнц подгруппы уровня $n$.
$$
\Gamma(n)=\left\{\begin{pmatrix}
	a&b\\
	c&d
\end{pmatrix} \in SL_2(\mathbb{Z}): \begin{pmatrix}
a&b\\
c&d
\end{pmatrix} \equiv\begin{pmatrix}
1&0\\
0&1
\end{pmatrix} (mod\, n)\right\},
$$
$$
\Gamma(1) = SL_2(\mathbb{Z}),
$$
$\Gamma \subset SL_2(\mathbb{Z})$ называют конгруэнц подгруппой, если $\exists n: \Gamma(n)\subset\Gamma$.

\noindent $\circ$ Для любой конгруэнц подгруппы $y(\Gamma)=\mathbb{H}/\Gamma$ --- РП. Она не компактна.

\noindent $X(\Gamma)=\overline{\mathbb{H}}/\Gamma$, где $\overline{\mathbb{H}}=\mathbb{H}\cup\mathbb{Q}\cup\{\infty\}$ --- компактификация. 

\noindent $\circ$ Фундаментальные области --- множество представителей классов эквивалентности. $\Gamma=\Gamma(1)=SL_2(\mathbb{Z})$.

\begin{center}
\begin{tikzpicture}[scale=2]
	% ЛЕВАЯ КАРТИНКА
	\begin{scope}[xshift=-2cm]
		\draw[->, thin] (-1.2,0) -- (1.2,0) node[right] {};
		\draw[->, thin] (0,-0.2) -- (0,2.1) node[above] {$i_{\infty}$};
		\draw[blue, thick, domain=-0.5:0, samples=100] 
		plot (\x, {sqrt(1-\x*\x)});
		\draw[dashed, blue, thick, domain=0:0.5, samples=100] 
		plot (\x, {sqrt(1-\x*\x)});
		\draw[blue, thick] (-0.5,0.866) -- (-0.5,2);
		\draw[dashed, blue, thick] (0.5,0.866) -- (0.5,2);		
		\draw[dashed] (-0.5,0) -- (-0.5,0.866) node[midway,left] {};
		\draw[dashed] (0.5,0) -- (0.5,0.866) node[midway,right] {};
		\node[below] at (-0.5,0) {$-0.5$};
		\node[below] at (0.5,0) {$0.5$};
		\node[left] at (0,1.1) {$i$};
		\node[left] at (-0.5,0.8) {$\omega$};
	\end{scope}
	
	% ПРАВАЯ КАРТИНКА
	\begin{scope}[xshift=2cm]
		\draw[->, thin] (-1.2,0) -- (1.2,0) node[right] {};
		\draw[->, thin] (0,-0.2) -- (0,1.2) node[above] {};
		\draw[red, thick, domain=60:120, samples=100] 
		plot ({-cos(\x)}, {sin(\x)});
		\draw[red, thick, domain=0:0.5, samples=100] 
		plot (\x, {sqrt(1-(\x-1)*(\x-1)});
		\draw[red, thick, domain=-0.5:0, samples=100] 
		plot (\x, {sqrt(1-(\x+1)*(\x+1)});
		\draw[dashed] (-0.5,0) -- (-0.5,0.866);
		\draw[dashed] (0.5,0) -- (0.5,0.866);
		\node[below] at (-0.5,0) {$0.5$};
		\node[below] at (-0.1,0) {$0$};
		\node[below] at (0.5,0) {$-0.5$};
		%\draw[red, thick] (0,0) -- (-0.5,0.866);
		%\draw[red, thick] (0,0) -- (0.5,0.866);
	\end{scope}
	% СТРЕЛКА И ОБОЗНАЧЕНИЕ ПРЕОБРАЗОВАНИЯ
	\draw[->, very thick] (-0.5,0.5) -- (0.5,0.5) node[midway,above] {$-\dfrac{1}{z}$};
\end{tikzpicture}
\end{center}

\vspace{-5mm}
\noindent $i_{\infty}$ или $0$ называется параболической точкой (casp). Точки $i$, $\omega$ называются эллиптическими точками.

\noindent $\circ$ $\Gamma$ --- конгруэнц подгруппа, если $z: h_z = |\Gamma_z|>1$, то $z$ называется эллиптической точкой для $\Gamma$.

 $h_z$ называется периодом
($\Gamma(1)$, $h_i=2$, $h_{\omega}=3$).

\noindent $\circ$ $\Gamma$ --- конгруэнц подгруппа, класс $\Gamma z$, где $z\in\mathbb{Q}\cup\{\infty\}$, называется параболической точкой.

$$
\forall z \in\mathbb{Q}\cup\{\infty\} \qquad \exists g\in SL_2(\mathbb{Z}): \quad gz=i_{\infty}
$$
\vspace{-3mm}
$$
h_z=|\Gamma_{i_{\infty}}/(g((\pm z)\Gamma)g^{-1})_{i_{\infty}}|
$$
\vspace{-2mm}

Рассмотрим отображение $q=\exp\left(2\pi i \dfrac{z}{n}\right)$.

\noindent $\circ$ $X(\Gamma)=\overline{\mathbb{H}}/\Gamma$.
\vspace{1mm}
$$
g = 1 + \dfrac{d}{12} + \dfrac{e_2}{3} + \dfrac{e_1}{4} + \dfrac{e_{\infty}}{2}
$$ 
\noindent $e_2$, $e_1$ называются эллиптическими точками периода 2, 1.
$e_{\infty}$ называется параболической точкой.

\hbox{$d=\deg F$, $F: \; X(\Gamma) \mapsto X(\Gamma(1)): \; \Gamma z \mapsto \Gamma(1/z)$. А также $g(X(\Gamma(1))) = 0$.}

\noindent $\circ$ Мероморфная на плоскости $\mathbb{H}$ функция $f$ называется слабо модулярной функцией веса $2k$, если
 для любого $g\in SL_2{\mathbb{Z}}$ и $ z \in \mathbb{H}$, справедливо соотношение
$$
f\left(\dfrac{az+b}{cz+d}\right) = (cz+d)^{2k}f(z).
$$

$f|_{g,2k} = (cz+d)^{-2k}f(gz)$ --- значение функции $f|_{g,2k}=f$ в точке $z$.

\begin{thebibliography}{00}
	
	\bibitem{B1}
	Miranda R. Algebraic Curves and Riemann Surfaces, AMS, 1995. xxi+390~p.

	
\end{thebibliography}



%----------------------------------------------------------------------

%\newpage

%\begin{remark}[\bf Замечание]

%\end{remark}

%\begin{difinition}[\bf Определение]

%\end{difinition}

%\begin{theorem}[\bf Теорема]
	
%\end{theorem}

%\begin{example}[\bf Пример]
	
%\end{example}

%\begin{property}[\bf Свойство]
	
%\end{property} % лекция

%\addcontentsline{toc}{section}{Список литературы}
%\begin{thebibliography}{99}  

%\end{thebibliography}
\end{document} %конец документа

