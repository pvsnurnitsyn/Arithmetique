%%% Работа с русским языком
\usepackage{cmap}					% поиск в PDF
\usepackage{mathtext} 				% русские буквы в формулах
\usepackage[T2A]{fontenc}			% кодировка
\usepackage[utf8]{inputenc}			% кодировка исходного текста
\usepackage[english,russian]{babel}	% локализация и переносы
\usepackage{indentfirst} %Абзацный отступ первого после заголовка абзаца
\frenchspacing
\usepackage{misccorr}%Точки после номеров заголовков 


\renewcommand{\epsilon}{\ensuremath{\varepsilon}}
\renewcommand{\phi}{\ensuremath{\varphi}}
\renewcommand{\kappa}{\ensuremath{\varkappa}}
\renewcommand{\le}{\ensuremath{\leqslant}}
\renewcommand{\leq}{\ensuremath{\leqslant}}
\renewcommand{\ge}{\ensuremath{\geqslant}}
\renewcommand{\geq}{\ensuremath{\geqslant}}
\renewcommand{\emptyset}{\varnothing}

%%% Дополнительная работа с математикой
\usepackage{amsmath,amsfonts,amssymb,amsthm,mathtools} % AMS
\usepackage{icomma} % "Умная" запятая: $0,2$ --- число, $0, 2$ --- перечисление
\usepackage{mathrsfs}

%% Номера формул
%\mathtoolsset{showonlyrefs=true} % Показывать номера только у тех формул, на которые есть \eqref{} в тексте.
%\usepackage{leqno} % Нумереация формул слева

%% Свои команды
\DeclareMathOperator{\sgn}{\mathop{sgn}}
\newcommand{\nullcounter}{
\setcounter{theorem}{0}
\setcounter{lemma}{0}
\setcounter{proposition}{0}
\setcounter{definition}{0}
\setcounter{corollary}{0}
\setcounter{task}{0}
\setcounter{example}{0}
\setcounter{remark}{0}
\setcounter{equation}{0}
} % Обнуление всех счётчиков типа теорема

%% Перенос знаков в формулах (по Львовскому)
\newcommand*{\hm}[1]{#1\nobreak\discretionary{}
{\hbox{$\mathsurround=0pt #1$}}{}}

%%% Работа с картинками
\usepackage{graphicx}  % Для вставки рисунков
\graphicspath{{images/}{images2/}}  % папки с картинками
\setlength\fboxsep{3pt} % Отступ рамки \fbox{} от рисунка
\setlength\fboxrule{3pt} % Толщина линий рамки \fbox{}
\usepackage{wrapfig} % Обтекание рисунков текстом

%%% Работа с таблицами
\usepackage{array,tabularx,tabulary,booktabs} % Дополнительная работа с таблицами
\usepackage{longtable}  % Длинные таблицы
\usepackage{multirow} % Слияние строк в таблице

%%% Теоремы
\theoremstyle{plain} %Стиль курсивом 
%\newtheorem{theorem}{Теорема}
%\newtheorem{lemma}{Лемма}
\newtheorem{proposition}{Предложение}
%-------------------------------------------------
\theoremstyle{definition} %Стиль без курсива
%\newtheorem{definition}{Определение}
\newtheorem{corollary}{Следствие}
\newtheorem{task}{№}
%\newtheorem{example}{Пример}
\newtheorem{axiom}{Аксиома}

\theoremstyle{remark}
%\newtheorem{remark}{Замечание}
%\newtheorem*{remark*}{Замечание}
\newtheorem*{solution}{Решение}

%%% Программирование
\usepackage{etoolbox} % логические операторы

%%% Страница
\usepackage{extsizes} % Возможность сделать 14-й шрифт
\usepackage{geometry} % Простой способ задавать поля
	\geometry{top=25mm}
	\geometry{bottom=25mm}
	\geometry{left=25mm}
	\geometry{right=20mm}
 %
\usepackage{fancyhdr} % Колонтитулы
 \pagestyle{fancy}
	\renewcommand{\headrulewidth}{1pt}  % Толщина линейки, отчеркивающей верхний колонтитул
	\lfoot{Лектор: Снурницын П.\,В.}
\rfoot{Спецкурс A-III <<Римановы поверхности>>}
	\rhead{НЕКОТОРЫЕ ПРИЛОЖЕНИЯ ТЕОРЕМЫ РИМАНА-РОХА}
%	\chead{}
 	\lhead{ЛЕКЦИЯ №11}
%	\cfoot{Нижний в центре} % По умолчанию здесь номер страницы

\usepackage{setspace} % Интерлиньяж
%\onehalfspacing % Интерлиньяж 1.5
%\doublespacing % Интерлиньяж 2
%\singlespacing % Интерлиньяж 1

\usepackage{statmath}

\usepackage{lastpage} % Узнать, сколько всего страниц в документе.

\usepackage{soul} % Модификаторы начертания

\usepackage{hyperref}
\usepackage[usenames,dvipsnames,svgnames,x11names, table,rgb]{xcolor}


\hypersetup{				% Гиперссылки
    unicode=true,           % русские буквы в раздела PDF
    pdftitle={Заголовок},   % Заголовок
    pdfauthor={Автор},      % Автор
    pdfsubject={Тема},      % Тема
    pdfcreator={Создатель}, % Создатель
    pdfproducer={Производитель}, % Производитель
    pdfkeywords={keyword1} {key2} {key3}, % Ключевые слова
    colorlinks=true,       	% false: ссылки в рамках; true: цветные ссылки
    linkcolor=black,          % внутренние ссылки
    citecolor=black,        % на библиографию
    filecolor=magenta,      % на файлы
    urlcolor=blue           % на URL
}

\usepackage{csquotes} % Еще инструменты для ссылок

%\usepackage[style=authoryear,maxcitenames=2,backend=biber,sorting=nty]{biblatex}

\usepackage{multicol} % Несколько колонок

\usepackage{tikz} % Работа с графикой
\usepackage{pgfplots}
\usepackage{pgfplotstable}

%GeoGebra
\usepackage{pgf}
\pgfplotsset{compat=1.15}
\usepackage{mathrsfs}
\usetikzlibrary{arrows}
\usetikzlibrary{patterns}

\usepackage {lscape}
\usepackage{multicol}

\usepackage{tikzsymbols}
\usetikzlibrary {arrows.meta}
\usetikzlibrary{graphs}
\usetikzlibrary{graphs.standard}
\usetikzlibrary{patterns}

\usepackage{asymptote}
%\usepackage{tcolorbox} 
\usepackage{rotating}

\newcommand{\answer}[1]{\begin{flushright}\textit{Ответы:} \texttt{#1}\end{flushright}}

\usepackage[utf8]{inputenc}
\usepackage[most]{tcolorbox}
\usepackage{lipsum}
% counters
\newcounter{theorem}
\newcounter{lemma}
\newcounter{difinition}
\newcounter{mark}
%\counterwithin{theorem}{section}
%\counterwithin{lemma}{section}

% names for the structures
\newcommand\theoname{Théorème}
\newcommand\lemmname{Lemme}
\newcommand\difname{difinition}

\usepackage{amsmath}
\usepackage{tikz}


\makeatletter
% environment for theorems
\newtcolorbox{theorem}[1][]{
	breakable,
	enhanced,
	colback=White!100,
	colframe=Coral3!70,
	top=\baselineskip,
	enlarge top by=\topsep,
	overlay unbroken and first={
		\node[thick,draw=Coral3!80!black,fill=Coral1!10,rounded corners] at (frame.north) %
		{\strut{}\if#1\@empty\relax\relax\else~(#1)\fi};
		\vspace{1mm}
	}
}

% environment for property
\newtcolorbox{property}[1][]{
	breakable,
	enhanced,
	colback=White!100,
	colframe=LightBlue3!100!black,
	top=\baselineskip,
	enlarge top by=\topsep,
	overlay unbroken and first={
		\node[thick,draw=LightBlue3!85!black,fill=LightBlue2!25,rounded corners] at (frame.north) %
		{\strut{}\if#1\@empty\relax\relax\else~(#1)\fi};
		\vspace{1mm}
	}
}
% environment for lemas
\newtcolorbox{difinition}[1][]{
	breakable,
	enhanced,
	colback=White!100,
	colframe=OliveDrab4!60,
	top=\baselineskip,
	enlarge top by=\topsep,
	overlay unbroken and first={
		\node[thick,draw=OliveDrab3!80!black,fill=OliveDrab1!10,rounded corners] at (frame.north) %
		{\strut{}\if#1\@empty\relax\relax\else~(#1)\fi};
		\vspace{1mm}
	}
}

\newtcolorbox{remark}[1][]{
	breakable,
	enhanced,
	colback=White!100,
	colframe=Goldenrod3!60,
	top=\baselineskip,
	enlarge top by=\topsep,
	overlay unbroken and first={
		\node[thick,draw=Goldenrod3!90,fill=Goldenrod1!10,rounded corners] at (frame.north) %
		{\strut{}\if#1\@empty\relax\relax\else~(#1)\fi};
		\vspace{1mm}
	}
}

\newtcolorbox{example}[1][]{
	breakable,
	enhanced,
	colback=White!100,
	colframe=MediumPurple2!60,
	top=\baselineskip,
	enlarge top by=\topsep,
	overlay unbroken and first={
		\node[thick,draw=MediumPurple2!90,fill=MediumPurple1!10,rounded corners] at (frame.north) %
		{\strut{}\if#1\@empty\relax\relax\else~(#1)\fi};
		\vspace{1mm}
	}
}

\makeatother