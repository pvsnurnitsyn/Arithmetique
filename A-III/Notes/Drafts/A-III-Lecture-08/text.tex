%%%%%%%%%%%%%%%%%%%%%%%%%%%%%%%%%%%%%%%%%%%%%%%%%%%%%%%%%%%%%%%
%
% Welcome to Overleaf --- just edit your LaTeX on the left,
% and we'll compile it for you on the right. If you open the
% 'Share' menu, you can invite other users to edit at the same
% time. See www.overleaf.com/learn for more info. Enjoy!
%
%%%%%%%%%%%%%%%%%%%%%%%%%%%%%%%%%%%%%%%%%%%%%%%%%%%%%%%%%%%%%%%
\documentclass{report}
\usepackage[T2A]{fontenc}
\usepackage{xcolor}
\usepackage{amsmath}
\usepackage{amsthm}
\usepackage{amssymb}
\usepackage{csquotes}
\usepackage{stmaryrd}
\usepackage{tikz}
\usepackage[normalem]{ulem}
\usepackage{hyperref}
\usepackage{bussproofs}
\usepackage{cancel}
\usepackage{lplfitch}


%Hyphenation rules
%--------------------------------------
\usepackage{hyphenat}
\hyphenation{ма-те-ма-ти-ка вос-ста-нав-ли-вать}
%--------------------------------------
\usepackage[english, russian]{babel}
\newcommand*{\abstraction}[2]{\lambda#1.#2}
\newcommand*{\bred}{\rightarrow_\beta}
\newcommand*{\lbred}{\leftarrow_\beta}
\newcommand*{\mbred}{\twoheadrightarrow_\beta}
\newcommand*{\combK}{\abstraction{xy}{x}}
\newcommand*{\combS}{\abstraction{fgx}{fx(gx)}}
\newcommand*{\aeq}{=_\alpha}
\newcommand*{\beq}{=_\beta}
\newcommand*{\equals}{\,\mathbf{equals}\,}
\newcommand*{\iszero}{\,\mathbf{is\_zero}\,}
\newcommand*{\zero}{\,\mathbf{zero}\,}
\newcommand*{\two}{\,\mathbf{two}\,}
\newcommand*{\isprime}{\,\mathbf{is\_prime}\,}
\newcommand*{\nil}{\,\mathbf{nil}\,}
\newcommand*{\minus}{\,\mathbf{minus}\,}
\newcommand*{\mult}{\,\mathbf{mult}\,}
\newcommand*{\plus}{\,\mathbf{plus}\,}
\newcommand*{\ifs}{\,\mathbf{if}\,}
\newcommand*{\pred}{\,\mathbf{pred}\,}
\newcommand*{\first}{\,\mathbf{first}\,}
\newcommand*{\second}{\,\mathbf{second}\,}
\newcommand*{\pair}{\,\mathbf{pair}\,}
\newcommand*{\lt}{\,\mathbf{lt}\,}
\newcommand*{\gt}{\,\mathbf{gt}\,}
\newcommand*{\nots}{\,\mathbf{not}\,}
\newcommand*{\isnil}{\,\mathbf{is\_nil}\,}
\newcommand*{\fold}{\,\mathbf{fold}\,}
\newcommand*{\ors}{\,\mathbf{or}\,}
\newcommand*{\unbound}[1]{\underline{#1}}
\newcommand*{\bound}[1]{\underline{\underline{#1}}}

\newcommand*{\ERule}{\RightLabel{$\rightarrow \mathbf{E}$}}
\newcommand*{\IRule}{\RightLabel{$\rightarrow \mathbf{I}$}}
\newcommand*{\VRule}{\RightLabel{\textbf{Var}}}



\renewcommand*{\min}{\,\mathbf{min}\,}
\renewcommand*{\max}{\,\mathbf{max}\,}
\renewcommand*{\and}{\,\mathbf{and}\,}

\renewcommand*{\t}{\,\mathbf{true}\,}
\renewcommand*{\f}{\,\mathbf{false}\,}
\begin{document}

\counterwithout{section}{chapter}
\usetikzlibrary {graphs, fit, decorations}
\newtheorem{definition}{Определение}[section]
\newtheorem{statement}{Утверждение}[section]
\newtheorem{theorem}{Теорема}[section]
\newtheorem{example}{Пример}[section]
\newtheorem{lemma}{Лемма}[section]
\newtheorem{note}{Замечание}[section]
\newtheorem{corollary}{Следствие}[section]

\setlength{\parindent}{0pt}

\newcommand*{\Deg}{\mathbf{Deg \,}}
\newcommand*{\Div}{\mathbf{Div \,}}
\newcommand*{\Dim}{\mathbf{Dim \,}}
\newcommand*{\PDiv}{\mathbf{PDiv \,}}


\newpage
\section{Пространство Римана-Роха и Линейные системы}

\begin{itemize}
    \item Пусть $X$ - компактная риманова поверхность.
    \item $\Div X = \{D = \sum n_p \cdot p\}$ \\
          $n_p = D(p), D: X \rightarrow \mathbb{Z}$ \\
          $\Deg D = \sum {n_p}$
    \item $D = p$ -- простые дивизоры \\
          $D = (f) = \sum {\nu_p (f) \cdot p}$ -- главные \\
          $D = (w) = \sum {\nu_p (w) \cdot p}$ -- канонические \\
    \item $\mathbb{C} / X = \Div X / \PDiv X$
    \item $D_1 \sim D_2 \Leftrightarrow D_1 - D_2 = (f) \in \PDiv X$
\end{itemize}

\begin{definition}
    $f, g \in \mathcal{M}_X, D \in \Div X$. $f \equiv g \; (D) \leftrightharpoons (f - g) \geq D$ или, что тоже самое, $ \forall p \in X \hspace{10pt} \nu_p (f - g) \geq n_p = D(p)$
\end{definition}

\begin{note}
    $D = \prod_p {p^{n_p}}, D_1 \leq D_2 \sim D_1 | D_2$
\end{note}

\begin{definition}
    $L(D) = \{f \in \mathcal{M}_X: f \equiv 0 \; (-D)\}$, т.е. $f: (f) \geq -D$ или $\nu_p (f) \geq -n_p$. \\
    $L(D)$ -- линейное пространство функций с полюсами ограниченными дивизором $D$ [пространство Римана-Роха]. \\
    $(D(p) = n_p = n > 0 \hspace{10pt} f \in L(D) \Leftrightarrow \nu_p (f) \geq n$ т.е. полюс в $p$ порядка $\leq n )$
\end{definition}

По определению $L(0) = O_x$ -- пространство голоморфных функций.

\begin{lemma}
    $D_1 \leq D_2 \Rightarrow L(D_1) \subseteq L(D_2)$
\end{lemma}

\begin{lemma}
    $X$ -- компактная РП: 
    \begin{enumerate}
        \item $L(0) = \mathbb{C}$ \\
        \item $\Deg D < 0 \Rightarrow L(D) = \{0\}$
    \end{enumerate}
\end{lemma}

\begin{proof}
    \hspace{}
    \begin{enumerate}
        \item Доказано ранее
        \item $f \in L(D), \hspace{10pt} D_1 = (f) + D \geq 0 \hspace{5pt} (\nu_p (f) + n_p \geq 0) \Rightarrow \Deg D_1 \geq 0$, \\
        но $\Deg D_1 = \Deg (f) + \Deg D \leq 0$. Если $f \not \equiv 0$ получаем противоречие.
    \end{enumerate}
\end{proof}

\begin{lemma}
    $X$ -- компактная РП, $\Deg D = 0$ 
    \begin{enumerate}
        \item $D \sim 0 \Rightarrow \Dim L(D) = 1$
        \item $D \not \sim 0 \Rightarrow L(D) = \{0\}$
    \end{enumerate}
\end{lemma}

\begin{proof}
    \hspace{} \\
    \begin{enumerate}
        \item $D \sim 0 \Leftrightarrow D(f). \hspace{10pt} L(D)  = \{g: \underbrace{(g) + (f)}_{=(gf)} \geq 0\} \\
        (gf) \geq 0 \Rightarrow gf \in O_x \Rightarrow gf = c \in \mathbb{C}.$ \\
        Таким образом $L(D) = \{c \cdot \frac1f, c \in \mathbb{C}\} = <\frac1f> \Rightarrow \Dim L(D) = 1$
        \item $D \not \sim 0, \Deg D = 0$ \\
        $g \in L(D) \Leftrightarrow \underbrace{(g) + D}_E \geq 0$ \\ 
        $\Deg E = \Deg (g) + \Dev D = 0 = \sum n_p, \hspace{10pt} n_p \geq 0 \Rightarrow n_p = 0 \Rightarrow E = 0$ 
    \end{enumerate}
\end{proof}

\begin{lemma}
    $D_1 \sim D_2 \Rightarrow L(D_1) \cong L(D_2)$
\end{lemma}

\begin{proof}
    $D_1 = D_2 + (h) \hspace{10pt} f \in L(D_1) \Leftrightarrow (f) \geq -D_1 $ \\
    $(hf) = (h) + (f) \geq (h) - D_1 = -D_2 \Leftrightarrow hf \in L(D_2)$ \\
    Обратно: $f \in L(D_2) \rightarrow \frac{f}{h} \in L(D_1)$, т.е. умножение на $h$ задает изоморфизм.
\end{proof}

\begin{definition}
    $L^{(1)}(D) = \{w \in \mathcal{M}^{(1)}_X: (w) \geq -D\}$ \\ 
    $L^{(1)}(0)$ -- пространство голоморфных форм.
\end{definition}

\begin{lemma}
    $D_1 \sim D_2 \Rightarrow L^{(1)}(D_1) \cong L^{(1)}(D_2)$
\end{lemma}

\begin{proof}
    Аналогично
\end{proof}

\begin{lemma}
    $L^{(1)}(D) \cong L(D + K)$, где $K = (w)$ -- канонический дивизор.
\end{lemma}

\begin{proof}
    $f \in L(D + K) \Leftrightarrow (f) + D + K \geq 0$ \\
    $fw \in \mathcal{M}^{(1)}_X \hspace{10pt} (fw) = (f) + K \geq -D \Rightarrow fw \in L^{(1)}(D)$ \\ 
    Обратно $w' \in L^{(1)}(D) \Rightarrow \exists f \in \mathcal{M}_X $ (лемма лекция 7) \\
    $(f) + D + \underset{=(w)}{K} = (fw) + D = (w') + D \geq 0$ \\
    $f \in L(D + K)$
\end{proof}

\begin{theorem}
    $X = \mathbb{C}_\infty, \hspace{10pt} D \in \Div \mathbb{C}_\infty, \Deg D \geq 0$ \\
    $D = \sum_{i=1}^n {e_i \cdot \lambda_i} + e_\infty \cdot \infty$. $\hspace{10pt} \lambda_i \in \mathbb{C} = \mathbb{C}_\infty \setminus \{\infty\}$ \\
    $f_D(z) = \prod_{i=1}^n (z - \lambda_i)^{-e_i}$, тогда $L(D) = \{g(z)f_D(z)\}$, где $g \in \mathbb{C}[z], \Deg g \leq \Deg D$
\end{theorem}

\begin{proof}
    $L' = \{g(z)f_D(z)\}$, где $g \in \mathbb{C}[z], \Deg g \leq \Deg D$ \\
    Возьмём $g \in \mathbb{C}[z], \Deg g = d$ \\
    $(gf_D) + D = \underset{\geq -d \cdot \infty}{(g)} + \underset{=\sum(-e_i)\lambda_i + (\sum e_i) \cdot \infty}{(f_D)} + \underset{\sum e_i \lambda_i + e_\infty \cdot \infty}{D} = \\ = (-d + \sum e_i + e_\infty) \cdot \infty \geq 0 \Leftrightarrow d \leq \Deg D$ \\
    Таким образом $d \leq \Deg d \Rightarrow gf_D \in L(D), L' \subset L(D).$ \\
    Пусть $h = L(D) \setminus \{0\}, g = \frac{h}{f_D}$ \\
    $(g) = (h) - (f_D) \geq -D - (f_D) = -(D + (f_D)) = -\underbrace{(\sum e_i + e_\infty)}_{\Deg D \geq 0} \cdot \infty \Leftrightarrow$ у $g$ нет полюсов кроме полюса порядка $\leq \Deg d$ в точке $\infty \Rightarrow g \in \mathbb{C}[z]$
\end{proof}

\begin{corollary}
    $D = \Div \mathbb{C}_\infty$, тогда $\Dim L(D) =  \begin{cases}
        0, \Deg D < 0 \\
        1 + \Deg D, \Deg D > 0
    \end{cases}$
\end{corollary}

\begin{theorem}
    $D = \mathbb{C} / L$ \\
    \begin{enumerate}
        \item $\Deg D < 0 \Rightarrow L(D) = \{0\}$ 
        \item $\Deg D > 0 \Rightarrow \Dim L(D) = \Deg D$
        \item $\Deg D = 0$
        \begin{enumerate}
            \item $D \sim 0 \Rightarrow \Dim L(D) = 1$
            \item $D \not \sim 0 \Rightarrow L(D) = \{0\}$
        \end{enumerate}
    \end{enumerate}
\end{theorem}

\begin{lemma}
    $D \in \Div X. p \in X$
    \begin{itemize}
        \item Либо $L(D - p) = L(D)$
        \item либо $\underbrace{\Dim L(D) / L(D - p)}_{\mathbf{codim}_{L(D)}L(D-p)} = 1$
    \end{itemize}
    $(D - p \leq D \Rightarrow L(D - p) \subseteq L(D))$ \\
    $(\text{или } \Dim L(D) / L(D - p) \leq 1)$
\end{lemma}

\begin{proof}
    $n = - D(p). f \in L(D) \Leftrightarrow$ в окрестности $p, \hspace{5pt} f$ имеет вид $cz^m + \ldots$ \\
    Рассмотрим $\alpha: L(D) \rightarrow \mathbb{C}: f \mapsto c$ -- линейное отображение. \\
    $\mathbf{Ker \,} \alpha = \{f: c = 0, \text{т. е. } \nu_p(f) > n\} = L(D - p)$. \\
    Если $\alpha \equiv 0 \Rightarrow L(D) = \mathbf{Ker \,} \alpha = L(D - p)$ \\
    Либо $L(D) / L(D - p) \cong \mathbb{C}$
\end{proof}

\begin{theorem}
    $X$ -- компактная РП. $D \in \Div x$. Тогда $\Dim L(D) < \infty$. Именно, если $D = P - N \hspace{10pt} P, N > 0$ тогда $\Dim L(D) \leq 1 + \Deg P$
\end{theorem}

\begin{proof}
    Индукция по $\Deg P$. \\
    $\Deg P = 0 \Leftrightarrow P = 0, \hspace{5pt} L(P) = L(0) = O_x \cong \mathbb{C}$, т.е. $\Dim L(P) = 1$ \\ 
    $D \leq P \Rightarrow L(D) \subseteq L(P) \Rightarrow \Dim L(D) \leq \Dim L(P) = 1 = 1 + 0 = 1 + \Deg P $ \\
    Пусть верно для $\Deg P \leq k - 1$ \\
    Рассмотрим для $\Deg P = k \geq 1$ \\
    $\exists p \in supp \, P: P(p) \geq 1. \hspace{5pt} D - p = \underbrace{P - p}_{\geq 0} - N \hspace{30pt} \Deg(P - p) = k - 1$ \\
    $\Rightarrow (\text{индукция}) \, \Dim(D - p) \leq 1 + \Deg (P - p) = k = \Deg P$ \\
    $\Dim L(D) / L(D - p) = \Dim L(D) - \Dim L(D - p) \leq 1 \\$
    $\Dim L(D) \leq 1 + \Dim L(D - p) \leq 1 + \Deg P$
\end{proof}

\begin{corollary}
    $X$ -- компактная РП, $D \in \Div X$, тогда $\Dim L^{(1)} < \infty$
\end{corollary}

\begin{proof}
    Следует из изоморфизма $\Dim L^{(1)}(D) \sim L(D + K)$
\end{proof}

\subsection{Линейные системы}

\begin{definition}
    $D \in \Div X$. Полной линейной системой $|D| = \{E \in \Div X: E \sim D, E \geq 0\}$
\end{definition}

\begin{lemma}
    \hspace{} \\
    \begin{enumerate}
        \item $E \in |D|, |E| = |D|$
        \item $X$ -- компактно, $\Deg D < 0 \Rightarrow |D| = \emptyset$
    \end{enumerate}
\end{lemma}

\begin{definition}
    $V$ -- конечномерное векторное пространство, проективизация V: $\mathbb{P}(V) = \{$множество линейных подпространств $ W \subset V, \Dim W = 1\}$ \\
    $L(D)$, $\mathbb{P}(L(D))$ имеет смысл. 
\end{definition}

\begin{lemma}
    $X$ - компактно, $S: \mathbb{P}(L(D)) \rightarrow |D|: <f> \mapsto (f) + D$ -- 1 к 1
\end{lemma}

\begin{proof}
    $E \in |D|, E \sim D \Leftrightarrow E = (f) + D \geq 0 \Rightarrow (f) \in L(D)$ \\
    $S(f) = E$, т.е. $S$ -- сюръективное. \\
    Пусть $S(f) = S(g) \Leftrightarrow (f) + D = (g) + D \Rightarrow (f) = (g) \Rightarrow (\frac{f}{g}) = 0 \Rightarrow \\ \Rightarrow \frac{f}{g} \equiv \mathbf{const} \Rightarrow f = \lambda g \Rightarrow <f> = <g>$ 
\end{proof}

\begin{definition}
    Линейной системой $\Lambda$ называется $\Lambda \subset |D|:$ \\ 
    $S(\Lambda)$ -- линейное пространство $\mathbb{P}(L(D)).$
\end{definition}

\begin{definition}
    $X$ -- РП, отображение \\ $\phi: X \rightarrow \mathbb{P}^n = \mathbb{P}^n(\mathbb{C})$ называется голоморфным в точке $p \in X$, если \\ $\exists g_0,\ldots, g_n \in O_{n, p}: \phi(x) = [g_0(x), \ldots g_n(x) ]$ в окрестности $p \in X.$ \\
    $\phi$ голоморфное $\Leftrightarrow$ если оно голоморфно $\forall p \in X$.
\end{definition}

\begin{lemma}
    Пусть $f = (f_0, \ldots f_n): f_i \in \mathcal{M}_X$. Такие, что не все \\ $f_i \equiv 0.$ $\phi_f: X \rightarrow \mathbb{P}^n:p \mapsto [f_0(p), \ldots, f_n(p)].$ \\
    $\phi_f$ продолжается до голоморфного на всей РП $X$.
\end{lemma}

\begin{proof}
$\phi_f$ не определена в $p$ если
\begin{enumerate}
    \item $ \exists i$ такое, что $p$ это полюс $f_i$.
    \item $ \forall i$ такое, что $p$ ноль $f_i$.
\end{enumerate}
\end{proof}

Т. е. $n = \min_{1 \leq i \leq n} \nu_p (f_i)$
\begin{itemize}
    \item $n < 0 \Rightarrow 1)$
    \item $n > 0 \Rightarrow 2)$
    \item $n = 0$ -- определено
\end{itemize}

Пусть $p: n = \min_{1 \leq i \leq n} \nu_p (f_i) = 0$. \\ 
$z$ -- карта с центром в точке $p$ в окрестности $p$ без других нулей и полюсов. \\
$g_i(z) = z^{-n}f_i{z}, \hspace{10pt} z \not = 0$. (т. е. при $x \not = p$) \\
$\phi_f(z)=[f_0(z), \ldots, f_n(z)] = [z^n g_0(z), \ldots, z^n g_n(z)] = [g_0(z), \ldots g_n(z)]$, при $z \not = 0$.\\
Но при $z = 0$ определено значение $[g_0(z), \ldots, g_n(z)]$, более того $m = \min \nu_p (g_i) \not = 0$

\begin{lemma}
$\phi: X \rightarrow \mathbb{P}^n$ -- голоморфное отображение, $\exists f = (f_0, \ldots, f_n), f_i \in \mathcal{M}_X: \phi = \phi_f$. Причем в однозначном смысле $\mathbb{P}^n$, т.е. если $g = (g_0, \ldots, g_n): \phi = \phi_g$, то $\exists h \in \mathcal{M}_X: \forall{i} g_i = hf_i.$
\begin{proof}
    Идея. $z_0 \not = 0, \hspace{5pt} e_i = \frac{z_i}{z_0}.$ \\
    $(z_0, \ldots, z_n)$ -- однородные проективные координаты. \\
    $f_i = \phi \circ e_i$ -- мероморфная.
\end{proof} 
\end{lemma}

\begin{definition}
    Пусть $\phi: X \rightarrow \mathbb{P}^n$ -- голоморфное отображение. $\phi = (f_0, \ldots, f_n), D = - \min_i (f_i)$, т.е. $D: \forall i \: f_i \in L(D) \Leftrightarrow -D \leq (f_i)$. \\
    Пусть $V = <f_0, \ldots, f_n>$ -- линейная оболочка. \\
    Линейной системой отображения $\phi$ называется $|\phi| = \{(g) + D: g \in V \}$ \\ $(|\phi| \subset |D| )$
\end{definition}

\begin{lemma}
    $|\phi|$ корректно определена.
\end{lemma}

\begin{lemma}
    $\phi: X \rightarrow \mathbb{P}^n$ -- голоморфное отображение, то $\forall p \in X \: \exists E \in |\phi|: p \not \in supp E$
\end{lemma}

\begin{proof}
    $\phi = [f_0, \ldots, f_n]$. $D = - \min (f_i)$ \\
    $D(p) = - k$, $k = \min_i \nu_p (f_i) $. Пусть $j$ -- на котором достигается минимум: $\nu_p(f_j) = k$. \\
    Рассмотрим $E = (f_j) + D, E \in |\phi|$. \\
    $E(p) = \nu_p(f_j) + D(p) = k - k = 0 \Rightarrow p \not \in supp D$
\end{proof}

\begin{definition}
    $\Lambda$ - линейная система, $p \in X$ называется базисзной (особой) для $\Lambda$, если $\forall E \in \Lambda$. $p \in supp E$ \\

    $\Lambda$ называется линейной системой без базисных точек если не существует базисных точек.
\end{definition}

\begin{corollary}
    $|\phi|$ -- линейная система без базисных точек.
\end{corollary}

\begin{lemma}
    $X$ - компактно. $D \in \Div X$ \\
    $p \in X$ -- базисная точка линейной системы $|D| \Leftrightarrow \Dim L(D - p) = L(D)$ \\
    $|D|$ без базисных точек $\Leftrightarrow \forall p: \Dim L(D - p) = L(D) - 1$
\end{lemma}

\newpage

\begin{example}
    \hspace{} \\
    \begin{itemize}
        \item $C_\infty, \forall D, \Deg D \geq 0, \Rightarrow |D|$ без базисных точек.
        \item $\forall D \in \Div \mathbb{C} / L, \Deg D \geq 2 \Rightarrow |D|$ без базисных точек.
    \end{itemize} 
\end{example}

\begin{theorem}
    $X$ -- компактное, $\Lambda \subset |D|$ -- без базисных точек, $\Dim \Lambda = n$. Тогда существует голоморфное отображение $\phi: X \rightarrow \mathbb{P}^n: |\phi| = \Lambda$
\end{theorem}

Линейные системы без базисных точек соотвествуют голоморфным отображениям $\phi: X \rightarrow \mathbb{P}^n$

\begin{theorem}
    $X$ -- компактное, $|D|$ -- без базисных точек. $\phi_D: X \rightarrow \mathbb{P}^n$ голоморфным вложением, т. е. $\phi_D (X) = Y$ -- голоморфно вложенная поверхность.
\end{theorem}

\begin{example}
    \hspace{}
    \\
    \begin{enumerate}
        \item $\mathbb{C}_\infty \Deg D \geq 0, \phi_D$ определяет голоморфное вложение сферы.
        \item $\mathbb{C} / L \Deg D \geq 3, \phi_D$ определяет голоморфное вложение тора.
    \end{enumerate}
\end{example}
\end{document}