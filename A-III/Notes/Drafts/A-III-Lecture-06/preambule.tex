\documentclass[a4paper, 12pt]{article}

\usepackage[english, russian]{babel}
\usepackage[T2A]{fontenc}
\usepackage[utf8]{inputenc}
\usepackage{indentfirst}
\usepackage{amsthm, amsfonts, amssymb, mathtools}
\usepackage{svg}
\usepackage{mathrsfs}  
\usepackage{multirow}
\usepackage{comment}
\usepackage{ wasysym }

\theoremstyle{definition}
\newtheorem{theorem}{Теорема}[section]
\newtheorem{corollary}{Следствие}[theorem]
\newtheorem*{definition}{Определение}
\newtheorem{lemma}[theorem]{Лемма}

\usepackage{geometry}
\geometry{top=25mm}
\geometry{bottom=30mm}
\geometry{left=20mm}
\geometry{right=20mm}

\usepackage{color}   %May be necessary if you want to color links
\usepackage{hyperref}
\hypersetup{
    colorlinks=true, %set true if you want colored links
    linktoc=all,     %set to all if you want both sections and subsections 
    linkcolor=blue,  %choose some color if you want links to stand out
}

\linespread{1}

\usepackage{titleps}
\newpagestyle{main}{
    \setheadrule{0.4pt}
    \sethead{Дифференциальные формы. Дивизоры}{\thepage}{Осень 2025}
    \setfootrule{0.4pt}
    \setfoot{Спецкурс A-III}{\thepage}{ВМК МГУ}
}

\pagestyle{main}

\newcommand{\deriv}[2]{\frac{\partial #1}{\partial #2}}
\newcommand{\R}{\mathbb R}
\newcommand{\Row}{\sum\limits_{n=1}^\infty}
\newcommand{\Rowk}{\sum\limits_{k=1}^\infty}
\newcommand{\Prod}{\prod\limits_{n=1}^\infty}
\newcommand{\Prodk}{\prod\limits_{k=1}^\infty}
\newcommand{\eps}{\varepsilon}
\renewcommand{\phi}{\varphi}
\newcommand{\fall}{\:\forall\:}
\newcommand{\ex}{\:\exists\:}

\DeclareMathOperator{\const}{const}
\DeclareMathOperator{\Ker}{ker}
\DeclareMathOperator{\Image}{im}
\DeclareMathOperator{\Def}{def}
\DeclareMathOperator{\Rank}{rank}
\DeclareMathOperator{\Dim}{dim}
\DeclareMathOperator{\Argmin}{argmin}
\DeclareMathOperator{\Argmax}{argmax}
\DeclareMathOperator{\Exp}{Exp}
\DeclareMathOperator{\Softmax}{Softmax}

\usepackage{subfig}

\renewcommand{\thesubfigure}{Figure \arabic{subfigure}}
\captionsetup[subfigure]{labelformat=simple, labelsep=colon}