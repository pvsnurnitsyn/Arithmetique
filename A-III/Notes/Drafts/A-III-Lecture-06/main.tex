\documentclass[a4paper, 12pt]{article}

\usepackage[english, russian]{babel}
\usepackage[T2A]{fontenc}
\usepackage[utf8]{inputenc}
\usepackage{indentfirst}
\usepackage{amsthm, amsfonts, amssymb, mathtools}
\usepackage{svg}
\usepackage{mathrsfs}  
\usepackage{multirow}
\usepackage{comment}
\usepackage{ wasysym }

\theoremstyle{definition}
\newtheorem{theorem}{Теорема}[section]
\newtheorem{corollary}{Следствие}[theorem]
\newtheorem*{definition}{Определение}
\newtheorem{lemma}[theorem]{Лемма}

\usepackage{geometry}
\geometry{top=25mm}
\geometry{bottom=30mm}
\geometry{left=20mm}
\geometry{right=20mm}

\usepackage{color}   %May be necessary if you want to color links
\usepackage{hyperref}
\hypersetup{
    colorlinks=true, %set true if you want colored links
    linktoc=all,     %set to all if you want both sections and subsections 
    linkcolor=blue,  %choose some color if you want links to stand out
}

\linespread{1}

\usepackage{titleps}
\newpagestyle{main}{
    \setheadrule{0.4pt}
    \sethead{Дифференциальные формы. Дивизоры}{\thepage}{Осень 2025}
    \setfootrule{0.4pt}
    \setfoot{Спецкурс A-III}{\thepage}{ВМК МГУ}
}

\pagestyle{main}

\newcommand{\deriv}[2]{\frac{\partial #1}{\partial #2}}
\newcommand{\R}{\mathbb R}
\newcommand{\Row}{\sum\limits_{n=1}^\infty}
\newcommand{\Rowk}{\sum\limits_{k=1}^\infty}
\newcommand{\Prod}{\prod\limits_{n=1}^\infty}
\newcommand{\Prodk}{\prod\limits_{k=1}^\infty}
\newcommand{\eps}{\varepsilon}
\renewcommand{\phi}{\varphi}
\newcommand{\fall}{\:\forall\:}
\newcommand{\ex}{\:\exists\:}

\DeclareMathOperator{\const}{const}
\DeclareMathOperator{\Ker}{ker}
\DeclareMathOperator{\Image}{im}
\DeclareMathOperator{\Def}{def}
\DeclareMathOperator{\Rank}{rank}
\DeclareMathOperator{\Dim}{dim}
\DeclareMathOperator{\Argmin}{argmin}
\DeclareMathOperator{\Argmax}{argmax}
\DeclareMathOperator{\Exp}{Exp}
\DeclareMathOperator{\Softmax}{Softmax}

\usepackage{subfig}

\renewcommand{\thesubfigure}{Figure \arabic{subfigure}}
\captionsetup[subfigure]{labelformat=simple, labelsep=colon}

\begin{document}

\title{Конспект лекции 6:\\ 
<<Дифференциальные формы. Дивизоры>>}
\author{Иванов Егор Романович, 503 группа}
\date{Осень 2025}
\maketitle

\tableofcontents

\section{Дифференциальные формы}

\subsection{Перенос определений}

\begin{definition}
    Пусть $f:\mathbb C\to\mathbb C$
    -- голоморфна на открытом множестве
    $V\subset \mathbb C$.
    Выражение
    $\omega=f(z)dz$
    называют \textit{голоморфной
    дифференциальной 1-формой
    (голоморфным дифференциалом)}.
\end{definition}
\noindentРаспространим данное определение
из курса ТФКП на римановы поверхности. Сделать
это можно с помощью карт.
Если $z=\varphi(x)$, $w=\psi(x)$
-- карты, а $T$ -- функция склейки, то есть
$z=T(w)$, то $dz = T'(w)dw$.

\begin{definition}
    Пусть имеются
    две
    голоморфные дифференциальные 1-формы
    \begin{equation}
        \begin{aligned}
        &\omega_1=f(z)dz\\
        &\omega_2=g(w)dw
        \end{aligned}
    \end{equation}
    Если
    $g(w)=f(T(w))T'(w)$,
    где $T$ -- голоморфная функция, то
    будем говорить, что $\omega_1$ 
    \textit{переходит}
    в $\omega_2$.
\end{definition}

\begin{definition}
    Пусть $X$ -- риманова поверхность.
    \textit{Голоморфной 1-формой} над $X$
    называется такое семейство 
    $(\omega_\varphi)$, что
    для любых карт
    $\varphi_1,\varphi_2$
    верно, что
    $\omega_{\varphi_1}$ 
    переходит в $\omega_{\varphi_2}$
    под действием функции склейки
    $T=\varphi_1\circ\varphi_2^{-1}$.
\end{definition}

\noindent\textbf{Замечание 1.}
Достаточно задать на картах одного
атласа $\mathcal A$.

\noindent\textbf{Замечание 2.}
Множество голоморфных форм
обозначается буквой $\Omega$.

\noindent\textbf{Замечание 3.}
Определения и замечание 1 аналогичны в случае 
мероморфной функции~$f$.
Обозначение: $\mathcal M^{(1)}$.

\begin{definition}
    Пусть
    $\omega\in\mathcal M_p^{(1)}$, $p\in X$.
    Если 
    $z=\varphi(\cdot)$ -- локальная координата,
    в которой форма имеет вид
    $\omega=f(z)dz$, $f\in\mathcal M_p$,
    то число
    $\nu_p(\omega)=\nu_p(f)$ 
    называется
    \textit{порядком} формы~$\omega$.
\end{definition}

\noindent\textbf{Замечание.}
Аналогично можно показать, что порядок
формы корректно определен, то есть
не зависит от выбора карты.

\begin{lemma}
    $\ $
    \begin{enumerate}
    \item  
    Пусть
    $\omega\in\Omega$
    -- голоморфная форма,
    $h\in\mathcal O$
    -- голоморфная функция,
    тогда $h\omega\in\Omega$.
    \item 
    Аналогично, 
    если $\omega\in\mathcal M^{(1)}$,
    $h\in\mathcal M$,
    то $h\omega\in\mathcal M^{(1)}$.
    \item 
    Если $\omega\in\mathcal M_p^{(1)}$,
    $h\in\mathcal M$,
    то 
    $\nu_p(h\omega) 
    =\nu_p(h)+\nu_p(\omega)$.
    \end{enumerate}
\end{lemma}

\begin{definition}
    Рассмотрим отображение 
    $d:f\mapsto f'(z)dz$,
    действующее из
    $\mathcal M$ в $\mathcal M^{(1)}$.
    Если форма $\omega$
    содержится в $d$, то есть $\omega=df$,
    то она 
    называется \textit{точной}.
\end{definition}

\noindentРассмотрим \textbf{примеры}:
$\mathbb C_\infty$ -- риманова сфера
\begin{itemize}
    \item Пусть $\omega=dz$.
    В любой конечной точке $p\in\mathbb C$
    верно, что
    $\nu_p(\omega)>0$. В окрестности~$\infty$:
    \[
        z = \frac{1}{w}
        \Rightarrow
        dz = -\frac{1}{w^2}dw,
    \]
    откуда $\nu_\infty(dz)=-2$.
    \item Пусть $\omega=f\:dz$.
    Если $f$ голоморфна на $\mathbb C$
    и не является тождественной константой,
    то $\nu_\infty(f)<0$. В окрестности
    нуля верно, что
    \begin{equation}
        \omega=
        -f
        \left(\frac{1}{w}\right)
        \frac{1}{w^2}dw
        \Rightarrow
        \nu_\infty(w)=-2
    \end{equation}
    \item Пусть $\omega=\frac{1}{z}dz$.
    Эта форма
    не является точной, так
    как все мероморфные функции на сфере
    -- рациональные функции.
\end{itemize}

\subsection{Интегрирование 1-форм}

\noindentВ комплексном анализе 
дифференциальные формы
-- инструмент для контурного интегрирования,
поэтому определим
соответствующие понятия и для 
случая римановых поверхностей.

\noindentВ $\mathbb R^2$ 1-формы
имеют вид 
$f(x,y)dx+g(x,y)dy$. В этой
модели можно смотреть на $\mathbb C$
как на плоскость $\mathbb R^2$:
\begin{equation}
    z = x+iy, \quad \bar z=x-iy \quad
    \Leftrightarrow
    \quad
    dx = \frac{dz+d\bar z}{2},
    \quad
    dy = \frac{dz-d\bar z}{2i}
\end{equation}
В данных обозначениях условия
Коши-Римана можно записать, как 
\begin{equation}
    \frac{\partial f}{\partial \bar z} = 0
\end{equation}
При их выполнении функция 
$f(z)=f(x,y)=f(z,\bar z)$ является
бесконечно-дифференцируемой как отображение
на плоскости.
Будем обозначать класс таких функций
$C^\infty$.

\begin{definition}
    $C^\infty$-1-формы
    $\omega=f(z,\bar z)dz + g(z,\bar z)d\bar z$, где $f,g\in C^\infty$.
\end{definition}
\noindentАналогично 
данное понятие распространяется 
на римановы поверхности с помощью карт.

\begin{definition}
    $C^\infty$-формы это семейство
    форм $(\omega_\varphi)$ в каждой карте
    таких, что для любых карт $\varphi, \psi$
    $\omega_\varphi$ переходит
    в $\omega_\psi$. Под
    этим понимается, что
    $f_2(w,\bar w)
    =f_1(T(w), \overline{T(w)})T'(w)$.
    Аналогично для $g_2$.
\end{definition}

\begin{definition}
    Пусть $X$ -- риманова поверхность.
    \textit{Путь} на $X$ --
    $C^\infty$ функция $\gamma:[a,b]\to X$.
\end{definition}

\noindentТак как путь может
пролегать по нескольким картам,
необходим аппарат для перехода между ними.

\begin{definition}
    $\mathcal A$ -- атлас на
    римановой поверхности $X$, 
    $\gamma$ - путь на $X$.
    \textit{Разбиением} $\gamma$ 
    называется $(\gamma_i)_{i=1}^n$:
    $[a,b]=\sqcup_i\:[a_i,b_i]$, $\gamma([a_i,b_i])\subset U_i$,
    где $U_i$ -- область определения
    некоторой карты.
\end{definition}

\begin{definition}
    Пусть
    $\omega$ -- $C^\infty$-форма такая, что
    в локальных координатах 
    $(\varphi_i)_{i=1}^n$ она имеет вид
    $\omega=f_i(z,\bar z)dz
    + g_i(z,\bar z)d\bar z$,
    а $z = \varphi_i\circ\gamma_i:
    [a_i,b_i]\to V_i\subset \mathbb C$.
    Тогда \textit{интегралом}
    \textit{формы} $\omega$ 
    \textit{по пути} $\gamma$
    называется:
    \begin{equation}
        \int_\gamma\omega = 
        \sum_{i=1}^n
        \int_{a_i}^{b_i}
        f_i(z(t), \overline{z(t)}) z'(t)
        +
        g_i(z(t), \overline{z(t)}) z'(t)
        \: 
        dt
    \end{equation}
\end{definition}

\begin{lemma}
    Интеграл $\int_\gamma\omega$ не зависит от параметризации
    и разбиения $\gamma$.
\end{lemma}

\begin{lemma}
    Если $f$ -- $C^\infty$-форма,
    то 
    $\int_\gamma f = 
    f(\gamma(b)) - f(\gamma(a))$.
\end{lemma}

\begin{lemma}
    Интеграл аддитивен по пути,
    то есть если 
    $\gamma = \sqcup_{i=1}^n \gamma_i$,
    то $\int_\gamma\omega = 
    \sum_{i=1}^n\int_{\gamma_i}\omega$
\end{lemma}

\begin{lemma}
    $\int_{\gamma^-}\omega=-\int_\gamma\omega$,
    где $\gamma^-$ -- это кривая $\gamma$
    с обратным направлением обхода.
\end{lemma}

\begin{definition}
    Пусть $\omega\in\mathcal M_p^{(1)}$,
    $z=\varphi$ -- локальная координата
    с центром в точке $p$.
    Если $\nu_p(\omega)=-N<0$,
    то есть если $\omega=fdz$,
    то $f$ раскладывается в ряд Лорана
    со старшей отрицательной степенью $-N$,
    то $c_{-1}$ -- коэффициент 
    при $1/z$ -- называется  
    вычетом и обозначается
    $\mathrm{Res}_p(\omega)$.
\end{definition}

\begin{lemma}
    Вычет корректно определен
    и верно следующее представление:
    \begin{equation}
        \mathrm{Res}_p(\omega)
        =
        \frac{1}{2\pi i}
        \oint_\gamma\omega,
    \end{equation}
    где $\gamma$ -- замкнутый
    контур вокруг $p$, внутри которого
    нет полюсов, кроме, быть может, 
    самой точки $p$.
\end{lemma}

\noindentПерейдем
к рассмотрению формулы Стокса,
которая использует 2-формы.
В случае $C^\infty$ они имеют вид
$h(x,y)dxdy$, в комплексном анализе -- 
$f(z,\bar z)\:dz\:\wedge\:d\bar z$.
Данное определение
по аналогии так же распространяется 
на римановы поверхности.

\begin{theorem}[Формула Стокса]
    \begin{equation}
        \iint_D d\omega = 
        \int_{\partial D} \omega,
    \end{equation}
    где 
    \begin{equation}
        d\omega
        =
        \left(
            \frac{\partial g}{\partial z}
            -
            \frac{\partial f}{\partial \bar z}
        \right)
        \:dz\:\wedge\:d\bar z
    \end{equation}
\end{theorem}

\begin{theorem}[о вычетах]
    Пусть $X$ -- компактная риманова
    поверхность,
    $\omega\in\mathcal M_p^{(1)}$,
    тогда 
    \begin{equation}
        \sum_{p\in X}
        \mathrm{Res}_p(\omega)=0
    \end{equation}
\end{theorem}

\begin{proof}
    Так как $X$ -- компактно,
    то множество нулей и полюсов 
    конечно.
    Пусть $p_1,\dots,p_n$ -- полюса.
    Возьмем для каждого полюса $p_i$
    контур $\gamma_i$, не содержащий
    другие полюса $p_j,\:j\ne i$,
    обозначим за $U_i$ внутренности
    $\gamma_i$, то есть 
    $\partial U_i=\gamma_i$.
    Положим $D = X\:\backslash\:
    \sqcup_{i=1}^n U_i$, которое
    так же будет компактно, причем
    верно, что
    $\partial D = (\sqcup_i \gamma_i)^-$.
    Тогда
    \begin{equation}
        \sum_p \mathrm{Res}_p(\omega)
        =
        \frac{1}{2\pi i}
        \sum_{i=1}^n
        \oint_{\gamma_i}\omega
        = 
        -\frac{1}{2\pi i}
        \int_{(\sqcup\gamma_i)^-} \omega
        =
        -\frac{1}{2\pi i} \int_{\partial D}
        \omega
        =
        -\frac{1}{2\pi i} \iint_{D}\omega = 0
    \end{equation}
\end{proof}

\begin{lemma}
    Если $f\in\mathcal M_p$,
    $\omega=df/f$, то
    \begin{equation}
        \mathrm{Res}_p
        \left(
        \frac{df}{f}
        \right)
        =\nu_p(f)
    \end{equation}
\end{lemma}

\begin{theorem}
    Пусть $X$ -- компактная риманова
    поверхность, $f\in\mathcal M_X$.
    Тогда $\sum_p\nu_p(f)=0$.
\end{theorem}

\begin{proof}
    Рассмотрим форму 
    $df/f\in\mathcal M_X^{(1)}$. Тогда
    \begin{equation}
        0 = 
        \sum_p\mathrm{Res}_p
        \frac{df}{f} = 
        \sum_p \nu_p(f)
    \end{equation}
\end{proof}

\begin{lemma}
    Пусть 
    $\omega_1,\omega_2\in\mathcal M_X^{(1)}$,
    $\omega_1\ne 0$, тогда
    существует $f\in\mathcal M_X$ 
    такая, что $\omega_2=f\omega_1$.
\end{lemma}

\section{Дивизоры}

\noindentПусть $X$ -- компактная 
риманова поверхность.

\begin{definition}
    \textit{Группой дивизоров}
    $\mathrm{Div}(X)$
    называется
    свободная абелева группа,
    порожденная всеми точками
    $X$. Иными словами, если
    $D\in\mathrm{Div}(X)$, 
    то $D$ представим в следующем виде:
    \begin{equation}
        D = \sum_{p\in X} n_p\cdot p,
    \end{equation}
    где $n_p\ne 0$ для конечного
    числа слагаемых.
    \textit{Носителем} дивизора
    называется множество:
    \begin{equation}
        \mathrm{supp}\: D
        = 
        \{p\in X: n_p\ne0\}
    \end{equation}
    \textit{Степенью} дивизора
    \begin{equation}
        \mathrm{deg} D
        =
        \sum_{p\in D} n_p
    \end{equation}
\end{definition}

\noindent\textbf{Замечание.}
По сути $D$ -- это функция 
$X\to\mathbb Z$ с конечным носителем,
а выражение $\sum n_p \:p$ это удобная
запись для $\sum D(p)\:p$.

\begin{definition}
    В пространстве функций
    можно задать \textit{отношение 
    частичного порядка}:
    $D_1\le D_2$ тогда и только тогда,
    когда $D_1(p)\le D_2(p)$ для всех $p$.
\end{definition}

\begin{definition}
    Дивизор вида $D=p$ называется
    \textit{простым} дивизором.
\end{definition}
\begin{definition}
    Если $f$ -- некоторая мероморфная функция,
    то ее дивизор можно определить следующим
    образом:
    \begin{equation}
        \mathrm{div}(f)=(f) =
        \sum_p \nu_p(f)\:p
    \end{equation}
    Обратно, всякий дивизор
    вида $D=(f)$ для некоторой $f$
    называется \textit{главным} дивизором.
    \textit{Дивизором нулей} будем
    называть
    \begin{equation}
        (f)_0 = \sum_{\nu_p(f)>0}
        \nu_p(f) \: p
    \end{equation}
    Аналогично \textit{дивизор полюсов}
    \begin{equation}
        (f)_\infty = \sum_{\nu_p(f)<0}
        (-\nu_p(f)) \: p
    \end{equation}
    Несложно убедиться в том, что
    $(f)=(f)_0-(f)_{\infty}$.
\end{definition}

\begin{lemma}
    $\ $
    \begin{enumerate}
    \item 
    $(fg)=(f)+(g)$, $(f/g)=(f)-(g)$
    \item 
    $\mathrm{deg}\:(f)
    =
    \sum\nu_p(f)=0$
    \end{enumerate}
\end{lemma}

\noindentОпределение можем распространить
на формы.

\begin{definition}
    Если $\omega\in\mathcal M_X^{(1)}$,
    то
    ее \textit{дивизором} будем
    называть
    \begin{equation}
        (\omega)=
        \mathrm{div}(\omega)
        =
        \sum \nu_p(\omega)\:p
    \end{equation}
\end{definition}

\noindentРассмотрим \textbf{примеры}.
Пусть $f\in\mathcal M_{\mathbb C_{\infty}}$.
Тогда $f$ -- рациональная функция,
то есть она представима в виде
\begin{equation}
    f(z)=c\prod_{i=1}^n(z-\lambda_i)^{e_i},\:
    e_i\in\mathbb Z
\end{equation}
Тогда для ее дивизора верно
\begin{equation}
    (f)=\sum_{i=1}^n e_i\lambda_i-
    \left(\sum_{i=1}^ne_i\right)\infty
\end{equation}
Если дополнительно рассмотреть
форму $\omega=fdz$, то
\begin{equation}
    (\omega)=(f)+(dz) = 
    \sum e_i\lambda_i -
    \left(\sum e_i - 2\right)\infty,
\end{equation}
откуда $\mathrm{deg}(\omega)=-2$.

\noindentВведем дополнительные обозначения:
\begin{enumerate}
    \item 
    $\mathrm{Div}_0(X)
    = \left\{
    D\in\mathrm{Div}(X): \mathrm{deg} \:D=0
    \right\}$
    \item 
    $\mathrm{PDiv}(X)
    = \left\{
    D\in\mathrm{Div}(X): D=(f)
    \right\}$
    \item 
    $\mathrm{KDiv}(X)
    = \left\{
    D\in\mathrm{Div}(X): D=(\omega)
    \right\}$
\end{enumerate}

\begin{definition}
    Фактор-группа
    $\mathrm{Div}(X)/\mathrm{PDiv}(X)$
    называется \textit{группой
    классов дивизоров}
    и обозначается $\mathrm{Cl}(X)$.
    Это соответствует
    отношению эквивалентности $D_1\sim D_2$,
    если $(D_1)~-~(D_2)=(f)$.
\end{definition}

\begin{lemma}
    Степени эквивалентных дивизоров равны.
\end{lemma}

\begin{proof}
    Верно в силу аддитивности
    степени и равенства нулю
    степени любого главного дизизора.
\end{proof}

\begin{lemma}
    Дивизоры любых форм эквивалентны.
\end{lemma}

\begin{proof}
    По ранее доказанной лемме,
    существует такая $f$, что
    $\omega_2=f\omega_1$, откуда
    и следует утверждение леммы.
\end{proof}

\begin{definition}
    Дивизоры $D$ вида $(\omega)$
    называются \textit{каноническими}.
\end{definition}

\begin{theorem}
    Пусть $X=\mathbb C_\infty$,
    тогда 
    $D\in\mathrm{PDiv}(X)
    \Leftrightarrow 
    \mathrm{deg}\:D = 0
    $.
\end{theorem}

\begin{proof}
    Необходимость была доказана
    ранее,
    покажем достаточность.
    Представим дивизор в следующем виде
    \begin{equation}
        D=\sum e_i\lambda_i + e_{\infty}\infty
    \end{equation}
    Степень дивизора равна 0 тогда 
    и только тогда, когда 
    $e_\infty=-\sum e_i$. В таком
    случае искомой будет
    мероморфная функция
    \begin{equation}
        f=\prod_i (z_i-\lambda_i)^{e_i}
    \end{equation}
\end{proof}

\noindentРассмотрим аналогичную теорему для
тора $X=\mathbb C/L$. Для этого
введем 

\begin{definition}
    Отображение 
    $A:\mathrm{Div}(\mathbb C/L)\to
    \mathbb C/L$, ставящее 
    в соответствие формальной сумме
    сумму комплексных чисел по модулю решетки,
    являющееся гомоморфизмом этих групп,
    называется 
    \textit{отображением Абеля(-Якоби)}.
\end{definition}

\begin{theorem}
    Пусть $X=\mathbb C/L$,
    тогда 
    $D\in\mathrm{PDiv}(X)
    \Leftrightarrow 
    \mathrm{deg}\:D = 0
    $ и $A(D)=0$.
\end{theorem}

\begin{proof}
    Покажем необходимость.
    Пусть тор $L$ задается решеткой 
    $\langle1,\tau\rangle$,
    где $\tau~\in~\mathbb H$ -- 
    число из верхней комплексной полуплоскости,
    $\pi:\mathbb C\to\mathbb C/L$ --
    естественная проекция.
    Введем $h:\mathbb C\to\mathbb C$,
    как $h\circ\pi=f$.
    Для любой точки $z_0\in\mathbb C$
    мы можем рассмотреть следующий контур
    $\gamma_{z_0}$
    -- фундаментальный параллелепипед
    решетки на точках
    $z_0, z_0+1, z_0+1+\tau, z_0+\tau$.
    Обозначим его внутренность $D$.
    Так как множество нулей и полюсов дискретно,
    то можно выбрать $\gamma_{z_0}$ так, что
    внутри контура нет нулей и полюсов. 
    \begin{equation}
        L\ni\int_\gamma z\frac{h'(z)}{h(z)}dz
        =
        \sum_{z\in D}\nu_z(h) z,
    \end{equation}
    что и означает, что $(f)$ лежит
    в ядре отображения Абеля.

    \noindentПокажем достаточность.
    Запишем дивизор $D$ в следующем
    виде: $D=\sum_i(p_i-q_i)$,
    возьмем прообразы 
    \begin{equation}
        \begin{aligned}
            &\pi^{-1}(p_i)=z_i\\
            &\pi^{-1}(q_i)=w_i
        \end{aligned}
    \end{equation}
    Тогда искомой функцией будет
    \begin{equation}
        f=\frac{\prod_i \Theta^{(z_i)}(z)}
        {\prod_i \Theta^{(w_i)}(z)}
    \end{equation}
\end{proof}

\end{document}